\chapter{Alternative FMS Failure Events}\label{sec:application_events}
This section details alternative failure events investigated and their associated miss distance calculations when stress testing the trajectory predictions in a flight management system (FMS).
Ultimately, analysis of these failure events showed their inadequacy and arc length failures were selected as the primary event.

\section{Tangency Kinks}
Tangency kinks arise when a small angle threshold is exceeded between the inbound straight segment and the tangent of the turn segment at a given waypoint. The failure event occurs when this angle difference $\theta$ is above the threshold $\tau_k = 0.001$ \si{\radian}.

\begin{figure}[!ht]
\centering
\resizebox{0.5\columnwidth}{!}{\begin{tikzpicture}
  [
    scale=16, every node/.style={scale=2},
    >=stealth,
    point/.style = {draw, circle,  fill = black, inner sep = 1pt},
    dot/.style   = {draw, circle,  fill = black, inner sep = .2pt},
    latlon/.style = {draw, rectangle, minimum width=0.0001mm, minimum height=2mm, fill=white, fill opacity=0.75, fill = black, inner sep = 0.1pt},
    latlonstart/.style = {draw, rectangle, minimum width=0.0001mm, minimum height=2mm, fill = black, inner sep = 0.3pt},
    wpt/.style  = {diamond, fill=white, fill opacity=0.75},
  ]

  \tikzstyle{every node}=[font=\Large]
 
  % the circle
  \def\unit{0.5}
  \def\rad{0.3}
  \def\ang{45}
  \def\failang{7}
  \def\inner{0.3*\rad}
  \def\failinner{0.6*\rad}

  \node (origin) at (0,0) [point, label = {below:$c$}]{};

  \node (initial) at +(-\unit,\rad) [latlon, label = above:$p$] {};
  \node [rotate=-\failang] (failstart) at +(0, \rad) [latlonstart, red] {};
  \draw[red, rotate around={-\failang:(failstart)}] (failstart) arc (90:\ang:\rad);

  \node (start) at +(0, \rad) [latlonstart, label = above:$s$] {};
  \node [rotate=-\ang] (end) at +(\ang:\rad) [latlon, label ={[label distance=-1mm]above:$e$}] {};
  \node [rotate around={-\failang:(start)}] (failend) at +(\ang:\rad) [rotate=-\ang, latlon,red] {};

  \node (failorigin) at (0,0) [point, red, rotate around={-\failang:(start)}]{};

  \draw (start) arc (90:\ang:\rad);

  \draw[-] (initial) -- node (ls) [label = {above:$\ell_1$}] {} (start);

  \draw[dotted,thick,->] (start) -- node (tangent) [label={[label distance=-2.8mm,text=red, xshift=6.5mm, yshift=0.5mm]below right:\large$\theta$}] {} (1.25*\rad,\rad);


  \draw[dashed]
    ($ (origin) ! 1 ! (start) $)
    -- ($(start) ! 1 ! (origin)$ )
    node (center) [right,label={[pos=0.35]right:$r$}] {};

  \draw[dashed]
    ($ (origin) ! 1 ! (end) $)
    -- ($(end) ! 1 ! (origin)$ )
    node [right] {};

  \draw[dotted,thick, red]
    ($ (failorigin) ! 1 ! (start) $)
    -- ($(start) ! 1 ! (failorigin)$ )
    node (center) [right] {};

  \draw[dotted,thick, red]
    ($ (failorigin) ! 1 ! (failend) $)
    -- ($(failend) ! 1 ! (failorigin)$ )
    node [right] {};

  \draw[->, rotate around={\ang:(end)}] (end) -- node (nextstraight) [very near end, label = {above:$\ell_2$},] {} (end |-, 0);
  \draw[red, ->, rotate around={\ang-\failang:(failend)}] (failend) -- node (failnextstraight) [red] {} (failend |-, 0);
  \draw [red] (1.25*\rad*5.06/7,\rad) arc (0:-\failang*5.06/7:1.25*\rad); % "fail" angle (i.e., kink)
  \draw[dotted,thick,->,red, rotate around={-\failang:(start)}] (start) -- node (fail) [] {} (1.25*\rad,\rad);

  % Right angle symbols
  \def\ralen{.2ex}  % length of the short segment
  \foreach \inter/\first/\last in {start/origin/tangent}
    {
      \draw let \p1 = ($(\inter)!\ralen!(\first)$), % point along first path
                \p2 = ($(\inter)!\ralen!(\last)$),  % point along second path
                \p3 = ($(\p1)+(\p2)-(\inter)$)      % corner point
            in
              (\p1) -- (\p3) -- (\p2)               % path
              ;
    }

  \draw [dashed] (0, \inner) arc (90:\ang:\inner);
  \node (innerangle) at +(90-\ang+\ang/1.4:\inner) [label = {[label distance=-3mm]above right:$\alpha$}] {};

  \node (mid) at +(90-\ang/2:\rad) [draw, wpt, inner sep=1pt, rotate=90-\ang/2] {};

\end{tikzpicture}}
\caption{Tangency kink failure event and miss distance.}
\label{fig:tangency_kink}
\end{figure}

Miss distance for the tangency kinks is calculated by first determining the azimuth angle $z$ between the start point of the arc $s$ and the center point $c$. The azimuth is calculated using a WGS84 Earth flattening leveraging the Geodesic.jl\footnote{\url{https://github.com/anowacki/Geodesics.jl}} package which implements Vincenty's formula  \cite{vincenty}. The tangency angle $\theta$ is then calculated for each $\ell \in L$ where $L$ is the set of lateral packets that each have a straight segment and a turn arc with radius $r$. The sign of the radius $r$ indicates the direction of turn, where left is negative and right is positive. The course-in angle of the straight segment $\ell$ is denoted $\angle \ell$, which we then get a distance metric of:
\begin{equation*}
    d = \tau_k - \max\limits_{(\ell,r) \in L} \abs{z - \angle \ell + \sign(r)\frac{\pi}{2}}
\end{equation*}
\phantom{}

\vspace{-6mm}
\section{Course Directions}
In-bound and out-bound course directions may deviate from one another and can be classified as a failure event.
Closely related to the \textit{disconnection} failure event, the course direction failures can arise when two sequential lateral packets are disconnected. This failure specifically looks for angle differences between the course-out $\theta_{\text{out}}$ and course-in $\theta_\text{in}$ directions of the sequential lateral packets. If this angle difference $\omega = \abs{\theta_\text{out} - \theta_\text{in}}$ is above the threshold $\tau_c = 1\si{\degree}$ then it is classified as a failure.

\begin{figure}[!ht]
\centering
\resizebox{0.6\columnwidth}{!}{\begin{tikzpicture}
  [
    scale=16, every node/.style={scale=2},
    >=stealth,
    point/.style = {draw, circle,  fill = black, inner sep = 1pt},
    dot/.style   = {draw, circle,  fill = black, inner sep = .2pt},
    latlon/.style = {draw, rectangle, minimum width=0.0001mm, minimum height=2mm, fill=white, fill opacity=0.75, fill = black, inner sep = 0.1pt},
    latlonstart/.style = {draw, rectangle, minimum width=0.0001mm, minimum height=2mm, fill = black, inner sep = 0.3pt},
    wpt/.style  = {diamond, fill=white, fill opacity=0.75},
  ]
 
  \tikzstyle{every node}=[font=\Large]
 
  % the circle
  \def\unit{0.5}
  \def\rad{0.3}
  \def\ang{45}
  \def\failang{7}
  \def\inner{0.1*\rad}
  \def\failinner{0.6*\rad}

  \node (initial) at +(-\unit,\rad) [latlon, label = above:$p_1$] {};

  \node (start) at +(0, \rad) [latlonstart, label = above:$s_1$] {};
  \node [rotate=-\ang] (end) at +(\ang:\rad) [latlon, label ={[label distance=-1mm]above:$e_1$}] {};
  \draw[->, dashed, rotate around={\ang:(end)}] (end) -- node (nextstraight) [pos=0.75, label={below right:$\;\;\theta_{\text{out}}$},] {} (end |-, \rad/4.25);

  \node (failinitial) at (0,\rad/2) [latlon, label = above:$p_2$] {};
  \node (failstart) at ($(failinitial) + (-0.25, 0)$) [latlonstart, label = above:$s_2$] {};

  \draw (start) arc (90:\ang:\rad);

  \draw[-] (initial) -- node (ls) [label = {above:$\ell_1$}] {} (start);

  \draw[->,dashed] ($(failinitial) + (0.32, 0)$) -- node (dist) [label={right:$\;\theta_{\text{in}}$},pos=0.05] {} (failinitial);
  \draw[-] (failinitial) -- node (fs) [label={above:$\ell_2$}] {} (failstart);

  \draw [red,thick,dashed,rotate=180] (-0.24, -\rad/2) arc (0:135:\inner);
  \node (innerangle) at (0.225, \rad/3.6) [label = {[red,label distance=-3mm]above right:$\omega$}] {};

  \node (mid) at +(90-\ang/2:\rad) [draw, wpt, inner sep=1pt, rotate=90-\ang/2] {};

\end{tikzpicture}
}
\caption{Course direction failure event and miss distance.}
\label{fig:course_direction}
\end{figure}

Miss distance is calculated as how close to the threshold $\tau_c$ is the maximum wrapped angle difference between the course-out angle and course-in angle denoted by $\omega$, namely
\begin{equation*}
    d = \tau_c - \max\limits_{\substack{\theta_\text{out} \in L_i\\\theta_\text{in} \in L_{i+1}}}\abs{\theta_\text{out} - \theta_\text{in}}.
\end{equation*}

\section{Disconnections}
Disconnected lateral packets occur when two sequential lateral packets are not connected end-to-start, thus leaving a distance $\delta$ between them. If this geodesic distance $\delta = \lVert e_i - p_{i+1} \rVert$ is above the threshold $\tau_d = 10$ \si{ft} then a failure occurred.

\begin{figure}[!ht]
\centering
\resizebox{0.6\columnwidth}{!}{\begin{tikzpicture}
  [
    scale=16, every node/.style={scale=2},
    >=stealth,
    point/.style = {draw, circle,  fill = black, inner sep = 1pt},
    dot/.style   = {draw, circle,  fill = black, inner sep = .2pt},
    latlon/.style = {draw, rectangle, minimum width=0.0001mm, minimum height=2mm, fill=white, fill opacity=0.75, fill = black, inner sep = 0.1pt},
    latlonstart/.style = {draw, rectangle, minimum width=0.0001mm, minimum height=2mm, fill = black, inner sep = 0.3pt},
    wpt/.style  = {diamond, fill=white, fill opacity=0.75},
  ]
 
  \tikzstyle{every node}=[font=\large]
 
  % the circle
  \def\unit{0.5}
  \def\rad{0.3}
  \def\ang{45}
  \def\failang{7}
  \def\inner{0.3*\rad}
  \def\failinner{0.6*\rad}

  \node (initial) at +(-\unit,\rad) [latlon, label = above:$p_1$] {};

  \node (start) at +(0, \rad) [latlonstart, label = above:$s_1$] {};
  \node [rotate=-\ang] (end) at +(\ang:\rad) [latlon, label ={[label distance=-1mm]above:$e_1$}] {};

  \node (failinitial) at (0,\rad/2) [latlon, label = above:$p_2$] {};
  \node (failstart) at ($(failinitial) + (-0.25, 0)$) [latlonstart, label = above:$s_2$] {};

  \draw (start) arc (90:\ang:\rad);

  \draw[-] (initial) -- node (ls) [label = {above:$\ell_1$}] {} (start);

  \draw[dashed,red] (end) -- node (dist) [label={[label distance=-1mm]below:$\delta$},pos=0.45] {} (failinitial);
  \draw[-] (failinitial) -- node (fs) [label={above:$\ell_2$}] {} (failstart);

  \node (mid) at +(90-\ang/2:\rad) [draw, wpt, inner sep=1pt, rotate=90-\ang/2] {};

\end{tikzpicture}
}
\caption{Disconnected failure event and miss distance.}
\label{fig:disconnection}
\end{figure}

Miss distance is calculated as the threshold $\tau_d$ minus the maximum distance between the end points $e_i$ (or $s_i$ if no arc is provided) and the initial point $p_{i+1}$ of the next lateral packet:
\begin{equation*}
    d = \tau_d - \max\limits_{\substack{e_i \in L_i\\p_{i+1} \in L_{i+1}}}\lVert e_i - p_{i+1} \rVert
\end{equation*}
