% \begin{abstract}
% The cross-entropy (CE) method is a popular stochastic method for optimization due to its simplicity and effectiveness.
% Designed for rare-event simulations where the probability of a target event occurring is relatively small,
% the CE-method relies on enough objective function calls to accurately estimate the optimal parameters of the underlying distribution. 
% Certain objective functions may be computationally expensive to evaluate, and the CE-method could potentially get stuck in local minima.
% This is compounded with the need to have an initial covariance wide enough to cover the design space of interest.
% We introduce novel variants of the CE-method to address these concerns.
% To mitigate expensive function calls, during optimization we use every sample to build a surrogate model to approximate the objective function.
% The surrogate model augments the belief of the objective function with less expensive evaluations.
% We use a Gaussian process for our surrogate model to incorporate uncertainty in the predictions which is especially helpful when dealing with sparse data.
% To address local minima convergence, we use Gaussian mixture models to encourage exploration of the design space.
% We experiment with evaluation scheduling techniques to reallocate true objective function calls earlier in the optimization when the covariance is the largest.
% To test our approach, we created a parameterized test objective function with many local minima and a single global minimum. Our test function can be adjusted to control the spread and distinction of the minima.
% Experiments were run to stress the cross-entropy method variants and results indicate that the surrogate model-based approach reduces local minima convergence using the same number of function evaluations.
% \end{abstract}


\section{Introduction}
The cross-entropy (CE) method is a probabilistic optimization approach that attempts to iteratively fit a distribution to elite samples from an initial input distribution \cite{rubinstein2004cross,rubinstein1999cross}.
The goal is to estimate a rare-event probability by minimizing the \textit{cross-entropy} between the two distributions \cite{de2005tutorial}.
The CE-method has gained popularity in part due to its simplicity in implementation and straightforward derivation.
The technique uses \textit{importance sampling} which introduces a proposal distribution over the rare-events to sample from then re-weights the posterior likelihood by the \textit{likelihood ratio} of the true distribution over the proposal distribution.

There are a few key assumptions that make the CE-method work effectively.
Through random sampling, the CE-method assumes that there are enough objective function evaluations to accurately represent the objective. 
This may not be a problem for simple applications, but can be an issue for computationally expensive objective functions. 
Another assumption is that the initial parameters of the input distribution are wide enough to cover the design space of interest. For the case with a multivariate Gaussian distribution, this corresponds to an appropriate mean and wide covariance.
In rare-event simulations with many local minima, the CE-method can fail to find a global minima especially with sparse objective function evaluations.

This work aims to address the key assumptions of the CE-method.
We introduce variants of the CE-method that use surrogate modeling to approximate the objective function, thus updating the belief of the underlying objective through estimation.
As part of this approach, we introduce evaluation scheduling techniques to reallocate true objective function calls earlier in the optimization when we know the covariance will be large.
The evaluation schedules can be based on a distribution (e.g., the Geometric distribution) or can be prescribed manually depending on the problem.
We also use a Gaussian mixture model representation of the prior distribution as a method to explore competing local optima.
While the use of Gaussian mixture models in the CE-method is not novel, we connect the use of mixture models and surrogate modeling in the CE-method.
This connection uses each elite sample as the mean of a component distribution in the mixture, optimized through a subroutine call to the standard CE-method using the learned surrogate model.
To test our approach, we introduce a parameterized test objective function called \textit{sierra}.
The sierra function is built from a multivariate Gaussian mixture model with many local minima and a single global minimum.
Parameters for the sierra function allow control over both the spread and distinction of the minima.
Lastly, we provide an analysis of the weak areas of the CE-method compared to our proposed variants.


\section{Related Work} \label{sec:related_work}
The cross-entropy method is popular in the fields of operations research, machine learning, and optimization \cite{kochenderfer2015decision,Kochenderfer2019}.
The combination of the cross-entropy method, surrogate modeling, and mixture models has been explored in other work \cite{bardenet2010surrogating}. 
The work in \cite{bardenet2010surrogating} proposed an adaptive grid approach to accelerate Gaussian-process-based surrogate modeling using mixture models as the prior in the cross-entropy method. They showed that a mixture model performs better than a single Gaussian when the objective function is multimodal.
Our work differs in that we augment the ``elite'' samples both by an approximate surrogate model and by a subroutine call to the CE-method using the learned surrogate model.
Other related work use Gaussian processes and a modified cross-entropy method for receding-horizon trajectory optimization \cite{tan2018gaussian}.
Their cross-entropy method variant also incorporates the notion of exploration in the context of path finding applications.
An approach based on \textit{relative entropy}, described in \cref{sec:background_ce}, proposed a model-based stochastic search that seeks to minimize the relative entropy \cite{NIPS2015_5672}. They also explore the use of a simple quadratic surrogate model to approximate the objective function.
Prior work that relate cross-entropy-based adaptive importance sampling with Gaussian mixture models show that a mixture model require less objective function calls than a na\"ive Monte Carlo or standard unimodal cross-entropy-based importance sampling method \cite{kurtz2013cross,wang2016cross}.


\section{Background} \label{sec:background}
This section provides necessary background on techniques used in this work. We provide introductions to cross-entropy and the cross-entropy method, surrogate modeling using Gaussian processes, and multivariate Gaussian mixture models.

\subsection{Cross-Entropy} \label{sec:background_ce}
Before understanding the cross-entropy method, we first must understand the notion of \textit{cross-entropy}.
Cross-entropy is a metric used to measure the distance between two probability distributions, where the distance may not be symmetric \cite{de2005tutorial}.
The distance used to define cross-entropy is called the \textit{Kullback-Leibler (KL) distance} or \textit{KL divergence}.
The KL distance is also called the \textit{relative entropy}, and we can use this to derive the cross-entropy.
Formally, for a random variable $\mat{X} = (X_1, \ldots, X_n)$ with a support of $\mathcal{X}$, the KL distance between two continuous probability density functions $f$ and $g$ is defined to be:
\begin{align*}
    \mathcal{D}(f, g) &= \E_f\left[\log \frac{f(\vec{X})}{g(\vec{X})} \right]\\
                      &= \int\limits_{\vec{x} \in \mathcal{X}} f(\vec{x}) \log f(\vec{x}) d\vec{x} - \int\limits_{\vec{x} \in \mathcal{X}} f(\vec{x}) \log g(\vec{x}) d\vec{x}
\end{align*}
We denote the expectation of some function with respect to a distribution $f$ as $\E_f$.
Minimizing the KL distance $\mathcal{D}$ between our true distribution $f$ and our proposal distribution $g$ parameterized by $\vec{\theta}$, is equivalent to choosing $\vec\theta$ that minimizes the following, called the \textit{cross-entropy}:
\begin{align*}
    H(f,g) &= H(f) + \mathcal{D}(f,g)\\
           &= -\E_f[\log g(\vec{X})] \tag{using KL distance}\\
           &= - \int\limits_{\vec{x} \in \mathcal{X}} f(\vec{x}) \log g(\vec{x} \mid \vec{\theta}) d\vec{x}
\end{align*}
where $H(f)$ denotes the entropy of the distribution $f$ (where we conflate entropy and continuous entropy for convenience).
This assumes that $f$ and $g$ share the support $\mathcal{X}$ and are continuous with respect to $\vec{x}$.
The minimization problem then becomes:
\begin{equation} \label{eq:min}
\begin{aligned}
    \minimize_{\vec{\theta}} & & - \int\limits_{\vec{x} \in \mathcal{X}} f(\vec{x}) \log g(\vec{x} \mid \vec{\theta}) d\vec{x}
\end{aligned}
\end{equation}
Efficiently finding this minimum is the goal of the cross-entropy method algorithm.


\subsection{Cross-Entropy Method} \label{sec:background_cem}
Using the definition of cross-entropy, intuitively the \textit{cross-entropy method} (CEM or CE-method) aims to minimize the cross-entropy between the unknown true distribution $f$ and a proposal distribution $g$ parameterized by $\vec\theta$.
This technique reformulates the minimization problem as a probability estimation problem, and uses adaptive importance sampling to estimate the unknown expectation \cite{de2005tutorial}.
The cross-entropy method has been applied in the context of both discrete and continuous optimization problems \cite{rubinstein1999cross,kroese2006cross}.


The initial goal is to estimate the probability 
\begin{align*}
    \ell = P_{\vec{\theta}}(S(\vec{x}) \ge \gamma)
\end{align*}
where $S$ can the thought of as an objective function of $\vec{x}$, and $\vec{x}$ follows a distribution defined by $g(\vec{x} \mid \vec{\theta})$.
We want to find events where our objective function $S$ is above some threshold $\gamma$.
We can express this unknown probability as the expectation
\begin{align} \label{eq:expect}
    \ell = \E_{\vec{\theta}}[\mathbbm{1}_{(S(\vec{x}) \ge \gamma)}]
\end{align}
where $\mathbbm{1}$ denotes the indicator function.
A straightforward way to estimate \cref{eq:expect} can be done through Monte Carlo sampling.
But for rare-event simulations where the probability of a target event occurring is relatively small, this estimate becomes inadequate.
The challenge of the minimization in \cref{eq:min} then becomes choosing the density function for the true distribution $f(\vec{x})$. 
Importance sampling tells us that the optimal importance sampling density can be reduced to
\begin{align*}
    f^*(\vec{x}) = \frac{\mathbbm{1}_{(S(\vec{x}) \ge \gamma)}g(\vec{x} \mid \vec{\theta})}{\ell}
\end{align*}
thus resulting in the optimization problem:
\begin{align*}
    \vec{\theta}_g^* &= \argmin_{\vec{\theta}_g} - \int\limits_{\vec{x} \in \mathcal{X}} f^*(\vec{x})\log g(\vec{x} \mid \vec{\theta}_g) d\vec{x}\\
                   &= \argmin_{\vec{\theta}_g} - \int\limits_{\vec{x} \in \mathcal{X}} \frac{\mathbbm{1}_{(S(\vec{x}) \ge \gamma)}g(\vec{x} \mid \vec{\theta})}{\ell}\log g(\vec{x} \mid \vec{\theta}_g) d\vec{x}
\end{align*}
Note that since we assume $f$ and $g$ belong to the same family of distributions, we get that $f(\vec{x}) = g(\vec{x} \mid \vec{\theta}_g)$.
Now notice that $\ell$ is independent of $\vec{\theta}_g$, thus we can drop $\ell$ and get the final optimization problem of:
\begin{align} \label{eq:opt}
    \vec{\theta}_g^* &= \argmin_{\vec{\theta}_g} - \int\limits_{\vec{x} \in \mathcal{X}} \mathbbm{1}_{(S(\vec{x}) \ge \gamma)}g(\vec{x} \mid \vec{\theta}) \log g(\vec{x} \mid \vec{\theta}_g) d\vec{x}\\\nonumber
                   &= \argmin_{\vec{\theta}_g} - \E_{\vec{\theta}}[ \mathbbm{1}_{(S(\vec{x}) \ge \gamma)} \log g(\vec{x} \mid \vec{\theta}_g)]
\end{align}

The CE-method uses a multi-level algorithm to estimate $\vec{\theta}_g^*$ iteratively.
The parameter $\vec{\theta}_k$ at iteration $k$ is used to find new parameters $\vec{\theta}_{k^\prime}$ at the next iteration $k^\prime$.
The threshold $\gamma_k$ becomes smaller that its initial value, thus artificially making events \textit{less rare} under $\vec{X} \sim g(\vec{x} \mid \vec{\theta}_k)$.

In practice, the CE-method algorithm requires the user to specify a number of \textit{elite} samples $m_\text{elite}$ which are used when fitting the new parameters for iteration $k^\prime$.
Conveniently, if our distribution $g$ belongs to the \textit{natural exponential family} then the optimal parameters can be found analytically \cite{Kochenderfer2019}. For a multivariate Gaussian distribution parameterized by $\vec{\mu}$ and $\mat{\Sigma}$, the optimal parameters for the next iteration $k^\prime$ correspond to the maximum likelihood estimate (MLE):
\begin{align*}
    \vec{\mu}_{k^\prime} &= \frac{1}{m_\text{elite}} \sum_{i=1}^{m_\text{elite}} \vec{x}_i\\
    \vec{\Sigma}_{k^\prime} &= \frac{1}{m_\text{elite}} \sum_{i=1}^{m_\text{elite}} (\vec{x}_i - \vec{\mu}_{k^\prime})(\vec{x}_i - \vec{\mu}_{k^\prime})^\top
\end{align*}

The cross-entropy method algorithm is shown in \cref{alg:cem}.
For an objective function $S$ and input distribution $g$, the CE-method algorithm will run for $k_\text{max}$ iterations.
At each iteration, $m$ inputs are sampled from $g$ and evaluated using the objective function $S$.
The sampled inputs are denoted by $\mat{X}$ and the evaluated values are denoted by $\mat{Y}$.
Next, the top $m_\text{elite}$ samples are stored in the elite set $\e$, and the distribution $g$ is fit to the elites.
This process is repeated for $k_\text{max}$ iterations and the resulting parameters $\vec{\theta}_{k_\text{max}}$ are returned.
Note that a variety of input distributions for $g$ are supported, but we focus on the multivariate Gaussian distribution and the Gaussian mixture model in this work.

%% CE-method
\begin{algorithm}[ht]
  \begin{algorithmic}
  \Function{CrossEntropyMethod}{}($S, g, m, m_\text{elite}, k_\text{max}$)
    \For {$k \in [1,\ldots,k_\text{max}]$}
        \State $\mat{X} \sim g(\;\cdot \mid \vec{\theta}_k)$ where $\mat{X} \in \R^m$
        \State $\mat{Y} \leftarrow S(\vec{x})$ for $\vec{x} \in \mat{X}$
        \State $\e \leftarrow$ store top $m_\text{elite}$ from $\mat{Y}$
        \State $\vec{\theta}_{k^\prime} \leftarrow \textproc{Fit}(g(\;\cdot \mid \vec{\theta}_k), \e)$
    \EndFor
    \State \Return $g(\;\cdot \mid \vec{\theta}_{k_\text{max}})$
  \EndFunction
  \end{algorithmic}
  \caption{\label{alg:cem} Cross-entropy method.}
\end{algorithm}


\subsection{Mixture Models}
A standard Gaussian distribution is \textit{unimodal} and can have trouble generalizing over data that is \textit{multimodal}.
A \textit{mixture model} is a weighted mixture of component distributions used to represent continuous multimodal distributions \cite{kochenderfer2015decision}.
Formally, a Gaussian mixture model (GMM) is defined by its parameters $\vec{\mu}$ and $\mat{\Sigma}$ and associated weights $\w$ where $\sum_{i=1}^n w_i = 1$. We denote that a random variable $\mat{X}$ is distributed according to a mixture model as $\mat{X} \sim \operatorname{Mixture}(\vec{\mu}, \vec{\Sigma}, \vec{w})$.
The probability density of the GMM then becomes:
%% Mixture model PDF
\begin{gather*}
    P( \mat{X} = \vec{x} \mid \vec{\mu}, \mat{\Sigma}, \vec{w}) = \sum_{i=1}^n w_i \Normal(\vec{x} \mid \vec{\mu}_i, \mat{\Sigma}_i)
\end{gather*}

To fit the parameters of a Gaussian mixture model, it is well known that the \textit{expectation-maximization (EM)} algorithm can be used \cite{dempster1977maximum,aitkin1980mixture}. 
The EM algorithm seeks to find the maximum likelihood estimate of the hidden variable $H$ using the observed data defined by $E$.
Intuitively, the algorithm alternates between an expectation step (E-step) and a maximization step (M-step) to guarantee convergence to a local minima.
A simplified EM algorithm is provide in \cref{alg:em} for reference and we refer to \cite{dempster1977maximum,aitkin1980mixture} for further reading.

%% Expectation Maximization
\begin{algorithm}[ht]
  \begin{algorithmic}
  \Function{ExpectationMaximization}{$H, E, \vec{\theta}$}
    \For{\textbf{E-step}}
        \State Compute $Q(h) = P(H=h \mid E=e, \vec{\theta})$ for each $h$ % (use any probabilistic inference algorithm)
        \State Create weighted points: $(h,e)$ with weight $Q(h)$
    \EndFor
    \For{\textbf{M-step}}
        \State Compute $\mathbf{\hat{\vec{\theta}}}_{\text{MLE}}$
    \EndFor
    \State Repeat until convergence.
    \State \Return $\mathbf{\hat{\vec{\theta}}}_{\text{MLE}}$
  \EndFunction
  \end{algorithmic}
  \caption{\label{alg:em} Expectation-maximization.}
\end{algorithm}



\subsection{Surrogate Models}
In the context of optimization, a surrogate model $\hat{S}$ is used to estimate the true objective function and provide less expensive evaluations.
Surrogate models are a popular approach and have been used to evaluate rare-event probabilities in computationally expensive systems \cite{li2010evaluation,li2011efficient}.
The simplest example of a surrogate model is linear regression.
In this work, we focus on the \textit{Gaussian process} surrogate model.
A Gaussian process (GP) is a distribution over functions that predicts the underlying objective function $S$ and captures the uncertainty of the prediction using a probability distribution \cite{Kochenderfer2019}.
This means a GP can be sampled to generate random functions, which can then be fit to our given data $\mat{X}$.
A Gaussian process is parameterized by a mean function $\m(\mat{X})$ and kernel function $\mat{K}(\mat{X},\mat{X})$, which captures the relationship between data points as covariance values.
We denote a Gaussian process that produces estimates $\hat{\vec{y}}$ as:
\begin{align*}
\hat{\vec{y}} &\sim\mathcal{N}\left(\vec{m}(\mat{X}),\vec{K}(\mat{X},\mat{X})\right)\\
        &= \begin{bmatrix} % Changed `m` to `n`
            \hat{S}(\vec{x}_1), \ldots, \hat{S}(\vec{x}_n)
        \end{bmatrix}
\end{align*}
where
\begin{gather*}
\vec{m}(\mat{X}) = \begin{bmatrix} m(\vec{x}_1), \ldots, m(\vec{x}_n) \end{bmatrix}\\
\vec{K}(\mat{X}, \mat{X}) = \begin{bmatrix}
         k(\vec{x}_1, \vec{x}_1) & \cdots & k(\vec{x}_1, \vec{x}_n)\\
         \vdots & \ddots & \vdots\\
         k(\vec{x}_n, \vec{x}_1) & \cdots & k(\vec{x}_n, \vec{x}_n)
     \end{bmatrix}
\end{gather*}
We use the commonly used zero-mean function $m(\vec{x}_i) = \vec{0}$.
For the kernel function $k(\vec{x}_i, \vec{x}_i)$, we use the squared exponential kernel with variance $\sigma^2$ and characteristic scale-length $\ell$, where larger $\ell$ values increase the correlation between successive data points, thus smoothing out the generated functions. The squared exponential kernel is defined as:
% Isotropic Squared Exponential kernel (covariance): \exp(-\frac{r^2}{2\ell^2})
\begin{align*}
k(\vec{x},\vec{x}^\prime) = \sigma^2\exp\left(- \frac{(\vec{x} - \vec{x}^\prime)^\top(\vec{x} - \vec{x}^\prime)}{2\ell^2}\right)
\end{align*}
We refer to \cite{Kochenderfer2019} for a detailed overview of Gaussian processes and different kernel functions.



\section{Algorithms} \label{sec:algorithms}
We can now describe the cross-entropy method variants introduced in this work.\footnote{Code available at \href{https://github.com/mossr/CrossEntropyVariants.jl}{https://github.com/mossr/CrossEntropyVariants.jl}}
This section will first cover the main algorithm introduced, the cross-entropy surrogate method (CE-surrogate).
Then we introduce a modification to the CE-surrogate method, namely the cross-entropy mixture method (CE-mixture).
Lastly, we describe various evaluation schedules for redistributing objective function calls over the iterations.

\subsection{Cross-Entropy Surrogate Method} \label{sec:alg_ce_surrogate}
The main CE-method variant we introduce is the cross-entropy surrogate method (CE-surrogate).
The CE-surrogate method is a superset of the CE-method, where the differences lie in the evaluation scheduling and modeling of the elite set using a surrogate model.
The goal of the CE-surrogate algorithm is to address the shortcomings of the CE-method when the number of objective function calls is sparse and the underlying objective function $S$ has multiple local minima.

The CE-surrogate algorithm is shown in \cref{alg:ce_surrogate}.
It takes as input the objective function $S$, the distribution $\M$ parameterized by $\vec{\theta}$, the number of samples $m$, the number of elite samples $m_\text{elite}$, and the maximum iterations $k_\text{max}$.
For each iteration $k$, the number of samples $m$ are redistributed through a call to \smallcaps{EvaluationSchedule}, where $m$ controls the number of true objective function evaluations of $S$. % EvaluationSchedule.
Then, the algorithm samples from $\M$ parameterized by the current $\vec{\theta}_k$ given the adjusted number of samples $m$. % and clamped $m_\text{elite}$.
For each sample in $\mat{X}$, the objective function $S$ is evaluated and the results are stored in $\mat{Y}$.
The top $m_\text{elite}$ evaluations from $\mat{Y}$ are stored in $\e$. 
Using all of the current function evaluations $\mat{Y}$ from sampled inputs $\mat{X}$, a modeled elite set $\bfE$ is created to augment the sparse information provided by a low number of true objective function evaluations.
Finally, the distribution $\M$ is fit to the elite set $\bfE$ and the distribution with the final parameters $\vec{\theta}_{k_\text{max}}$ is returned.

%% CE-surrogate
\begin{algorithm}[ht]
  \begin{algorithmic}
  \Function{CE-Surrogate}{$S$, $\M$, $m$, $m_\text{elite}$, $k_\text{max}$}
    \For {$k \in [1,\ldots,k_\text{max}]$}
        \State $m, m_\text{elite} \leftarrow \textproc{EvaluationSchedule}(k, k_\text{max})$
        \State $\mat{X} \sim \M(\;\cdot \mid \vec{\theta}_k)$ where $\mat{X} \in \R^m$
        \State $\mat{Y} \leftarrow S(\vec{x})$ for $\vec{x} \in \mat{X}$
        \State $\e \leftarrow$ store top $m_\text{elite}$ from $\mat{Y}$
        \State $\bfE \leftarrow \textproc{ModelEliteSet}(\mat{X}, \mat{Y}, \M, \e, m, m_\text{elite})$
        \State $\vec{\theta}_{k^\prime} \leftarrow \textproc{Fit}(\M(\;\cdot \mid \vec{\theta}_k), \bfE)$
    \EndFor
    \State \Return $\M(\;\cdot \mid \vec{\theta}_{k_\text{max}})$
  \EndFunction
  \end{algorithmic}
  \caption{\label{alg:ce_surrogate} Cross-entropy surrogate method.}
\end{algorithm}


The main difference between the standard CE-method and the CE-surrogate variant lies in the call to \smallcaps{ModelEliteSet}.
The motivation is to use \textit{all} of the already evaluated objective function values $\mat{Y}$ from a set of sampled inputs $\mat{X}$.
This way the expensive function evaluations---otherwise discarded---can be used to build a surrogate model of the underlying objective function.
First, a surrogate model $\surrogate$ is constructed from the samples $\mat{X}$ and true objective function values $\mat{Y}$.
We used a Gaussian process with a specified kernel and optimizer, but other surrogate modeling techniques such as regression with basis functions can be used.
We chose a Gaussian process because it incorporates probabilistic uncertainty in the predictions, which may more accurately represent our objective function, or at least be sensitive to over-fitting to sparse data.
Now we have an approximated objective function $\surrogate$ that we can inexpensively call. 
We sample $10m$ values from the distribution $\M$ and evaluate them using the surrogate model.
We then store the top $10m_\text{elite}$ values from the estimates $\mathbf{\hat{\mat{Y}}}_\text{m}$.
We call these estimated elite values $\e_\text{model}$ the \textit{model-elites}.
The surrogate model is then passed to \smallcaps{SubEliteSet}, which returns more estimates for elite values.
Finally, the elite set $\bfE$ is built from the true-elites $\e$, the model-elites $\e_\text{model}$, and the subcomponent-elites $\e_\text{sub}$.
The resulting concatenated elite set $\bfE$ is returned.

\begin{algorithm}[ht]
  \begin{algorithmic}
  \Function{ModelEliteSet}{$\mat{X}, \mat{Y}, \M, \e, m, m_\text{elite}$}
    % Fit to entire population!
    \State $\surrogate \leftarrow \textproc{GaussianProcess}(\mat{X}, \mat{Y}, \text{kernel}, \text{optimizer})$ % Squared exponential, NelderMead
    \State $\mat{X}_\text{m} \sim \M(\;\cdot \mid \vec{\theta}_k)$ where $\mat{X}_\text{m} \in \R^{10m}$
    \State $\mathbf{\hat{\mat{Y}}}_\text{m} \leftarrow \surrogate(\vec{x}_\text{m})$ for $\vec{x}_\text{m} \in \mat{X}_\text{m}$
    \State $\e_\text{model} \leftarrow$ store top $10m_\text{elite}$ from $\mathbf{\hat{\mat{Y}}}_\text{m}$
    \State $\e_\text{sub} \leftarrow \textproc{SubEliteSet}(\surrogate, \M, \e)$
    \State $\bfE \leftarrow \{ \e \} \cup \{ \e_\text{model} \} \cup \{ \e_\text{sub} \}$ \algorithmiccomment{elite set}
    \State \Return $\bfE$
  \EndFunction
  \end{algorithmic}
  \caption{\label{alg:model_elite_set} Modeling elite set.}
\end{algorithm}

To encourage exploration of promising areas of the design space, the algorithm \smallcaps{SubEliteSet} focuses on the already marked true-elites $\e$.
Each elite $e_x \in \e$ is used as the mean of a new multivariate Gaussian distribution with covariance inherited from the distribution $\M$.
The collection of subcomponent distributions is stored in $\m$.
The idea is to use the information given to us by the true-elites to emphasize areas of the design space that look promising.
For each distribution $\m_i \in \m$ we run a subroutine call to the standard CE-method to fit the distribution $\m_i$ using the surrogate model $\surrogate$. 
Then the best objective function value is added to the subcomponent-elite set $\e_\text{sub}$, and after iterating the full set is returned.
Note that we use $\theta_\text{CE}$ to denote the parameters for the CE-method algorithm.
In our case, we recommend using a small $k_\text{max}$ of around $2$ so the subcomponent-elites do not over-fit to the surrogate model but have enough CE-method iterations to tend towards optimal.

\begin{algorithm}[ht]
  \begin{algorithmic}
  \Function{SubEliteSet}{$\surrogate, \M, \e$}
    \State $\e_\text{sub} \leftarrow \emptyset$
    \State $\m \leftarrow \{ e_x \in \e \mid \Normal(e_x, \M.\Sigma) \}$
    \For {$\m_i \in \m$}
        \State $\m_i \leftarrow \textproc{CrossEntropyMethod}(\surrogate, \m_i \mid \theta_{\text{CE}})$
        \State $\e_\text{sub} \leftarrow \{\e_\text{sub}\} \cup \{\textproc{Best}(\m_i)\}$
    \EndFor
    \State \Return $\e_\text{sub}$
  \EndFunction
  \end{algorithmic}
  \caption{\label{alg:sub_elite_set} Subcomponent elite set.}
\end{algorithm}


\subsection{Cross-Entropy Mixture Method} \label{sec:alg_ce_mixture}
We refer to the variant of our CE-surrogate method that takes an input \textit{mixture model} $\M$ as the cross-entropy mixture method (CE-mixture).
The CE-mixture algorithm is identical to the CE-surrogate algorithm, but calls a custom \smallcaps{Fit} function to fit a mixture model to the elite set $\bfE$.
The input distribution $\M$ is cast to a mixture model using the subcomponent distributions $\m$ as the components of the mixture.
We use the default uniform weighting for each mixture component.
The mixture model $\M$ is then fit using the expectation-maximization algorithm shown in \cref{alg:em}, and the resulting distribution is returned.
The idea is to use the distributions in $\m$ that are centered around each true-elite as the components of the casted mixture model.
Therefore, we would expect better performance of the CE-mixture method when the objective function has many competing local minima.
Results in \cref{sec:results} aim to show this behavior.

%% CE-mixture (fit)
\begin{algorithm}[ht]
  \begin{algorithmic}
  \Function{Fit}{$\M, \m, \bfE$}
    \State $\M \leftarrow \operatorname{Mixture}( \m )$
    \State $\mathbf{\hat{\vec{\theta}}} \leftarrow \textproc{ExpectationMaximization}(\M, \bfE)$
    \State \Return $\M(\;\cdot \mid \mathbf{\hat{\vec{\theta}}})$
  \EndFunction
  \end{algorithmic}
  \caption{\label{alg:ce_mixture_fit} Fitting mixture models (used by CE-mixture).}
\end{algorithm}


\subsection{Evaluation Scheduling} \label{sec:alg_eval_schedule}
Given the nature of the CE-method, we expect the covariance to shrink over time, thus resulting in a solution with higher confidence.
Yet if each iteration is given the same number of objective function evaluations $m$, there is the potential for elite samples from early iterations dominating the convergence.
Therefore, we would like to redistribute the objective function evaluations throughout the iterations to use more truth information early in the process.
We call these heuristics \textit{evaluation schedules}.
One way to achieve this is to reallocate the evaluations according to a Geometric distribution.
Evaluation schedules can also be ad-hoc and manually prescribed based on the current iteration.

We provide the evaluation schedule we use that follows a Geometric distribution with parameter $p$ in \cref{alg:evaluation_schedule}.
We denote $G \sim \Geo(p)$ to be a random variable that follows a truncated Geometric distribution with the probability mass function $p_G(k) = p(1 - p)^k$ for $k \in \{0, 1, 2, \ldots, k_\text{max}\}$. % Geo(p) PMF
Note the use of the integer rounding function (e.g., $\round{x}$), which we later have to compensate for towards the final iterations.
Results in \cref{sec:results} compare values of $p$ that control the redistribution of evaluations.


%% EvaluationSchedule
\begin{algorithm}[ht]
  \begin{algorithmic}
  \Function{EvaluationSchedule}{$k, k_\text{max}$}
    \State $G \sim \Geo(p)$
    \State $N_\text{max} \leftarrow k_\text{max} \cdot m$
    \State $m \leftarrow \round{N_\text{max} \cdot p_G(k)}$
    \If{$k = k_\text{max}$}
        \State $s \leftarrow \displaystyle\sum_{i=1}^{k_\text{max}-1} \round{N_\text{max} \cdot p_G(i)}$
        \State $m \leftarrow \min(N_\text{max} - s, N_\text{max} - m)$
    \EndIf
    \State $m_\text{elite} \leftarrow \min(m_\text{elite}, m)$
    \State \Return ($m, m_\text{elite}$) 
  \EndFunction
  \end{algorithmic}
  \caption{\label{alg:evaluation_schedule} Evaluation schedule using a Geometric distr.}
\end{algorithm}


\section{Experiments} \label{sec:experiments}
In this section, we detail the experiments we ran to compare the CE-method variants and evaluation schedules.
We first introduce a test objective function we created to stress the issue of converging to local minima. 
We then describe the experimental setup for each of our experiments and provide an analysis and results.


\subsection{Test Objective Function Generation}
\begin{figure*}[!t]
  \centering
  \resizebox{0.8\textwidth}{!}{\begin{tikzpicture}[,
scale=4]
\begin{groupplot}[group style={horizontal sep=1cm, y descriptions at=edge left, group size=3 by 2}]
\nextgroupplot [height = {8cm}, title = {$\eta=0.5$, decay=1}, xmajorticks=false, width = {8cm}, enlargelimits = false, axis on top]\addplot [point meta min=-0.00615171396062386, point meta max=-8.999491013175768e-6] graphics [xmin=-15, xmax=15, ymin=-15, ymax=15] {figures/cem_variants/tmp_10000000000056.png};

\nextgroupplot [height = {8cm}, title = {$\eta=2.0$, decay=1}, xmajorticks=false, width = {8cm}, enlargelimits = false, axis on top]\addplot [point meta min=-0.01017517566501975, point meta max=-3.188592122982145e-10] graphics [xmin=-15, xmax=15, ymin=-15, ymax=15] {figures/cem_variants/tmp_10000000000057.png};

\nextgroupplot [height = {8cm}, title = {$\eta=6.0$, decay=1}, xmajorticks=false, width = {8cm}, enlargelimits = false, axis on top, colormap={mycolormap}{ rgb(0cm)=(0.993248,0.906157,0.143936) rgb(1cm)=(0.983868,0.904867,0.136897) rgb(2cm)=(0.974417,0.90359,0.130215) rgb(3cm)=(0.964894,0.902323,0.123941) rgb(4cm)=(0.9553,0.901065,0.118128) rgb(5cm)=(0.945636,0.899815,0.112838) rgb(6cm)=(0.935904,0.89857,0.108131) rgb(7cm)=(0.926106,0.89733,0.104071) rgb(8cm)=(0.916242,0.896091,0.100717) rgb(9cm)=(0.906311,0.894855,0.098125) rgb(10cm)=(0.89632,0.893616,0.096335) rgb(11cm)=(0.886271,0.892374,0.095374) rgb(12cm)=(0.876168,0.891125,0.09525) rgb(13cm)=(0.866013,0.889868,0.095953) rgb(14cm)=(0.85581,0.888601,0.097452) rgb(15cm)=(0.845561,0.887322,0.099702) rgb(16cm)=(0.83527,0.886029,0.102646) rgb(17cm)=(0.82494,0.88472,0.106217) rgb(18cm)=(0.814576,0.883393,0.110347) rgb(19cm)=(0.804182,0.882046,0.114965) rgb(20cm)=(0.79376,0.880678,0.120005) rgb(21cm)=(0.783315,0.879285,0.125405) rgb(22cm)=(0.772852,0.877868,0.131109) rgb(23cm)=(0.762373,0.876424,0.137064) rgb(24cm)=(0.751884,0.874951,0.143228) rgb(25cm)=(0.741388,0.873449,0.149561) rgb(26cm)=(0.730889,0.871916,0.156029) rgb(27cm)=(0.720391,0.87035,0.162603) rgb(28cm)=(0.709898,0.868751,0.169257) rgb(29cm)=(0.699415,0.867117,0.175971) rgb(30cm)=(0.688944,0.865448,0.182725) rgb(31cm)=(0.678489,0.863742,0.189503) rgb(32cm)=(0.668054,0.861999,0.196293) rgb(33cm)=(0.657642,0.860219,0.203082) rgb(34cm)=(0.647257,0.8584,0.209861) rgb(35cm)=(0.636902,0.856542,0.21662) rgb(36cm)=(0.626579,0.854645,0.223353) rgb(37cm)=(0.616293,0.852709,0.230052) rgb(38cm)=(0.606045,0.850733,0.236712) rgb(39cm)=(0.595839,0.848717,0.243329) rgb(40cm)=(0.585678,0.846661,0.249897) rgb(41cm)=(0.575563,0.844566,0.256415) rgb(42cm)=(0.565498,0.84243,0.262877) rgb(43cm)=(0.555484,0.840254,0.269281) rgb(44cm)=(0.545524,0.838039,0.275626) rgb(45cm)=(0.535621,0.835785,0.281908) rgb(46cm)=(0.525776,0.833491,0.288127) rgb(47cm)=(0.515992,0.831158,0.294279) rgb(48cm)=(0.506271,0.828786,0.300362) rgb(49cm)=(0.496615,0.826376,0.306377) rgb(50cm)=(0.487026,0.823929,0.312321) rgb(51cm)=(0.477504,0.821444,0.318195) rgb(52cm)=(0.468053,0.818921,0.323998) rgb(53cm)=(0.458674,0.816363,0.329727) rgb(54cm)=(0.449368,0.813768,0.335384) rgb(55cm)=(0.440137,0.811138,0.340967) rgb(56cm)=(0.430983,0.808473,0.346476) rgb(57cm)=(0.421908,0.805774,0.35191) rgb(58cm)=(0.412913,0.803041,0.357269) rgb(59cm)=(0.404001,0.800275,0.362552) rgb(60cm)=(0.395174,0.797475,0.367757) rgb(61cm)=(0.386433,0.794644,0.372886) rgb(62cm)=(0.377779,0.791781,0.377939) rgb(63cm)=(0.369214,0.788888,0.382914) rgb(64cm)=(0.360741,0.785964,0.387814) rgb(65cm)=(0.35236,0.783011,0.392636) rgb(66cm)=(0.344074,0.780029,0.397381) rgb(67cm)=(0.335885,0.777018,0.402049) rgb(68cm)=(0.327796,0.77398,0.40664) rgb(69cm)=(0.319809,0.770914,0.411152) rgb(70cm)=(0.311925,0.767822,0.415586) rgb(71cm)=(0.304148,0.764704,0.419943) rgb(72cm)=(0.296479,0.761561,0.424223) rgb(73cm)=(0.288921,0.758394,0.428426) rgb(74cm)=(0.281477,0.755203,0.432552) rgb(75cm)=(0.274149,0.751988,0.436601) rgb(76cm)=(0.266941,0.748751,0.440573) rgb(77cm)=(0.259857,0.745492,0.444467) rgb(78cm)=(0.252899,0.742211,0.448284) rgb(79cm)=(0.24607,0.73891,0.452024) rgb(80cm)=(0.239374,0.735588,0.455688) rgb(81cm)=(0.232815,0.732247,0.459277) rgb(82cm)=(0.226397,0.728888,0.462789) rgb(83cm)=(0.220124,0.725509,0.466226) rgb(84cm)=(0.214,0.722114,0.469588) rgb(85cm)=(0.20803,0.718701,0.472873) rgb(86cm)=(0.202219,0.715272,0.476084) rgb(87cm)=(0.196571,0.711827,0.479221) rgb(88cm)=(0.19109,0.708366,0.482284) rgb(89cm)=(0.185783,0.704891,0.485273) rgb(90cm)=(0.180653,0.701402,0.488189) rgb(91cm)=(0.175707,0.6979,0.491033) rgb(92cm)=(0.170948,0.694384,0.493803) rgb(93cm)=(0.166383,0.690856,0.496502) rgb(94cm)=(0.162016,0.687316,0.499129) rgb(95cm)=(0.157851,0.683765,0.501686) rgb(96cm)=(0.153894,0.680203,0.504172) rgb(97cm)=(0.150148,0.676631,0.506589) rgb(98cm)=(0.146616,0.67305,0.508936) rgb(99cm)=(0.143303,0.669459,0.511215) rgb(100cm)=(0.14021,0.665859,0.513427) rgb(101cm)=(0.137339,0.662252,0.515571) rgb(102cm)=(0.134692,0.658636,0.517649) rgb(103cm)=(0.132268,0.655014,0.519661) rgb(104cm)=(0.130067,0.651384,0.521608) rgb(105cm)=(0.128087,0.647749,0.523491) rgb(106cm)=(0.126326,0.644107,0.525311) rgb(107cm)=(0.12478,0.640461,0.527068) rgb(108cm)=(0.123444,0.636809,0.528763) rgb(109cm)=(0.122312,0.633153,0.530398) rgb(110cm)=(0.12138,0.629492,0.531973) rgb(111cm)=(0.120638,0.625828,0.533488) rgb(112cm)=(0.120081,0.622161,0.534946) rgb(113cm)=(0.119699,0.61849,0.536347) rgb(114cm)=(0.119483,0.614817,0.537692) rgb(115cm)=(0.119423,0.611141,0.538982) rgb(116cm)=(0.119512,0.607464,0.540218) rgb(117cm)=(0.119738,0.603785,0.5414) rgb(118cm)=(0.120092,0.600104,0.54253) rgb(119cm)=(0.120565,0.596422,0.543611) rgb(120cm)=(0.121148,0.592739,0.544641) rgb(121cm)=(0.121831,0.589055,0.545623) rgb(122cm)=(0.122606,0.585371,0.546557) rgb(123cm)=(0.123463,0.581687,0.547445) rgb(124cm)=(0.124395,0.578002,0.548287) rgb(125cm)=(0.125394,0.574318,0.549086) rgb(126cm)=(0.126453,0.570633,0.549841) rgb(127cm)=(0.127568,0.566949,0.550556) rgb(128cm)=(0.128729,0.563265,0.551229) rgb(129cm)=(0.129933,0.559582,0.551864) rgb(130cm)=(0.131172,0.555899,0.552459) rgb(131cm)=(0.132444,0.552216,0.553018) rgb(132cm)=(0.133743,0.548535,0.553541) rgb(133cm)=(0.135066,0.544853,0.554029) rgb(134cm)=(0.136408,0.541173,0.554483) rgb(135cm)=(0.13777,0.537492,0.554906) rgb(136cm)=(0.139147,0.533812,0.555298) rgb(137cm)=(0.140536,0.530132,0.555659) rgb(138cm)=(0.141935,0.526453,0.555991) rgb(139cm)=(0.143343,0.522773,0.556295) rgb(140cm)=(0.144759,0.519093,0.556572) rgb(141cm)=(0.14618,0.515413,0.556823) rgb(142cm)=(0.147607,0.511733,0.557049) rgb(143cm)=(0.149039,0.508051,0.55725) rgb(144cm)=(0.150476,0.504369,0.55743) rgb(145cm)=(0.151918,0.500685,0.557587) rgb(146cm)=(0.153364,0.497,0.557724) rgb(147cm)=(0.154815,0.493313,0.55784) rgb(148cm)=(0.15627,0.489624,0.557936) rgb(149cm)=(0.157729,0.485932,0.558013) rgb(150cm)=(0.159194,0.482237,0.558073) rgb(151cm)=(0.160665,0.47854,0.558115) rgb(152cm)=(0.162142,0.474838,0.55814) rgb(153cm)=(0.163625,0.471133,0.558148) rgb(154cm)=(0.165117,0.467423,0.558141) rgb(155cm)=(0.166617,0.463708,0.558119) rgb(156cm)=(0.168126,0.459988,0.558082) rgb(157cm)=(0.169646,0.456262,0.55803) rgb(158cm)=(0.171176,0.45253,0.557965) rgb(159cm)=(0.172719,0.448791,0.557885) rgb(160cm)=(0.174274,0.445044,0.557792) rgb(161cm)=(0.175841,0.44129,0.557685) rgb(162cm)=(0.177423,0.437527,0.557565) rgb(163cm)=(0.179019,0.433756,0.55743) rgb(164cm)=(0.180629,0.429975,0.557282) rgb(165cm)=(0.182256,0.426184,0.55712) rgb(166cm)=(0.183898,0.422383,0.556944) rgb(167cm)=(0.185556,0.41857,0.556753) rgb(168cm)=(0.187231,0.414746,0.556547) rgb(169cm)=(0.188923,0.41091,0.556326) rgb(170cm)=(0.190631,0.407061,0.556089) rgb(171cm)=(0.192357,0.403199,0.555836) rgb(172cm)=(0.1941,0.399323,0.555565) rgb(173cm)=(0.19586,0.395433,0.555276) rgb(174cm)=(0.197636,0.391528,0.554969) rgb(175cm)=(0.19943,0.387607,0.554642) rgb(176cm)=(0.201239,0.38367,0.554294) rgb(177cm)=(0.203063,0.379716,0.553925) rgb(178cm)=(0.204903,0.375746,0.553533) rgb(179cm)=(0.206756,0.371758,0.553117) rgb(180cm)=(0.208623,0.367752,0.552675) rgb(181cm)=(0.210503,0.363727,0.552206) rgb(182cm)=(0.212395,0.359683,0.55171) rgb(183cm)=(0.214298,0.355619,0.551184) rgb(184cm)=(0.21621,0.351535,0.550627) rgb(185cm)=(0.21813,0.347432,0.550038) rgb(186cm)=(0.220057,0.343307,0.549413) rgb(187cm)=(0.221989,0.339161,0.548752) rgb(188cm)=(0.223925,0.334994,0.548053) rgb(189cm)=(0.225863,0.330805,0.547314) rgb(190cm)=(0.227802,0.326594,0.546532) rgb(191cm)=(0.229739,0.322361,0.545706) rgb(192cm)=(0.231674,0.318106,0.544834) rgb(193cm)=(0.233603,0.313828,0.543914) rgb(194cm)=(0.235526,0.309527,0.542944) rgb(195cm)=(0.237441,0.305202,0.541921) rgb(196cm)=(0.239346,0.300855,0.540844) rgb(197cm)=(0.241237,0.296485,0.539709) rgb(198cm)=(0.243113,0.292092,0.538516) rgb(199cm)=(0.244972,0.287675,0.53726) rgb(200cm)=(0.246811,0.283237,0.535941) rgb(201cm)=(0.248629,0.278775,0.534556) rgb(202cm)=(0.250425,0.27429,0.533103) rgb(203cm)=(0.252194,0.269783,0.531579) rgb(204cm)=(0.253935,0.265254,0.529983) rgb(205cm)=(0.255645,0.260703,0.528312) rgb(206cm)=(0.257322,0.25613,0.526563) rgb(207cm)=(0.258965,0.251537,0.524736) rgb(208cm)=(0.260571,0.246922,0.522828) rgb(209cm)=(0.262138,0.242286,0.520837) rgb(210cm)=(0.263663,0.237631,0.518762) rgb(211cm)=(0.265145,0.232956,0.516599) rgb(212cm)=(0.26658,0.228262,0.514349) rgb(213cm)=(0.267968,0.223549,0.512008) rgb(214cm)=(0.269308,0.218818,0.509577) rgb(215cm)=(0.270595,0.214069,0.507052) rgb(216cm)=(0.271828,0.209303,0.504434) rgb(217cm)=(0.273006,0.20452,0.501721) rgb(218cm)=(0.274128,0.199721,0.498911) rgb(219cm)=(0.275191,0.194905,0.496005) rgb(220cm)=(0.276194,0.190074,0.493001) rgb(221cm)=(0.277134,0.185228,0.489898) rgb(222cm)=(0.278012,0.180367,0.486697) rgb(223cm)=(0.278826,0.17549,0.483397) rgb(224cm)=(0.279574,0.170599,0.479997) rgb(225cm)=(0.280255,0.165693,0.476498) rgb(226cm)=(0.280868,0.160771,0.472899) rgb(227cm)=(0.281412,0.155834,0.469201) rgb(228cm)=(0.281887,0.150881,0.465405) rgb(229cm)=(0.28229,0.145912,0.46151) rgb(230cm)=(0.282623,0.140926,0.457517) rgb(231cm)=(0.282884,0.13592,0.453427) rgb(232cm)=(0.283072,0.130895,0.449241) rgb(233cm)=(0.283187,0.125848,0.44496) rgb(234cm)=(0.283229,0.120777,0.440584) rgb(235cm)=(0.283197,0.11568,0.436115) rgb(236cm)=(0.283091,0.110553,0.431554) rgb(237cm)=(0.28291,0.105393,0.426902) rgb(238cm)=(0.282656,0.100196,0.42216) rgb(239cm)=(0.282327,0.094955,0.417331) rgb(240cm)=(0.281924,0.089666,0.412415) rgb(241cm)=(0.281446,0.08432,0.407414) rgb(242cm)=(0.280894,0.078907,0.402329) rgb(243cm)=(0.280267,0.073417,0.397163) rgb(244cm)=(0.279566,0.067836,0.391917) rgb(245cm)=(0.278791,0.062145,0.386592) rgb(246cm)=(0.277941,0.056324,0.381191) rgb(247cm)=(0.277018,0.050344,0.375715) rgb(248cm)=(0.276022,0.044167,0.370164) rgb(249cm)=(0.274952,0.037752,0.364543) rgb(250cm)=(0.273809,0.031497,0.358853) rgb(251cm)=(0.272594,0.025563,0.353093) rgb(252cm)=(0.271305,0.019942,0.347269) rgb(253cm)=(0.269944,0.014625,0.341379) rgb(254cm)=(0.26851,0.009605,0.335427) rgb(255cm)=(0.267004,0.004874,0.329415) }, colorbar]\addplot [point meta min=-0.022002368421582257, point meta max=-1.7302938158670046e-22] graphics [xmin=-15, xmax=15, ymin=-15, ymax=15] {figures/cem_variants/tmp_10000000000058.png};

\nextgroupplot [height = {8cm}, title = {$\eta=0.5$, decay=0}, , width = {8cm}, enlargelimits = false, axis on top]\addplot [point meta min=-0.005899996159458542, point meta max=-1.4151795673351503e-11] graphics [xmin=-15, xmax=15, ymin=-15, ymax=15] {figures/cem_variants/tmp_10000000000059.png};

\nextgroupplot [height = {8cm}, title = {$\eta=2.0$, decay=0}, , width = {8cm}, enlargelimits = false, axis on top]\addplot [point meta min=-0.009946798929399269, point meta max=-1.196273392353628e-34] graphics [xmin=-15, xmax=15, ymin=-15, ymax=15] {figures/cem_variants/tmp_10000000000060.png};

\nextgroupplot [height = {8cm}, title = {$\eta=6.0$, decay=0}, , width = {8cm}, enlargelimits = false, axis on top, colormap={mycolormap}{ rgb(0cm)=(0.993248,0.906157,0.143936) rgb(1cm)=(0.983868,0.904867,0.136897) rgb(2cm)=(0.974417,0.90359,0.130215) rgb(3cm)=(0.964894,0.902323,0.123941) rgb(4cm)=(0.9553,0.901065,0.118128) rgb(5cm)=(0.945636,0.899815,0.112838) rgb(6cm)=(0.935904,0.89857,0.108131) rgb(7cm)=(0.926106,0.89733,0.104071) rgb(8cm)=(0.916242,0.896091,0.100717) rgb(9cm)=(0.906311,0.894855,0.098125) rgb(10cm)=(0.89632,0.893616,0.096335) rgb(11cm)=(0.886271,0.892374,0.095374) rgb(12cm)=(0.876168,0.891125,0.09525) rgb(13cm)=(0.866013,0.889868,0.095953) rgb(14cm)=(0.85581,0.888601,0.097452) rgb(15cm)=(0.845561,0.887322,0.099702) rgb(16cm)=(0.83527,0.886029,0.102646) rgb(17cm)=(0.82494,0.88472,0.106217) rgb(18cm)=(0.814576,0.883393,0.110347) rgb(19cm)=(0.804182,0.882046,0.114965) rgb(20cm)=(0.79376,0.880678,0.120005) rgb(21cm)=(0.783315,0.879285,0.125405) rgb(22cm)=(0.772852,0.877868,0.131109) rgb(23cm)=(0.762373,0.876424,0.137064) rgb(24cm)=(0.751884,0.874951,0.143228) rgb(25cm)=(0.741388,0.873449,0.149561) rgb(26cm)=(0.730889,0.871916,0.156029) rgb(27cm)=(0.720391,0.87035,0.162603) rgb(28cm)=(0.709898,0.868751,0.169257) rgb(29cm)=(0.699415,0.867117,0.175971) rgb(30cm)=(0.688944,0.865448,0.182725) rgb(31cm)=(0.678489,0.863742,0.189503) rgb(32cm)=(0.668054,0.861999,0.196293) rgb(33cm)=(0.657642,0.860219,0.203082) rgb(34cm)=(0.647257,0.8584,0.209861) rgb(35cm)=(0.636902,0.856542,0.21662) rgb(36cm)=(0.626579,0.854645,0.223353) rgb(37cm)=(0.616293,0.852709,0.230052) rgb(38cm)=(0.606045,0.850733,0.236712) rgb(39cm)=(0.595839,0.848717,0.243329) rgb(40cm)=(0.585678,0.846661,0.249897) rgb(41cm)=(0.575563,0.844566,0.256415) rgb(42cm)=(0.565498,0.84243,0.262877) rgb(43cm)=(0.555484,0.840254,0.269281) rgb(44cm)=(0.545524,0.838039,0.275626) rgb(45cm)=(0.535621,0.835785,0.281908) rgb(46cm)=(0.525776,0.833491,0.288127) rgb(47cm)=(0.515992,0.831158,0.294279) rgb(48cm)=(0.506271,0.828786,0.300362) rgb(49cm)=(0.496615,0.826376,0.306377) rgb(50cm)=(0.487026,0.823929,0.312321) rgb(51cm)=(0.477504,0.821444,0.318195) rgb(52cm)=(0.468053,0.818921,0.323998) rgb(53cm)=(0.458674,0.816363,0.329727) rgb(54cm)=(0.449368,0.813768,0.335384) rgb(55cm)=(0.440137,0.811138,0.340967) rgb(56cm)=(0.430983,0.808473,0.346476) rgb(57cm)=(0.421908,0.805774,0.35191) rgb(58cm)=(0.412913,0.803041,0.357269) rgb(59cm)=(0.404001,0.800275,0.362552) rgb(60cm)=(0.395174,0.797475,0.367757) rgb(61cm)=(0.386433,0.794644,0.372886) rgb(62cm)=(0.377779,0.791781,0.377939) rgb(63cm)=(0.369214,0.788888,0.382914) rgb(64cm)=(0.360741,0.785964,0.387814) rgb(65cm)=(0.35236,0.783011,0.392636) rgb(66cm)=(0.344074,0.780029,0.397381) rgb(67cm)=(0.335885,0.777018,0.402049) rgb(68cm)=(0.327796,0.77398,0.40664) rgb(69cm)=(0.319809,0.770914,0.411152) rgb(70cm)=(0.311925,0.767822,0.415586) rgb(71cm)=(0.304148,0.764704,0.419943) rgb(72cm)=(0.296479,0.761561,0.424223) rgb(73cm)=(0.288921,0.758394,0.428426) rgb(74cm)=(0.281477,0.755203,0.432552) rgb(75cm)=(0.274149,0.751988,0.436601) rgb(76cm)=(0.266941,0.748751,0.440573) rgb(77cm)=(0.259857,0.745492,0.444467) rgb(78cm)=(0.252899,0.742211,0.448284) rgb(79cm)=(0.24607,0.73891,0.452024) rgb(80cm)=(0.239374,0.735588,0.455688) rgb(81cm)=(0.232815,0.732247,0.459277) rgb(82cm)=(0.226397,0.728888,0.462789) rgb(83cm)=(0.220124,0.725509,0.466226) rgb(84cm)=(0.214,0.722114,0.469588) rgb(85cm)=(0.20803,0.718701,0.472873) rgb(86cm)=(0.202219,0.715272,0.476084) rgb(87cm)=(0.196571,0.711827,0.479221) rgb(88cm)=(0.19109,0.708366,0.482284) rgb(89cm)=(0.185783,0.704891,0.485273) rgb(90cm)=(0.180653,0.701402,0.488189) rgb(91cm)=(0.175707,0.6979,0.491033) rgb(92cm)=(0.170948,0.694384,0.493803) rgb(93cm)=(0.166383,0.690856,0.496502) rgb(94cm)=(0.162016,0.687316,0.499129) rgb(95cm)=(0.157851,0.683765,0.501686) rgb(96cm)=(0.153894,0.680203,0.504172) rgb(97cm)=(0.150148,0.676631,0.506589) rgb(98cm)=(0.146616,0.67305,0.508936) rgb(99cm)=(0.143303,0.669459,0.511215) rgb(100cm)=(0.14021,0.665859,0.513427) rgb(101cm)=(0.137339,0.662252,0.515571) rgb(102cm)=(0.134692,0.658636,0.517649) rgb(103cm)=(0.132268,0.655014,0.519661) rgb(104cm)=(0.130067,0.651384,0.521608) rgb(105cm)=(0.128087,0.647749,0.523491) rgb(106cm)=(0.126326,0.644107,0.525311) rgb(107cm)=(0.12478,0.640461,0.527068) rgb(108cm)=(0.123444,0.636809,0.528763) rgb(109cm)=(0.122312,0.633153,0.530398) rgb(110cm)=(0.12138,0.629492,0.531973) rgb(111cm)=(0.120638,0.625828,0.533488) rgb(112cm)=(0.120081,0.622161,0.534946) rgb(113cm)=(0.119699,0.61849,0.536347) rgb(114cm)=(0.119483,0.614817,0.537692) rgb(115cm)=(0.119423,0.611141,0.538982) rgb(116cm)=(0.119512,0.607464,0.540218) rgb(117cm)=(0.119738,0.603785,0.5414) rgb(118cm)=(0.120092,0.600104,0.54253) rgb(119cm)=(0.120565,0.596422,0.543611) rgb(120cm)=(0.121148,0.592739,0.544641) rgb(121cm)=(0.121831,0.589055,0.545623) rgb(122cm)=(0.122606,0.585371,0.546557) rgb(123cm)=(0.123463,0.581687,0.547445) rgb(124cm)=(0.124395,0.578002,0.548287) rgb(125cm)=(0.125394,0.574318,0.549086) rgb(126cm)=(0.126453,0.570633,0.549841) rgb(127cm)=(0.127568,0.566949,0.550556) rgb(128cm)=(0.128729,0.563265,0.551229) rgb(129cm)=(0.129933,0.559582,0.551864) rgb(130cm)=(0.131172,0.555899,0.552459) rgb(131cm)=(0.132444,0.552216,0.553018) rgb(132cm)=(0.133743,0.548535,0.553541) rgb(133cm)=(0.135066,0.544853,0.554029) rgb(134cm)=(0.136408,0.541173,0.554483) rgb(135cm)=(0.13777,0.537492,0.554906) rgb(136cm)=(0.139147,0.533812,0.555298) rgb(137cm)=(0.140536,0.530132,0.555659) rgb(138cm)=(0.141935,0.526453,0.555991) rgb(139cm)=(0.143343,0.522773,0.556295) rgb(140cm)=(0.144759,0.519093,0.556572) rgb(141cm)=(0.14618,0.515413,0.556823) rgb(142cm)=(0.147607,0.511733,0.557049) rgb(143cm)=(0.149039,0.508051,0.55725) rgb(144cm)=(0.150476,0.504369,0.55743) rgb(145cm)=(0.151918,0.500685,0.557587) rgb(146cm)=(0.153364,0.497,0.557724) rgb(147cm)=(0.154815,0.493313,0.55784) rgb(148cm)=(0.15627,0.489624,0.557936) rgb(149cm)=(0.157729,0.485932,0.558013) rgb(150cm)=(0.159194,0.482237,0.558073) rgb(151cm)=(0.160665,0.47854,0.558115) rgb(152cm)=(0.162142,0.474838,0.55814) rgb(153cm)=(0.163625,0.471133,0.558148) rgb(154cm)=(0.165117,0.467423,0.558141) rgb(155cm)=(0.166617,0.463708,0.558119) rgb(156cm)=(0.168126,0.459988,0.558082) rgb(157cm)=(0.169646,0.456262,0.55803) rgb(158cm)=(0.171176,0.45253,0.557965) rgb(159cm)=(0.172719,0.448791,0.557885) rgb(160cm)=(0.174274,0.445044,0.557792) rgb(161cm)=(0.175841,0.44129,0.557685) rgb(162cm)=(0.177423,0.437527,0.557565) rgb(163cm)=(0.179019,0.433756,0.55743) rgb(164cm)=(0.180629,0.429975,0.557282) rgb(165cm)=(0.182256,0.426184,0.55712) rgb(166cm)=(0.183898,0.422383,0.556944) rgb(167cm)=(0.185556,0.41857,0.556753) rgb(168cm)=(0.187231,0.414746,0.556547) rgb(169cm)=(0.188923,0.41091,0.556326) rgb(170cm)=(0.190631,0.407061,0.556089) rgb(171cm)=(0.192357,0.403199,0.555836) rgb(172cm)=(0.1941,0.399323,0.555565) rgb(173cm)=(0.19586,0.395433,0.555276) rgb(174cm)=(0.197636,0.391528,0.554969) rgb(175cm)=(0.19943,0.387607,0.554642) rgb(176cm)=(0.201239,0.38367,0.554294) rgb(177cm)=(0.203063,0.379716,0.553925) rgb(178cm)=(0.204903,0.375746,0.553533) rgb(179cm)=(0.206756,0.371758,0.553117) rgb(180cm)=(0.208623,0.367752,0.552675) rgb(181cm)=(0.210503,0.363727,0.552206) rgb(182cm)=(0.212395,0.359683,0.55171) rgb(183cm)=(0.214298,0.355619,0.551184) rgb(184cm)=(0.21621,0.351535,0.550627) rgb(185cm)=(0.21813,0.347432,0.550038) rgb(186cm)=(0.220057,0.343307,0.549413) rgb(187cm)=(0.221989,0.339161,0.548752) rgb(188cm)=(0.223925,0.334994,0.548053) rgb(189cm)=(0.225863,0.330805,0.547314) rgb(190cm)=(0.227802,0.326594,0.546532) rgb(191cm)=(0.229739,0.322361,0.545706) rgb(192cm)=(0.231674,0.318106,0.544834) rgb(193cm)=(0.233603,0.313828,0.543914) rgb(194cm)=(0.235526,0.309527,0.542944) rgb(195cm)=(0.237441,0.305202,0.541921) rgb(196cm)=(0.239346,0.300855,0.540844) rgb(197cm)=(0.241237,0.296485,0.539709) rgb(198cm)=(0.243113,0.292092,0.538516) rgb(199cm)=(0.244972,0.287675,0.53726) rgb(200cm)=(0.246811,0.283237,0.535941) rgb(201cm)=(0.248629,0.278775,0.534556) rgb(202cm)=(0.250425,0.27429,0.533103) rgb(203cm)=(0.252194,0.269783,0.531579) rgb(204cm)=(0.253935,0.265254,0.529983) rgb(205cm)=(0.255645,0.260703,0.528312) rgb(206cm)=(0.257322,0.25613,0.526563) rgb(207cm)=(0.258965,0.251537,0.524736) rgb(208cm)=(0.260571,0.246922,0.522828) rgb(209cm)=(0.262138,0.242286,0.520837) rgb(210cm)=(0.263663,0.237631,0.518762) rgb(211cm)=(0.265145,0.232956,0.516599) rgb(212cm)=(0.26658,0.228262,0.514349) rgb(213cm)=(0.267968,0.223549,0.512008) rgb(214cm)=(0.269308,0.218818,0.509577) rgb(215cm)=(0.270595,0.214069,0.507052) rgb(216cm)=(0.271828,0.209303,0.504434) rgb(217cm)=(0.273006,0.20452,0.501721) rgb(218cm)=(0.274128,0.199721,0.498911) rgb(219cm)=(0.275191,0.194905,0.496005) rgb(220cm)=(0.276194,0.190074,0.493001) rgb(221cm)=(0.277134,0.185228,0.489898) rgb(222cm)=(0.278012,0.180367,0.486697) rgb(223cm)=(0.278826,0.17549,0.483397) rgb(224cm)=(0.279574,0.170599,0.479997) rgb(225cm)=(0.280255,0.165693,0.476498) rgb(226cm)=(0.280868,0.160771,0.472899) rgb(227cm)=(0.281412,0.155834,0.469201) rgb(228cm)=(0.281887,0.150881,0.465405) rgb(229cm)=(0.28229,0.145912,0.46151) rgb(230cm)=(0.282623,0.140926,0.457517) rgb(231cm)=(0.282884,0.13592,0.453427) rgb(232cm)=(0.283072,0.130895,0.449241) rgb(233cm)=(0.283187,0.125848,0.44496) rgb(234cm)=(0.283229,0.120777,0.440584) rgb(235cm)=(0.283197,0.11568,0.436115) rgb(236cm)=(0.283091,0.110553,0.431554) rgb(237cm)=(0.28291,0.105393,0.426902) rgb(238cm)=(0.282656,0.100196,0.42216) rgb(239cm)=(0.282327,0.094955,0.417331) rgb(240cm)=(0.281924,0.089666,0.412415) rgb(241cm)=(0.281446,0.08432,0.407414) rgb(242cm)=(0.280894,0.078907,0.402329) rgb(243cm)=(0.280267,0.073417,0.397163) rgb(244cm)=(0.279566,0.067836,0.391917) rgb(245cm)=(0.278791,0.062145,0.386592) rgb(246cm)=(0.277941,0.056324,0.381191) rgb(247cm)=(0.277018,0.050344,0.375715) rgb(248cm)=(0.276022,0.044167,0.370164) rgb(249cm)=(0.274952,0.037752,0.364543) rgb(250cm)=(0.273809,0.031497,0.358853) rgb(251cm)=(0.272594,0.025563,0.353093) rgb(252cm)=(0.271305,0.019942,0.347269) rgb(253cm)=(0.269944,0.014625,0.341379) rgb(254cm)=(0.26851,0.009605,0.335427) rgb(255cm)=(0.267004,0.004874,0.329415) }, colorbar]\addplot [point meta min=-0.02126410264451447, point meta max=-2.738337589257707e-97] graphics [xmin=-15, xmax=15, ymin=-15, ymax=15] {figures/cem_variants/tmp_10000000000061.png};

\end{groupplot}

\end{tikzpicture}
}
  \caption{
    \label{fig:sierra}
    Example test objective functions generated using the sierra function.  
  }
\end{figure*}
To stress the cross-entropy method and its variants, we created a test objective function called \textit{sierra} that is generated from a mixture model comprised of $49$ multivariate Gaussian distributions.
We chose this construction so that we can use the negative peeks of the component distributions as local minima and can force a global minimum centered at our desired $\mathbf{\tilde{\vec{\mu}}}$.
The construction of the sierra test function can be controlled by parameters that define the spread of the local minima.
We first start with the center defined by a mean vector $\mathbf{\tilde{\vec{\mu}}}$ and we use a common covariance $\mathbf{\tilde{\mat{\Sigma}}}$:
\begin{align*}
    \mathbf{\tilde{\vec{\mu}}} &= [\mu_1, \mu_2], \quad \mathbf{\tilde{\mat{\Sigma}}} = \begin{bmatrix}\sigma & 0\\ 0 & \sigma \end{bmatrix}
\end{align*}
Next, we use the parameter $\delta$ that controls the clustered distance between symmetric points:
\begin{align*}
    \mat{G} &= \left\{[+\delta, +\delta], [+\delta, -\delta], [-\delta, +\delta], [-\delta, -\delta]\right\}
\end{align*}
We chose points $\mat{P}$ to fan out the clustered minima relative to the center defined by $\mathbf{\tilde{\vec{\mu}}}$:
\begin{align*}
    \mat{P} &= \left\{[0, 0], [1, 1], [2, 0], [3, 1], [0, 2], [1, 3]\right\}
\end{align*}
The vector $\vec{s}$ is used to control the $\pm$ distance to create an `s' shape comprised of minima, using the standard deviation $\sigma$:
$\vec{s} = \begin{bmatrix}+\sigma, -\sigma \end{bmatrix}$.
We set the following default parameters: standard deviation $\sigma=3$, spread rate $\eta=6$, and cluster distance $\delta=2$.
We can also control if the local minima clusters ``decay'', thus making those local minima less distinct (where $\text{decay} \in \{0, 1\})$.
The parameters that define the sierra function are collected into $\vec{\theta} = \langle \mathbf{\tilde{\vec{\mu}}}, \mathbf{\tilde{\mat{\Sigma}}}, \mat{G}, \mat{P}, \vec{s} \rangle$.
Using these parameters, we can define the mixture model used by the sierra function as:
\begin{gather*}
    \Sierra \sim \operatorname{Mixture}\left(\left\{ \vec{\theta} ~\Big|~ \Normal\left(\vec{g} +  s\vec{p}_i + \mathbf{\tilde{\vec{\mu}}},\; \mathbf{\tilde{\mat{\Sigma}}} \cdot i^{\text{decay}}/\eta \right) \right\} \right)\\
    \text{for } (\vec{g}, \vec{p}_i, s) \in (\mat{G}, \mat{P}, \vec{s})
\end{gather*}
We add a final component to be our global minimum centered at $\mathbf{\tilde{\vec{\mu}}}$ and with a covariance scaled by $\sigma\eta$. Namely, the global minimum is $\vec{x}^* = \E[\Normal(\mathbf{\tilde{\vec{\mu}}}, \mathbf{\tilde{\mat{\Sigma}}}/(\sigma\eta))] = \mathbf{\tilde{\vec{\mu}}}$.
We can now use this constant mixture model with $49$ components and define the sierra objective function $\mathcal{S}(\vec{x})$ to be the negative probability density of the mixture at input $\vec{x}$ with uniform weights:

\begin{align*}
    \mathcal{S}(\vec{x}) &= -P(\Sierra = \vec{x}) = -\frac{1}{|\Sierra|}\sum_{j=1}^{n}\Normal(\vec{x} \mid \vec{\mu}_j, \mat{\Sigma}_j)
\end{align*}
An example of six different objective functions generated using the sierra function are shown in \cref{fig:sierra}, sweeping over the spread rate $\eta$, with and without decay.

\subsection{Experimental Setup} \label{sec:experiment_setup}
Experiments were run to stress a variety of behaviors of each CE-method variant.
The experiments are split into two categories: algorithmic and scheduling.
The algorithmic category aims to compare features of each CE-method variant while holding common parameters constant (for a better comparison).
While the scheduling category experiments with evaluation scheduling heuristics.

\begin{figure*}[!t]
  \centering
    \subfloat[The cross-entropy method.]{%
    \resizebox{0.3\textwidth}{!}{%% Creator: Matplotlib, PGF backend
%%
%% To include the figure in your LaTeX document, write
%%   \input{<filename>.pgf}
%%
%% Make sure the required packages are loaded in your preamble
%%   \usepackage{pgf}
%%
%% Figures using additional raster images can only be included by \input if
%% they are in the same directory as the main LaTeX file. For loading figures
%% from other directories you can use the `import` package
%%   \usepackage{import}
%% and then include the figures with
%%   \import{<path to file>}{<filename>.pgf}
%%
%% Matplotlib used the following preamble
%%   \usepackage{fontspec}
%%   \setmainfont{DejaVuSans.ttf}[Path=C:/Users/mossr/.julia/conda/3/lib/site-packages/matplotlib/mpl-data/fonts/ttf/]
%%   \setsansfont{DejaVuSans.ttf}[Path=C:/Users/mossr/.julia/conda/3/lib/site-packages/matplotlib/mpl-data/fonts/ttf/]
%%   \setmonofont{DejaVuSansMono.ttf}[Path=C:/Users/mossr/.julia/conda/3/lib/site-packages/matplotlib/mpl-data/fonts/ttf/]
%%
\begingroup%
\makeatletter%
\begin{pgfpicture}%
% \pgfpathrectangle{\pgfqpoint{1.432000in}{0.528000in}}{\pgfqpoint{3.696000in}{3.696000in}}%
\pgfpathrectangle{\pgfqpoint{1.0in}{0.0in}}{\pgfqpoint{4.0in}{4.5in}}%
% \pgfpathrectangle{\pgfpointorigin}{\pgfqpoint{5.400000in}{4.800000in}}%
\pgfusepath{use as bounding box, clip}%
\begin{pgfscope}%
\pgfsetbuttcap%
\pgfsetmiterjoin%
\definecolor{currentfill}{rgb}{1.000000,1.000000,1.000000}%
\pgfsetfillcolor{currentfill}%
\pgfsetlinewidth{0.000000pt}%
\definecolor{currentstroke}{rgb}{1.000000,1.000000,1.000000}%
\pgfsetstrokecolor{currentstroke}%
\pgfsetdash{}{0pt}%
\pgfpathmoveto{\pgfqpoint{0.000000in}{0.000000in}}%
\pgfpathlineto{\pgfqpoint{6.400000in}{0.000000in}}%
\pgfpathlineto{\pgfqpoint{6.400000in}{4.800000in}}%
\pgfpathlineto{\pgfqpoint{0.000000in}{4.800000in}}%
\pgfpathclose%
\pgfusepath{fill}%
\end{pgfscope}%
\begin{pgfscope}%
\pgfsetbuttcap%
\pgfsetmiterjoin%
\definecolor{currentfill}{rgb}{1.000000,1.000000,1.000000}%
\pgfsetfillcolor{currentfill}%
\pgfsetlinewidth{0.000000pt}%
\definecolor{currentstroke}{rgb}{0.000000,0.000000,0.000000}%
\pgfsetstrokecolor{currentstroke}%
\pgfsetstrokeopacity{0.000000}%
\pgfsetdash{}{0pt}%
\pgfpathmoveto{\pgfqpoint{1.432000in}{0.528000in}}%
\pgfpathlineto{\pgfqpoint{5.128000in}{0.528000in}}%
\pgfpathlineto{\pgfqpoint{5.128000in}{4.224000in}}%
\pgfpathlineto{\pgfqpoint{1.432000in}{4.224000in}}%
\pgfpathclose%
\pgfusepath{fill}%
\end{pgfscope}%
\begin{pgfscope}%
\pgfpathrectangle{\pgfqpoint{1.432000in}{0.528000in}}{\pgfqpoint{3.696000in}{3.696000in}}%
\pgfusepath{clip}%
\pgfsys@transformshift{1.432000in}{0.528000in}%
\pgftext[left,bottom]{\pgfimage[interpolate=true,width=3.700000in,height=3.700000in]{figures/cem_variants/k5_ce_method-img0.png}}%
\end{pgfscope}%
\begin{pgfscope}%
\pgfpathrectangle{\pgfqpoint{1.432000in}{0.528000in}}{\pgfqpoint{3.696000in}{3.696000in}}%
\pgfusepath{clip}%
\pgfsetbuttcap%
\pgfsetroundjoin%
\definecolor{currentfill}{rgb}{0.000000,0.000000,0.000000}%
\pgfsetfillcolor{currentfill}%
\pgfsetlinewidth{0.501875pt}%
\definecolor{currentstroke}{rgb}{1.000000,1.000000,1.000000}%
\pgfsetstrokecolor{currentstroke}%
\pgfsetdash{}{0pt}%
\pgfsys@defobject{currentmarker}{\pgfqpoint{-0.018373in}{-0.018373in}}{\pgfqpoint{0.018373in}{0.018373in}}{%
\pgfpathmoveto{\pgfqpoint{0.000000in}{-0.018373in}}%
\pgfpathcurveto{\pgfqpoint{0.004873in}{-0.018373in}}{\pgfqpoint{0.009546in}{-0.016437in}}{\pgfqpoint{0.012992in}{-0.012992in}}%
\pgfpathcurveto{\pgfqpoint{0.016437in}{-0.009546in}}{\pgfqpoint{0.018373in}{-0.004873in}}{\pgfqpoint{0.018373in}{0.000000in}}%
\pgfpathcurveto{\pgfqpoint{0.018373in}{0.004873in}}{\pgfqpoint{0.016437in}{0.009546in}}{\pgfqpoint{0.012992in}{0.012992in}}%
\pgfpathcurveto{\pgfqpoint{0.009546in}{0.016437in}}{\pgfqpoint{0.004873in}{0.018373in}}{\pgfqpoint{0.000000in}{0.018373in}}%
\pgfpathcurveto{\pgfqpoint{-0.004873in}{0.018373in}}{\pgfqpoint{-0.009546in}{0.016437in}}{\pgfqpoint{-0.012992in}{0.012992in}}%
\pgfpathcurveto{\pgfqpoint{-0.016437in}{0.009546in}}{\pgfqpoint{-0.018373in}{0.004873in}}{\pgfqpoint{-0.018373in}{0.000000in}}%
\pgfpathcurveto{\pgfqpoint{-0.018373in}{-0.004873in}}{\pgfqpoint{-0.016437in}{-0.009546in}}{\pgfqpoint{-0.012992in}{-0.012992in}}%
\pgfpathcurveto{\pgfqpoint{-0.009546in}{-0.016437in}}{\pgfqpoint{-0.004873in}{-0.018373in}}{\pgfqpoint{0.000000in}{-0.018373in}}%
\pgfpathclose%
\pgfusepath{stroke,fill}%
}%
\begin{pgfscope}%
\pgfsys@transformshift{3.226538in}{2.443021in}%
\pgfsys@useobject{currentmarker}{}%
\end{pgfscope}%
\begin{pgfscope}%
\pgfsys@transformshift{2.855500in}{2.409783in}%
\pgfsys@useobject{currentmarker}{}%
\end{pgfscope}%
\begin{pgfscope}%
\pgfsys@transformshift{3.134620in}{2.419218in}%
\pgfsys@useobject{currentmarker}{}%
\end{pgfscope}%
\begin{pgfscope}%
\pgfsys@transformshift{2.802524in}{2.255363in}%
\pgfsys@useobject{currentmarker}{}%
\end{pgfscope}%
\begin{pgfscope}%
\pgfsys@transformshift{3.019718in}{2.396610in}%
\pgfsys@useobject{currentmarker}{}%
\end{pgfscope}%
\begin{pgfscope}%
\pgfsys@transformshift{2.993512in}{2.394083in}%
\pgfsys@useobject{currentmarker}{}%
\end{pgfscope}%
\begin{pgfscope}%
\pgfsys@transformshift{2.980602in}{2.487810in}%
\pgfsys@useobject{currentmarker}{}%
\end{pgfscope}%
\begin{pgfscope}%
\pgfsys@transformshift{3.255902in}{2.493239in}%
\pgfsys@useobject{currentmarker}{}%
\end{pgfscope}%
\begin{pgfscope}%
\pgfsys@transformshift{2.966496in}{2.256087in}%
\pgfsys@useobject{currentmarker}{}%
\end{pgfscope}%
\begin{pgfscope}%
\pgfsys@transformshift{2.477142in}{2.180839in}%
\pgfsys@useobject{currentmarker}{}%
\end{pgfscope}%
\end{pgfscope}%
\begin{pgfscope}%
\pgfpathrectangle{\pgfqpoint{1.432000in}{0.528000in}}{\pgfqpoint{3.696000in}{3.696000in}}%
\pgfusepath{clip}%
\pgfsetbuttcap%
\pgfsetroundjoin%
\definecolor{currentfill}{rgb}{1.000000,0.000000,0.000000}%
\pgfsetfillcolor{currentfill}%
\pgfsetlinewidth{0.501875pt}%
\definecolor{currentstroke}{rgb}{1.000000,1.000000,1.000000}%
\pgfsetstrokecolor{currentstroke}%
\pgfsetdash{}{0pt}%
\pgfsys@defobject{currentmarker}{\pgfqpoint{-0.018373in}{-0.018373in}}{\pgfqpoint{0.018373in}{0.018373in}}{%
\pgfpathmoveto{\pgfqpoint{0.000000in}{-0.018373in}}%
\pgfpathcurveto{\pgfqpoint{0.004873in}{-0.018373in}}{\pgfqpoint{0.009546in}{-0.016437in}}{\pgfqpoint{0.012992in}{-0.012992in}}%
\pgfpathcurveto{\pgfqpoint{0.016437in}{-0.009546in}}{\pgfqpoint{0.018373in}{-0.004873in}}{\pgfqpoint{0.018373in}{0.000000in}}%
\pgfpathcurveto{\pgfqpoint{0.018373in}{0.004873in}}{\pgfqpoint{0.016437in}{0.009546in}}{\pgfqpoint{0.012992in}{0.012992in}}%
\pgfpathcurveto{\pgfqpoint{0.009546in}{0.016437in}}{\pgfqpoint{0.004873in}{0.018373in}}{\pgfqpoint{0.000000in}{0.018373in}}%
\pgfpathcurveto{\pgfqpoint{-0.004873in}{0.018373in}}{\pgfqpoint{-0.009546in}{0.016437in}}{\pgfqpoint{-0.012992in}{0.012992in}}%
\pgfpathcurveto{\pgfqpoint{-0.016437in}{0.009546in}}{\pgfqpoint{-0.018373in}{0.004873in}}{\pgfqpoint{-0.018373in}{0.000000in}}%
\pgfpathcurveto{\pgfqpoint{-0.018373in}{-0.004873in}}{\pgfqpoint{-0.016437in}{-0.009546in}}{\pgfqpoint{-0.012992in}{-0.012992in}}%
\pgfpathcurveto{\pgfqpoint{-0.009546in}{-0.016437in}}{\pgfqpoint{-0.004873in}{-0.018373in}}{\pgfqpoint{0.000000in}{-0.018373in}}%
\pgfpathclose%
\pgfusepath{stroke,fill}%
}%
\begin{pgfscope}%
\pgfsys@transformshift{3.226538in}{2.443021in}%
\pgfsys@useobject{currentmarker}{}%
\end{pgfscope}%
\begin{pgfscope}%
\pgfsys@transformshift{2.966496in}{2.256087in}%
\pgfsys@useobject{currentmarker}{}%
\end{pgfscope}%
\begin{pgfscope}%
\pgfsys@transformshift{2.980602in}{2.487810in}%
\pgfsys@useobject{currentmarker}{}%
\end{pgfscope}%
\begin{pgfscope}%
\pgfsys@transformshift{2.802524in}{2.255363in}%
\pgfsys@useobject{currentmarker}{}%
\end{pgfscope}%
\begin{pgfscope}%
\pgfsys@transformshift{2.477142in}{2.180839in}%
\pgfsys@useobject{currentmarker}{}%
\end{pgfscope}%
\end{pgfscope}%
\begin{pgfscope}%
\pgfsetbuttcap%
\pgfsetroundjoin%
\definecolor{currentfill}{rgb}{0.000000,0.000000,0.000000}%
\pgfsetfillcolor{currentfill}%
\pgfsetlinewidth{0.803000pt}%
\definecolor{currentstroke}{rgb}{0.000000,0.000000,0.000000}%
\pgfsetstrokecolor{currentstroke}%
\pgfsetdash{}{0pt}%
\pgfsys@defobject{currentmarker}{\pgfqpoint{0.000000in}{-0.048611in}}{\pgfqpoint{0.000000in}{0.000000in}}{%
\pgfpathmoveto{\pgfqpoint{0.000000in}{0.000000in}}%
\pgfpathlineto{\pgfqpoint{0.000000in}{-0.048611in}}%
\pgfusepath{stroke,fill}%
}%
\begin{pgfscope}%
\pgfsys@transformshift{1.432000in}{0.528000in}%
\pgfsys@useobject{currentmarker}{}%
\end{pgfscope}%
\end{pgfscope}%
\begin{pgfscope}%
\definecolor{textcolor}{rgb}{0.000000,0.000000,0.000000}%
\pgfsetstrokecolor{textcolor}%
\pgfsetfillcolor{textcolor}%
\pgftext[x=1.432000in,y=0.430778in,,top]{\color{textcolor}\rmfamily\fontsize{10.000000}{12.000000}\selectfont \(\displaystyle -15\)}%
\end{pgfscope}%
\begin{pgfscope}%
\pgfsetbuttcap%
\pgfsetroundjoin%
\definecolor{currentfill}{rgb}{0.000000,0.000000,0.000000}%
\pgfsetfillcolor{currentfill}%
\pgfsetlinewidth{0.803000pt}%
\definecolor{currentstroke}{rgb}{0.000000,0.000000,0.000000}%
\pgfsetstrokecolor{currentstroke}%
\pgfsetdash{}{0pt}%
\pgfsys@defobject{currentmarker}{\pgfqpoint{0.000000in}{-0.048611in}}{\pgfqpoint{0.000000in}{0.000000in}}{%
\pgfpathmoveto{\pgfqpoint{0.000000in}{0.000000in}}%
\pgfpathlineto{\pgfqpoint{0.000000in}{-0.048611in}}%
\pgfusepath{stroke,fill}%
}%
\begin{pgfscope}%
\pgfsys@transformshift{2.048000in}{0.528000in}%
\pgfsys@useobject{currentmarker}{}%
\end{pgfscope}%
\end{pgfscope}%
\begin{pgfscope}%
\definecolor{textcolor}{rgb}{0.000000,0.000000,0.000000}%
\pgfsetstrokecolor{textcolor}%
\pgfsetfillcolor{textcolor}%
\pgftext[x=2.048000in,y=0.430778in,,top]{\color{textcolor}\rmfamily\fontsize{10.000000}{12.000000}\selectfont \(\displaystyle -10\)}%
\end{pgfscope}%
\begin{pgfscope}%
\pgfsetbuttcap%
\pgfsetroundjoin%
\definecolor{currentfill}{rgb}{0.000000,0.000000,0.000000}%
\pgfsetfillcolor{currentfill}%
\pgfsetlinewidth{0.803000pt}%
\definecolor{currentstroke}{rgb}{0.000000,0.000000,0.000000}%
\pgfsetstrokecolor{currentstroke}%
\pgfsetdash{}{0pt}%
\pgfsys@defobject{currentmarker}{\pgfqpoint{0.000000in}{-0.048611in}}{\pgfqpoint{0.000000in}{0.000000in}}{%
\pgfpathmoveto{\pgfqpoint{0.000000in}{0.000000in}}%
\pgfpathlineto{\pgfqpoint{0.000000in}{-0.048611in}}%
\pgfusepath{stroke,fill}%
}%
\begin{pgfscope}%
\pgfsys@transformshift{2.664000in}{0.528000in}%
\pgfsys@useobject{currentmarker}{}%
\end{pgfscope}%
\end{pgfscope}%
\begin{pgfscope}%
\definecolor{textcolor}{rgb}{0.000000,0.000000,0.000000}%
\pgfsetstrokecolor{textcolor}%
\pgfsetfillcolor{textcolor}%
\pgftext[x=2.664000in,y=0.430778in,,top]{\color{textcolor}\rmfamily\fontsize{10.000000}{12.000000}\selectfont \(\displaystyle -5\)}%
\end{pgfscope}%
\begin{pgfscope}%
\pgfsetbuttcap%
\pgfsetroundjoin%
\definecolor{currentfill}{rgb}{0.000000,0.000000,0.000000}%
\pgfsetfillcolor{currentfill}%
\pgfsetlinewidth{0.803000pt}%
\definecolor{currentstroke}{rgb}{0.000000,0.000000,0.000000}%
\pgfsetstrokecolor{currentstroke}%
\pgfsetdash{}{0pt}%
\pgfsys@defobject{currentmarker}{\pgfqpoint{0.000000in}{-0.048611in}}{\pgfqpoint{0.000000in}{0.000000in}}{%
\pgfpathmoveto{\pgfqpoint{0.000000in}{0.000000in}}%
\pgfpathlineto{\pgfqpoint{0.000000in}{-0.048611in}}%
\pgfusepath{stroke,fill}%
}%
\begin{pgfscope}%
\pgfsys@transformshift{3.280000in}{0.528000in}%
\pgfsys@useobject{currentmarker}{}%
\end{pgfscope}%
\end{pgfscope}%
\begin{pgfscope}%
\definecolor{textcolor}{rgb}{0.000000,0.000000,0.000000}%
\pgfsetstrokecolor{textcolor}%
\pgfsetfillcolor{textcolor}%
\pgftext[x=3.280000in,y=0.430778in,,top]{\color{textcolor}\rmfamily\fontsize{10.000000}{12.000000}\selectfont \(\displaystyle 0\)}%
\end{pgfscope}%
\begin{pgfscope}%
\pgfsetbuttcap%
\pgfsetroundjoin%
\definecolor{currentfill}{rgb}{0.000000,0.000000,0.000000}%
\pgfsetfillcolor{currentfill}%
\pgfsetlinewidth{0.803000pt}%
\definecolor{currentstroke}{rgb}{0.000000,0.000000,0.000000}%
\pgfsetstrokecolor{currentstroke}%
\pgfsetdash{}{0pt}%
\pgfsys@defobject{currentmarker}{\pgfqpoint{0.000000in}{-0.048611in}}{\pgfqpoint{0.000000in}{0.000000in}}{%
\pgfpathmoveto{\pgfqpoint{0.000000in}{0.000000in}}%
\pgfpathlineto{\pgfqpoint{0.000000in}{-0.048611in}}%
\pgfusepath{stroke,fill}%
}%
\begin{pgfscope}%
\pgfsys@transformshift{3.896000in}{0.528000in}%
\pgfsys@useobject{currentmarker}{}%
\end{pgfscope}%
\end{pgfscope}%
\begin{pgfscope}%
\definecolor{textcolor}{rgb}{0.000000,0.000000,0.000000}%
\pgfsetstrokecolor{textcolor}%
\pgfsetfillcolor{textcolor}%
\pgftext[x=3.896000in,y=0.430778in,,top]{\color{textcolor}\rmfamily\fontsize{10.000000}{12.000000}\selectfont \(\displaystyle 5\)}%
\end{pgfscope}%
\begin{pgfscope}%
\pgfsetbuttcap%
\pgfsetroundjoin%
\definecolor{currentfill}{rgb}{0.000000,0.000000,0.000000}%
\pgfsetfillcolor{currentfill}%
\pgfsetlinewidth{0.803000pt}%
\definecolor{currentstroke}{rgb}{0.000000,0.000000,0.000000}%
\pgfsetstrokecolor{currentstroke}%
\pgfsetdash{}{0pt}%
\pgfsys@defobject{currentmarker}{\pgfqpoint{0.000000in}{-0.048611in}}{\pgfqpoint{0.000000in}{0.000000in}}{%
\pgfpathmoveto{\pgfqpoint{0.000000in}{0.000000in}}%
\pgfpathlineto{\pgfqpoint{0.000000in}{-0.048611in}}%
\pgfusepath{stroke,fill}%
}%
\begin{pgfscope}%
\pgfsys@transformshift{4.512000in}{0.528000in}%
\pgfsys@useobject{currentmarker}{}%
\end{pgfscope}%
\end{pgfscope}%
\begin{pgfscope}%
\definecolor{textcolor}{rgb}{0.000000,0.000000,0.000000}%
\pgfsetstrokecolor{textcolor}%
\pgfsetfillcolor{textcolor}%
\pgftext[x=4.512000in,y=0.430778in,,top]{\color{textcolor}\rmfamily\fontsize{10.000000}{12.000000}\selectfont \(\displaystyle 10\)}%
\end{pgfscope}%
\begin{pgfscope}%
\pgfsetbuttcap%
\pgfsetroundjoin%
\definecolor{currentfill}{rgb}{0.000000,0.000000,0.000000}%
\pgfsetfillcolor{currentfill}%
\pgfsetlinewidth{0.803000pt}%
\definecolor{currentstroke}{rgb}{0.000000,0.000000,0.000000}%
\pgfsetstrokecolor{currentstroke}%
\pgfsetdash{}{0pt}%
\pgfsys@defobject{currentmarker}{\pgfqpoint{0.000000in}{-0.048611in}}{\pgfqpoint{0.000000in}{0.000000in}}{%
\pgfpathmoveto{\pgfqpoint{0.000000in}{0.000000in}}%
\pgfpathlineto{\pgfqpoint{0.000000in}{-0.048611in}}%
\pgfusepath{stroke,fill}%
}%
\begin{pgfscope}%
\pgfsys@transformshift{5.128000in}{0.528000in}%
\pgfsys@useobject{currentmarker}{}%
\end{pgfscope}%
\end{pgfscope}%
\begin{pgfscope}%
\definecolor{textcolor}{rgb}{0.000000,0.000000,0.000000}%
\pgfsetstrokecolor{textcolor}%
\pgfsetfillcolor{textcolor}%
\pgftext[x=5.128000in,y=0.430778in,,top]{\color{textcolor}\rmfamily\fontsize{10.000000}{12.000000}\selectfont \(\displaystyle 15\)}%
\end{pgfscope}%
\begin{pgfscope}%
\pgfsetbuttcap%
\pgfsetroundjoin%
\definecolor{currentfill}{rgb}{0.000000,0.000000,0.000000}%
\pgfsetfillcolor{currentfill}%
\pgfsetlinewidth{0.803000pt}%
\definecolor{currentstroke}{rgb}{0.000000,0.000000,0.000000}%
\pgfsetstrokecolor{currentstroke}%
\pgfsetdash{}{0pt}%
\pgfsys@defobject{currentmarker}{\pgfqpoint{-0.048611in}{0.000000in}}{\pgfqpoint{0.000000in}{0.000000in}}{%
\pgfpathmoveto{\pgfqpoint{0.000000in}{0.000000in}}%
\pgfpathlineto{\pgfqpoint{-0.048611in}{0.000000in}}%
\pgfusepath{stroke,fill}%
}%
\begin{pgfscope}%
\pgfsys@transformshift{1.432000in}{0.528000in}%
\pgfsys@useobject{currentmarker}{}%
\end{pgfscope}%
\end{pgfscope}%
\begin{pgfscope}%
\definecolor{textcolor}{rgb}{0.000000,0.000000,0.000000}%
\pgfsetstrokecolor{textcolor}%
\pgfsetfillcolor{textcolor}%
\pgftext[x=1.087863in,y=0.475238in,left,base]{\color{textcolor}\rmfamily\fontsize{10.000000}{12.000000}\selectfont \(\displaystyle -15\)}%
\end{pgfscope}%
\begin{pgfscope}%
\pgfsetbuttcap%
\pgfsetroundjoin%
\definecolor{currentfill}{rgb}{0.000000,0.000000,0.000000}%
\pgfsetfillcolor{currentfill}%
\pgfsetlinewidth{0.803000pt}%
\definecolor{currentstroke}{rgb}{0.000000,0.000000,0.000000}%
\pgfsetstrokecolor{currentstroke}%
\pgfsetdash{}{0pt}%
\pgfsys@defobject{currentmarker}{\pgfqpoint{-0.048611in}{0.000000in}}{\pgfqpoint{0.000000in}{0.000000in}}{%
\pgfpathmoveto{\pgfqpoint{0.000000in}{0.000000in}}%
\pgfpathlineto{\pgfqpoint{-0.048611in}{0.000000in}}%
\pgfusepath{stroke,fill}%
}%
\begin{pgfscope}%
\pgfsys@transformshift{1.432000in}{1.144000in}%
\pgfsys@useobject{currentmarker}{}%
\end{pgfscope}%
\end{pgfscope}%
\begin{pgfscope}%
\definecolor{textcolor}{rgb}{0.000000,0.000000,0.000000}%
\pgfsetstrokecolor{textcolor}%
\pgfsetfillcolor{textcolor}%
\pgftext[x=1.087863in,y=1.091238in,left,base]{\color{textcolor}\rmfamily\fontsize{10.000000}{12.000000}\selectfont \(\displaystyle -10\)}%
\end{pgfscope}%
\begin{pgfscope}%
\pgfsetbuttcap%
\pgfsetroundjoin%
\definecolor{currentfill}{rgb}{0.000000,0.000000,0.000000}%
\pgfsetfillcolor{currentfill}%
\pgfsetlinewidth{0.803000pt}%
\definecolor{currentstroke}{rgb}{0.000000,0.000000,0.000000}%
\pgfsetstrokecolor{currentstroke}%
\pgfsetdash{}{0pt}%
\pgfsys@defobject{currentmarker}{\pgfqpoint{-0.048611in}{0.000000in}}{\pgfqpoint{0.000000in}{0.000000in}}{%
\pgfpathmoveto{\pgfqpoint{0.000000in}{0.000000in}}%
\pgfpathlineto{\pgfqpoint{-0.048611in}{0.000000in}}%
\pgfusepath{stroke,fill}%
}%
\begin{pgfscope}%
\pgfsys@transformshift{1.432000in}{1.760000in}%
\pgfsys@useobject{currentmarker}{}%
\end{pgfscope}%
\end{pgfscope}%
\begin{pgfscope}%
\definecolor{textcolor}{rgb}{0.000000,0.000000,0.000000}%
\pgfsetstrokecolor{textcolor}%
\pgfsetfillcolor{textcolor}%
\pgftext[x=1.157308in,y=1.707238in,left,base]{\color{textcolor}\rmfamily\fontsize{10.000000}{12.000000}\selectfont \(\displaystyle -5\)}%
\end{pgfscope}%
\begin{pgfscope}%
\pgfsetbuttcap%
\pgfsetroundjoin%
\definecolor{currentfill}{rgb}{0.000000,0.000000,0.000000}%
\pgfsetfillcolor{currentfill}%
\pgfsetlinewidth{0.803000pt}%
\definecolor{currentstroke}{rgb}{0.000000,0.000000,0.000000}%
\pgfsetstrokecolor{currentstroke}%
\pgfsetdash{}{0pt}%
\pgfsys@defobject{currentmarker}{\pgfqpoint{-0.048611in}{0.000000in}}{\pgfqpoint{0.000000in}{0.000000in}}{%
\pgfpathmoveto{\pgfqpoint{0.000000in}{0.000000in}}%
\pgfpathlineto{\pgfqpoint{-0.048611in}{0.000000in}}%
\pgfusepath{stroke,fill}%
}%
\begin{pgfscope}%
\pgfsys@transformshift{1.432000in}{2.376000in}%
\pgfsys@useobject{currentmarker}{}%
\end{pgfscope}%
\end{pgfscope}%
\begin{pgfscope}%
\definecolor{textcolor}{rgb}{0.000000,0.000000,0.000000}%
\pgfsetstrokecolor{textcolor}%
\pgfsetfillcolor{textcolor}%
\pgftext[x=1.265333in,y=2.323238in,left,base]{\color{textcolor}\rmfamily\fontsize{10.000000}{12.000000}\selectfont \(\displaystyle 0\)}%
\end{pgfscope}%
\begin{pgfscope}%
\pgfsetbuttcap%
\pgfsetroundjoin%
\definecolor{currentfill}{rgb}{0.000000,0.000000,0.000000}%
\pgfsetfillcolor{currentfill}%
\pgfsetlinewidth{0.803000pt}%
\definecolor{currentstroke}{rgb}{0.000000,0.000000,0.000000}%
\pgfsetstrokecolor{currentstroke}%
\pgfsetdash{}{0pt}%
\pgfsys@defobject{currentmarker}{\pgfqpoint{-0.048611in}{0.000000in}}{\pgfqpoint{0.000000in}{0.000000in}}{%
\pgfpathmoveto{\pgfqpoint{0.000000in}{0.000000in}}%
\pgfpathlineto{\pgfqpoint{-0.048611in}{0.000000in}}%
\pgfusepath{stroke,fill}%
}%
\begin{pgfscope}%
\pgfsys@transformshift{1.432000in}{2.992000in}%
\pgfsys@useobject{currentmarker}{}%
\end{pgfscope}%
\end{pgfscope}%
\begin{pgfscope}%
\definecolor{textcolor}{rgb}{0.000000,0.000000,0.000000}%
\pgfsetstrokecolor{textcolor}%
\pgfsetfillcolor{textcolor}%
\pgftext[x=1.265333in,y=2.939238in,left,base]{\color{textcolor}\rmfamily\fontsize{10.000000}{12.000000}\selectfont \(\displaystyle 5\)}%
\end{pgfscope}%
\begin{pgfscope}%
\pgfsetbuttcap%
\pgfsetroundjoin%
\definecolor{currentfill}{rgb}{0.000000,0.000000,0.000000}%
\pgfsetfillcolor{currentfill}%
\pgfsetlinewidth{0.803000pt}%
\definecolor{currentstroke}{rgb}{0.000000,0.000000,0.000000}%
\pgfsetstrokecolor{currentstroke}%
\pgfsetdash{}{0pt}%
\pgfsys@defobject{currentmarker}{\pgfqpoint{-0.048611in}{0.000000in}}{\pgfqpoint{0.000000in}{0.000000in}}{%
\pgfpathmoveto{\pgfqpoint{0.000000in}{0.000000in}}%
\pgfpathlineto{\pgfqpoint{-0.048611in}{0.000000in}}%
\pgfusepath{stroke,fill}%
}%
\begin{pgfscope}%
\pgfsys@transformshift{1.432000in}{3.608000in}%
\pgfsys@useobject{currentmarker}{}%
\end{pgfscope}%
\end{pgfscope}%
\begin{pgfscope}%
\definecolor{textcolor}{rgb}{0.000000,0.000000,0.000000}%
\pgfsetstrokecolor{textcolor}%
\pgfsetfillcolor{textcolor}%
\pgftext[x=1.195888in,y=3.555238in,left,base]{\color{textcolor}\rmfamily\fontsize{10.000000}{12.000000}\selectfont \(\displaystyle 10\)}%
\end{pgfscope}%
\begin{pgfscope}%
\pgfsetbuttcap%
\pgfsetroundjoin%
\definecolor{currentfill}{rgb}{0.000000,0.000000,0.000000}%
\pgfsetfillcolor{currentfill}%
\pgfsetlinewidth{0.803000pt}%
\definecolor{currentstroke}{rgb}{0.000000,0.000000,0.000000}%
\pgfsetstrokecolor{currentstroke}%
\pgfsetdash{}{0pt}%
\pgfsys@defobject{currentmarker}{\pgfqpoint{-0.048611in}{0.000000in}}{\pgfqpoint{0.000000in}{0.000000in}}{%
\pgfpathmoveto{\pgfqpoint{0.000000in}{0.000000in}}%
\pgfpathlineto{\pgfqpoint{-0.048611in}{0.000000in}}%
\pgfusepath{stroke,fill}%
}%
\begin{pgfscope}%
\pgfsys@transformshift{1.432000in}{4.224000in}%
\pgfsys@useobject{currentmarker}{}%
\end{pgfscope}%
\end{pgfscope}%
\begin{pgfscope}%
\definecolor{textcolor}{rgb}{0.000000,0.000000,0.000000}%
\pgfsetstrokecolor{textcolor}%
\pgfsetfillcolor{textcolor}%
\pgftext[x=1.195888in,y=4.171238in,left,base]{\color{textcolor}\rmfamily\fontsize{10.000000}{12.000000}\selectfont \(\displaystyle 15\)}%
\end{pgfscope}%
\begin{pgfscope}%
\pgfpathrectangle{\pgfqpoint{1.432000in}{0.528000in}}{\pgfqpoint{3.696000in}{3.696000in}}%
\pgfusepath{clip}%
\pgfsetbuttcap%
\pgfsetroundjoin%
\pgfsetlinewidth{1.505625pt}%
\definecolor{currentstroke}{rgb}{0.412214,0.000000,0.000000}%
\pgfsetstrokecolor{currentstroke}%
\pgfsetstrokeopacity{0.300000}%
\pgfsetdash{}{0pt}%
\pgfpathmoveto{\pgfqpoint{2.440000in}{1.965838in}}%
\pgfpathlineto{\pgfqpoint{2.477333in}{1.952694in}}%
\pgfpathlineto{\pgfqpoint{2.514667in}{1.952765in}}%
\pgfpathlineto{\pgfqpoint{2.552000in}{1.957969in}}%
\pgfpathlineto{\pgfqpoint{2.589333in}{1.966486in}}%
\pgfpathlineto{\pgfqpoint{2.626667in}{1.978641in}}%
\pgfpathlineto{\pgfqpoint{2.639205in}{1.984000in}}%
\pgfpathlineto{\pgfqpoint{2.664000in}{1.991810in}}%
\pgfpathlineto{\pgfqpoint{2.701333in}{2.006158in}}%
\pgfpathlineto{\pgfqpoint{2.731361in}{2.021333in}}%
\pgfpathlineto{\pgfqpoint{2.738667in}{2.024119in}}%
\pgfpathlineto{\pgfqpoint{2.776000in}{2.040167in}}%
\pgfpathlineto{\pgfqpoint{2.808241in}{2.058667in}}%
\pgfpathlineto{\pgfqpoint{2.813333in}{2.060924in}}%
\pgfpathlineto{\pgfqpoint{2.850667in}{2.079049in}}%
\pgfpathlineto{\pgfqpoint{2.877400in}{2.096000in}}%
\pgfpathlineto{\pgfqpoint{2.888000in}{2.101320in}}%
\pgfpathlineto{\pgfqpoint{2.925333in}{2.122679in}}%
\pgfpathlineto{\pgfqpoint{2.940837in}{2.133333in}}%
\pgfpathlineto{\pgfqpoint{2.962667in}{2.145502in}}%
\pgfpathlineto{\pgfqpoint{2.998377in}{2.170667in}}%
\pgfpathlineto{\pgfqpoint{3.000000in}{2.171642in}}%
\pgfpathlineto{\pgfqpoint{3.037333in}{2.194848in}}%
\pgfpathlineto{\pgfqpoint{3.054787in}{2.208000in}}%
\pgfpathlineto{\pgfqpoint{3.074667in}{2.221094in}}%
\pgfpathlineto{\pgfqpoint{3.105854in}{2.245333in}}%
\pgfpathlineto{\pgfqpoint{3.112000in}{2.249727in}}%
\pgfpathlineto{\pgfqpoint{3.149333in}{2.278474in}}%
\pgfpathlineto{\pgfqpoint{3.154437in}{2.282667in}}%
\pgfpathlineto{\pgfqpoint{3.186667in}{2.307614in}}%
\pgfpathlineto{\pgfqpoint{3.201203in}{2.320000in}}%
\pgfpathlineto{\pgfqpoint{3.224000in}{2.339273in}}%
\pgfpathlineto{\pgfqpoint{3.244299in}{2.357333in}}%
\pgfpathlineto{\pgfqpoint{3.261333in}{2.373195in}}%
\pgfpathlineto{\pgfqpoint{3.284253in}{2.394667in}}%
\pgfpathlineto{\pgfqpoint{3.298667in}{2.409606in}}%
\pgfpathlineto{\pgfqpoint{3.321133in}{2.432000in}}%
\pgfpathlineto{\pgfqpoint{3.336000in}{2.449385in}}%
\pgfpathlineto{\pgfqpoint{3.354557in}{2.469333in}}%
\pgfpathlineto{\pgfqpoint{3.373333in}{2.494550in}}%
\pgfpathlineto{\pgfqpoint{3.383601in}{2.506667in}}%
\pgfpathlineto{\pgfqpoint{3.407368in}{2.544000in}}%
\pgfpathlineto{\pgfqpoint{3.410667in}{2.552310in}}%
\pgfpathlineto{\pgfqpoint{3.425371in}{2.581333in}}%
\pgfpathlineto{\pgfqpoint{3.431284in}{2.618667in}}%
\pgfpathlineto{\pgfqpoint{3.410667in}{2.654598in}}%
\pgfpathlineto{\pgfqpoint{3.408477in}{2.656000in}}%
\pgfpathlineto{\pgfqpoint{3.373333in}{2.665730in}}%
\pgfpathlineto{\pgfqpoint{3.336000in}{2.665695in}}%
\pgfpathlineto{\pgfqpoint{3.298667in}{2.659749in}}%
\pgfpathlineto{\pgfqpoint{3.285797in}{2.656000in}}%
\pgfpathlineto{\pgfqpoint{3.261333in}{2.650680in}}%
\pgfpathlineto{\pgfqpoint{3.224000in}{2.640086in}}%
\pgfpathlineto{\pgfqpoint{3.186667in}{2.626128in}}%
\pgfpathlineto{\pgfqpoint{3.170943in}{2.618667in}}%
\pgfpathlineto{\pgfqpoint{3.149333in}{2.611078in}}%
\pgfpathlineto{\pgfqpoint{3.112000in}{2.595259in}}%
\pgfpathlineto{\pgfqpoint{3.086485in}{2.581333in}}%
\pgfpathlineto{\pgfqpoint{3.074667in}{2.576445in}}%
\pgfpathlineto{\pgfqpoint{3.037333in}{2.558702in}}%
\pgfpathlineto{\pgfqpoint{3.013174in}{2.544000in}}%
\pgfpathlineto{\pgfqpoint{3.000000in}{2.537775in}}%
\pgfpathlineto{\pgfqpoint{2.962667in}{2.517250in}}%
\pgfpathlineto{\pgfqpoint{2.946726in}{2.506667in}}%
\pgfpathlineto{\pgfqpoint{2.925333in}{2.495364in}}%
\pgfpathlineto{\pgfqpoint{2.888000in}{2.470263in}}%
\pgfpathlineto{\pgfqpoint{2.886696in}{2.469333in}}%
\pgfpathlineto{\pgfqpoint{2.850667in}{2.448380in}}%
\pgfpathlineto{\pgfqpoint{2.828257in}{2.432000in}}%
\pgfpathlineto{\pgfqpoint{2.813333in}{2.422646in}}%
\pgfpathlineto{\pgfqpoint{2.776175in}{2.394667in}}%
\pgfpathlineto{\pgfqpoint{2.776000in}{2.394548in}}%
\pgfpathlineto{\pgfqpoint{2.738667in}{2.369038in}}%
\pgfpathlineto{\pgfqpoint{2.724027in}{2.357333in}}%
\pgfpathlineto{\pgfqpoint{2.701333in}{2.340562in}}%
\pgfpathlineto{\pgfqpoint{2.676513in}{2.320000in}}%
\pgfpathlineto{\pgfqpoint{2.664000in}{2.309908in}}%
\pgfpathlineto{\pgfqpoint{2.632479in}{2.282667in}}%
\pgfpathlineto{\pgfqpoint{2.626667in}{2.277509in}}%
\pgfpathlineto{\pgfqpoint{2.591274in}{2.245333in}}%
\pgfpathlineto{\pgfqpoint{2.589333in}{2.243419in}}%
\pgfpathlineto{\pgfqpoint{2.552681in}{2.208000in}}%
\pgfpathlineto{\pgfqpoint{2.552000in}{2.207243in}}%
\pgfpathlineto{\pgfqpoint{2.516867in}{2.170667in}}%
\pgfpathlineto{\pgfqpoint{2.514667in}{2.167858in}}%
\pgfpathlineto{\pgfqpoint{2.484430in}{2.133333in}}%
\pgfpathlineto{\pgfqpoint{2.477333in}{2.122677in}}%
\pgfpathlineto{\pgfqpoint{2.456547in}{2.096000in}}%
\pgfpathlineto{\pgfqpoint{2.440000in}{2.065711in}}%
\pgfpathlineto{\pgfqpoint{2.435287in}{2.058667in}}%
\pgfpathlineto{\pgfqpoint{2.421437in}{2.021333in}}%
\pgfpathlineto{\pgfqpoint{2.423336in}{1.984000in}}%
\pgfpathlineto{\pgfqpoint{2.440000in}{1.965838in}}%
\pgfusepath{stroke}%
\end{pgfscope}%
\begin{pgfscope}%
\pgfpathrectangle{\pgfqpoint{1.432000in}{0.528000in}}{\pgfqpoint{3.696000in}{3.696000in}}%
\pgfusepath{clip}%
\pgfsetbuttcap%
\pgfsetroundjoin%
\pgfsetlinewidth{1.505625pt}%
\definecolor{currentstroke}{rgb}{0.793124,0.000000,0.000000}%
\pgfsetstrokecolor{currentstroke}%
\pgfsetstrokeopacity{0.300000}%
\pgfsetdash{}{0pt}%
\pgfpathmoveto{\pgfqpoint{2.552000in}{2.032209in}}%
\pgfpathlineto{\pgfqpoint{2.589333in}{2.026606in}}%
\pgfpathlineto{\pgfqpoint{2.626667in}{2.031026in}}%
\pgfpathlineto{\pgfqpoint{2.664000in}{2.039610in}}%
\pgfpathlineto{\pgfqpoint{2.701333in}{2.051719in}}%
\pgfpathlineto{\pgfqpoint{2.717410in}{2.058667in}}%
\pgfpathlineto{\pgfqpoint{2.738667in}{2.066221in}}%
\pgfpathlineto{\pgfqpoint{2.776000in}{2.081843in}}%
\pgfpathlineto{\pgfqpoint{2.803088in}{2.096000in}}%
\pgfpathlineto{\pgfqpoint{2.813333in}{2.100498in}}%
\pgfpathlineto{\pgfqpoint{2.850667in}{2.118865in}}%
\pgfpathlineto{\pgfqpoint{2.874759in}{2.133333in}}%
\pgfpathlineto{\pgfqpoint{2.888000in}{2.140160in}}%
\pgfpathlineto{\pgfqpoint{2.925333in}{2.161778in}}%
\pgfpathlineto{\pgfqpoint{2.938641in}{2.170667in}}%
\pgfpathlineto{\pgfqpoint{2.962667in}{2.184757in}}%
\pgfpathlineto{\pgfqpoint{2.996253in}{2.208000in}}%
\pgfpathlineto{\pgfqpoint{3.000000in}{2.210405in}}%
\pgfpathlineto{\pgfqpoint{3.037333in}{2.235383in}}%
\pgfpathlineto{\pgfqpoint{3.050539in}{2.245333in}}%
\pgfpathlineto{\pgfqpoint{3.074667in}{2.262566in}}%
\pgfpathlineto{\pgfqpoint{3.100178in}{2.282667in}}%
\pgfpathlineto{\pgfqpoint{3.112000in}{2.292001in}}%
\pgfpathlineto{\pgfqpoint{3.145645in}{2.320000in}}%
\pgfpathlineto{\pgfqpoint{3.149333in}{2.323263in}}%
\pgfpathlineto{\pgfqpoint{3.186667in}{2.356300in}}%
\pgfpathlineto{\pgfqpoint{3.187829in}{2.357333in}}%
\pgfpathlineto{\pgfqpoint{3.224000in}{2.392189in}}%
\pgfpathlineto{\pgfqpoint{3.226611in}{2.394667in}}%
\pgfpathlineto{\pgfqpoint{3.260836in}{2.432000in}}%
\pgfpathlineto{\pgfqpoint{3.261333in}{2.432709in}}%
\pgfpathlineto{\pgfqpoint{3.290630in}{2.469333in}}%
\pgfpathlineto{\pgfqpoint{3.298667in}{2.484011in}}%
\pgfpathlineto{\pgfqpoint{3.313492in}{2.506667in}}%
\pgfpathlineto{\pgfqpoint{3.325413in}{2.544000in}}%
\pgfpathlineto{\pgfqpoint{3.310346in}{2.581333in}}%
\pgfpathlineto{\pgfqpoint{3.298667in}{2.587225in}}%
\pgfpathlineto{\pgfqpoint{3.261333in}{2.591640in}}%
\pgfpathlineto{\pgfqpoint{3.224000in}{2.587317in}}%
\pgfpathlineto{\pgfqpoint{3.201830in}{2.581333in}}%
\pgfpathlineto{\pgfqpoint{3.186667in}{2.577890in}}%
\pgfpathlineto{\pgfqpoint{3.149333in}{2.566366in}}%
\pgfpathlineto{\pgfqpoint{3.112000in}{2.551944in}}%
\pgfpathlineto{\pgfqpoint{3.095663in}{2.544000in}}%
\pgfpathlineto{\pgfqpoint{3.074667in}{2.535586in}}%
\pgfpathlineto{\pgfqpoint{3.037333in}{2.518149in}}%
\pgfpathlineto{\pgfqpoint{3.017112in}{2.506667in}}%
\pgfpathlineto{\pgfqpoint{3.000000in}{2.498482in}}%
\pgfpathlineto{\pgfqpoint{2.962667in}{2.478007in}}%
\pgfpathlineto{\pgfqpoint{2.949080in}{2.469333in}}%
\pgfpathlineto{\pgfqpoint{2.925333in}{2.456273in}}%
\pgfpathlineto{\pgfqpoint{2.888552in}{2.432000in}}%
\pgfpathlineto{\pgfqpoint{2.888000in}{2.431669in}}%
\pgfpathlineto{\pgfqpoint{2.850667in}{2.408567in}}%
\pgfpathlineto{\pgfqpoint{2.831484in}{2.394667in}}%
\pgfpathlineto{\pgfqpoint{2.813333in}{2.382435in}}%
\pgfpathlineto{\pgfqpoint{2.780145in}{2.357333in}}%
\pgfpathlineto{\pgfqpoint{2.776000in}{2.354250in}}%
\pgfpathlineto{\pgfqpoint{2.738667in}{2.325700in}}%
\pgfpathlineto{\pgfqpoint{2.731669in}{2.320000in}}%
\pgfpathlineto{\pgfqpoint{2.701333in}{2.295178in}}%
\pgfpathlineto{\pgfqpoint{2.686752in}{2.282667in}}%
\pgfpathlineto{\pgfqpoint{2.664000in}{2.261902in}}%
\pgfpathlineto{\pgfqpoint{2.645882in}{2.245333in}}%
\pgfpathlineto{\pgfqpoint{2.626667in}{2.225470in}}%
\pgfpathlineto{\pgfqpoint{2.609013in}{2.208000in}}%
\pgfpathlineto{\pgfqpoint{2.589333in}{2.184475in}}%
\pgfpathlineto{\pgfqpoint{2.576704in}{2.170667in}}%
\pgfpathlineto{\pgfqpoint{2.552000in}{2.135495in}}%
\pgfpathlineto{\pgfqpoint{2.550262in}{2.133333in}}%
\pgfpathlineto{\pgfqpoint{2.531023in}{2.096000in}}%
\pgfpathlineto{\pgfqpoint{2.526202in}{2.058667in}}%
\pgfpathlineto{\pgfqpoint{2.552000in}{2.032209in}}%
\pgfusepath{stroke}%
\end{pgfscope}%
\begin{pgfscope}%
\pgfpathrectangle{\pgfqpoint{1.432000in}{0.528000in}}{\pgfqpoint{3.696000in}{3.696000in}}%
\pgfusepath{clip}%
\pgfsetbuttcap%
\pgfsetroundjoin%
\pgfsetlinewidth{1.505625pt}%
\definecolor{currentstroke}{rgb}{1.000000,0.163726,0.000000}%
\pgfsetstrokecolor{currentstroke}%
\pgfsetstrokeopacity{0.300000}%
\pgfsetdash{}{0pt}%
\pgfpathmoveto{\pgfqpoint{2.626667in}{2.085065in}}%
\pgfpathlineto{\pgfqpoint{2.664000in}{2.082181in}}%
\pgfpathlineto{\pgfqpoint{2.701333in}{2.088321in}}%
\pgfpathlineto{\pgfqpoint{2.725689in}{2.096000in}}%
\pgfpathlineto{\pgfqpoint{2.738667in}{2.099753in}}%
\pgfpathlineto{\pgfqpoint{2.776000in}{2.113624in}}%
\pgfpathlineto{\pgfqpoint{2.813333in}{2.130260in}}%
\pgfpathlineto{\pgfqpoint{2.819063in}{2.133333in}}%
\pgfpathlineto{\pgfqpoint{2.850667in}{2.148417in}}%
\pgfpathlineto{\pgfqpoint{2.888000in}{2.169026in}}%
\pgfpathlineto{\pgfqpoint{2.890644in}{2.170667in}}%
\pgfpathlineto{\pgfqpoint{2.925333in}{2.190205in}}%
\pgfpathlineto{\pgfqpoint{2.952853in}{2.208000in}}%
\pgfpathlineto{\pgfqpoint{2.962667in}{2.214112in}}%
\pgfpathlineto{\pgfqpoint{3.000000in}{2.238846in}}%
\pgfpathlineto{\pgfqpoint{3.008971in}{2.245333in}}%
\pgfpathlineto{\pgfqpoint{3.037333in}{2.265474in}}%
\pgfpathlineto{\pgfqpoint{3.059835in}{2.282667in}}%
\pgfpathlineto{\pgfqpoint{3.074667in}{2.294495in}}%
\pgfpathlineto{\pgfqpoint{3.105804in}{2.320000in}}%
\pgfpathlineto{\pgfqpoint{3.112000in}{2.325662in}}%
\pgfpathlineto{\pgfqpoint{3.147355in}{2.357333in}}%
\pgfpathlineto{\pgfqpoint{3.149333in}{2.359466in}}%
\pgfpathlineto{\pgfqpoint{3.184251in}{2.394667in}}%
\pgfpathlineto{\pgfqpoint{3.186667in}{2.397877in}}%
\pgfpathlineto{\pgfqpoint{3.215456in}{2.432000in}}%
\pgfpathlineto{\pgfqpoint{3.224000in}{2.447077in}}%
\pgfpathlineto{\pgfqpoint{3.238831in}{2.469333in}}%
\pgfpathlineto{\pgfqpoint{3.248106in}{2.506667in}}%
\pgfpathlineto{\pgfqpoint{3.224000in}{2.532642in}}%
\pgfpathlineto{\pgfqpoint{3.186667in}{2.535813in}}%
\pgfpathlineto{\pgfqpoint{3.149333in}{2.529223in}}%
\pgfpathlineto{\pgfqpoint{3.112000in}{2.518501in}}%
\pgfpathlineto{\pgfqpoint{3.082114in}{2.506667in}}%
\pgfpathlineto{\pgfqpoint{3.074667in}{2.503974in}}%
\pgfpathlineto{\pgfqpoint{3.037333in}{2.487934in}}%
\pgfpathlineto{\pgfqpoint{3.000928in}{2.469333in}}%
\pgfpathlineto{\pgfqpoint{3.000000in}{2.468893in}}%
\pgfpathlineto{\pgfqpoint{2.962667in}{2.449441in}}%
\pgfpathlineto{\pgfqpoint{2.933935in}{2.432000in}}%
\pgfpathlineto{\pgfqpoint{2.925333in}{2.427065in}}%
\pgfpathlineto{\pgfqpoint{2.888000in}{2.404142in}}%
\pgfpathlineto{\pgfqpoint{2.874225in}{2.394667in}}%
\pgfpathlineto{\pgfqpoint{2.850667in}{2.379051in}}%
\pgfpathlineto{\pgfqpoint{2.820705in}{2.357333in}}%
\pgfpathlineto{\pgfqpoint{2.813333in}{2.351860in}}%
\pgfpathlineto{\pgfqpoint{2.776000in}{2.323338in}}%
\pgfpathlineto{\pgfqpoint{2.771814in}{2.320000in}}%
\pgfpathlineto{\pgfqpoint{2.738667in}{2.292430in}}%
\pgfpathlineto{\pgfqpoint{2.727203in}{2.282667in}}%
\pgfpathlineto{\pgfqpoint{2.701333in}{2.258179in}}%
\pgfpathlineto{\pgfqpoint{2.687453in}{2.245333in}}%
\pgfpathlineto{\pgfqpoint{2.664000in}{2.219444in}}%
\pgfpathlineto{\pgfqpoint{2.652893in}{2.208000in}}%
\pgfpathlineto{\pgfqpoint{2.626667in}{2.173010in}}%
\pgfpathlineto{\pgfqpoint{2.624695in}{2.170667in}}%
\pgfpathlineto{\pgfqpoint{2.606116in}{2.133333in}}%
\pgfpathlineto{\pgfqpoint{2.608607in}{2.096000in}}%
\pgfpathlineto{\pgfqpoint{2.626667in}{2.085065in}}%
\pgfusepath{stroke}%
\end{pgfscope}%
\begin{pgfscope}%
\pgfpathrectangle{\pgfqpoint{1.432000in}{0.528000in}}{\pgfqpoint{3.696000in}{3.696000in}}%
\pgfusepath{clip}%
\pgfsetbuttcap%
\pgfsetroundjoin%
\pgfsetlinewidth{1.505625pt}%
\definecolor{currentstroke}{rgb}{1.000000,0.544608,0.000000}%
\pgfsetstrokecolor{currentstroke}%
\pgfsetstrokeopacity{0.300000}%
\pgfsetdash{}{0pt}%
\pgfpathmoveto{\pgfqpoint{2.701333in}{2.129571in}}%
\pgfpathlineto{\pgfqpoint{2.738667in}{2.132521in}}%
\pgfpathlineto{\pgfqpoint{2.741287in}{2.133333in}}%
\pgfpathlineto{\pgfqpoint{2.776000in}{2.143998in}}%
\pgfpathlineto{\pgfqpoint{2.813333in}{2.157823in}}%
\pgfpathlineto{\pgfqpoint{2.841204in}{2.170667in}}%
\pgfpathlineto{\pgfqpoint{2.850667in}{2.175025in}}%
\pgfpathlineto{\pgfqpoint{2.888000in}{2.194289in}}%
\pgfpathlineto{\pgfqpoint{2.911535in}{2.208000in}}%
\pgfpathlineto{\pgfqpoint{2.925333in}{2.216132in}}%
\pgfpathlineto{\pgfqpoint{2.962667in}{2.239353in}}%
\pgfpathlineto{\pgfqpoint{2.971435in}{2.245333in}}%
\pgfpathlineto{\pgfqpoint{3.000000in}{2.265333in}}%
\pgfpathlineto{\pgfqpoint{3.023710in}{2.282667in}}%
\pgfpathlineto{\pgfqpoint{3.037333in}{2.293641in}}%
\pgfpathlineto{\pgfqpoint{3.070214in}{2.320000in}}%
\pgfpathlineto{\pgfqpoint{3.074667in}{2.324273in}}%
\pgfpathlineto{\pgfqpoint{3.111017in}{2.357333in}}%
\pgfpathlineto{\pgfqpoint{3.112000in}{2.358521in}}%
\pgfpathlineto{\pgfqpoint{3.145148in}{2.394667in}}%
\pgfpathlineto{\pgfqpoint{3.149333in}{2.401732in}}%
\pgfpathlineto{\pgfqpoint{3.170235in}{2.432000in}}%
\pgfpathlineto{\pgfqpoint{3.178953in}{2.469333in}}%
\pgfpathlineto{\pgfqpoint{3.149333in}{2.486301in}}%
\pgfpathlineto{\pgfqpoint{3.112000in}{2.484154in}}%
\pgfpathlineto{\pgfqpoint{3.074667in}{2.474657in}}%
\pgfpathlineto{\pgfqpoint{3.061439in}{2.469333in}}%
\pgfpathlineto{\pgfqpoint{3.037333in}{2.459814in}}%
\pgfpathlineto{\pgfqpoint{3.000000in}{2.443085in}}%
\pgfpathlineto{\pgfqpoint{2.979157in}{2.432000in}}%
\pgfpathlineto{\pgfqpoint{2.962667in}{2.423291in}}%
\pgfpathlineto{\pgfqpoint{2.925333in}{2.402096in}}%
\pgfpathlineto{\pgfqpoint{2.913682in}{2.394667in}}%
\pgfpathlineto{\pgfqpoint{2.888000in}{2.378169in}}%
\pgfpathlineto{\pgfqpoint{2.857369in}{2.357333in}}%
\pgfpathlineto{\pgfqpoint{2.850667in}{2.352401in}}%
\pgfpathlineto{\pgfqpoint{2.813333in}{2.324690in}}%
\pgfpathlineto{\pgfqpoint{2.807280in}{2.320000in}}%
\pgfpathlineto{\pgfqpoint{2.776000in}{2.293457in}}%
\pgfpathlineto{\pgfqpoint{2.763226in}{2.282667in}}%
\pgfpathlineto{\pgfqpoint{2.738667in}{2.258152in}}%
\pgfpathlineto{\pgfqpoint{2.725139in}{2.245333in}}%
\pgfpathlineto{\pgfqpoint{2.701333in}{2.216116in}}%
\pgfpathlineto{\pgfqpoint{2.694026in}{2.208000in}}%
\pgfpathlineto{\pgfqpoint{2.674171in}{2.170667in}}%
\pgfpathlineto{\pgfqpoint{2.687690in}{2.133333in}}%
\pgfpathlineto{\pgfqpoint{2.701333in}{2.129571in}}%
\pgfusepath{stroke}%
\end{pgfscope}%
\begin{pgfscope}%
\pgfpathrectangle{\pgfqpoint{1.432000in}{0.528000in}}{\pgfqpoint{3.696000in}{3.696000in}}%
\pgfusepath{clip}%
\pgfsetbuttcap%
\pgfsetroundjoin%
\pgfsetlinewidth{1.505625pt}%
\definecolor{currentstroke}{rgb}{1.000000,0.915196,0.000000}%
\pgfsetstrokecolor{currentstroke}%
\pgfsetstrokeopacity{0.300000}%
\pgfsetdash{}{0pt}%
\pgfpathmoveto{\pgfqpoint{2.776000in}{2.181559in}}%
\pgfpathlineto{\pgfqpoint{2.813333in}{2.189391in}}%
\pgfpathlineto{\pgfqpoint{2.850667in}{2.202321in}}%
\pgfpathlineto{\pgfqpoint{2.862425in}{2.208000in}}%
\pgfpathlineto{\pgfqpoint{2.888000in}{2.221425in}}%
\pgfpathlineto{\pgfqpoint{2.925333in}{2.241467in}}%
\pgfpathlineto{\pgfqpoint{2.931555in}{2.245333in}}%
\pgfpathlineto{\pgfqpoint{2.962667in}{2.266443in}}%
\pgfpathlineto{\pgfqpoint{2.986531in}{2.282667in}}%
\pgfpathlineto{\pgfqpoint{3.000000in}{2.293634in}}%
\pgfpathlineto{\pgfqpoint{3.034049in}{2.320000in}}%
\pgfpathlineto{\pgfqpoint{3.037333in}{2.323452in}}%
\pgfpathlineto{\pgfqpoint{3.073120in}{2.357333in}}%
\pgfpathlineto{\pgfqpoint{3.074667in}{2.359775in}}%
\pgfpathlineto{\pgfqpoint{3.100614in}{2.394667in}}%
\pgfpathlineto{\pgfqpoint{3.105691in}{2.432000in}}%
\pgfpathlineto{\pgfqpoint{3.074667in}{2.438113in}}%
\pgfpathlineto{\pgfqpoint{3.043738in}{2.432000in}}%
\pgfpathlineto{\pgfqpoint{3.037333in}{2.430458in}}%
\pgfpathlineto{\pgfqpoint{3.000000in}{2.414479in}}%
\pgfpathlineto{\pgfqpoint{2.962667in}{2.397663in}}%
\pgfpathlineto{\pgfqpoint{2.957339in}{2.394667in}}%
\pgfpathlineto{\pgfqpoint{2.925333in}{2.375363in}}%
\pgfpathlineto{\pgfqpoint{2.895747in}{2.357333in}}%
\pgfpathlineto{\pgfqpoint{2.888000in}{2.351777in}}%
\pgfpathlineto{\pgfqpoint{2.850667in}{2.325185in}}%
\pgfpathlineto{\pgfqpoint{2.843679in}{2.320000in}}%
\pgfpathlineto{\pgfqpoint{2.813333in}{2.293525in}}%
\pgfpathlineto{\pgfqpoint{2.800363in}{2.282667in}}%
\pgfpathlineto{\pgfqpoint{2.776000in}{2.256019in}}%
\pgfpathlineto{\pgfqpoint{2.765291in}{2.245333in}}%
\pgfpathlineto{\pgfqpoint{2.742524in}{2.208000in}}%
\pgfpathlineto{\pgfqpoint{2.776000in}{2.181559in}}%
\pgfusepath{stroke}%
\end{pgfscope}%
\begin{pgfscope}%
\pgfpathrectangle{\pgfqpoint{1.432000in}{0.528000in}}{\pgfqpoint{3.696000in}{3.696000in}}%
\pgfusepath{clip}%
\pgfsetbuttcap%
\pgfsetroundjoin%
\pgfsetlinewidth{1.505625pt}%
\definecolor{currentstroke}{rgb}{1.000000,1.000000,0.444117}%
\pgfsetstrokecolor{currentstroke}%
\pgfsetstrokeopacity{0.300000}%
\pgfsetdash{}{0pt}%
\pgfpathmoveto{\pgfqpoint{2.850667in}{2.242641in}}%
\pgfpathlineto{\pgfqpoint{2.859656in}{2.245333in}}%
\pgfpathlineto{\pgfqpoint{2.888000in}{2.258370in}}%
\pgfpathlineto{\pgfqpoint{2.925333in}{2.275733in}}%
\pgfpathlineto{\pgfqpoint{2.937225in}{2.282667in}}%
\pgfpathlineto{\pgfqpoint{2.962667in}{2.303668in}}%
\pgfpathlineto{\pgfqpoint{2.985495in}{2.320000in}}%
\pgfpathlineto{\pgfqpoint{3.000000in}{2.340148in}}%
\pgfpathlineto{\pgfqpoint{3.015459in}{2.357333in}}%
\pgfpathlineto{\pgfqpoint{3.000000in}{2.369628in}}%
\pgfpathlineto{\pgfqpoint{2.962667in}{2.362419in}}%
\pgfpathlineto{\pgfqpoint{2.951906in}{2.357333in}}%
\pgfpathlineto{\pgfqpoint{2.925333in}{2.339810in}}%
\pgfpathlineto{\pgfqpoint{2.889613in}{2.320000in}}%
\pgfpathlineto{\pgfqpoint{2.888000in}{2.318272in}}%
\pgfpathlineto{\pgfqpoint{2.850667in}{2.285032in}}%
\pgfpathlineto{\pgfqpoint{2.847810in}{2.282667in}}%
\pgfpathlineto{\pgfqpoint{2.839055in}{2.245333in}}%
\pgfpathlineto{\pgfqpoint{2.850667in}{2.242641in}}%
\pgfusepath{stroke}%
\end{pgfscope}%
\begin{pgfscope}%
\pgfsetrectcap%
\pgfsetmiterjoin%
\pgfsetlinewidth{0.803000pt}%
\definecolor{currentstroke}{rgb}{0.000000,0.000000,0.000000}%
\pgfsetstrokecolor{currentstroke}%
\pgfsetdash{}{0pt}%
\pgfpathmoveto{\pgfqpoint{1.432000in}{0.528000in}}%
\pgfpathlineto{\pgfqpoint{1.432000in}{4.224000in}}%
\pgfusepath{stroke}%
\end{pgfscope}%
\begin{pgfscope}%
\pgfsetrectcap%
\pgfsetmiterjoin%
\pgfsetlinewidth{0.803000pt}%
\definecolor{currentstroke}{rgb}{0.000000,0.000000,0.000000}%
\pgfsetstrokecolor{currentstroke}%
\pgfsetdash{}{0pt}%
\pgfpathmoveto{\pgfqpoint{5.128000in}{0.528000in}}%
\pgfpathlineto{\pgfqpoint{5.128000in}{4.224000in}}%
\pgfusepath{stroke}%
\end{pgfscope}%
\begin{pgfscope}%
\pgfsetrectcap%
\pgfsetmiterjoin%
\pgfsetlinewidth{0.803000pt}%
\definecolor{currentstroke}{rgb}{0.000000,0.000000,0.000000}%
\pgfsetstrokecolor{currentstroke}%
\pgfsetdash{}{0pt}%
\pgfpathmoveto{\pgfqpoint{1.432000in}{0.528000in}}%
\pgfpathlineto{\pgfqpoint{5.128000in}{0.528000in}}%
\pgfusepath{stroke}%
\end{pgfscope}%
\begin{pgfscope}%
\pgfsetrectcap%
\pgfsetmiterjoin%
\pgfsetlinewidth{0.803000pt}%
\definecolor{currentstroke}{rgb}{0.000000,0.000000,0.000000}%
\pgfsetstrokecolor{currentstroke}%
\pgfsetdash{}{0pt}%
\pgfpathmoveto{\pgfqpoint{1.432000in}{4.224000in}}%
\pgfpathlineto{\pgfqpoint{5.128000in}{4.224000in}}%
\pgfusepath{stroke}%
\end{pgfscope}%
\begin{pgfscope}%
\definecolor{textcolor}{rgb}{0.000000,0.000000,0.000000}%
\pgfsetstrokecolor{textcolor}%
\pgfsetfillcolor{textcolor}%
\pgftext[x=3.280000in,y=4.307333in,,base]{\color{textcolor}\rmfamily\fontsize{12.000000}{14.400000}\selectfont Experiment 1A: CE-method (\(\displaystyle k=5\))}%
\end{pgfscope}%
\end{pgfpicture}%
\makeatother%
\endgroup%
}
    }
    \subfloat[The cross-entropy surrogate method.]{%
    \resizebox{0.3\textwidth}{!}{%% Creator: Matplotlib, PGF backend
%%
%% To include the figure in your LaTeX document, write
%%   \input{<filename>.pgf}
%%
%% Make sure the required packages are loaded in your preamble
%%   \usepackage{pgf}
%%
%% Figures using additional raster images can only be included by \input if
%% they are in the same directory as the main LaTeX file. For loading figures
%% from other directories you can use the `import` package
%%   \usepackage{import}
%% and then include the figures with
%%   \import{<path to file>}{<filename>.pgf}
%%
%% Matplotlib used the following preamble
%%   \usepackage{fontspec}
%%   \setmainfont{DejaVuSans.ttf}[Path=C:/Users/mossr/.julia/conda/3/lib/site-packages/matplotlib/mpl-data/fonts/ttf/]
%%   \setsansfont{DejaVuSans.ttf}[Path=C:/Users/mossr/.julia/conda/3/lib/site-packages/matplotlib/mpl-data/fonts/ttf/]
%%   \setmonofont{DejaVuSansMono.ttf}[Path=C:/Users/mossr/.julia/conda/3/lib/site-packages/matplotlib/mpl-data/fonts/ttf/]
%%
\begingroup%
\makeatletter%
\begin{pgfpicture}%
% \pgfpathrectangle{\pgfpointorigin}{\pgfqpoint{5.400000in}{4.800000in}}%
% \pgfpathrectangle{\pgfqpoint{1.432000in}{0.0in}}{\pgfqpoint{3.696000in}{4.0in}}%
\pgfpathrectangle{\pgfqpoint{1.0in}{0.0in}}{\pgfqpoint{4.0in}{4.5in}}%
\pgfusepath{use as bounding box, clip}%
\begin{pgfscope}%
\pgfsetbuttcap%
\pgfsetmiterjoin%
\definecolor{currentfill}{rgb}{1.000000,1.000000,1.000000}%
\pgfsetfillcolor{currentfill}%
\pgfsetlinewidth{0.000000pt}%
\definecolor{currentstroke}{rgb}{1.000000,1.000000,1.000000}%
\pgfsetstrokecolor{currentstroke}%
\pgfsetdash{}{0pt}%
\pgfpathmoveto{\pgfqpoint{0.000000in}{0.000000in}}%
\pgfpathlineto{\pgfqpoint{6.400000in}{0.000000in}}%
\pgfpathlineto{\pgfqpoint{6.400000in}{4.800000in}}%
\pgfpathlineto{\pgfqpoint{0.000000in}{4.800000in}}%
\pgfpathclose%
\pgfusepath{fill}%
\end{pgfscope}%
\begin{pgfscope}%
\pgfsetbuttcap%
\pgfsetmiterjoin%
\definecolor{currentfill}{rgb}{1.000000,1.000000,1.000000}%
\pgfsetfillcolor{currentfill}%
\pgfsetlinewidth{0.000000pt}%
\definecolor{currentstroke}{rgb}{0.000000,0.000000,0.000000}%
\pgfsetstrokecolor{currentstroke}%
\pgfsetstrokeopacity{0.000000}%
\pgfsetdash{}{0pt}%
\pgfpathmoveto{\pgfqpoint{1.432000in}{0.528000in}}%
\pgfpathlineto{\pgfqpoint{5.128000in}{0.528000in}}%
\pgfpathlineto{\pgfqpoint{5.128000in}{4.224000in}}%
\pgfpathlineto{\pgfqpoint{1.432000in}{4.224000in}}%
\pgfpathclose%
\pgfusepath{fill}%
\end{pgfscope}%
\begin{pgfscope}%
\pgfpathrectangle{\pgfqpoint{1.432000in}{0.528000in}}{\pgfqpoint{3.696000in}{3.696000in}}%
\pgfusepath{clip}%
\pgfsys@transformshift{1.432000in}{0.528000in}%
\pgftext[left,bottom]{\pgfimage[interpolate=true,width=3.700000in,height=3.700000in]{figures/cem_variants/k5_ce_surrogate-img0.png}}%
\end{pgfscope}%
\begin{pgfscope}%
\pgfpathrectangle{\pgfqpoint{1.432000in}{0.528000in}}{\pgfqpoint{3.696000in}{3.696000in}}%
\pgfusepath{clip}%
\pgfsetbuttcap%
\pgfsetroundjoin%
\definecolor{currentfill}{rgb}{0.000000,0.000000,0.000000}%
\pgfsetfillcolor{currentfill}%
\pgfsetlinewidth{0.501875pt}%
\definecolor{currentstroke}{rgb}{1.000000,1.000000,1.000000}%
\pgfsetstrokecolor{currentstroke}%
\pgfsetdash{}{0pt}%
\pgfsys@defobject{currentmarker}{\pgfqpoint{-0.018373in}{-0.018373in}}{\pgfqpoint{0.018373in}{0.018373in}}{%
\pgfpathmoveto{\pgfqpoint{0.000000in}{-0.018373in}}%
\pgfpathcurveto{\pgfqpoint{0.004873in}{-0.018373in}}{\pgfqpoint{0.009546in}{-0.016437in}}{\pgfqpoint{0.012992in}{-0.012992in}}%
\pgfpathcurveto{\pgfqpoint{0.016437in}{-0.009546in}}{\pgfqpoint{0.018373in}{-0.004873in}}{\pgfqpoint{0.018373in}{0.000000in}}%
\pgfpathcurveto{\pgfqpoint{0.018373in}{0.004873in}}{\pgfqpoint{0.016437in}{0.009546in}}{\pgfqpoint{0.012992in}{0.012992in}}%
\pgfpathcurveto{\pgfqpoint{0.009546in}{0.016437in}}{\pgfqpoint{0.004873in}{0.018373in}}{\pgfqpoint{0.000000in}{0.018373in}}%
\pgfpathcurveto{\pgfqpoint{-0.004873in}{0.018373in}}{\pgfqpoint{-0.009546in}{0.016437in}}{\pgfqpoint{-0.012992in}{0.012992in}}%
\pgfpathcurveto{\pgfqpoint{-0.016437in}{0.009546in}}{\pgfqpoint{-0.018373in}{0.004873in}}{\pgfqpoint{-0.018373in}{0.000000in}}%
\pgfpathcurveto{\pgfqpoint{-0.018373in}{-0.004873in}}{\pgfqpoint{-0.016437in}{-0.009546in}}{\pgfqpoint{-0.012992in}{-0.012992in}}%
\pgfpathcurveto{\pgfqpoint{-0.009546in}{-0.016437in}}{\pgfqpoint{-0.004873in}{-0.018373in}}{\pgfqpoint{0.000000in}{-0.018373in}}%
\pgfpathclose%
\pgfusepath{stroke,fill}%
}%
\begin{pgfscope}%
\pgfsys@transformshift{3.869958in}{2.569666in}%
\pgfsys@useobject{currentmarker}{}%
\end{pgfscope}%
\end{pgfscope}%
\begin{pgfscope}%
\pgfpathrectangle{\pgfqpoint{1.432000in}{0.528000in}}{\pgfqpoint{3.696000in}{3.696000in}}%
\pgfusepath{clip}%
\pgfsetbuttcap%
\pgfsetroundjoin%
\definecolor{currentfill}{rgb}{0.000000,0.000000,0.000000}%
\pgfsetfillcolor{currentfill}%
\pgfsetlinewidth{0.501875pt}%
\definecolor{currentstroke}{rgb}{1.000000,1.000000,1.000000}%
\pgfsetstrokecolor{currentstroke}%
\pgfsetdash{}{0pt}%
\pgfsys@defobject{currentmarker}{\pgfqpoint{-0.018373in}{-0.018373in}}{\pgfqpoint{0.018373in}{0.018373in}}{%
\pgfpathmoveto{\pgfqpoint{0.000000in}{-0.018373in}}%
\pgfpathcurveto{\pgfqpoint{0.004873in}{-0.018373in}}{\pgfqpoint{0.009546in}{-0.016437in}}{\pgfqpoint{0.012992in}{-0.012992in}}%
\pgfpathcurveto{\pgfqpoint{0.016437in}{-0.009546in}}{\pgfqpoint{0.018373in}{-0.004873in}}{\pgfqpoint{0.018373in}{0.000000in}}%
\pgfpathcurveto{\pgfqpoint{0.018373in}{0.004873in}}{\pgfqpoint{0.016437in}{0.009546in}}{\pgfqpoint{0.012992in}{0.012992in}}%
\pgfpathcurveto{\pgfqpoint{0.009546in}{0.016437in}}{\pgfqpoint{0.004873in}{0.018373in}}{\pgfqpoint{0.000000in}{0.018373in}}%
\pgfpathcurveto{\pgfqpoint{-0.004873in}{0.018373in}}{\pgfqpoint{-0.009546in}{0.016437in}}{\pgfqpoint{-0.012992in}{0.012992in}}%
\pgfpathcurveto{\pgfqpoint{-0.016437in}{0.009546in}}{\pgfqpoint{-0.018373in}{0.004873in}}{\pgfqpoint{-0.018373in}{0.000000in}}%
\pgfpathcurveto{\pgfqpoint{-0.018373in}{-0.004873in}}{\pgfqpoint{-0.016437in}{-0.009546in}}{\pgfqpoint{-0.012992in}{-0.012992in}}%
\pgfpathcurveto{\pgfqpoint{-0.009546in}{-0.016437in}}{\pgfqpoint{-0.004873in}{-0.018373in}}{\pgfqpoint{0.000000in}{-0.018373in}}%
\pgfpathclose%
\pgfusepath{stroke,fill}%
}%
\begin{pgfscope}%
\pgfsys@transformshift{3.289556in}{2.412084in}%
\pgfsys@useobject{currentmarker}{}%
\end{pgfscope}%
\end{pgfscope}%
\begin{pgfscope}%
\pgfpathrectangle{\pgfqpoint{1.432000in}{0.528000in}}{\pgfqpoint{3.696000in}{3.696000in}}%
\pgfusepath{clip}%
\pgfsetbuttcap%
\pgfsetroundjoin%
\definecolor{currentfill}{rgb}{0.000000,0.000000,0.000000}%
\pgfsetfillcolor{currentfill}%
\pgfsetlinewidth{0.501875pt}%
\definecolor{currentstroke}{rgb}{1.000000,1.000000,1.000000}%
\pgfsetstrokecolor{currentstroke}%
\pgfsetdash{}{0pt}%
\pgfsys@defobject{currentmarker}{\pgfqpoint{-0.018373in}{-0.018373in}}{\pgfqpoint{0.018373in}{0.018373in}}{%
\pgfpathmoveto{\pgfqpoint{0.000000in}{-0.018373in}}%
\pgfpathcurveto{\pgfqpoint{0.004873in}{-0.018373in}}{\pgfqpoint{0.009546in}{-0.016437in}}{\pgfqpoint{0.012992in}{-0.012992in}}%
\pgfpathcurveto{\pgfqpoint{0.016437in}{-0.009546in}}{\pgfqpoint{0.018373in}{-0.004873in}}{\pgfqpoint{0.018373in}{0.000000in}}%
\pgfpathcurveto{\pgfqpoint{0.018373in}{0.004873in}}{\pgfqpoint{0.016437in}{0.009546in}}{\pgfqpoint{0.012992in}{0.012992in}}%
\pgfpathcurveto{\pgfqpoint{0.009546in}{0.016437in}}{\pgfqpoint{0.004873in}{0.018373in}}{\pgfqpoint{0.000000in}{0.018373in}}%
\pgfpathcurveto{\pgfqpoint{-0.004873in}{0.018373in}}{\pgfqpoint{-0.009546in}{0.016437in}}{\pgfqpoint{-0.012992in}{0.012992in}}%
\pgfpathcurveto{\pgfqpoint{-0.016437in}{0.009546in}}{\pgfqpoint{-0.018373in}{0.004873in}}{\pgfqpoint{-0.018373in}{0.000000in}}%
\pgfpathcurveto{\pgfqpoint{-0.018373in}{-0.004873in}}{\pgfqpoint{-0.016437in}{-0.009546in}}{\pgfqpoint{-0.012992in}{-0.012992in}}%
\pgfpathcurveto{\pgfqpoint{-0.009546in}{-0.016437in}}{\pgfqpoint{-0.004873in}{-0.018373in}}{\pgfqpoint{0.000000in}{-0.018373in}}%
\pgfpathclose%
\pgfusepath{stroke,fill}%
}%
\begin{pgfscope}%
\pgfsys@transformshift{2.950585in}{2.694697in}%
\pgfsys@useobject{currentmarker}{}%
\end{pgfscope}%
\end{pgfscope}%
\begin{pgfscope}%
\pgfpathrectangle{\pgfqpoint{1.432000in}{0.528000in}}{\pgfqpoint{3.696000in}{3.696000in}}%
\pgfusepath{clip}%
\pgfsetbuttcap%
\pgfsetroundjoin%
\definecolor{currentfill}{rgb}{0.000000,0.000000,0.000000}%
\pgfsetfillcolor{currentfill}%
\pgfsetlinewidth{0.501875pt}%
\definecolor{currentstroke}{rgb}{1.000000,1.000000,1.000000}%
\pgfsetstrokecolor{currentstroke}%
\pgfsetdash{}{0pt}%
\pgfsys@defobject{currentmarker}{\pgfqpoint{-0.018373in}{-0.018373in}}{\pgfqpoint{0.018373in}{0.018373in}}{%
\pgfpathmoveto{\pgfqpoint{0.000000in}{-0.018373in}}%
\pgfpathcurveto{\pgfqpoint{0.004873in}{-0.018373in}}{\pgfqpoint{0.009546in}{-0.016437in}}{\pgfqpoint{0.012992in}{-0.012992in}}%
\pgfpathcurveto{\pgfqpoint{0.016437in}{-0.009546in}}{\pgfqpoint{0.018373in}{-0.004873in}}{\pgfqpoint{0.018373in}{0.000000in}}%
\pgfpathcurveto{\pgfqpoint{0.018373in}{0.004873in}}{\pgfqpoint{0.016437in}{0.009546in}}{\pgfqpoint{0.012992in}{0.012992in}}%
\pgfpathcurveto{\pgfqpoint{0.009546in}{0.016437in}}{\pgfqpoint{0.004873in}{0.018373in}}{\pgfqpoint{0.000000in}{0.018373in}}%
\pgfpathcurveto{\pgfqpoint{-0.004873in}{0.018373in}}{\pgfqpoint{-0.009546in}{0.016437in}}{\pgfqpoint{-0.012992in}{0.012992in}}%
\pgfpathcurveto{\pgfqpoint{-0.016437in}{0.009546in}}{\pgfqpoint{-0.018373in}{0.004873in}}{\pgfqpoint{-0.018373in}{0.000000in}}%
\pgfpathcurveto{\pgfqpoint{-0.018373in}{-0.004873in}}{\pgfqpoint{-0.016437in}{-0.009546in}}{\pgfqpoint{-0.012992in}{-0.012992in}}%
\pgfpathcurveto{\pgfqpoint{-0.009546in}{-0.016437in}}{\pgfqpoint{-0.004873in}{-0.018373in}}{\pgfqpoint{0.000000in}{-0.018373in}}%
\pgfpathclose%
\pgfusepath{stroke,fill}%
}%
\begin{pgfscope}%
\pgfsys@transformshift{3.547797in}{2.484532in}%
\pgfsys@useobject{currentmarker}{}%
\end{pgfscope}%
\end{pgfscope}%
\begin{pgfscope}%
\pgfpathrectangle{\pgfqpoint{1.432000in}{0.528000in}}{\pgfqpoint{3.696000in}{3.696000in}}%
\pgfusepath{clip}%
\pgfsetbuttcap%
\pgfsetroundjoin%
\definecolor{currentfill}{rgb}{0.000000,0.000000,0.000000}%
\pgfsetfillcolor{currentfill}%
\pgfsetlinewidth{0.501875pt}%
\definecolor{currentstroke}{rgb}{1.000000,1.000000,1.000000}%
\pgfsetstrokecolor{currentstroke}%
\pgfsetdash{}{0pt}%
\pgfsys@defobject{currentmarker}{\pgfqpoint{-0.018373in}{-0.018373in}}{\pgfqpoint{0.018373in}{0.018373in}}{%
\pgfpathmoveto{\pgfqpoint{0.000000in}{-0.018373in}}%
\pgfpathcurveto{\pgfqpoint{0.004873in}{-0.018373in}}{\pgfqpoint{0.009546in}{-0.016437in}}{\pgfqpoint{0.012992in}{-0.012992in}}%
\pgfpathcurveto{\pgfqpoint{0.016437in}{-0.009546in}}{\pgfqpoint{0.018373in}{-0.004873in}}{\pgfqpoint{0.018373in}{0.000000in}}%
\pgfpathcurveto{\pgfqpoint{0.018373in}{0.004873in}}{\pgfqpoint{0.016437in}{0.009546in}}{\pgfqpoint{0.012992in}{0.012992in}}%
\pgfpathcurveto{\pgfqpoint{0.009546in}{0.016437in}}{\pgfqpoint{0.004873in}{0.018373in}}{\pgfqpoint{0.000000in}{0.018373in}}%
\pgfpathcurveto{\pgfqpoint{-0.004873in}{0.018373in}}{\pgfqpoint{-0.009546in}{0.016437in}}{\pgfqpoint{-0.012992in}{0.012992in}}%
\pgfpathcurveto{\pgfqpoint{-0.016437in}{0.009546in}}{\pgfqpoint{-0.018373in}{0.004873in}}{\pgfqpoint{-0.018373in}{0.000000in}}%
\pgfpathcurveto{\pgfqpoint{-0.018373in}{-0.004873in}}{\pgfqpoint{-0.016437in}{-0.009546in}}{\pgfqpoint{-0.012992in}{-0.012992in}}%
\pgfpathcurveto{\pgfqpoint{-0.009546in}{-0.016437in}}{\pgfqpoint{-0.004873in}{-0.018373in}}{\pgfqpoint{0.000000in}{-0.018373in}}%
\pgfpathclose%
\pgfusepath{stroke,fill}%
}%
\begin{pgfscope}%
\pgfsys@transformshift{3.281881in}{2.554559in}%
\pgfsys@useobject{currentmarker}{}%
\end{pgfscope}%
\end{pgfscope}%
\begin{pgfscope}%
\pgfpathrectangle{\pgfqpoint{1.432000in}{0.528000in}}{\pgfqpoint{3.696000in}{3.696000in}}%
\pgfusepath{clip}%
\pgfsetbuttcap%
\pgfsetroundjoin%
\definecolor{currentfill}{rgb}{0.000000,0.000000,0.000000}%
\pgfsetfillcolor{currentfill}%
\pgfsetlinewidth{0.501875pt}%
\definecolor{currentstroke}{rgb}{1.000000,1.000000,1.000000}%
\pgfsetstrokecolor{currentstroke}%
\pgfsetdash{}{0pt}%
\pgfsys@defobject{currentmarker}{\pgfqpoint{-0.018373in}{-0.018373in}}{\pgfqpoint{0.018373in}{0.018373in}}{%
\pgfpathmoveto{\pgfqpoint{0.000000in}{-0.018373in}}%
\pgfpathcurveto{\pgfqpoint{0.004873in}{-0.018373in}}{\pgfqpoint{0.009546in}{-0.016437in}}{\pgfqpoint{0.012992in}{-0.012992in}}%
\pgfpathcurveto{\pgfqpoint{0.016437in}{-0.009546in}}{\pgfqpoint{0.018373in}{-0.004873in}}{\pgfqpoint{0.018373in}{0.000000in}}%
\pgfpathcurveto{\pgfqpoint{0.018373in}{0.004873in}}{\pgfqpoint{0.016437in}{0.009546in}}{\pgfqpoint{0.012992in}{0.012992in}}%
\pgfpathcurveto{\pgfqpoint{0.009546in}{0.016437in}}{\pgfqpoint{0.004873in}{0.018373in}}{\pgfqpoint{0.000000in}{0.018373in}}%
\pgfpathcurveto{\pgfqpoint{-0.004873in}{0.018373in}}{\pgfqpoint{-0.009546in}{0.016437in}}{\pgfqpoint{-0.012992in}{0.012992in}}%
\pgfpathcurveto{\pgfqpoint{-0.016437in}{0.009546in}}{\pgfqpoint{-0.018373in}{0.004873in}}{\pgfqpoint{-0.018373in}{0.000000in}}%
\pgfpathcurveto{\pgfqpoint{-0.018373in}{-0.004873in}}{\pgfqpoint{-0.016437in}{-0.009546in}}{\pgfqpoint{-0.012992in}{-0.012992in}}%
\pgfpathcurveto{\pgfqpoint{-0.009546in}{-0.016437in}}{\pgfqpoint{-0.004873in}{-0.018373in}}{\pgfqpoint{0.000000in}{-0.018373in}}%
\pgfpathclose%
\pgfusepath{stroke,fill}%
}%
\begin{pgfscope}%
\pgfsys@transformshift{3.892231in}{2.559505in}%
\pgfsys@useobject{currentmarker}{}%
\end{pgfscope}%
\end{pgfscope}%
\begin{pgfscope}%
\pgfpathrectangle{\pgfqpoint{1.432000in}{0.528000in}}{\pgfqpoint{3.696000in}{3.696000in}}%
\pgfusepath{clip}%
\pgfsetbuttcap%
\pgfsetroundjoin%
\definecolor{currentfill}{rgb}{0.000000,0.000000,0.000000}%
\pgfsetfillcolor{currentfill}%
\pgfsetlinewidth{0.501875pt}%
\definecolor{currentstroke}{rgb}{1.000000,1.000000,1.000000}%
\pgfsetstrokecolor{currentstroke}%
\pgfsetdash{}{0pt}%
\pgfsys@defobject{currentmarker}{\pgfqpoint{-0.018373in}{-0.018373in}}{\pgfqpoint{0.018373in}{0.018373in}}{%
\pgfpathmoveto{\pgfqpoint{0.000000in}{-0.018373in}}%
\pgfpathcurveto{\pgfqpoint{0.004873in}{-0.018373in}}{\pgfqpoint{0.009546in}{-0.016437in}}{\pgfqpoint{0.012992in}{-0.012992in}}%
\pgfpathcurveto{\pgfqpoint{0.016437in}{-0.009546in}}{\pgfqpoint{0.018373in}{-0.004873in}}{\pgfqpoint{0.018373in}{0.000000in}}%
\pgfpathcurveto{\pgfqpoint{0.018373in}{0.004873in}}{\pgfqpoint{0.016437in}{0.009546in}}{\pgfqpoint{0.012992in}{0.012992in}}%
\pgfpathcurveto{\pgfqpoint{0.009546in}{0.016437in}}{\pgfqpoint{0.004873in}{0.018373in}}{\pgfqpoint{0.000000in}{0.018373in}}%
\pgfpathcurveto{\pgfqpoint{-0.004873in}{0.018373in}}{\pgfqpoint{-0.009546in}{0.016437in}}{\pgfqpoint{-0.012992in}{0.012992in}}%
\pgfpathcurveto{\pgfqpoint{-0.016437in}{0.009546in}}{\pgfqpoint{-0.018373in}{0.004873in}}{\pgfqpoint{-0.018373in}{0.000000in}}%
\pgfpathcurveto{\pgfqpoint{-0.018373in}{-0.004873in}}{\pgfqpoint{-0.016437in}{-0.009546in}}{\pgfqpoint{-0.012992in}{-0.012992in}}%
\pgfpathcurveto{\pgfqpoint{-0.009546in}{-0.016437in}}{\pgfqpoint{-0.004873in}{-0.018373in}}{\pgfqpoint{0.000000in}{-0.018373in}}%
\pgfpathclose%
\pgfusepath{stroke,fill}%
}%
\begin{pgfscope}%
\pgfsys@transformshift{3.411789in}{2.556180in}%
\pgfsys@useobject{currentmarker}{}%
\end{pgfscope}%
\end{pgfscope}%
\begin{pgfscope}%
\pgfpathrectangle{\pgfqpoint{1.432000in}{0.528000in}}{\pgfqpoint{3.696000in}{3.696000in}}%
\pgfusepath{clip}%
\pgfsetbuttcap%
\pgfsetroundjoin%
\definecolor{currentfill}{rgb}{0.000000,0.000000,0.000000}%
\pgfsetfillcolor{currentfill}%
\pgfsetlinewidth{0.501875pt}%
\definecolor{currentstroke}{rgb}{1.000000,1.000000,1.000000}%
\pgfsetstrokecolor{currentstroke}%
\pgfsetdash{}{0pt}%
\pgfsys@defobject{currentmarker}{\pgfqpoint{-0.018373in}{-0.018373in}}{\pgfqpoint{0.018373in}{0.018373in}}{%
\pgfpathmoveto{\pgfqpoint{0.000000in}{-0.018373in}}%
\pgfpathcurveto{\pgfqpoint{0.004873in}{-0.018373in}}{\pgfqpoint{0.009546in}{-0.016437in}}{\pgfqpoint{0.012992in}{-0.012992in}}%
\pgfpathcurveto{\pgfqpoint{0.016437in}{-0.009546in}}{\pgfqpoint{0.018373in}{-0.004873in}}{\pgfqpoint{0.018373in}{0.000000in}}%
\pgfpathcurveto{\pgfqpoint{0.018373in}{0.004873in}}{\pgfqpoint{0.016437in}{0.009546in}}{\pgfqpoint{0.012992in}{0.012992in}}%
\pgfpathcurveto{\pgfqpoint{0.009546in}{0.016437in}}{\pgfqpoint{0.004873in}{0.018373in}}{\pgfqpoint{0.000000in}{0.018373in}}%
\pgfpathcurveto{\pgfqpoint{-0.004873in}{0.018373in}}{\pgfqpoint{-0.009546in}{0.016437in}}{\pgfqpoint{-0.012992in}{0.012992in}}%
\pgfpathcurveto{\pgfqpoint{-0.016437in}{0.009546in}}{\pgfqpoint{-0.018373in}{0.004873in}}{\pgfqpoint{-0.018373in}{0.000000in}}%
\pgfpathcurveto{\pgfqpoint{-0.018373in}{-0.004873in}}{\pgfqpoint{-0.016437in}{-0.009546in}}{\pgfqpoint{-0.012992in}{-0.012992in}}%
\pgfpathcurveto{\pgfqpoint{-0.009546in}{-0.016437in}}{\pgfqpoint{-0.004873in}{-0.018373in}}{\pgfqpoint{0.000000in}{-0.018373in}}%
\pgfpathclose%
\pgfusepath{stroke,fill}%
}%
\begin{pgfscope}%
\pgfsys@transformshift{3.332165in}{2.424052in}%
\pgfsys@useobject{currentmarker}{}%
\end{pgfscope}%
\end{pgfscope}%
\begin{pgfscope}%
\pgfpathrectangle{\pgfqpoint{1.432000in}{0.528000in}}{\pgfqpoint{3.696000in}{3.696000in}}%
\pgfusepath{clip}%
\pgfsetbuttcap%
\pgfsetroundjoin%
\definecolor{currentfill}{rgb}{0.000000,0.000000,0.000000}%
\pgfsetfillcolor{currentfill}%
\pgfsetlinewidth{0.501875pt}%
\definecolor{currentstroke}{rgb}{1.000000,1.000000,1.000000}%
\pgfsetstrokecolor{currentstroke}%
\pgfsetdash{}{0pt}%
\pgfsys@defobject{currentmarker}{\pgfqpoint{-0.018373in}{-0.018373in}}{\pgfqpoint{0.018373in}{0.018373in}}{%
\pgfpathmoveto{\pgfqpoint{0.000000in}{-0.018373in}}%
\pgfpathcurveto{\pgfqpoint{0.004873in}{-0.018373in}}{\pgfqpoint{0.009546in}{-0.016437in}}{\pgfqpoint{0.012992in}{-0.012992in}}%
\pgfpathcurveto{\pgfqpoint{0.016437in}{-0.009546in}}{\pgfqpoint{0.018373in}{-0.004873in}}{\pgfqpoint{0.018373in}{0.000000in}}%
\pgfpathcurveto{\pgfqpoint{0.018373in}{0.004873in}}{\pgfqpoint{0.016437in}{0.009546in}}{\pgfqpoint{0.012992in}{0.012992in}}%
\pgfpathcurveto{\pgfqpoint{0.009546in}{0.016437in}}{\pgfqpoint{0.004873in}{0.018373in}}{\pgfqpoint{0.000000in}{0.018373in}}%
\pgfpathcurveto{\pgfqpoint{-0.004873in}{0.018373in}}{\pgfqpoint{-0.009546in}{0.016437in}}{\pgfqpoint{-0.012992in}{0.012992in}}%
\pgfpathcurveto{\pgfqpoint{-0.016437in}{0.009546in}}{\pgfqpoint{-0.018373in}{0.004873in}}{\pgfqpoint{-0.018373in}{0.000000in}}%
\pgfpathcurveto{\pgfqpoint{-0.018373in}{-0.004873in}}{\pgfqpoint{-0.016437in}{-0.009546in}}{\pgfqpoint{-0.012992in}{-0.012992in}}%
\pgfpathcurveto{\pgfqpoint{-0.009546in}{-0.016437in}}{\pgfqpoint{-0.004873in}{-0.018373in}}{\pgfqpoint{0.000000in}{-0.018373in}}%
\pgfpathclose%
\pgfusepath{stroke,fill}%
}%
\begin{pgfscope}%
\pgfsys@transformshift{3.057383in}{2.621550in}%
\pgfsys@useobject{currentmarker}{}%
\end{pgfscope}%
\end{pgfscope}%
\begin{pgfscope}%
\pgfpathrectangle{\pgfqpoint{1.432000in}{0.528000in}}{\pgfqpoint{3.696000in}{3.696000in}}%
\pgfusepath{clip}%
\pgfsetbuttcap%
\pgfsetroundjoin%
\definecolor{currentfill}{rgb}{0.000000,0.000000,0.000000}%
\pgfsetfillcolor{currentfill}%
\pgfsetlinewidth{0.501875pt}%
\definecolor{currentstroke}{rgb}{1.000000,1.000000,1.000000}%
\pgfsetstrokecolor{currentstroke}%
\pgfsetdash{}{0pt}%
\pgfsys@defobject{currentmarker}{\pgfqpoint{-0.018373in}{-0.018373in}}{\pgfqpoint{0.018373in}{0.018373in}}{%
\pgfpathmoveto{\pgfqpoint{0.000000in}{-0.018373in}}%
\pgfpathcurveto{\pgfqpoint{0.004873in}{-0.018373in}}{\pgfqpoint{0.009546in}{-0.016437in}}{\pgfqpoint{0.012992in}{-0.012992in}}%
\pgfpathcurveto{\pgfqpoint{0.016437in}{-0.009546in}}{\pgfqpoint{0.018373in}{-0.004873in}}{\pgfqpoint{0.018373in}{0.000000in}}%
\pgfpathcurveto{\pgfqpoint{0.018373in}{0.004873in}}{\pgfqpoint{0.016437in}{0.009546in}}{\pgfqpoint{0.012992in}{0.012992in}}%
\pgfpathcurveto{\pgfqpoint{0.009546in}{0.016437in}}{\pgfqpoint{0.004873in}{0.018373in}}{\pgfqpoint{0.000000in}{0.018373in}}%
\pgfpathcurveto{\pgfqpoint{-0.004873in}{0.018373in}}{\pgfqpoint{-0.009546in}{0.016437in}}{\pgfqpoint{-0.012992in}{0.012992in}}%
\pgfpathcurveto{\pgfqpoint{-0.016437in}{0.009546in}}{\pgfqpoint{-0.018373in}{0.004873in}}{\pgfqpoint{-0.018373in}{0.000000in}}%
\pgfpathcurveto{\pgfqpoint{-0.018373in}{-0.004873in}}{\pgfqpoint{-0.016437in}{-0.009546in}}{\pgfqpoint{-0.012992in}{-0.012992in}}%
\pgfpathcurveto{\pgfqpoint{-0.009546in}{-0.016437in}}{\pgfqpoint{-0.004873in}{-0.018373in}}{\pgfqpoint{0.000000in}{-0.018373in}}%
\pgfpathclose%
\pgfusepath{stroke,fill}%
}%
\begin{pgfscope}%
\pgfsys@transformshift{3.152055in}{2.754986in}%
\pgfsys@useobject{currentmarker}{}%
\end{pgfscope}%
\end{pgfscope}%
\begin{pgfscope}%
\pgfpathrectangle{\pgfqpoint{1.432000in}{0.528000in}}{\pgfqpoint{3.696000in}{3.696000in}}%
\pgfusepath{clip}%
\pgfsetbuttcap%
\pgfsetroundjoin%
\definecolor{currentfill}{rgb}{1.000000,1.000000,1.000000}%
\pgfsetfillcolor{currentfill}%
\pgfsetlinewidth{0.501875pt}%
\definecolor{currentstroke}{rgb}{0.000000,0.000000,0.000000}%
\pgfsetstrokecolor{currentstroke}%
\pgfsetdash{}{0pt}%
\pgfsys@defobject{currentmarker}{\pgfqpoint{-0.018373in}{-0.018373in}}{\pgfqpoint{0.018373in}{0.018373in}}{%
\pgfpathmoveto{\pgfqpoint{0.000000in}{-0.018373in}}%
\pgfpathcurveto{\pgfqpoint{0.004873in}{-0.018373in}}{\pgfqpoint{0.009546in}{-0.016437in}}{\pgfqpoint{0.012992in}{-0.012992in}}%
\pgfpathcurveto{\pgfqpoint{0.016437in}{-0.009546in}}{\pgfqpoint{0.018373in}{-0.004873in}}{\pgfqpoint{0.018373in}{0.000000in}}%
\pgfpathcurveto{\pgfqpoint{0.018373in}{0.004873in}}{\pgfqpoint{0.016437in}{0.009546in}}{\pgfqpoint{0.012992in}{0.012992in}}%
\pgfpathcurveto{\pgfqpoint{0.009546in}{0.016437in}}{\pgfqpoint{0.004873in}{0.018373in}}{\pgfqpoint{0.000000in}{0.018373in}}%
\pgfpathcurveto{\pgfqpoint{-0.004873in}{0.018373in}}{\pgfqpoint{-0.009546in}{0.016437in}}{\pgfqpoint{-0.012992in}{0.012992in}}%
\pgfpathcurveto{\pgfqpoint{-0.016437in}{0.009546in}}{\pgfqpoint{-0.018373in}{0.004873in}}{\pgfqpoint{-0.018373in}{0.000000in}}%
\pgfpathcurveto{\pgfqpoint{-0.018373in}{-0.004873in}}{\pgfqpoint{-0.016437in}{-0.009546in}}{\pgfqpoint{-0.012992in}{-0.012992in}}%
\pgfpathcurveto{\pgfqpoint{-0.009546in}{-0.016437in}}{\pgfqpoint{-0.004873in}{-0.018373in}}{\pgfqpoint{0.000000in}{-0.018373in}}%
\pgfpathclose%
\pgfusepath{stroke,fill}%
}%
\begin{pgfscope}%
\pgfsys@transformshift{3.283292in}{2.354501in}%
\pgfsys@useobject{currentmarker}{}%
\end{pgfscope}%
\begin{pgfscope}%
\pgfsys@transformshift{3.273021in}{2.344113in}%
\pgfsys@useobject{currentmarker}{}%
\end{pgfscope}%
\begin{pgfscope}%
\pgfsys@transformshift{3.319048in}{2.404178in}%
\pgfsys@useobject{currentmarker}{}%
\end{pgfscope}%
\begin{pgfscope}%
\pgfsys@transformshift{3.233071in}{2.424603in}%
\pgfsys@useobject{currentmarker}{}%
\end{pgfscope}%
\begin{pgfscope}%
\pgfsys@transformshift{3.039480in}{2.585090in}%
\pgfsys@useobject{currentmarker}{}%
\end{pgfscope}%
\begin{pgfscope}%
\pgfsys@transformshift{3.054800in}{2.657985in}%
\pgfsys@useobject{currentmarker}{}%
\end{pgfscope}%
\begin{pgfscope}%
\pgfsys@transformshift{3.276301in}{2.312427in}%
\pgfsys@useobject{currentmarker}{}%
\end{pgfscope}%
\begin{pgfscope}%
\pgfsys@transformshift{3.035734in}{2.683244in}%
\pgfsys@useobject{currentmarker}{}%
\end{pgfscope}%
\begin{pgfscope}%
\pgfsys@transformshift{2.948045in}{2.629035in}%
\pgfsys@useobject{currentmarker}{}%
\end{pgfscope}%
\begin{pgfscope}%
\pgfsys@transformshift{2.950072in}{2.646289in}%
\pgfsys@useobject{currentmarker}{}%
\end{pgfscope}%
\begin{pgfscope}%
\pgfsys@transformshift{2.999372in}{2.701313in}%
\pgfsys@useobject{currentmarker}{}%
\end{pgfscope}%
\begin{pgfscope}%
\pgfsys@transformshift{3.103530in}{2.674313in}%
\pgfsys@useobject{currentmarker}{}%
\end{pgfscope}%
\begin{pgfscope}%
\pgfsys@transformshift{3.425671in}{2.557898in}%
\pgfsys@useobject{currentmarker}{}%
\end{pgfscope}%
\begin{pgfscope}%
\pgfsys@transformshift{2.966100in}{2.174734in}%
\pgfsys@useobject{currentmarker}{}%
\end{pgfscope}%
\begin{pgfscope}%
\pgfsys@transformshift{3.399755in}{2.606349in}%
\pgfsys@useobject{currentmarker}{}%
\end{pgfscope}%
\begin{pgfscope}%
\pgfsys@transformshift{3.459670in}{2.541984in}%
\pgfsys@useobject{currentmarker}{}%
\end{pgfscope}%
\begin{pgfscope}%
\pgfsys@transformshift{3.075318in}{2.709145in}%
\pgfsys@useobject{currentmarker}{}%
\end{pgfscope}%
\begin{pgfscope}%
\pgfsys@transformshift{3.433755in}{2.543333in}%
\pgfsys@useobject{currentmarker}{}%
\end{pgfscope}%
\begin{pgfscope}%
\pgfsys@transformshift{2.983248in}{2.169458in}%
\pgfsys@useobject{currentmarker}{}%
\end{pgfscope}%
\begin{pgfscope}%
\pgfsys@transformshift{3.504693in}{2.565738in}%
\pgfsys@useobject{currentmarker}{}%
\end{pgfscope}%
\begin{pgfscope}%
\pgfsys@transformshift{3.228771in}{2.461407in}%
\pgfsys@useobject{currentmarker}{}%
\end{pgfscope}%
\begin{pgfscope}%
\pgfsys@transformshift{3.519662in}{2.587057in}%
\pgfsys@useobject{currentmarker}{}%
\end{pgfscope}%
\begin{pgfscope}%
\pgfsys@transformshift{3.419082in}{2.531620in}%
\pgfsys@useobject{currentmarker}{}%
\end{pgfscope}%
\begin{pgfscope}%
\pgfsys@transformshift{3.190525in}{2.317681in}%
\pgfsys@useobject{currentmarker}{}%
\end{pgfscope}%
\begin{pgfscope}%
\pgfsys@transformshift{3.576152in}{2.515948in}%
\pgfsys@useobject{currentmarker}{}%
\end{pgfscope}%
\begin{pgfscope}%
\pgfsys@transformshift{3.634785in}{2.579710in}%
\pgfsys@useobject{currentmarker}{}%
\end{pgfscope}%
\begin{pgfscope}%
\pgfsys@transformshift{3.557265in}{2.597375in}%
\pgfsys@useobject{currentmarker}{}%
\end{pgfscope}%
\begin{pgfscope}%
\pgfsys@transformshift{3.585356in}{2.538750in}%
\pgfsys@useobject{currentmarker}{}%
\end{pgfscope}%
\begin{pgfscope}%
\pgfsys@transformshift{3.633137in}{2.596459in}%
\pgfsys@useobject{currentmarker}{}%
\end{pgfscope}%
\begin{pgfscope}%
\pgfsys@transformshift{3.585093in}{2.603143in}%
\pgfsys@useobject{currentmarker}{}%
\end{pgfscope}%
\begin{pgfscope}%
\pgfsys@transformshift{3.022230in}{2.260224in}%
\pgfsys@useobject{currentmarker}{}%
\end{pgfscope}%
\begin{pgfscope}%
\pgfsys@transformshift{2.933578in}{2.546695in}%
\pgfsys@useobject{currentmarker}{}%
\end{pgfscope}%
\begin{pgfscope}%
\pgfsys@transformshift{2.958308in}{2.287580in}%
\pgfsys@useobject{currentmarker}{}%
\end{pgfscope}%
\begin{pgfscope}%
\pgfsys@transformshift{3.396007in}{2.682468in}%
\pgfsys@useobject{currentmarker}{}%
\end{pgfscope}%
\begin{pgfscope}%
\pgfsys@transformshift{3.355827in}{2.299261in}%
\pgfsys@useobject{currentmarker}{}%
\end{pgfscope}%
\begin{pgfscope}%
\pgfsys@transformshift{3.182134in}{2.312724in}%
\pgfsys@useobject{currentmarker}{}%
\end{pgfscope}%
\begin{pgfscope}%
\pgfsys@transformshift{2.939087in}{2.134911in}%
\pgfsys@useobject{currentmarker}{}%
\end{pgfscope}%
\begin{pgfscope}%
\pgfsys@transformshift{3.159261in}{2.427724in}%
\pgfsys@useobject{currentmarker}{}%
\end{pgfscope}%
\begin{pgfscope}%
\pgfsys@transformshift{3.821675in}{2.491402in}%
\pgfsys@useobject{currentmarker}{}%
\end{pgfscope}%
\begin{pgfscope}%
\pgfsys@transformshift{3.786106in}{2.558550in}%
\pgfsys@useobject{currentmarker}{}%
\end{pgfscope}%
\begin{pgfscope}%
\pgfsys@transformshift{3.856556in}{2.466622in}%
\pgfsys@useobject{currentmarker}{}%
\end{pgfscope}%
\begin{pgfscope}%
\pgfsys@transformshift{3.243255in}{2.482748in}%
\pgfsys@useobject{currentmarker}{}%
\end{pgfscope}%
\begin{pgfscope}%
\pgfsys@transformshift{3.371106in}{2.681308in}%
\pgfsys@useobject{currentmarker}{}%
\end{pgfscope}%
\begin{pgfscope}%
\pgfsys@transformshift{3.344234in}{2.659058in}%
\pgfsys@useobject{currentmarker}{}%
\end{pgfscope}%
\begin{pgfscope}%
\pgfsys@transformshift{3.492977in}{2.445895in}%
\pgfsys@useobject{currentmarker}{}%
\end{pgfscope}%
\begin{pgfscope}%
\pgfsys@transformshift{3.272924in}{2.255793in}%
\pgfsys@useobject{currentmarker}{}%
\end{pgfscope}%
\begin{pgfscope}%
\pgfsys@transformshift{3.943385in}{2.466196in}%
\pgfsys@useobject{currentmarker}{}%
\end{pgfscope}%
\begin{pgfscope}%
\pgfsys@transformshift{3.404287in}{2.491013in}%
\pgfsys@useobject{currentmarker}{}%
\end{pgfscope}%
\begin{pgfscope}%
\pgfsys@transformshift{3.389585in}{2.417220in}%
\pgfsys@useobject{currentmarker}{}%
\end{pgfscope}%
\begin{pgfscope}%
\pgfsys@transformshift{3.002773in}{2.323183in}%
\pgfsys@useobject{currentmarker}{}%
\end{pgfscope}%
\end{pgfscope}%
\begin{pgfscope}%
\pgfpathrectangle{\pgfqpoint{1.432000in}{0.528000in}}{\pgfqpoint{3.696000in}{3.696000in}}%
\pgfusepath{clip}%
\pgfsetbuttcap%
\pgfsetroundjoin%
\definecolor{currentfill}{rgb}{1.000000,0.000000,0.000000}%
\pgfsetfillcolor{currentfill}%
\pgfsetlinewidth{0.501875pt}%
\definecolor{currentstroke}{rgb}{1.000000,1.000000,1.000000}%
\pgfsetstrokecolor{currentstroke}%
\pgfsetdash{}{0pt}%
\pgfsys@defobject{currentmarker}{\pgfqpoint{-0.018373in}{-0.018373in}}{\pgfqpoint{0.018373in}{0.018373in}}{%
\pgfpathmoveto{\pgfqpoint{0.000000in}{-0.018373in}}%
\pgfpathcurveto{\pgfqpoint{0.004873in}{-0.018373in}}{\pgfqpoint{0.009546in}{-0.016437in}}{\pgfqpoint{0.012992in}{-0.012992in}}%
\pgfpathcurveto{\pgfqpoint{0.016437in}{-0.009546in}}{\pgfqpoint{0.018373in}{-0.004873in}}{\pgfqpoint{0.018373in}{0.000000in}}%
\pgfpathcurveto{\pgfqpoint{0.018373in}{0.004873in}}{\pgfqpoint{0.016437in}{0.009546in}}{\pgfqpoint{0.012992in}{0.012992in}}%
\pgfpathcurveto{\pgfqpoint{0.009546in}{0.016437in}}{\pgfqpoint{0.004873in}{0.018373in}}{\pgfqpoint{0.000000in}{0.018373in}}%
\pgfpathcurveto{\pgfqpoint{-0.004873in}{0.018373in}}{\pgfqpoint{-0.009546in}{0.016437in}}{\pgfqpoint{-0.012992in}{0.012992in}}%
\pgfpathcurveto{\pgfqpoint{-0.016437in}{0.009546in}}{\pgfqpoint{-0.018373in}{0.004873in}}{\pgfqpoint{-0.018373in}{0.000000in}}%
\pgfpathcurveto{\pgfqpoint{-0.018373in}{-0.004873in}}{\pgfqpoint{-0.016437in}{-0.009546in}}{\pgfqpoint{-0.012992in}{-0.012992in}}%
\pgfpathcurveto{\pgfqpoint{-0.009546in}{-0.016437in}}{\pgfqpoint{-0.004873in}{-0.018373in}}{\pgfqpoint{0.000000in}{-0.018373in}}%
\pgfpathclose%
\pgfusepath{stroke,fill}%
}%
\begin{pgfscope}%
\pgfsys@transformshift{3.289556in}{2.412084in}%
\pgfsys@useobject{currentmarker}{}%
\end{pgfscope}%
\begin{pgfscope}%
\pgfsys@transformshift{3.057383in}{2.621550in}%
\pgfsys@useobject{currentmarker}{}%
\end{pgfscope}%
\begin{pgfscope}%
\pgfsys@transformshift{3.332165in}{2.424052in}%
\pgfsys@useobject{currentmarker}{}%
\end{pgfscope}%
\begin{pgfscope}%
\pgfsys@transformshift{2.950585in}{2.694697in}%
\pgfsys@useobject{currentmarker}{}%
\end{pgfscope}%
\begin{pgfscope}%
\pgfsys@transformshift{3.411789in}{2.556180in}%
\pgfsys@useobject{currentmarker}{}%
\end{pgfscope}%
\end{pgfscope}%
\begin{pgfscope}%
\pgfsetbuttcap%
\pgfsetroundjoin%
\definecolor{currentfill}{rgb}{0.000000,0.000000,0.000000}%
\pgfsetfillcolor{currentfill}%
\pgfsetlinewidth{0.803000pt}%
\definecolor{currentstroke}{rgb}{0.000000,0.000000,0.000000}%
\pgfsetstrokecolor{currentstroke}%
\pgfsetdash{}{0pt}%
\pgfsys@defobject{currentmarker}{\pgfqpoint{0.000000in}{-0.048611in}}{\pgfqpoint{0.000000in}{0.000000in}}{%
\pgfpathmoveto{\pgfqpoint{0.000000in}{0.000000in}}%
\pgfpathlineto{\pgfqpoint{0.000000in}{-0.048611in}}%
\pgfusepath{stroke,fill}%
}%
\begin{pgfscope}%
\pgfsys@transformshift{1.432000in}{0.528000in}%
\pgfsys@useobject{currentmarker}{}%
\end{pgfscope}%
\end{pgfscope}%
\begin{pgfscope}%
\definecolor{textcolor}{rgb}{0.000000,0.000000,0.000000}%
\pgfsetstrokecolor{textcolor}%
\pgfsetfillcolor{textcolor}%
\pgftext[x=1.432000in,y=0.430778in,,top]{\color{textcolor}\rmfamily\fontsize{10.000000}{12.000000}\selectfont \(\displaystyle -15\)}%
\end{pgfscope}%
\begin{pgfscope}%
\pgfsetbuttcap%
\pgfsetroundjoin%
\definecolor{currentfill}{rgb}{0.000000,0.000000,0.000000}%
\pgfsetfillcolor{currentfill}%
\pgfsetlinewidth{0.803000pt}%
\definecolor{currentstroke}{rgb}{0.000000,0.000000,0.000000}%
\pgfsetstrokecolor{currentstroke}%
\pgfsetdash{}{0pt}%
\pgfsys@defobject{currentmarker}{\pgfqpoint{0.000000in}{-0.048611in}}{\pgfqpoint{0.000000in}{0.000000in}}{%
\pgfpathmoveto{\pgfqpoint{0.000000in}{0.000000in}}%
\pgfpathlineto{\pgfqpoint{0.000000in}{-0.048611in}}%
\pgfusepath{stroke,fill}%
}%
\begin{pgfscope}%
\pgfsys@transformshift{2.048000in}{0.528000in}%
\pgfsys@useobject{currentmarker}{}%
\end{pgfscope}%
\end{pgfscope}%
\begin{pgfscope}%
\definecolor{textcolor}{rgb}{0.000000,0.000000,0.000000}%
\pgfsetstrokecolor{textcolor}%
\pgfsetfillcolor{textcolor}%
\pgftext[x=2.048000in,y=0.430778in,,top]{\color{textcolor}\rmfamily\fontsize{10.000000}{12.000000}\selectfont \(\displaystyle -10\)}%
\end{pgfscope}%
\begin{pgfscope}%
\pgfsetbuttcap%
\pgfsetroundjoin%
\definecolor{currentfill}{rgb}{0.000000,0.000000,0.000000}%
\pgfsetfillcolor{currentfill}%
\pgfsetlinewidth{0.803000pt}%
\definecolor{currentstroke}{rgb}{0.000000,0.000000,0.000000}%
\pgfsetstrokecolor{currentstroke}%
\pgfsetdash{}{0pt}%
\pgfsys@defobject{currentmarker}{\pgfqpoint{0.000000in}{-0.048611in}}{\pgfqpoint{0.000000in}{0.000000in}}{%
\pgfpathmoveto{\pgfqpoint{0.000000in}{0.000000in}}%
\pgfpathlineto{\pgfqpoint{0.000000in}{-0.048611in}}%
\pgfusepath{stroke,fill}%
}%
\begin{pgfscope}%
\pgfsys@transformshift{2.664000in}{0.528000in}%
\pgfsys@useobject{currentmarker}{}%
\end{pgfscope}%
\end{pgfscope}%
\begin{pgfscope}%
\definecolor{textcolor}{rgb}{0.000000,0.000000,0.000000}%
\pgfsetstrokecolor{textcolor}%
\pgfsetfillcolor{textcolor}%
\pgftext[x=2.664000in,y=0.430778in,,top]{\color{textcolor}\rmfamily\fontsize{10.000000}{12.000000}\selectfont \(\displaystyle -5\)}%
\end{pgfscope}%
\begin{pgfscope}%
\pgfsetbuttcap%
\pgfsetroundjoin%
\definecolor{currentfill}{rgb}{0.000000,0.000000,0.000000}%
\pgfsetfillcolor{currentfill}%
\pgfsetlinewidth{0.803000pt}%
\definecolor{currentstroke}{rgb}{0.000000,0.000000,0.000000}%
\pgfsetstrokecolor{currentstroke}%
\pgfsetdash{}{0pt}%
\pgfsys@defobject{currentmarker}{\pgfqpoint{0.000000in}{-0.048611in}}{\pgfqpoint{0.000000in}{0.000000in}}{%
\pgfpathmoveto{\pgfqpoint{0.000000in}{0.000000in}}%
\pgfpathlineto{\pgfqpoint{0.000000in}{-0.048611in}}%
\pgfusepath{stroke,fill}%
}%
\begin{pgfscope}%
\pgfsys@transformshift{3.280000in}{0.528000in}%
\pgfsys@useobject{currentmarker}{}%
\end{pgfscope}%
\end{pgfscope}%
\begin{pgfscope}%
\definecolor{textcolor}{rgb}{0.000000,0.000000,0.000000}%
\pgfsetstrokecolor{textcolor}%
\pgfsetfillcolor{textcolor}%
\pgftext[x=3.280000in,y=0.430778in,,top]{\color{textcolor}\rmfamily\fontsize{10.000000}{12.000000}\selectfont \(\displaystyle 0\)}%
\end{pgfscope}%
\begin{pgfscope}%
\pgfsetbuttcap%
\pgfsetroundjoin%
\definecolor{currentfill}{rgb}{0.000000,0.000000,0.000000}%
\pgfsetfillcolor{currentfill}%
\pgfsetlinewidth{0.803000pt}%
\definecolor{currentstroke}{rgb}{0.000000,0.000000,0.000000}%
\pgfsetstrokecolor{currentstroke}%
\pgfsetdash{}{0pt}%
\pgfsys@defobject{currentmarker}{\pgfqpoint{0.000000in}{-0.048611in}}{\pgfqpoint{0.000000in}{0.000000in}}{%
\pgfpathmoveto{\pgfqpoint{0.000000in}{0.000000in}}%
\pgfpathlineto{\pgfqpoint{0.000000in}{-0.048611in}}%
\pgfusepath{stroke,fill}%
}%
\begin{pgfscope}%
\pgfsys@transformshift{3.896000in}{0.528000in}%
\pgfsys@useobject{currentmarker}{}%
\end{pgfscope}%
\end{pgfscope}%
\begin{pgfscope}%
\definecolor{textcolor}{rgb}{0.000000,0.000000,0.000000}%
\pgfsetstrokecolor{textcolor}%
\pgfsetfillcolor{textcolor}%
\pgftext[x=3.896000in,y=0.430778in,,top]{\color{textcolor}\rmfamily\fontsize{10.000000}{12.000000}\selectfont \(\displaystyle 5\)}%
\end{pgfscope}%
\begin{pgfscope}%
\pgfsetbuttcap%
\pgfsetroundjoin%
\definecolor{currentfill}{rgb}{0.000000,0.000000,0.000000}%
\pgfsetfillcolor{currentfill}%
\pgfsetlinewidth{0.803000pt}%
\definecolor{currentstroke}{rgb}{0.000000,0.000000,0.000000}%
\pgfsetstrokecolor{currentstroke}%
\pgfsetdash{}{0pt}%
\pgfsys@defobject{currentmarker}{\pgfqpoint{0.000000in}{-0.048611in}}{\pgfqpoint{0.000000in}{0.000000in}}{%
\pgfpathmoveto{\pgfqpoint{0.000000in}{0.000000in}}%
\pgfpathlineto{\pgfqpoint{0.000000in}{-0.048611in}}%
\pgfusepath{stroke,fill}%
}%
\begin{pgfscope}%
\pgfsys@transformshift{4.512000in}{0.528000in}%
\pgfsys@useobject{currentmarker}{}%
\end{pgfscope}%
\end{pgfscope}%
\begin{pgfscope}%
\definecolor{textcolor}{rgb}{0.000000,0.000000,0.000000}%
\pgfsetstrokecolor{textcolor}%
\pgfsetfillcolor{textcolor}%
\pgftext[x=4.512000in,y=0.430778in,,top]{\color{textcolor}\rmfamily\fontsize{10.000000}{12.000000}\selectfont \(\displaystyle 10\)}%
\end{pgfscope}%
\begin{pgfscope}%
\pgfsetbuttcap%
\pgfsetroundjoin%
\definecolor{currentfill}{rgb}{0.000000,0.000000,0.000000}%
\pgfsetfillcolor{currentfill}%
\pgfsetlinewidth{0.803000pt}%
\definecolor{currentstroke}{rgb}{0.000000,0.000000,0.000000}%
\pgfsetstrokecolor{currentstroke}%
\pgfsetdash{}{0pt}%
\pgfsys@defobject{currentmarker}{\pgfqpoint{0.000000in}{-0.048611in}}{\pgfqpoint{0.000000in}{0.000000in}}{%
\pgfpathmoveto{\pgfqpoint{0.000000in}{0.000000in}}%
\pgfpathlineto{\pgfqpoint{0.000000in}{-0.048611in}}%
\pgfusepath{stroke,fill}%
}%
\begin{pgfscope}%
\pgfsys@transformshift{5.128000in}{0.528000in}%
\pgfsys@useobject{currentmarker}{}%
\end{pgfscope}%
\end{pgfscope}%
\begin{pgfscope}%
\definecolor{textcolor}{rgb}{0.000000,0.000000,0.000000}%
\pgfsetstrokecolor{textcolor}%
\pgfsetfillcolor{textcolor}%
\pgftext[x=5.128000in,y=0.430778in,,top]{\color{textcolor}\rmfamily\fontsize{10.000000}{12.000000}\selectfont \(\displaystyle 15\)}%
\end{pgfscope}%
\begin{pgfscope}%
\pgfsetbuttcap%
\pgfsetroundjoin%
\definecolor{currentfill}{rgb}{0.000000,0.000000,0.000000}%
\pgfsetfillcolor{currentfill}%
\pgfsetlinewidth{0.803000pt}%
\definecolor{currentstroke}{rgb}{0.000000,0.000000,0.000000}%
\pgfsetstrokecolor{currentstroke}%
\pgfsetdash{}{0pt}%
\pgfsys@defobject{currentmarker}{\pgfqpoint{-0.048611in}{0.000000in}}{\pgfqpoint{0.000000in}{0.000000in}}{%
\pgfpathmoveto{\pgfqpoint{0.000000in}{0.000000in}}%
\pgfpathlineto{\pgfqpoint{-0.048611in}{0.000000in}}%
\pgfusepath{stroke,fill}%
}%
\begin{pgfscope}%
\pgfsys@transformshift{1.432000in}{0.528000in}%
\pgfsys@useobject{currentmarker}{}%
\end{pgfscope}%
\end{pgfscope}%
\begin{pgfscope}%
\definecolor{textcolor}{rgb}{0.000000,0.000000,0.000000}%
\pgfsetstrokecolor{textcolor}%
\pgfsetfillcolor{textcolor}%
\pgftext[x=1.087863in,y=0.475238in,left,base]{\color{textcolor}\rmfamily\fontsize{10.000000}{12.000000}\selectfont \(\displaystyle -15\)}%
\end{pgfscope}%
\begin{pgfscope}%
\pgfsetbuttcap%
\pgfsetroundjoin%
\definecolor{currentfill}{rgb}{0.000000,0.000000,0.000000}%
\pgfsetfillcolor{currentfill}%
\pgfsetlinewidth{0.803000pt}%
\definecolor{currentstroke}{rgb}{0.000000,0.000000,0.000000}%
\pgfsetstrokecolor{currentstroke}%
\pgfsetdash{}{0pt}%
\pgfsys@defobject{currentmarker}{\pgfqpoint{-0.048611in}{0.000000in}}{\pgfqpoint{0.000000in}{0.000000in}}{%
\pgfpathmoveto{\pgfqpoint{0.000000in}{0.000000in}}%
\pgfpathlineto{\pgfqpoint{-0.048611in}{0.000000in}}%
\pgfusepath{stroke,fill}%
}%
\begin{pgfscope}%
\pgfsys@transformshift{1.432000in}{1.144000in}%
\pgfsys@useobject{currentmarker}{}%
\end{pgfscope}%
\end{pgfscope}%
\begin{pgfscope}%
\definecolor{textcolor}{rgb}{0.000000,0.000000,0.000000}%
\pgfsetstrokecolor{textcolor}%
\pgfsetfillcolor{textcolor}%
\pgftext[x=1.087863in,y=1.091238in,left,base]{\color{textcolor}\rmfamily\fontsize{10.000000}{12.000000}\selectfont \(\displaystyle -10\)}%
\end{pgfscope}%
\begin{pgfscope}%
\pgfsetbuttcap%
\pgfsetroundjoin%
\definecolor{currentfill}{rgb}{0.000000,0.000000,0.000000}%
\pgfsetfillcolor{currentfill}%
\pgfsetlinewidth{0.803000pt}%
\definecolor{currentstroke}{rgb}{0.000000,0.000000,0.000000}%
\pgfsetstrokecolor{currentstroke}%
\pgfsetdash{}{0pt}%
\pgfsys@defobject{currentmarker}{\pgfqpoint{-0.048611in}{0.000000in}}{\pgfqpoint{0.000000in}{0.000000in}}{%
\pgfpathmoveto{\pgfqpoint{0.000000in}{0.000000in}}%
\pgfpathlineto{\pgfqpoint{-0.048611in}{0.000000in}}%
\pgfusepath{stroke,fill}%
}%
\begin{pgfscope}%
\pgfsys@transformshift{1.432000in}{1.760000in}%
\pgfsys@useobject{currentmarker}{}%
\end{pgfscope}%
\end{pgfscope}%
\begin{pgfscope}%
\definecolor{textcolor}{rgb}{0.000000,0.000000,0.000000}%
\pgfsetstrokecolor{textcolor}%
\pgfsetfillcolor{textcolor}%
\pgftext[x=1.157308in,y=1.707238in,left,base]{\color{textcolor}\rmfamily\fontsize{10.000000}{12.000000}\selectfont \(\displaystyle -5\)}%
\end{pgfscope}%
\begin{pgfscope}%
\pgfsetbuttcap%
\pgfsetroundjoin%
\definecolor{currentfill}{rgb}{0.000000,0.000000,0.000000}%
\pgfsetfillcolor{currentfill}%
\pgfsetlinewidth{0.803000pt}%
\definecolor{currentstroke}{rgb}{0.000000,0.000000,0.000000}%
\pgfsetstrokecolor{currentstroke}%
\pgfsetdash{}{0pt}%
\pgfsys@defobject{currentmarker}{\pgfqpoint{-0.048611in}{0.000000in}}{\pgfqpoint{0.000000in}{0.000000in}}{%
\pgfpathmoveto{\pgfqpoint{0.000000in}{0.000000in}}%
\pgfpathlineto{\pgfqpoint{-0.048611in}{0.000000in}}%
\pgfusepath{stroke,fill}%
}%
\begin{pgfscope}%
\pgfsys@transformshift{1.432000in}{2.376000in}%
\pgfsys@useobject{currentmarker}{}%
\end{pgfscope}%
\end{pgfscope}%
\begin{pgfscope}%
\definecolor{textcolor}{rgb}{0.000000,0.000000,0.000000}%
\pgfsetstrokecolor{textcolor}%
\pgfsetfillcolor{textcolor}%
\pgftext[x=1.265333in,y=2.323238in,left,base]{\color{textcolor}\rmfamily\fontsize{10.000000}{12.000000}\selectfont \(\displaystyle 0\)}%
\end{pgfscope}%
\begin{pgfscope}%
\pgfsetbuttcap%
\pgfsetroundjoin%
\definecolor{currentfill}{rgb}{0.000000,0.000000,0.000000}%
\pgfsetfillcolor{currentfill}%
\pgfsetlinewidth{0.803000pt}%
\definecolor{currentstroke}{rgb}{0.000000,0.000000,0.000000}%
\pgfsetstrokecolor{currentstroke}%
\pgfsetdash{}{0pt}%
\pgfsys@defobject{currentmarker}{\pgfqpoint{-0.048611in}{0.000000in}}{\pgfqpoint{0.000000in}{0.000000in}}{%
\pgfpathmoveto{\pgfqpoint{0.000000in}{0.000000in}}%
\pgfpathlineto{\pgfqpoint{-0.048611in}{0.000000in}}%
\pgfusepath{stroke,fill}%
}%
\begin{pgfscope}%
\pgfsys@transformshift{1.432000in}{2.992000in}%
\pgfsys@useobject{currentmarker}{}%
\end{pgfscope}%
\end{pgfscope}%
\begin{pgfscope}%
\definecolor{textcolor}{rgb}{0.000000,0.000000,0.000000}%
\pgfsetstrokecolor{textcolor}%
\pgfsetfillcolor{textcolor}%
\pgftext[x=1.265333in,y=2.939238in,left,base]{\color{textcolor}\rmfamily\fontsize{10.000000}{12.000000}\selectfont \(\displaystyle 5\)}%
\end{pgfscope}%
\begin{pgfscope}%
\pgfsetbuttcap%
\pgfsetroundjoin%
\definecolor{currentfill}{rgb}{0.000000,0.000000,0.000000}%
\pgfsetfillcolor{currentfill}%
\pgfsetlinewidth{0.803000pt}%
\definecolor{currentstroke}{rgb}{0.000000,0.000000,0.000000}%
\pgfsetstrokecolor{currentstroke}%
\pgfsetdash{}{0pt}%
\pgfsys@defobject{currentmarker}{\pgfqpoint{-0.048611in}{0.000000in}}{\pgfqpoint{0.000000in}{0.000000in}}{%
\pgfpathmoveto{\pgfqpoint{0.000000in}{0.000000in}}%
\pgfpathlineto{\pgfqpoint{-0.048611in}{0.000000in}}%
\pgfusepath{stroke,fill}%
}%
\begin{pgfscope}%
\pgfsys@transformshift{1.432000in}{3.608000in}%
\pgfsys@useobject{currentmarker}{}%
\end{pgfscope}%
\end{pgfscope}%
\begin{pgfscope}%
\definecolor{textcolor}{rgb}{0.000000,0.000000,0.000000}%
\pgfsetstrokecolor{textcolor}%
\pgfsetfillcolor{textcolor}%
\pgftext[x=1.195888in,y=3.555238in,left,base]{\color{textcolor}\rmfamily\fontsize{10.000000}{12.000000}\selectfont \(\displaystyle 10\)}%
\end{pgfscope}%
\begin{pgfscope}%
\pgfsetbuttcap%
\pgfsetroundjoin%
\definecolor{currentfill}{rgb}{0.000000,0.000000,0.000000}%
\pgfsetfillcolor{currentfill}%
\pgfsetlinewidth{0.803000pt}%
\definecolor{currentstroke}{rgb}{0.000000,0.000000,0.000000}%
\pgfsetstrokecolor{currentstroke}%
\pgfsetdash{}{0pt}%
\pgfsys@defobject{currentmarker}{\pgfqpoint{-0.048611in}{0.000000in}}{\pgfqpoint{0.000000in}{0.000000in}}{%
\pgfpathmoveto{\pgfqpoint{0.000000in}{0.000000in}}%
\pgfpathlineto{\pgfqpoint{-0.048611in}{0.000000in}}%
\pgfusepath{stroke,fill}%
}%
\begin{pgfscope}%
\pgfsys@transformshift{1.432000in}{4.224000in}%
\pgfsys@useobject{currentmarker}{}%
\end{pgfscope}%
\end{pgfscope}%
\begin{pgfscope}%
\definecolor{textcolor}{rgb}{0.000000,0.000000,0.000000}%
\pgfsetstrokecolor{textcolor}%
\pgfsetfillcolor{textcolor}%
\pgftext[x=1.195888in,y=4.171238in,left,base]{\color{textcolor}\rmfamily\fontsize{10.000000}{12.000000}\selectfont \(\displaystyle 15\)}%
\end{pgfscope}%
\begin{pgfscope}%
\pgfpathrectangle{\pgfqpoint{1.432000in}{0.528000in}}{\pgfqpoint{3.696000in}{3.696000in}}%
\pgfusepath{clip}%
\pgfsetbuttcap%
\pgfsetroundjoin%
\pgfsetlinewidth{1.505625pt}%
\definecolor{currentstroke}{rgb}{0.371035,0.000000,0.000000}%
\pgfsetstrokecolor{currentstroke}%
\pgfsetstrokeopacity{0.300000}%
\pgfsetdash{}{0pt}%
\pgfpathmoveto{\pgfqpoint{3.074667in}{2.094094in}}%
\pgfpathlineto{\pgfqpoint{3.112000in}{2.091209in}}%
\pgfpathlineto{\pgfqpoint{3.149333in}{2.089811in}}%
\pgfpathlineto{\pgfqpoint{3.186667in}{2.089821in}}%
\pgfpathlineto{\pgfqpoint{3.224000in}{2.091213in}}%
\pgfpathlineto{\pgfqpoint{3.261333in}{2.094010in}}%
\pgfpathlineto{\pgfqpoint{3.278988in}{2.096000in}}%
\pgfpathlineto{\pgfqpoint{3.298667in}{2.097755in}}%
\pgfpathlineto{\pgfqpoint{3.336000in}{2.102253in}}%
\pgfpathlineto{\pgfqpoint{3.373333in}{2.108093in}}%
\pgfpathlineto{\pgfqpoint{3.410667in}{2.115463in}}%
\pgfpathlineto{\pgfqpoint{3.448000in}{2.124618in}}%
\pgfpathlineto{\pgfqpoint{3.477391in}{2.133333in}}%
\pgfpathlineto{\pgfqpoint{3.485333in}{2.135342in}}%
\pgfpathlineto{\pgfqpoint{3.522667in}{2.146155in}}%
\pgfpathlineto{\pgfqpoint{3.560000in}{2.159349in}}%
\pgfpathlineto{\pgfqpoint{3.587060in}{2.170667in}}%
\pgfpathlineto{\pgfqpoint{3.597333in}{2.174553in}}%
\pgfpathlineto{\pgfqpoint{3.634667in}{2.190466in}}%
\pgfpathlineto{\pgfqpoint{3.668606in}{2.208000in}}%
\pgfpathlineto{\pgfqpoint{3.672000in}{2.209670in}}%
\pgfpathlineto{\pgfqpoint{3.709333in}{2.229693in}}%
\pgfpathlineto{\pgfqpoint{3.734204in}{2.245333in}}%
\pgfpathlineto{\pgfqpoint{3.746667in}{2.253204in}}%
\pgfpathlineto{\pgfqpoint{3.784000in}{2.279705in}}%
\pgfpathlineto{\pgfqpoint{3.787902in}{2.282667in}}%
\pgfpathlineto{\pgfqpoint{3.821333in}{2.309579in}}%
\pgfpathlineto{\pgfqpoint{3.833243in}{2.320000in}}%
\pgfpathlineto{\pgfqpoint{3.858667in}{2.344786in}}%
\pgfpathlineto{\pgfqpoint{3.870961in}{2.357333in}}%
\pgfpathlineto{\pgfqpoint{3.896000in}{2.387368in}}%
\pgfpathlineto{\pgfqpoint{3.902034in}{2.394667in}}%
\pgfpathlineto{\pgfqpoint{3.927585in}{2.432000in}}%
\pgfpathlineto{\pgfqpoint{3.933333in}{2.442858in}}%
\pgfpathlineto{\pgfqpoint{3.948141in}{2.469333in}}%
\pgfpathlineto{\pgfqpoint{3.963102in}{2.506667in}}%
\pgfpathlineto{\pgfqpoint{3.970667in}{2.535820in}}%
\pgfpathlineto{\pgfqpoint{3.973055in}{2.544000in}}%
\pgfpathlineto{\pgfqpoint{3.978810in}{2.581333in}}%
\pgfpathlineto{\pgfqpoint{3.979320in}{2.618667in}}%
\pgfpathlineto{\pgfqpoint{3.974456in}{2.656000in}}%
\pgfpathlineto{\pgfqpoint{3.970667in}{2.669595in}}%
\pgfpathlineto{\pgfqpoint{3.964590in}{2.693333in}}%
\pgfpathlineto{\pgfqpoint{3.948994in}{2.730667in}}%
\pgfpathlineto{\pgfqpoint{3.933333in}{2.756434in}}%
\pgfpathlineto{\pgfqpoint{3.926316in}{2.768000in}}%
\pgfpathlineto{\pgfqpoint{3.896000in}{2.805248in}}%
\pgfpathlineto{\pgfqpoint{3.895927in}{2.805333in}}%
\pgfpathlineto{\pgfqpoint{3.858667in}{2.840939in}}%
\pgfpathlineto{\pgfqpoint{3.856663in}{2.842667in}}%
\pgfpathlineto{\pgfqpoint{3.821333in}{2.869577in}}%
\pgfpathlineto{\pgfqpoint{3.805324in}{2.880000in}}%
\pgfpathlineto{\pgfqpoint{3.784000in}{2.893133in}}%
\pgfpathlineto{\pgfqpoint{3.746667in}{2.912453in}}%
\pgfpathlineto{\pgfqpoint{3.735692in}{2.917333in}}%
\pgfpathlineto{\pgfqpoint{3.709333in}{2.929326in}}%
\pgfpathlineto{\pgfqpoint{3.672000in}{2.943299in}}%
\pgfpathlineto{\pgfqpoint{3.634667in}{2.954332in}}%
\pgfpathlineto{\pgfqpoint{3.633341in}{2.954667in}}%
\pgfpathlineto{\pgfqpoint{3.597333in}{2.964844in}}%
\pgfpathlineto{\pgfqpoint{3.560000in}{2.973106in}}%
\pgfpathlineto{\pgfqpoint{3.522667in}{2.979420in}}%
\pgfpathlineto{\pgfqpoint{3.485333in}{2.984072in}}%
\pgfpathlineto{\pgfqpoint{3.448000in}{2.987273in}}%
\pgfpathlineto{\pgfqpoint{3.410667in}{2.989176in}}%
\pgfpathlineto{\pgfqpoint{3.373333in}{2.989881in}}%
\pgfpathlineto{\pgfqpoint{3.336000in}{2.989445in}}%
\pgfpathlineto{\pgfqpoint{3.298667in}{2.987884in}}%
\pgfpathlineto{\pgfqpoint{3.261333in}{2.985173in}}%
\pgfpathlineto{\pgfqpoint{3.224000in}{2.981244in}}%
\pgfpathlineto{\pgfqpoint{3.186667in}{2.975987in}}%
\pgfpathlineto{\pgfqpoint{3.149333in}{2.969237in}}%
\pgfpathlineto{\pgfqpoint{3.112000in}{2.960767in}}%
\pgfpathlineto{\pgfqpoint{3.089599in}{2.954667in}}%
\pgfpathlineto{\pgfqpoint{3.074667in}{2.951242in}}%
\pgfpathlineto{\pgfqpoint{3.037333in}{2.941214in}}%
\pgfpathlineto{\pgfqpoint{3.000000in}{2.928967in}}%
\pgfpathlineto{\pgfqpoint{2.970312in}{2.917333in}}%
\pgfpathlineto{\pgfqpoint{2.962667in}{2.914657in}}%
\pgfpathlineto{\pgfqpoint{2.925333in}{2.900002in}}%
\pgfpathlineto{\pgfqpoint{2.888000in}{2.882048in}}%
\pgfpathlineto{\pgfqpoint{2.884078in}{2.880000in}}%
\pgfpathlineto{\pgfqpoint{2.850667in}{2.863456in}}%
\pgfpathlineto{\pgfqpoint{2.815805in}{2.842667in}}%
\pgfpathlineto{\pgfqpoint{2.813333in}{2.841206in}}%
\pgfpathlineto{\pgfqpoint{2.776000in}{2.817391in}}%
\pgfpathlineto{\pgfqpoint{2.759301in}{2.805333in}}%
\pgfpathlineto{\pgfqpoint{2.738667in}{2.789743in}}%
\pgfpathlineto{\pgfqpoint{2.712535in}{2.768000in}}%
\pgfpathlineto{\pgfqpoint{2.701333in}{2.757760in}}%
\pgfpathlineto{\pgfqpoint{2.673385in}{2.730667in}}%
\pgfpathlineto{\pgfqpoint{2.664000in}{2.720131in}}%
\pgfpathlineto{\pgfqpoint{2.640627in}{2.693333in}}%
\pgfpathlineto{\pgfqpoint{2.626667in}{2.673687in}}%
\pgfpathlineto{\pgfqpoint{2.613892in}{2.656000in}}%
\pgfpathlineto{\pgfqpoint{2.592933in}{2.618667in}}%
\pgfpathlineto{\pgfqpoint{2.589333in}{2.610096in}}%
\pgfpathlineto{\pgfqpoint{2.576254in}{2.581333in}}%
\pgfpathlineto{\pgfqpoint{2.564912in}{2.544000in}}%
\pgfpathlineto{\pgfqpoint{2.558686in}{2.506667in}}%
\pgfpathlineto{\pgfqpoint{2.557205in}{2.469333in}}%
\pgfpathlineto{\pgfqpoint{2.560524in}{2.432000in}}%
\pgfpathlineto{\pgfqpoint{2.569136in}{2.394667in}}%
\pgfpathlineto{\pgfqpoint{2.584049in}{2.357333in}}%
\pgfpathlineto{\pgfqpoint{2.589333in}{2.347888in}}%
\pgfpathlineto{\pgfqpoint{2.604691in}{2.320000in}}%
\pgfpathlineto{\pgfqpoint{2.626667in}{2.291026in}}%
\pgfpathlineto{\pgfqpoint{2.633238in}{2.282667in}}%
\pgfpathlineto{\pgfqpoint{2.664000in}{2.251336in}}%
\pgfpathlineto{\pgfqpoint{2.670437in}{2.245333in}}%
\pgfpathlineto{\pgfqpoint{2.701333in}{2.220345in}}%
\pgfpathlineto{\pgfqpoint{2.718923in}{2.208000in}}%
\pgfpathlineto{\pgfqpoint{2.738667in}{2.195109in}}%
\pgfpathlineto{\pgfqpoint{2.776000in}{2.174492in}}%
\pgfpathlineto{\pgfqpoint{2.783959in}{2.170667in}}%
\pgfpathlineto{\pgfqpoint{2.813333in}{2.156448in}}%
\pgfpathlineto{\pgfqpoint{2.850667in}{2.141734in}}%
\pgfpathlineto{\pgfqpoint{2.876854in}{2.133333in}}%
\pgfpathlineto{\pgfqpoint{2.888000in}{2.129452in}}%
\pgfpathlineto{\pgfqpoint{2.925333in}{2.118385in}}%
\pgfpathlineto{\pgfqpoint{2.962667in}{2.109714in}}%
\pgfpathlineto{\pgfqpoint{3.000000in}{2.103036in}}%
\pgfpathlineto{\pgfqpoint{3.037333in}{2.098043in}}%
\pgfpathlineto{\pgfqpoint{3.058488in}{2.096000in}}%
\pgfpathlineto{\pgfqpoint{3.074667in}{2.094094in}}%
\pgfusepath{stroke}%
\end{pgfscope}%
\begin{pgfscope}%
\pgfpathrectangle{\pgfqpoint{1.432000in}{0.528000in}}{\pgfqpoint{3.696000in}{3.696000in}}%
\pgfusepath{clip}%
\pgfsetbuttcap%
\pgfsetroundjoin%
\pgfsetlinewidth{1.505625pt}%
\definecolor{currentstroke}{rgb}{0.700470,0.000000,0.000000}%
\pgfsetstrokecolor{currentstroke}%
\pgfsetstrokeopacity{0.300000}%
\pgfsetdash{}{0pt}%
\pgfpathmoveto{\pgfqpoint{2.962667in}{2.204144in}}%
\pgfpathlineto{\pgfqpoint{3.000000in}{2.193769in}}%
\pgfpathlineto{\pgfqpoint{3.037333in}{2.185890in}}%
\pgfpathlineto{\pgfqpoint{3.074667in}{2.180138in}}%
\pgfpathlineto{\pgfqpoint{3.112000in}{2.176239in}}%
\pgfpathlineto{\pgfqpoint{3.149333in}{2.173996in}}%
\pgfpathlineto{\pgfqpoint{3.186667in}{2.173279in}}%
\pgfpathlineto{\pgfqpoint{3.224000in}{2.174013in}}%
\pgfpathlineto{\pgfqpoint{3.261333in}{2.176177in}}%
\pgfpathlineto{\pgfqpoint{3.298667in}{2.179803in}}%
\pgfpathlineto{\pgfqpoint{3.336000in}{2.184974in}}%
\pgfpathlineto{\pgfqpoint{3.373333in}{2.191831in}}%
\pgfpathlineto{\pgfqpoint{3.410667in}{2.200584in}}%
\pgfpathlineto{\pgfqpoint{3.436507in}{2.208000in}}%
\pgfpathlineto{\pgfqpoint{3.448000in}{2.210992in}}%
\pgfpathlineto{\pgfqpoint{3.485333in}{2.222418in}}%
\pgfpathlineto{\pgfqpoint{3.522667in}{2.236360in}}%
\pgfpathlineto{\pgfqpoint{3.543277in}{2.245333in}}%
\pgfpathlineto{\pgfqpoint{3.560000in}{2.252359in}}%
\pgfpathlineto{\pgfqpoint{3.597333in}{2.270435in}}%
\pgfpathlineto{\pgfqpoint{3.619158in}{2.282667in}}%
\pgfpathlineto{\pgfqpoint{3.634667in}{2.291536in}}%
\pgfpathlineto{\pgfqpoint{3.672000in}{2.315846in}}%
\pgfpathlineto{\pgfqpoint{3.677856in}{2.320000in}}%
\pgfpathlineto{\pgfqpoint{3.709333in}{2.344154in}}%
\pgfpathlineto{\pgfqpoint{3.724940in}{2.357333in}}%
\pgfpathlineto{\pgfqpoint{3.746667in}{2.378314in}}%
\pgfpathlineto{\pgfqpoint{3.762663in}{2.394667in}}%
\pgfpathlineto{\pgfqpoint{3.784000in}{2.421225in}}%
\pgfpathlineto{\pgfqpoint{3.792498in}{2.432000in}}%
\pgfpathlineto{\pgfqpoint{3.815497in}{2.469333in}}%
\pgfpathlineto{\pgfqpoint{3.821333in}{2.482589in}}%
\pgfpathlineto{\pgfqpoint{3.832426in}{2.506667in}}%
\pgfpathlineto{\pgfqpoint{3.843211in}{2.544000in}}%
\pgfpathlineto{\pgfqpoint{3.847869in}{2.581333in}}%
\pgfpathlineto{\pgfqpoint{3.846540in}{2.618667in}}%
\pgfpathlineto{\pgfqpoint{3.838820in}{2.656000in}}%
\pgfpathlineto{\pgfqpoint{3.823698in}{2.693333in}}%
\pgfpathlineto{\pgfqpoint{3.821333in}{2.697343in}}%
\pgfpathlineto{\pgfqpoint{3.801199in}{2.730667in}}%
\pgfpathlineto{\pgfqpoint{3.784000in}{2.751146in}}%
\pgfpathlineto{\pgfqpoint{3.768687in}{2.768000in}}%
\pgfpathlineto{\pgfqpoint{3.746667in}{2.787465in}}%
\pgfpathlineto{\pgfqpoint{3.723527in}{2.805333in}}%
\pgfpathlineto{\pgfqpoint{3.709333in}{2.814931in}}%
\pgfpathlineto{\pgfqpoint{3.672000in}{2.836540in}}%
\pgfpathlineto{\pgfqpoint{3.659443in}{2.842667in}}%
\pgfpathlineto{\pgfqpoint{3.634667in}{2.854251in}}%
\pgfpathlineto{\pgfqpoint{3.597333in}{2.868506in}}%
\pgfpathlineto{\pgfqpoint{3.560000in}{2.879611in}}%
\pgfpathlineto{\pgfqpoint{3.558410in}{2.880000in}}%
\pgfpathlineto{\pgfqpoint{3.522667in}{2.889233in}}%
\pgfpathlineto{\pgfqpoint{3.485333in}{2.896501in}}%
\pgfpathlineto{\pgfqpoint{3.448000in}{2.901699in}}%
\pgfpathlineto{\pgfqpoint{3.410667in}{2.905074in}}%
\pgfpathlineto{\pgfqpoint{3.373333in}{2.906797in}}%
\pgfpathlineto{\pgfqpoint{3.336000in}{2.906979in}}%
\pgfpathlineto{\pgfqpoint{3.298667in}{2.905670in}}%
\pgfpathlineto{\pgfqpoint{3.261333in}{2.902869in}}%
\pgfpathlineto{\pgfqpoint{3.224000in}{2.898521in}}%
\pgfpathlineto{\pgfqpoint{3.186667in}{2.892512in}}%
\pgfpathlineto{\pgfqpoint{3.149333in}{2.884667in}}%
\pgfpathlineto{\pgfqpoint{3.131241in}{2.880000in}}%
\pgfpathlineto{\pgfqpoint{3.112000in}{2.875555in}}%
\pgfpathlineto{\pgfqpoint{3.074667in}{2.865174in}}%
\pgfpathlineto{\pgfqpoint{3.037333in}{2.852451in}}%
\pgfpathlineto{\pgfqpoint{3.013044in}{2.842667in}}%
\pgfpathlineto{\pgfqpoint{3.000000in}{2.837663in}}%
\pgfpathlineto{\pgfqpoint{2.962667in}{2.821241in}}%
\pgfpathlineto{\pgfqpoint{2.932275in}{2.805333in}}%
\pgfpathlineto{\pgfqpoint{2.925333in}{2.801677in}}%
\pgfpathlineto{\pgfqpoint{2.888000in}{2.779800in}}%
\pgfpathlineto{\pgfqpoint{2.870288in}{2.768000in}}%
\pgfpathlineto{\pgfqpoint{2.850667in}{2.754060in}}%
\pgfpathlineto{\pgfqpoint{2.821166in}{2.730667in}}%
\pgfpathlineto{\pgfqpoint{2.813333in}{2.723672in}}%
\pgfpathlineto{\pgfqpoint{2.781681in}{2.693333in}}%
\pgfpathlineto{\pgfqpoint{2.776000in}{2.686814in}}%
\pgfpathlineto{\pgfqpoint{2.749995in}{2.656000in}}%
\pgfpathlineto{\pgfqpoint{2.738667in}{2.638727in}}%
\pgfpathlineto{\pgfqpoint{2.725427in}{2.618667in}}%
\pgfpathlineto{\pgfqpoint{2.707514in}{2.581333in}}%
\pgfpathlineto{\pgfqpoint{2.701333in}{2.561452in}}%
\pgfpathlineto{\pgfqpoint{2.695514in}{2.544000in}}%
\pgfpathlineto{\pgfqpoint{2.689297in}{2.506667in}}%
\pgfpathlineto{\pgfqpoint{2.689316in}{2.469333in}}%
\pgfpathlineto{\pgfqpoint{2.695844in}{2.432000in}}%
\pgfpathlineto{\pgfqpoint{2.701333in}{2.416615in}}%
\pgfpathlineto{\pgfqpoint{2.708982in}{2.394667in}}%
\pgfpathlineto{\pgfqpoint{2.729897in}{2.357333in}}%
\pgfpathlineto{\pgfqpoint{2.738667in}{2.345969in}}%
\pgfpathlineto{\pgfqpoint{2.760016in}{2.320000in}}%
\pgfpathlineto{\pgfqpoint{2.776000in}{2.304758in}}%
\pgfpathlineto{\pgfqpoint{2.802097in}{2.282667in}}%
\pgfpathlineto{\pgfqpoint{2.813333in}{2.274503in}}%
\pgfpathlineto{\pgfqpoint{2.850667in}{2.250990in}}%
\pgfpathlineto{\pgfqpoint{2.861227in}{2.245333in}}%
\pgfpathlineto{\pgfqpoint{2.888000in}{2.231832in}}%
\pgfpathlineto{\pgfqpoint{2.925333in}{2.216516in}}%
\pgfpathlineto{\pgfqpoint{2.951211in}{2.208000in}}%
\pgfpathlineto{\pgfqpoint{2.962667in}{2.204144in}}%
\pgfusepath{stroke}%
\end{pgfscope}%
\begin{pgfscope}%
\pgfpathrectangle{\pgfqpoint{1.432000in}{0.528000in}}{\pgfqpoint{3.696000in}{3.696000in}}%
\pgfusepath{clip}%
\pgfsetbuttcap%
\pgfsetroundjoin%
\pgfsetlinewidth{1.505625pt}%
\definecolor{currentstroke}{rgb}{1.000000,0.029903,0.000000}%
\pgfsetstrokecolor{currentstroke}%
\pgfsetstrokeopacity{0.300000}%
\pgfsetdash{}{0pt}%
\pgfpathmoveto{\pgfqpoint{3.074667in}{2.243235in}}%
\pgfpathlineto{\pgfqpoint{3.112000in}{2.237333in}}%
\pgfpathlineto{\pgfqpoint{3.149333in}{2.233664in}}%
\pgfpathlineto{\pgfqpoint{3.186667in}{2.232031in}}%
\pgfpathlineto{\pgfqpoint{3.224000in}{2.232313in}}%
\pgfpathlineto{\pgfqpoint{3.261333in}{2.234463in}}%
\pgfpathlineto{\pgfqpoint{3.298667in}{2.238497in}}%
\pgfpathlineto{\pgfqpoint{3.336000in}{2.244503in}}%
\pgfpathlineto{\pgfqpoint{3.339943in}{2.245333in}}%
\pgfpathlineto{\pgfqpoint{3.373333in}{2.251860in}}%
\pgfpathlineto{\pgfqpoint{3.410667in}{2.261187in}}%
\pgfpathlineto{\pgfqpoint{3.448000in}{2.272849in}}%
\pgfpathlineto{\pgfqpoint{3.474049in}{2.282667in}}%
\pgfpathlineto{\pgfqpoint{3.485333in}{2.286877in}}%
\pgfpathlineto{\pgfqpoint{3.522667in}{2.303041in}}%
\pgfpathlineto{\pgfqpoint{3.555361in}{2.320000in}}%
\pgfpathlineto{\pgfqpoint{3.560000in}{2.322529in}}%
\pgfpathlineto{\pgfqpoint{3.597333in}{2.345269in}}%
\pgfpathlineto{\pgfqpoint{3.614737in}{2.357333in}}%
\pgfpathlineto{\pgfqpoint{3.634667in}{2.372821in}}%
\pgfpathlineto{\pgfqpoint{3.660113in}{2.394667in}}%
\pgfpathlineto{\pgfqpoint{3.672000in}{2.406882in}}%
\pgfpathlineto{\pgfqpoint{3.695032in}{2.432000in}}%
\pgfpathlineto{\pgfqpoint{3.709333in}{2.452223in}}%
\pgfpathlineto{\pgfqpoint{3.721198in}{2.469333in}}%
\pgfpathlineto{\pgfqpoint{3.739529in}{2.506667in}}%
\pgfpathlineto{\pgfqpoint{3.746667in}{2.531058in}}%
\pgfpathlineto{\pgfqpoint{3.750628in}{2.544000in}}%
\pgfpathlineto{\pgfqpoint{3.754767in}{2.581333in}}%
\pgfpathlineto{\pgfqpoint{3.751401in}{2.618667in}}%
\pgfpathlineto{\pgfqpoint{3.746667in}{2.634492in}}%
\pgfpathlineto{\pgfqpoint{3.740229in}{2.656000in}}%
\pgfpathlineto{\pgfqpoint{3.720088in}{2.693333in}}%
\pgfpathlineto{\pgfqpoint{3.709333in}{2.706952in}}%
\pgfpathlineto{\pgfqpoint{3.688810in}{2.730667in}}%
\pgfpathlineto{\pgfqpoint{3.672000in}{2.745532in}}%
\pgfpathlineto{\pgfqpoint{3.642418in}{2.768000in}}%
\pgfpathlineto{\pgfqpoint{3.634667in}{2.772993in}}%
\pgfpathlineto{\pgfqpoint{3.597333in}{2.793663in}}%
\pgfpathlineto{\pgfqpoint{3.570879in}{2.805333in}}%
\pgfpathlineto{\pgfqpoint{3.560000in}{2.809828in}}%
\pgfpathlineto{\pgfqpoint{3.522667in}{2.822543in}}%
\pgfpathlineto{\pgfqpoint{3.485333in}{2.832146in}}%
\pgfpathlineto{\pgfqpoint{3.448000in}{2.839114in}}%
\pgfpathlineto{\pgfqpoint{3.420002in}{2.842667in}}%
\pgfpathlineto{\pgfqpoint{3.410667in}{2.843916in}}%
\pgfpathlineto{\pgfqpoint{3.373333in}{2.846864in}}%
\pgfpathlineto{\pgfqpoint{3.336000in}{2.847782in}}%
\pgfpathlineto{\pgfqpoint{3.298667in}{2.846761in}}%
\pgfpathlineto{\pgfqpoint{3.261333in}{2.843819in}}%
\pgfpathlineto{\pgfqpoint{3.252437in}{2.842667in}}%
\pgfpathlineto{\pgfqpoint{3.224000in}{2.839320in}}%
\pgfpathlineto{\pgfqpoint{3.186667in}{2.833122in}}%
\pgfpathlineto{\pgfqpoint{3.149333in}{2.824947in}}%
\pgfpathlineto{\pgfqpoint{3.112000in}{2.814560in}}%
\pgfpathlineto{\pgfqpoint{3.084879in}{2.805333in}}%
\pgfpathlineto{\pgfqpoint{3.074667in}{2.801942in}}%
\pgfpathlineto{\pgfqpoint{3.037333in}{2.787438in}}%
\pgfpathlineto{\pgfqpoint{3.000000in}{2.769814in}}%
\pgfpathlineto{\pgfqpoint{2.996551in}{2.768000in}}%
\pgfpathlineto{\pgfqpoint{2.962667in}{2.749406in}}%
\pgfpathlineto{\pgfqpoint{2.933472in}{2.730667in}}%
\pgfpathlineto{\pgfqpoint{2.925333in}{2.724902in}}%
\pgfpathlineto{\pgfqpoint{2.888000in}{2.695451in}}%
\pgfpathlineto{\pgfqpoint{2.885513in}{2.693333in}}%
\pgfpathlineto{\pgfqpoint{2.850667in}{2.658220in}}%
\pgfpathlineto{\pgfqpoint{2.848590in}{2.656000in}}%
\pgfpathlineto{\pgfqpoint{2.820551in}{2.618667in}}%
\pgfpathlineto{\pgfqpoint{2.813333in}{2.605246in}}%
\pgfpathlineto{\pgfqpoint{2.800403in}{2.581333in}}%
\pgfpathlineto{\pgfqpoint{2.787714in}{2.544000in}}%
\pgfpathlineto{\pgfqpoint{2.782234in}{2.506667in}}%
\pgfpathlineto{\pgfqpoint{2.783873in}{2.469333in}}%
\pgfpathlineto{\pgfqpoint{2.793196in}{2.432000in}}%
\pgfpathlineto{\pgfqpoint{2.811509in}{2.394667in}}%
\pgfpathlineto{\pgfqpoint{2.813333in}{2.392104in}}%
\pgfpathlineto{\pgfqpoint{2.840020in}{2.357333in}}%
\pgfpathlineto{\pgfqpoint{2.850667in}{2.347029in}}%
\pgfpathlineto{\pgfqpoint{2.882612in}{2.320000in}}%
\pgfpathlineto{\pgfqpoint{2.888000in}{2.316222in}}%
\pgfpathlineto{\pgfqpoint{2.925333in}{2.293455in}}%
\pgfpathlineto{\pgfqpoint{2.947212in}{2.282667in}}%
\pgfpathlineto{\pgfqpoint{2.962667in}{2.275665in}}%
\pgfpathlineto{\pgfqpoint{3.000000in}{2.261773in}}%
\pgfpathlineto{\pgfqpoint{3.037333in}{2.251194in}}%
\pgfpathlineto{\pgfqpoint{3.064970in}{2.245333in}}%
\pgfpathlineto{\pgfqpoint{3.074667in}{2.243235in}}%
\pgfusepath{stroke}%
\end{pgfscope}%
\begin{pgfscope}%
\pgfpathrectangle{\pgfqpoint{1.432000in}{0.528000in}}{\pgfqpoint{3.696000in}{3.696000in}}%
\pgfusepath{clip}%
\pgfsetbuttcap%
\pgfsetroundjoin%
\pgfsetlinewidth{1.505625pt}%
\definecolor{currentstroke}{rgb}{1.000000,0.359314,0.000000}%
\pgfsetstrokecolor{currentstroke}%
\pgfsetstrokeopacity{0.300000}%
\pgfsetdash{}{0pt}%
\pgfpathmoveto{\pgfqpoint{3.224000in}{2.282383in}}%
\pgfpathlineto{\pgfqpoint{3.229444in}{2.282667in}}%
\pgfpathlineto{\pgfqpoint{3.261333in}{2.284238in}}%
\pgfpathlineto{\pgfqpoint{3.298667in}{2.288194in}}%
\pgfpathlineto{\pgfqpoint{3.336000in}{2.294308in}}%
\pgfpathlineto{\pgfqpoint{3.373333in}{2.302705in}}%
\pgfpathlineto{\pgfqpoint{3.410667in}{2.313597in}}%
\pgfpathlineto{\pgfqpoint{3.428614in}{2.320000in}}%
\pgfpathlineto{\pgfqpoint{3.448000in}{2.327118in}}%
\pgfpathlineto{\pgfqpoint{3.485333in}{2.343530in}}%
\pgfpathlineto{\pgfqpoint{3.511736in}{2.357333in}}%
\pgfpathlineto{\pgfqpoint{3.522667in}{2.363618in}}%
\pgfpathlineto{\pgfqpoint{3.560000in}{2.388175in}}%
\pgfpathlineto{\pgfqpoint{3.568816in}{2.394667in}}%
\pgfpathlineto{\pgfqpoint{3.597333in}{2.419537in}}%
\pgfpathlineto{\pgfqpoint{3.610376in}{2.432000in}}%
\pgfpathlineto{\pgfqpoint{3.634667in}{2.461897in}}%
\pgfpathlineto{\pgfqpoint{3.640414in}{2.469333in}}%
\pgfpathlineto{\pgfqpoint{3.660891in}{2.506667in}}%
\pgfpathlineto{\pgfqpoint{3.672000in}{2.542433in}}%
\pgfpathlineto{\pgfqpoint{3.672495in}{2.544000in}}%
\pgfpathlineto{\pgfqpoint{3.675738in}{2.581333in}}%
\pgfpathlineto{\pgfqpoint{3.672000in}{2.605917in}}%
\pgfpathlineto{\pgfqpoint{3.670033in}{2.618667in}}%
\pgfpathlineto{\pgfqpoint{3.654504in}{2.656000in}}%
\pgfpathlineto{\pgfqpoint{3.634667in}{2.683539in}}%
\pgfpathlineto{\pgfqpoint{3.626784in}{2.693333in}}%
\pgfpathlineto{\pgfqpoint{3.597333in}{2.719408in}}%
\pgfpathlineto{\pgfqpoint{3.582136in}{2.730667in}}%
\pgfpathlineto{\pgfqpoint{3.560000in}{2.743984in}}%
\pgfpathlineto{\pgfqpoint{3.522667in}{2.761969in}}%
\pgfpathlineto{\pgfqpoint{3.506733in}{2.768000in}}%
\pgfpathlineto{\pgfqpoint{3.485333in}{2.775372in}}%
\pgfpathlineto{\pgfqpoint{3.448000in}{2.785129in}}%
\pgfpathlineto{\pgfqpoint{3.410667in}{2.791879in}}%
\pgfpathlineto{\pgfqpoint{3.373333in}{2.795989in}}%
\pgfpathlineto{\pgfqpoint{3.336000in}{2.797714in}}%
\pgfpathlineto{\pgfqpoint{3.298667in}{2.797218in}}%
\pgfpathlineto{\pgfqpoint{3.261333in}{2.794577in}}%
\pgfpathlineto{\pgfqpoint{3.224000in}{2.789790in}}%
\pgfpathlineto{\pgfqpoint{3.186667in}{2.782777in}}%
\pgfpathlineto{\pgfqpoint{3.149333in}{2.773373in}}%
\pgfpathlineto{\pgfqpoint{3.132253in}{2.768000in}}%
\pgfpathlineto{\pgfqpoint{3.112000in}{2.761540in}}%
\pgfpathlineto{\pgfqpoint{3.074667in}{2.747066in}}%
\pgfpathlineto{\pgfqpoint{3.039929in}{2.730667in}}%
\pgfpathlineto{\pgfqpoint{3.037333in}{2.729341in}}%
\pgfpathlineto{\pgfqpoint{3.000000in}{2.707654in}}%
\pgfpathlineto{\pgfqpoint{2.978717in}{2.693333in}}%
\pgfpathlineto{\pgfqpoint{2.962667in}{2.680771in}}%
\pgfpathlineto{\pgfqpoint{2.934258in}{2.656000in}}%
\pgfpathlineto{\pgfqpoint{2.925333in}{2.646190in}}%
\pgfpathlineto{\pgfqpoint{2.901883in}{2.618667in}}%
\pgfpathlineto{\pgfqpoint{2.888000in}{2.595775in}}%
\pgfpathlineto{\pgfqpoint{2.879452in}{2.581333in}}%
\pgfpathlineto{\pgfqpoint{2.865864in}{2.544000in}}%
\pgfpathlineto{\pgfqpoint{2.860859in}{2.506667in}}%
\pgfpathlineto{\pgfqpoint{2.864551in}{2.469333in}}%
\pgfpathlineto{\pgfqpoint{2.877872in}{2.432000in}}%
\pgfpathlineto{\pgfqpoint{2.888000in}{2.415935in}}%
\pgfpathlineto{\pgfqpoint{2.902710in}{2.394667in}}%
\pgfpathlineto{\pgfqpoint{2.925333in}{2.372285in}}%
\pgfpathlineto{\pgfqpoint{2.943032in}{2.357333in}}%
\pgfpathlineto{\pgfqpoint{2.962667in}{2.344229in}}%
\pgfpathlineto{\pgfqpoint{3.000000in}{2.324035in}}%
\pgfpathlineto{\pgfqpoint{3.009269in}{2.320000in}}%
\pgfpathlineto{\pgfqpoint{3.037333in}{2.309119in}}%
\pgfpathlineto{\pgfqpoint{3.074667in}{2.298119in}}%
\pgfpathlineto{\pgfqpoint{3.112000in}{2.290313in}}%
\pgfpathlineto{\pgfqpoint{3.149333in}{2.285275in}}%
\pgfpathlineto{\pgfqpoint{3.186667in}{2.282704in}}%
\pgfpathlineto{\pgfqpoint{3.191159in}{2.282667in}}%
\pgfpathlineto{\pgfqpoint{3.224000in}{2.282383in}}%
\pgfusepath{stroke}%
\end{pgfscope}%
\begin{pgfscope}%
\pgfpathrectangle{\pgfqpoint{1.432000in}{0.528000in}}{\pgfqpoint{3.696000in}{3.696000in}}%
\pgfusepath{clip}%
\pgfsetbuttcap%
\pgfsetroundjoin%
\pgfsetlinewidth{1.505625pt}%
\definecolor{currentstroke}{rgb}{1.000000,0.688725,0.000000}%
\pgfsetstrokecolor{currentstroke}%
\pgfsetstrokeopacity{0.300000}%
\pgfsetdash{}{0pt}%
\pgfpathmoveto{\pgfqpoint{3.074667in}{2.352557in}}%
\pgfpathlineto{\pgfqpoint{3.112000in}{2.341574in}}%
\pgfpathlineto{\pgfqpoint{3.149333in}{2.334258in}}%
\pgfpathlineto{\pgfqpoint{3.186667in}{2.330163in}}%
\pgfpathlineto{\pgfqpoint{3.224000in}{2.328979in}}%
\pgfpathlineto{\pgfqpoint{3.261333in}{2.330517in}}%
\pgfpathlineto{\pgfqpoint{3.298667in}{2.334689in}}%
\pgfpathlineto{\pgfqpoint{3.336000in}{2.341510in}}%
\pgfpathlineto{\pgfqpoint{3.373333in}{2.351092in}}%
\pgfpathlineto{\pgfqpoint{3.392224in}{2.357333in}}%
\pgfpathlineto{\pgfqpoint{3.410667in}{2.363913in}}%
\pgfpathlineto{\pgfqpoint{3.448000in}{2.380308in}}%
\pgfpathlineto{\pgfqpoint{3.475064in}{2.394667in}}%
\pgfpathlineto{\pgfqpoint{3.485333in}{2.401025in}}%
\pgfpathlineto{\pgfqpoint{3.522667in}{2.427784in}}%
\pgfpathlineto{\pgfqpoint{3.527903in}{2.432000in}}%
\pgfpathlineto{\pgfqpoint{3.560000in}{2.465100in}}%
\pgfpathlineto{\pgfqpoint{3.563767in}{2.469333in}}%
\pgfpathlineto{\pgfqpoint{3.587011in}{2.506667in}}%
\pgfpathlineto{\pgfqpoint{3.597333in}{2.537377in}}%
\pgfpathlineto{\pgfqpoint{3.599519in}{2.544000in}}%
\pgfpathlineto{\pgfqpoint{3.601537in}{2.581333in}}%
\pgfpathlineto{\pgfqpoint{3.597333in}{2.599253in}}%
\pgfpathlineto{\pgfqpoint{3.592445in}{2.618667in}}%
\pgfpathlineto{\pgfqpoint{3.570157in}{2.656000in}}%
\pgfpathlineto{\pgfqpoint{3.560000in}{2.666625in}}%
\pgfpathlineto{\pgfqpoint{3.529483in}{2.693333in}}%
\pgfpathlineto{\pgfqpoint{3.522667in}{2.697786in}}%
\pgfpathlineto{\pgfqpoint{3.485333in}{2.717687in}}%
\pgfpathlineto{\pgfqpoint{3.452365in}{2.730667in}}%
\pgfpathlineto{\pgfqpoint{3.448000in}{2.732145in}}%
\pgfpathlineto{\pgfqpoint{3.410667in}{2.741580in}}%
\pgfpathlineto{\pgfqpoint{3.373333in}{2.747587in}}%
\pgfpathlineto{\pgfqpoint{3.336000in}{2.750541in}}%
\pgfpathlineto{\pgfqpoint{3.298667in}{2.750692in}}%
\pgfpathlineto{\pgfqpoint{3.261333in}{2.748178in}}%
\pgfpathlineto{\pgfqpoint{3.224000in}{2.743039in}}%
\pgfpathlineto{\pgfqpoint{3.186667in}{2.735215in}}%
\pgfpathlineto{\pgfqpoint{3.170473in}{2.730667in}}%
\pgfpathlineto{\pgfqpoint{3.149333in}{2.724361in}}%
\pgfpathlineto{\pgfqpoint{3.112000in}{2.710276in}}%
\pgfpathlineto{\pgfqpoint{3.075820in}{2.693333in}}%
\pgfpathlineto{\pgfqpoint{3.074667in}{2.692715in}}%
\pgfpathlineto{\pgfqpoint{3.037333in}{2.669493in}}%
\pgfpathlineto{\pgfqpoint{3.018647in}{2.656000in}}%
\pgfpathlineto{\pgfqpoint{3.000000in}{2.639137in}}%
\pgfpathlineto{\pgfqpoint{2.979602in}{2.618667in}}%
\pgfpathlineto{\pgfqpoint{2.962667in}{2.594742in}}%
\pgfpathlineto{\pgfqpoint{2.953724in}{2.581333in}}%
\pgfpathlineto{\pgfqpoint{2.939024in}{2.544000in}}%
\pgfpathlineto{\pgfqpoint{2.934730in}{2.506667in}}%
\pgfpathlineto{\pgfqpoint{2.941310in}{2.469333in}}%
\pgfpathlineto{\pgfqpoint{2.960297in}{2.432000in}}%
\pgfpathlineto{\pgfqpoint{2.962667in}{2.429154in}}%
\pgfpathlineto{\pgfqpoint{2.996236in}{2.394667in}}%
\pgfpathlineto{\pgfqpoint{3.000000in}{2.391887in}}%
\pgfpathlineto{\pgfqpoint{3.037333in}{2.368991in}}%
\pgfpathlineto{\pgfqpoint{3.062410in}{2.357333in}}%
\pgfpathlineto{\pgfqpoint{3.074667in}{2.352557in}}%
\pgfusepath{stroke}%
\end{pgfscope}%
\begin{pgfscope}%
\pgfpathrectangle{\pgfqpoint{1.432000in}{0.528000in}}{\pgfqpoint{3.696000in}{3.696000in}}%
\pgfusepath{clip}%
\pgfsetbuttcap%
\pgfsetroundjoin%
\pgfsetlinewidth{1.505625pt}%
\definecolor{currentstroke}{rgb}{1.000000,1.000000,0.027205}%
\pgfsetstrokecolor{currentstroke}%
\pgfsetstrokeopacity{0.300000}%
\pgfsetdash{}{0pt}%
\pgfpathmoveto{\pgfqpoint{3.149333in}{2.386152in}}%
\pgfpathlineto{\pgfqpoint{3.186667in}{2.379707in}}%
\pgfpathlineto{\pgfqpoint{3.224000in}{2.377133in}}%
\pgfpathlineto{\pgfqpoint{3.261333in}{2.378109in}}%
\pgfpathlineto{\pgfqpoint{3.298667in}{2.382454in}}%
\pgfpathlineto{\pgfqpoint{3.336000in}{2.390121in}}%
\pgfpathlineto{\pgfqpoint{3.351499in}{2.394667in}}%
\pgfpathlineto{\pgfqpoint{3.373333in}{2.402042in}}%
\pgfpathlineto{\pgfqpoint{3.410667in}{2.418457in}}%
\pgfpathlineto{\pgfqpoint{3.435511in}{2.432000in}}%
\pgfpathlineto{\pgfqpoint{3.448000in}{2.440672in}}%
\pgfpathlineto{\pgfqpoint{3.482992in}{2.469333in}}%
\pgfpathlineto{\pgfqpoint{3.485333in}{2.472158in}}%
\pgfpathlineto{\pgfqpoint{3.510928in}{2.506667in}}%
\pgfpathlineto{\pgfqpoint{3.522667in}{2.537878in}}%
\pgfpathlineto{\pgfqpoint{3.524832in}{2.544000in}}%
\pgfpathlineto{\pgfqpoint{3.524944in}{2.581333in}}%
\pgfpathlineto{\pgfqpoint{3.522667in}{2.587623in}}%
\pgfpathlineto{\pgfqpoint{3.509683in}{2.618667in}}%
\pgfpathlineto{\pgfqpoint{3.485333in}{2.645838in}}%
\pgfpathlineto{\pgfqpoint{3.473812in}{2.656000in}}%
\pgfpathlineto{\pgfqpoint{3.448000in}{2.671296in}}%
\pgfpathlineto{\pgfqpoint{3.410667in}{2.687256in}}%
\pgfpathlineto{\pgfqpoint{3.389202in}{2.693333in}}%
\pgfpathlineto{\pgfqpoint{3.373333in}{2.696993in}}%
\pgfpathlineto{\pgfqpoint{3.336000in}{2.701808in}}%
\pgfpathlineto{\pgfqpoint{3.298667in}{2.702983in}}%
\pgfpathlineto{\pgfqpoint{3.261333in}{2.700765in}}%
\pgfpathlineto{\pgfqpoint{3.224000in}{2.695263in}}%
\pgfpathlineto{\pgfqpoint{3.215730in}{2.693333in}}%
\pgfpathlineto{\pgfqpoint{3.186667in}{2.685674in}}%
\pgfpathlineto{\pgfqpoint{3.149333in}{2.672186in}}%
\pgfpathlineto{\pgfqpoint{3.114494in}{2.656000in}}%
\pgfpathlineto{\pgfqpoint{3.112000in}{2.654560in}}%
\pgfpathlineto{\pgfqpoint{3.074667in}{2.628694in}}%
\pgfpathlineto{\pgfqpoint{3.062272in}{2.618667in}}%
\pgfpathlineto{\pgfqpoint{3.037333in}{2.590125in}}%
\pgfpathlineto{\pgfqpoint{3.030419in}{2.581333in}}%
\pgfpathlineto{\pgfqpoint{3.013903in}{2.544000in}}%
\pgfpathlineto{\pgfqpoint{3.010803in}{2.506667in}}%
\pgfpathlineto{\pgfqpoint{3.022291in}{2.469333in}}%
\pgfpathlineto{\pgfqpoint{3.037333in}{2.449002in}}%
\pgfpathlineto{\pgfqpoint{3.052778in}{2.432000in}}%
\pgfpathlineto{\pgfqpoint{3.074667in}{2.416720in}}%
\pgfpathlineto{\pgfqpoint{3.112000in}{2.397424in}}%
\pgfpathlineto{\pgfqpoint{3.119592in}{2.394667in}}%
\pgfpathlineto{\pgfqpoint{3.149333in}{2.386152in}}%
\pgfusepath{stroke}%
\end{pgfscope}%
\begin{pgfscope}%
\pgfpathrectangle{\pgfqpoint{1.432000in}{0.528000in}}{\pgfqpoint{3.696000in}{3.696000in}}%
\pgfusepath{clip}%
\pgfsetbuttcap%
\pgfsetroundjoin%
\pgfsetlinewidth{1.505625pt}%
\definecolor{currentstroke}{rgb}{1.000000,1.000000,0.521323}%
\pgfsetstrokecolor{currentstroke}%
\pgfsetstrokeopacity{0.300000}%
\pgfsetdash{}{0pt}%
\pgfpathmoveto{\pgfqpoint{3.149333in}{2.454905in}}%
\pgfpathlineto{\pgfqpoint{3.186667in}{2.440148in}}%
\pgfpathlineto{\pgfqpoint{3.224000in}{2.433194in}}%
\pgfpathlineto{\pgfqpoint{3.261333in}{2.432911in}}%
\pgfpathlineto{\pgfqpoint{3.298667in}{2.438536in}}%
\pgfpathlineto{\pgfqpoint{3.336000in}{2.449618in}}%
\pgfpathlineto{\pgfqpoint{3.373333in}{2.465966in}}%
\pgfpathlineto{\pgfqpoint{3.379191in}{2.469333in}}%
\pgfpathlineto{\pgfqpoint{3.410667in}{2.496378in}}%
\pgfpathlineto{\pgfqpoint{3.420558in}{2.506667in}}%
\pgfpathlineto{\pgfqpoint{3.437043in}{2.544000in}}%
\pgfpathlineto{\pgfqpoint{3.432404in}{2.581333in}}%
\pgfpathlineto{\pgfqpoint{3.410667in}{2.609110in}}%
\pgfpathlineto{\pgfqpoint{3.399998in}{2.618667in}}%
\pgfpathlineto{\pgfqpoint{3.373333in}{2.631755in}}%
\pgfpathlineto{\pgfqpoint{3.336000in}{2.642541in}}%
\pgfpathlineto{\pgfqpoint{3.298667in}{2.646607in}}%
\pgfpathlineto{\pgfqpoint{3.261333in}{2.644786in}}%
\pgfpathlineto{\pgfqpoint{3.224000in}{2.637612in}}%
\pgfpathlineto{\pgfqpoint{3.186667in}{2.625361in}}%
\pgfpathlineto{\pgfqpoint{3.172147in}{2.618667in}}%
\pgfpathlineto{\pgfqpoint{3.149333in}{2.603575in}}%
\pgfpathlineto{\pgfqpoint{3.122708in}{2.581333in}}%
\pgfpathlineto{\pgfqpoint{3.112000in}{2.564045in}}%
\pgfpathlineto{\pgfqpoint{3.101402in}{2.544000in}}%
\pgfpathlineto{\pgfqpoint{3.101222in}{2.506667in}}%
\pgfpathlineto{\pgfqpoint{3.112000in}{2.487831in}}%
\pgfpathlineto{\pgfqpoint{3.126332in}{2.469333in}}%
\pgfpathlineto{\pgfqpoint{3.149333in}{2.454905in}}%
\pgfusepath{stroke}%
\end{pgfscope}%
\begin{pgfscope}%
\pgfsetrectcap%
\pgfsetmiterjoin%
\pgfsetlinewidth{0.803000pt}%
\definecolor{currentstroke}{rgb}{0.000000,0.000000,0.000000}%
\pgfsetstrokecolor{currentstroke}%
\pgfsetdash{}{0pt}%
\pgfpathmoveto{\pgfqpoint{1.432000in}{0.528000in}}%
\pgfpathlineto{\pgfqpoint{1.432000in}{4.224000in}}%
\pgfusepath{stroke}%
\end{pgfscope}%
\begin{pgfscope}%
\pgfsetrectcap%
\pgfsetmiterjoin%
\pgfsetlinewidth{0.803000pt}%
\definecolor{currentstroke}{rgb}{0.000000,0.000000,0.000000}%
\pgfsetstrokecolor{currentstroke}%
\pgfsetdash{}{0pt}%
\pgfpathmoveto{\pgfqpoint{5.128000in}{0.528000in}}%
\pgfpathlineto{\pgfqpoint{5.128000in}{4.224000in}}%
\pgfusepath{stroke}%
\end{pgfscope}%
\begin{pgfscope}%
\pgfsetrectcap%
\pgfsetmiterjoin%
\pgfsetlinewidth{0.803000pt}%
\definecolor{currentstroke}{rgb}{0.000000,0.000000,0.000000}%
\pgfsetstrokecolor{currentstroke}%
\pgfsetdash{}{0pt}%
\pgfpathmoveto{\pgfqpoint{1.432000in}{0.528000in}}%
\pgfpathlineto{\pgfqpoint{5.128000in}{0.528000in}}%
\pgfusepath{stroke}%
\end{pgfscope}%
\begin{pgfscope}%
\pgfsetrectcap%
\pgfsetmiterjoin%
\pgfsetlinewidth{0.803000pt}%
\definecolor{currentstroke}{rgb}{0.000000,0.000000,0.000000}%
\pgfsetstrokecolor{currentstroke}%
\pgfsetdash{}{0pt}%
\pgfpathmoveto{\pgfqpoint{1.432000in}{4.224000in}}%
\pgfpathlineto{\pgfqpoint{5.128000in}{4.224000in}}%
\pgfusepath{stroke}%
\end{pgfscope}%
\begin{pgfscope}%
\definecolor{textcolor}{rgb}{0.000000,0.000000,0.000000}%
\pgfsetstrokecolor{textcolor}%
\pgfsetfillcolor{textcolor}%
\pgftext[x=3.280000in,y=4.307333in,,base]{\color{textcolor}\rmfamily\fontsize{12.000000}{14.400000}\selectfont Experiment 1A: CE-surrogate (\(\displaystyle k=5\))}%
\end{pgfscope}%
\end{pgfpicture}%
\makeatother%
\endgroup%
}
    }
    \subfloat[The cross-entropy mixture method.]{%
    \resizebox{0.3\textwidth}{!}{%% Creator: Matplotlib, PGF backend
%%
%% To include the figure in your LaTeX document, write
%%   \input{<filename>.pgf}
%%
%% Make sure the required packages are loaded in your preamble
%%   \usepackage{pgf}
%%
%% Figures using additional raster images can only be included by \input if
%% they are in the same directory as the main LaTeX file. For loading figures
%% from other directories you can use the `import` package
%%   \usepackage{import}
%% and then include the figures with
%%   \import{<path to file>}{<filename>.pgf}
%%
%% Matplotlib used the following preamble
%%   \usepackage{fontspec}
%%   \setmainfont{DejaVuSans.ttf}[Path=C:/Users/mossr/.julia/conda/3/lib/site-packages/matplotlib/mpl-data/fonts/ttf/]
%%   \setsansfont{DejaVuSans.ttf}[Path=C:/Users/mossr/.julia/conda/3/lib/site-packages/matplotlib/mpl-data/fonts/ttf/]
%%   \setmonofont{DejaVuSansMono.ttf}[Path=C:/Users/mossr/.julia/conda/3/lib/site-packages/matplotlib/mpl-data/fonts/ttf/]
%%
\begingroup%
\makeatletter%
\begin{pgfpicture}%
% \pgfpathrectangle{\pgfpointorigin}{\pgfqpoint{5.400000in}{4.800000in}}%
% \pgfpathrectangle{\pgfqpoint{1.432000in}{0.0in}}{\pgfqpoint{3.696000in}{4.0in}}%
\pgfpathrectangle{\pgfqpoint{1.0in}{0.0in}}{\pgfqpoint{4.0in}{4.5in}}%
\pgfusepath{use as bounding box, clip}%
\begin{pgfscope}%
\pgfsetbuttcap%
\pgfsetmiterjoin%
\definecolor{currentfill}{rgb}{1.000000,1.000000,1.000000}%
\pgfsetfillcolor{currentfill}%
\pgfsetlinewidth{0.000000pt}%
\definecolor{currentstroke}{rgb}{1.000000,1.000000,1.000000}%
\pgfsetstrokecolor{currentstroke}%
\pgfsetdash{}{0pt}%
\pgfpathmoveto{\pgfqpoint{0.000000in}{0.000000in}}%
\pgfpathlineto{\pgfqpoint{6.400000in}{0.000000in}}%
\pgfpathlineto{\pgfqpoint{6.400000in}{4.800000in}}%
\pgfpathlineto{\pgfqpoint{0.000000in}{4.800000in}}%
\pgfpathclose%
\pgfusepath{fill}%
\end{pgfscope}%
\begin{pgfscope}%
\pgfsetbuttcap%
\pgfsetmiterjoin%
\definecolor{currentfill}{rgb}{1.000000,1.000000,1.000000}%
\pgfsetfillcolor{currentfill}%
\pgfsetlinewidth{0.000000pt}%
\definecolor{currentstroke}{rgb}{0.000000,0.000000,0.000000}%
\pgfsetstrokecolor{currentstroke}%
\pgfsetstrokeopacity{0.000000}%
\pgfsetdash{}{0pt}%
\pgfpathmoveto{\pgfqpoint{1.432000in}{0.528000in}}%
\pgfpathlineto{\pgfqpoint{5.128000in}{0.528000in}}%
\pgfpathlineto{\pgfqpoint{5.128000in}{4.224000in}}%
\pgfpathlineto{\pgfqpoint{1.432000in}{4.224000in}}%
\pgfpathclose%
\pgfusepath{fill}%
\end{pgfscope}%
\begin{pgfscope}%
\pgfpathrectangle{\pgfqpoint{1.432000in}{0.528000in}}{\pgfqpoint{3.696000in}{3.696000in}}%
\pgfusepath{clip}%
\pgfsys@transformshift{1.432000in}{0.528000in}%
\pgftext[left,bottom]{\pgfimage[interpolate=true,width=3.700000in,height=3.700000in]{figures/cem_variants/k5_ce_mixture-img0.png}}%
\end{pgfscope}%
\begin{pgfscope}%
\pgfpathrectangle{\pgfqpoint{1.432000in}{0.528000in}}{\pgfqpoint{3.696000in}{3.696000in}}%
\pgfusepath{clip}%
\pgfsetbuttcap%
\pgfsetroundjoin%
\definecolor{currentfill}{rgb}{0.000000,0.000000,0.000000}%
\pgfsetfillcolor{currentfill}%
\pgfsetlinewidth{0.501875pt}%
\definecolor{currentstroke}{rgb}{1.000000,1.000000,1.000000}%
\pgfsetstrokecolor{currentstroke}%
\pgfsetdash{}{0pt}%
\pgfsys@defobject{currentmarker}{\pgfqpoint{-0.018373in}{-0.018373in}}{\pgfqpoint{0.018373in}{0.018373in}}{%
\pgfpathmoveto{\pgfqpoint{0.000000in}{-0.018373in}}%
\pgfpathcurveto{\pgfqpoint{0.004873in}{-0.018373in}}{\pgfqpoint{0.009546in}{-0.016437in}}{\pgfqpoint{0.012992in}{-0.012992in}}%
\pgfpathcurveto{\pgfqpoint{0.016437in}{-0.009546in}}{\pgfqpoint{0.018373in}{-0.004873in}}{\pgfqpoint{0.018373in}{0.000000in}}%
\pgfpathcurveto{\pgfqpoint{0.018373in}{0.004873in}}{\pgfqpoint{0.016437in}{0.009546in}}{\pgfqpoint{0.012992in}{0.012992in}}%
\pgfpathcurveto{\pgfqpoint{0.009546in}{0.016437in}}{\pgfqpoint{0.004873in}{0.018373in}}{\pgfqpoint{0.000000in}{0.018373in}}%
\pgfpathcurveto{\pgfqpoint{-0.004873in}{0.018373in}}{\pgfqpoint{-0.009546in}{0.016437in}}{\pgfqpoint{-0.012992in}{0.012992in}}%
\pgfpathcurveto{\pgfqpoint{-0.016437in}{0.009546in}}{\pgfqpoint{-0.018373in}{0.004873in}}{\pgfqpoint{-0.018373in}{0.000000in}}%
\pgfpathcurveto{\pgfqpoint{-0.018373in}{-0.004873in}}{\pgfqpoint{-0.016437in}{-0.009546in}}{\pgfqpoint{-0.012992in}{-0.012992in}}%
\pgfpathcurveto{\pgfqpoint{-0.009546in}{-0.016437in}}{\pgfqpoint{-0.004873in}{-0.018373in}}{\pgfqpoint{0.000000in}{-0.018373in}}%
\pgfpathclose%
\pgfusepath{stroke,fill}%
}%
\begin{pgfscope}%
\pgfsys@transformshift{2.878656in}{2.695467in}%
\pgfsys@useobject{currentmarker}{}%
\end{pgfscope}%
\end{pgfscope}%
\begin{pgfscope}%
\pgfpathrectangle{\pgfqpoint{1.432000in}{0.528000in}}{\pgfqpoint{3.696000in}{3.696000in}}%
\pgfusepath{clip}%
\pgfsetbuttcap%
\pgfsetroundjoin%
\definecolor{currentfill}{rgb}{0.000000,0.000000,0.000000}%
\pgfsetfillcolor{currentfill}%
\pgfsetlinewidth{0.501875pt}%
\definecolor{currentstroke}{rgb}{1.000000,1.000000,1.000000}%
\pgfsetstrokecolor{currentstroke}%
\pgfsetdash{}{0pt}%
\pgfsys@defobject{currentmarker}{\pgfqpoint{-0.018373in}{-0.018373in}}{\pgfqpoint{0.018373in}{0.018373in}}{%
\pgfpathmoveto{\pgfqpoint{0.000000in}{-0.018373in}}%
\pgfpathcurveto{\pgfqpoint{0.004873in}{-0.018373in}}{\pgfqpoint{0.009546in}{-0.016437in}}{\pgfqpoint{0.012992in}{-0.012992in}}%
\pgfpathcurveto{\pgfqpoint{0.016437in}{-0.009546in}}{\pgfqpoint{0.018373in}{-0.004873in}}{\pgfqpoint{0.018373in}{0.000000in}}%
\pgfpathcurveto{\pgfqpoint{0.018373in}{0.004873in}}{\pgfqpoint{0.016437in}{0.009546in}}{\pgfqpoint{0.012992in}{0.012992in}}%
\pgfpathcurveto{\pgfqpoint{0.009546in}{0.016437in}}{\pgfqpoint{0.004873in}{0.018373in}}{\pgfqpoint{0.000000in}{0.018373in}}%
\pgfpathcurveto{\pgfqpoint{-0.004873in}{0.018373in}}{\pgfqpoint{-0.009546in}{0.016437in}}{\pgfqpoint{-0.012992in}{0.012992in}}%
\pgfpathcurveto{\pgfqpoint{-0.016437in}{0.009546in}}{\pgfqpoint{-0.018373in}{0.004873in}}{\pgfqpoint{-0.018373in}{0.000000in}}%
\pgfpathcurveto{\pgfqpoint{-0.018373in}{-0.004873in}}{\pgfqpoint{-0.016437in}{-0.009546in}}{\pgfqpoint{-0.012992in}{-0.012992in}}%
\pgfpathcurveto{\pgfqpoint{-0.009546in}{-0.016437in}}{\pgfqpoint{-0.004873in}{-0.018373in}}{\pgfqpoint{0.000000in}{-0.018373in}}%
\pgfpathclose%
\pgfusepath{stroke,fill}%
}%
\begin{pgfscope}%
\pgfsys@transformshift{3.413110in}{2.166836in}%
\pgfsys@useobject{currentmarker}{}%
\end{pgfscope}%
\end{pgfscope}%
\begin{pgfscope}%
\pgfpathrectangle{\pgfqpoint{1.432000in}{0.528000in}}{\pgfqpoint{3.696000in}{3.696000in}}%
\pgfusepath{clip}%
\pgfsetbuttcap%
\pgfsetroundjoin%
\definecolor{currentfill}{rgb}{0.000000,0.000000,0.000000}%
\pgfsetfillcolor{currentfill}%
\pgfsetlinewidth{0.501875pt}%
\definecolor{currentstroke}{rgb}{1.000000,1.000000,1.000000}%
\pgfsetstrokecolor{currentstroke}%
\pgfsetdash{}{0pt}%
\pgfsys@defobject{currentmarker}{\pgfqpoint{-0.018373in}{-0.018373in}}{\pgfqpoint{0.018373in}{0.018373in}}{%
\pgfpathmoveto{\pgfqpoint{0.000000in}{-0.018373in}}%
\pgfpathcurveto{\pgfqpoint{0.004873in}{-0.018373in}}{\pgfqpoint{0.009546in}{-0.016437in}}{\pgfqpoint{0.012992in}{-0.012992in}}%
\pgfpathcurveto{\pgfqpoint{0.016437in}{-0.009546in}}{\pgfqpoint{0.018373in}{-0.004873in}}{\pgfqpoint{0.018373in}{0.000000in}}%
\pgfpathcurveto{\pgfqpoint{0.018373in}{0.004873in}}{\pgfqpoint{0.016437in}{0.009546in}}{\pgfqpoint{0.012992in}{0.012992in}}%
\pgfpathcurveto{\pgfqpoint{0.009546in}{0.016437in}}{\pgfqpoint{0.004873in}{0.018373in}}{\pgfqpoint{0.000000in}{0.018373in}}%
\pgfpathcurveto{\pgfqpoint{-0.004873in}{0.018373in}}{\pgfqpoint{-0.009546in}{0.016437in}}{\pgfqpoint{-0.012992in}{0.012992in}}%
\pgfpathcurveto{\pgfqpoint{-0.016437in}{0.009546in}}{\pgfqpoint{-0.018373in}{0.004873in}}{\pgfqpoint{-0.018373in}{0.000000in}}%
\pgfpathcurveto{\pgfqpoint{-0.018373in}{-0.004873in}}{\pgfqpoint{-0.016437in}{-0.009546in}}{\pgfqpoint{-0.012992in}{-0.012992in}}%
\pgfpathcurveto{\pgfqpoint{-0.009546in}{-0.016437in}}{\pgfqpoint{-0.004873in}{-0.018373in}}{\pgfqpoint{0.000000in}{-0.018373in}}%
\pgfpathclose%
\pgfusepath{stroke,fill}%
}%
\begin{pgfscope}%
\pgfsys@transformshift{3.050309in}{2.661601in}%
\pgfsys@useobject{currentmarker}{}%
\end{pgfscope}%
\end{pgfscope}%
\begin{pgfscope}%
\pgfpathrectangle{\pgfqpoint{1.432000in}{0.528000in}}{\pgfqpoint{3.696000in}{3.696000in}}%
\pgfusepath{clip}%
\pgfsetbuttcap%
\pgfsetroundjoin%
\definecolor{currentfill}{rgb}{0.000000,0.000000,0.000000}%
\pgfsetfillcolor{currentfill}%
\pgfsetlinewidth{0.501875pt}%
\definecolor{currentstroke}{rgb}{1.000000,1.000000,1.000000}%
\pgfsetstrokecolor{currentstroke}%
\pgfsetdash{}{0pt}%
\pgfsys@defobject{currentmarker}{\pgfqpoint{-0.018373in}{-0.018373in}}{\pgfqpoint{0.018373in}{0.018373in}}{%
\pgfpathmoveto{\pgfqpoint{0.000000in}{-0.018373in}}%
\pgfpathcurveto{\pgfqpoint{0.004873in}{-0.018373in}}{\pgfqpoint{0.009546in}{-0.016437in}}{\pgfqpoint{0.012992in}{-0.012992in}}%
\pgfpathcurveto{\pgfqpoint{0.016437in}{-0.009546in}}{\pgfqpoint{0.018373in}{-0.004873in}}{\pgfqpoint{0.018373in}{0.000000in}}%
\pgfpathcurveto{\pgfqpoint{0.018373in}{0.004873in}}{\pgfqpoint{0.016437in}{0.009546in}}{\pgfqpoint{0.012992in}{0.012992in}}%
\pgfpathcurveto{\pgfqpoint{0.009546in}{0.016437in}}{\pgfqpoint{0.004873in}{0.018373in}}{\pgfqpoint{0.000000in}{0.018373in}}%
\pgfpathcurveto{\pgfqpoint{-0.004873in}{0.018373in}}{\pgfqpoint{-0.009546in}{0.016437in}}{\pgfqpoint{-0.012992in}{0.012992in}}%
\pgfpathcurveto{\pgfqpoint{-0.016437in}{0.009546in}}{\pgfqpoint{-0.018373in}{0.004873in}}{\pgfqpoint{-0.018373in}{0.000000in}}%
\pgfpathcurveto{\pgfqpoint{-0.018373in}{-0.004873in}}{\pgfqpoint{-0.016437in}{-0.009546in}}{\pgfqpoint{-0.012992in}{-0.012992in}}%
\pgfpathcurveto{\pgfqpoint{-0.009546in}{-0.016437in}}{\pgfqpoint{-0.004873in}{-0.018373in}}{\pgfqpoint{0.000000in}{-0.018373in}}%
\pgfpathclose%
\pgfusepath{stroke,fill}%
}%
\begin{pgfscope}%
\pgfsys@transformshift{3.281074in}{2.166690in}%
\pgfsys@useobject{currentmarker}{}%
\end{pgfscope}%
\end{pgfscope}%
\begin{pgfscope}%
\pgfpathrectangle{\pgfqpoint{1.432000in}{0.528000in}}{\pgfqpoint{3.696000in}{3.696000in}}%
\pgfusepath{clip}%
\pgfsetbuttcap%
\pgfsetroundjoin%
\definecolor{currentfill}{rgb}{0.000000,0.000000,0.000000}%
\pgfsetfillcolor{currentfill}%
\pgfsetlinewidth{0.501875pt}%
\definecolor{currentstroke}{rgb}{1.000000,1.000000,1.000000}%
\pgfsetstrokecolor{currentstroke}%
\pgfsetdash{}{0pt}%
\pgfsys@defobject{currentmarker}{\pgfqpoint{-0.018373in}{-0.018373in}}{\pgfqpoint{0.018373in}{0.018373in}}{%
\pgfpathmoveto{\pgfqpoint{0.000000in}{-0.018373in}}%
\pgfpathcurveto{\pgfqpoint{0.004873in}{-0.018373in}}{\pgfqpoint{0.009546in}{-0.016437in}}{\pgfqpoint{0.012992in}{-0.012992in}}%
\pgfpathcurveto{\pgfqpoint{0.016437in}{-0.009546in}}{\pgfqpoint{0.018373in}{-0.004873in}}{\pgfqpoint{0.018373in}{0.000000in}}%
\pgfpathcurveto{\pgfqpoint{0.018373in}{0.004873in}}{\pgfqpoint{0.016437in}{0.009546in}}{\pgfqpoint{0.012992in}{0.012992in}}%
\pgfpathcurveto{\pgfqpoint{0.009546in}{0.016437in}}{\pgfqpoint{0.004873in}{0.018373in}}{\pgfqpoint{0.000000in}{0.018373in}}%
\pgfpathcurveto{\pgfqpoint{-0.004873in}{0.018373in}}{\pgfqpoint{-0.009546in}{0.016437in}}{\pgfqpoint{-0.012992in}{0.012992in}}%
\pgfpathcurveto{\pgfqpoint{-0.016437in}{0.009546in}}{\pgfqpoint{-0.018373in}{0.004873in}}{\pgfqpoint{-0.018373in}{0.000000in}}%
\pgfpathcurveto{\pgfqpoint{-0.018373in}{-0.004873in}}{\pgfqpoint{-0.016437in}{-0.009546in}}{\pgfqpoint{-0.012992in}{-0.012992in}}%
\pgfpathcurveto{\pgfqpoint{-0.009546in}{-0.016437in}}{\pgfqpoint{-0.004873in}{-0.018373in}}{\pgfqpoint{0.000000in}{-0.018373in}}%
\pgfpathclose%
\pgfusepath{stroke,fill}%
}%
\begin{pgfscope}%
\pgfsys@transformshift{3.313443in}{2.064678in}%
\pgfsys@useobject{currentmarker}{}%
\end{pgfscope}%
\end{pgfscope}%
\begin{pgfscope}%
\pgfpathrectangle{\pgfqpoint{1.432000in}{0.528000in}}{\pgfqpoint{3.696000in}{3.696000in}}%
\pgfusepath{clip}%
\pgfsetbuttcap%
\pgfsetroundjoin%
\definecolor{currentfill}{rgb}{0.000000,0.000000,0.000000}%
\pgfsetfillcolor{currentfill}%
\pgfsetlinewidth{0.501875pt}%
\definecolor{currentstroke}{rgb}{1.000000,1.000000,1.000000}%
\pgfsetstrokecolor{currentstroke}%
\pgfsetdash{}{0pt}%
\pgfsys@defobject{currentmarker}{\pgfqpoint{-0.018373in}{-0.018373in}}{\pgfqpoint{0.018373in}{0.018373in}}{%
\pgfpathmoveto{\pgfqpoint{0.000000in}{-0.018373in}}%
\pgfpathcurveto{\pgfqpoint{0.004873in}{-0.018373in}}{\pgfqpoint{0.009546in}{-0.016437in}}{\pgfqpoint{0.012992in}{-0.012992in}}%
\pgfpathcurveto{\pgfqpoint{0.016437in}{-0.009546in}}{\pgfqpoint{0.018373in}{-0.004873in}}{\pgfqpoint{0.018373in}{0.000000in}}%
\pgfpathcurveto{\pgfqpoint{0.018373in}{0.004873in}}{\pgfqpoint{0.016437in}{0.009546in}}{\pgfqpoint{0.012992in}{0.012992in}}%
\pgfpathcurveto{\pgfqpoint{0.009546in}{0.016437in}}{\pgfqpoint{0.004873in}{0.018373in}}{\pgfqpoint{0.000000in}{0.018373in}}%
\pgfpathcurveto{\pgfqpoint{-0.004873in}{0.018373in}}{\pgfqpoint{-0.009546in}{0.016437in}}{\pgfqpoint{-0.012992in}{0.012992in}}%
\pgfpathcurveto{\pgfqpoint{-0.016437in}{0.009546in}}{\pgfqpoint{-0.018373in}{0.004873in}}{\pgfqpoint{-0.018373in}{0.000000in}}%
\pgfpathcurveto{\pgfqpoint{-0.018373in}{-0.004873in}}{\pgfqpoint{-0.016437in}{-0.009546in}}{\pgfqpoint{-0.012992in}{-0.012992in}}%
\pgfpathcurveto{\pgfqpoint{-0.009546in}{-0.016437in}}{\pgfqpoint{-0.004873in}{-0.018373in}}{\pgfqpoint{0.000000in}{-0.018373in}}%
\pgfpathclose%
\pgfusepath{stroke,fill}%
}%
\begin{pgfscope}%
\pgfsys@transformshift{3.128731in}{2.707201in}%
\pgfsys@useobject{currentmarker}{}%
\end{pgfscope}%
\end{pgfscope}%
\begin{pgfscope}%
\pgfpathrectangle{\pgfqpoint{1.432000in}{0.528000in}}{\pgfqpoint{3.696000in}{3.696000in}}%
\pgfusepath{clip}%
\pgfsetbuttcap%
\pgfsetroundjoin%
\definecolor{currentfill}{rgb}{0.000000,0.000000,0.000000}%
\pgfsetfillcolor{currentfill}%
\pgfsetlinewidth{0.501875pt}%
\definecolor{currentstroke}{rgb}{1.000000,1.000000,1.000000}%
\pgfsetstrokecolor{currentstroke}%
\pgfsetdash{}{0pt}%
\pgfsys@defobject{currentmarker}{\pgfqpoint{-0.018373in}{-0.018373in}}{\pgfqpoint{0.018373in}{0.018373in}}{%
\pgfpathmoveto{\pgfqpoint{0.000000in}{-0.018373in}}%
\pgfpathcurveto{\pgfqpoint{0.004873in}{-0.018373in}}{\pgfqpoint{0.009546in}{-0.016437in}}{\pgfqpoint{0.012992in}{-0.012992in}}%
\pgfpathcurveto{\pgfqpoint{0.016437in}{-0.009546in}}{\pgfqpoint{0.018373in}{-0.004873in}}{\pgfqpoint{0.018373in}{0.000000in}}%
\pgfpathcurveto{\pgfqpoint{0.018373in}{0.004873in}}{\pgfqpoint{0.016437in}{0.009546in}}{\pgfqpoint{0.012992in}{0.012992in}}%
\pgfpathcurveto{\pgfqpoint{0.009546in}{0.016437in}}{\pgfqpoint{0.004873in}{0.018373in}}{\pgfqpoint{0.000000in}{0.018373in}}%
\pgfpathcurveto{\pgfqpoint{-0.004873in}{0.018373in}}{\pgfqpoint{-0.009546in}{0.016437in}}{\pgfqpoint{-0.012992in}{0.012992in}}%
\pgfpathcurveto{\pgfqpoint{-0.016437in}{0.009546in}}{\pgfqpoint{-0.018373in}{0.004873in}}{\pgfqpoint{-0.018373in}{0.000000in}}%
\pgfpathcurveto{\pgfqpoint{-0.018373in}{-0.004873in}}{\pgfqpoint{-0.016437in}{-0.009546in}}{\pgfqpoint{-0.012992in}{-0.012992in}}%
\pgfpathcurveto{\pgfqpoint{-0.009546in}{-0.016437in}}{\pgfqpoint{-0.004873in}{-0.018373in}}{\pgfqpoint{0.000000in}{-0.018373in}}%
\pgfpathclose%
\pgfusepath{stroke,fill}%
}%
\begin{pgfscope}%
\pgfsys@transformshift{3.573224in}{2.250984in}%
\pgfsys@useobject{currentmarker}{}%
\end{pgfscope}%
\end{pgfscope}%
\begin{pgfscope}%
\pgfpathrectangle{\pgfqpoint{1.432000in}{0.528000in}}{\pgfqpoint{3.696000in}{3.696000in}}%
\pgfusepath{clip}%
\pgfsetbuttcap%
\pgfsetroundjoin%
\definecolor{currentfill}{rgb}{0.000000,0.000000,0.000000}%
\pgfsetfillcolor{currentfill}%
\pgfsetlinewidth{0.501875pt}%
\definecolor{currentstroke}{rgb}{1.000000,1.000000,1.000000}%
\pgfsetstrokecolor{currentstroke}%
\pgfsetdash{}{0pt}%
\pgfsys@defobject{currentmarker}{\pgfqpoint{-0.018373in}{-0.018373in}}{\pgfqpoint{0.018373in}{0.018373in}}{%
\pgfpathmoveto{\pgfqpoint{0.000000in}{-0.018373in}}%
\pgfpathcurveto{\pgfqpoint{0.004873in}{-0.018373in}}{\pgfqpoint{0.009546in}{-0.016437in}}{\pgfqpoint{0.012992in}{-0.012992in}}%
\pgfpathcurveto{\pgfqpoint{0.016437in}{-0.009546in}}{\pgfqpoint{0.018373in}{-0.004873in}}{\pgfqpoint{0.018373in}{0.000000in}}%
\pgfpathcurveto{\pgfqpoint{0.018373in}{0.004873in}}{\pgfqpoint{0.016437in}{0.009546in}}{\pgfqpoint{0.012992in}{0.012992in}}%
\pgfpathcurveto{\pgfqpoint{0.009546in}{0.016437in}}{\pgfqpoint{0.004873in}{0.018373in}}{\pgfqpoint{0.000000in}{0.018373in}}%
\pgfpathcurveto{\pgfqpoint{-0.004873in}{0.018373in}}{\pgfqpoint{-0.009546in}{0.016437in}}{\pgfqpoint{-0.012992in}{0.012992in}}%
\pgfpathcurveto{\pgfqpoint{-0.016437in}{0.009546in}}{\pgfqpoint{-0.018373in}{0.004873in}}{\pgfqpoint{-0.018373in}{0.000000in}}%
\pgfpathcurveto{\pgfqpoint{-0.018373in}{-0.004873in}}{\pgfqpoint{-0.016437in}{-0.009546in}}{\pgfqpoint{-0.012992in}{-0.012992in}}%
\pgfpathcurveto{\pgfqpoint{-0.009546in}{-0.016437in}}{\pgfqpoint{-0.004873in}{-0.018373in}}{\pgfqpoint{0.000000in}{-0.018373in}}%
\pgfpathclose%
\pgfusepath{stroke,fill}%
}%
\begin{pgfscope}%
\pgfsys@transformshift{3.370721in}{2.231546in}%
\pgfsys@useobject{currentmarker}{}%
\end{pgfscope}%
\end{pgfscope}%
\begin{pgfscope}%
\pgfpathrectangle{\pgfqpoint{1.432000in}{0.528000in}}{\pgfqpoint{3.696000in}{3.696000in}}%
\pgfusepath{clip}%
\pgfsetbuttcap%
\pgfsetroundjoin%
\definecolor{currentfill}{rgb}{0.000000,0.000000,0.000000}%
\pgfsetfillcolor{currentfill}%
\pgfsetlinewidth{0.501875pt}%
\definecolor{currentstroke}{rgb}{1.000000,1.000000,1.000000}%
\pgfsetstrokecolor{currentstroke}%
\pgfsetdash{}{0pt}%
\pgfsys@defobject{currentmarker}{\pgfqpoint{-0.018373in}{-0.018373in}}{\pgfqpoint{0.018373in}{0.018373in}}{%
\pgfpathmoveto{\pgfqpoint{0.000000in}{-0.018373in}}%
\pgfpathcurveto{\pgfqpoint{0.004873in}{-0.018373in}}{\pgfqpoint{0.009546in}{-0.016437in}}{\pgfqpoint{0.012992in}{-0.012992in}}%
\pgfpathcurveto{\pgfqpoint{0.016437in}{-0.009546in}}{\pgfqpoint{0.018373in}{-0.004873in}}{\pgfqpoint{0.018373in}{0.000000in}}%
\pgfpathcurveto{\pgfqpoint{0.018373in}{0.004873in}}{\pgfqpoint{0.016437in}{0.009546in}}{\pgfqpoint{0.012992in}{0.012992in}}%
\pgfpathcurveto{\pgfqpoint{0.009546in}{0.016437in}}{\pgfqpoint{0.004873in}{0.018373in}}{\pgfqpoint{0.000000in}{0.018373in}}%
\pgfpathcurveto{\pgfqpoint{-0.004873in}{0.018373in}}{\pgfqpoint{-0.009546in}{0.016437in}}{\pgfqpoint{-0.012992in}{0.012992in}}%
\pgfpathcurveto{\pgfqpoint{-0.016437in}{0.009546in}}{\pgfqpoint{-0.018373in}{0.004873in}}{\pgfqpoint{-0.018373in}{0.000000in}}%
\pgfpathcurveto{\pgfqpoint{-0.018373in}{-0.004873in}}{\pgfqpoint{-0.016437in}{-0.009546in}}{\pgfqpoint{-0.012992in}{-0.012992in}}%
\pgfpathcurveto{\pgfqpoint{-0.009546in}{-0.016437in}}{\pgfqpoint{-0.004873in}{-0.018373in}}{\pgfqpoint{0.000000in}{-0.018373in}}%
\pgfpathclose%
\pgfusepath{stroke,fill}%
}%
\begin{pgfscope}%
\pgfsys@transformshift{3.098016in}{2.689295in}%
\pgfsys@useobject{currentmarker}{}%
\end{pgfscope}%
\end{pgfscope}%
\begin{pgfscope}%
\pgfpathrectangle{\pgfqpoint{1.432000in}{0.528000in}}{\pgfqpoint{3.696000in}{3.696000in}}%
\pgfusepath{clip}%
\pgfsetbuttcap%
\pgfsetroundjoin%
\definecolor{currentfill}{rgb}{0.000000,0.000000,0.000000}%
\pgfsetfillcolor{currentfill}%
\pgfsetlinewidth{0.501875pt}%
\definecolor{currentstroke}{rgb}{1.000000,1.000000,1.000000}%
\pgfsetstrokecolor{currentstroke}%
\pgfsetdash{}{0pt}%
\pgfsys@defobject{currentmarker}{\pgfqpoint{-0.018373in}{-0.018373in}}{\pgfqpoint{0.018373in}{0.018373in}}{%
\pgfpathmoveto{\pgfqpoint{0.000000in}{-0.018373in}}%
\pgfpathcurveto{\pgfqpoint{0.004873in}{-0.018373in}}{\pgfqpoint{0.009546in}{-0.016437in}}{\pgfqpoint{0.012992in}{-0.012992in}}%
\pgfpathcurveto{\pgfqpoint{0.016437in}{-0.009546in}}{\pgfqpoint{0.018373in}{-0.004873in}}{\pgfqpoint{0.018373in}{0.000000in}}%
\pgfpathcurveto{\pgfqpoint{0.018373in}{0.004873in}}{\pgfqpoint{0.016437in}{0.009546in}}{\pgfqpoint{0.012992in}{0.012992in}}%
\pgfpathcurveto{\pgfqpoint{0.009546in}{0.016437in}}{\pgfqpoint{0.004873in}{0.018373in}}{\pgfqpoint{0.000000in}{0.018373in}}%
\pgfpathcurveto{\pgfqpoint{-0.004873in}{0.018373in}}{\pgfqpoint{-0.009546in}{0.016437in}}{\pgfqpoint{-0.012992in}{0.012992in}}%
\pgfpathcurveto{\pgfqpoint{-0.016437in}{0.009546in}}{\pgfqpoint{-0.018373in}{0.004873in}}{\pgfqpoint{-0.018373in}{0.000000in}}%
\pgfpathcurveto{\pgfqpoint{-0.018373in}{-0.004873in}}{\pgfqpoint{-0.016437in}{-0.009546in}}{\pgfqpoint{-0.012992in}{-0.012992in}}%
\pgfpathcurveto{\pgfqpoint{-0.009546in}{-0.016437in}}{\pgfqpoint{-0.004873in}{-0.018373in}}{\pgfqpoint{0.000000in}{-0.018373in}}%
\pgfpathclose%
\pgfusepath{stroke,fill}%
}%
\begin{pgfscope}%
\pgfsys@transformshift{3.366879in}{2.207284in}%
\pgfsys@useobject{currentmarker}{}%
\end{pgfscope}%
\end{pgfscope}%
\begin{pgfscope}%
\pgfpathrectangle{\pgfqpoint{1.432000in}{0.528000in}}{\pgfqpoint{3.696000in}{3.696000in}}%
\pgfusepath{clip}%
\pgfsetbuttcap%
\pgfsetroundjoin%
\definecolor{currentfill}{rgb}{1.000000,1.000000,1.000000}%
\pgfsetfillcolor{currentfill}%
\pgfsetlinewidth{0.501875pt}%
\definecolor{currentstroke}{rgb}{0.000000,0.000000,0.000000}%
\pgfsetstrokecolor{currentstroke}%
\pgfsetdash{}{0pt}%
\pgfsys@defobject{currentmarker}{\pgfqpoint{-0.018373in}{-0.018373in}}{\pgfqpoint{0.018373in}{0.018373in}}{%
\pgfpathmoveto{\pgfqpoint{0.000000in}{-0.018373in}}%
\pgfpathcurveto{\pgfqpoint{0.004873in}{-0.018373in}}{\pgfqpoint{0.009546in}{-0.016437in}}{\pgfqpoint{0.012992in}{-0.012992in}}%
\pgfpathcurveto{\pgfqpoint{0.016437in}{-0.009546in}}{\pgfqpoint{0.018373in}{-0.004873in}}{\pgfqpoint{0.018373in}{0.000000in}}%
\pgfpathcurveto{\pgfqpoint{0.018373in}{0.004873in}}{\pgfqpoint{0.016437in}{0.009546in}}{\pgfqpoint{0.012992in}{0.012992in}}%
\pgfpathcurveto{\pgfqpoint{0.009546in}{0.016437in}}{\pgfqpoint{0.004873in}{0.018373in}}{\pgfqpoint{0.000000in}{0.018373in}}%
\pgfpathcurveto{\pgfqpoint{-0.004873in}{0.018373in}}{\pgfqpoint{-0.009546in}{0.016437in}}{\pgfqpoint{-0.012992in}{0.012992in}}%
\pgfpathcurveto{\pgfqpoint{-0.016437in}{0.009546in}}{\pgfqpoint{-0.018373in}{0.004873in}}{\pgfqpoint{-0.018373in}{0.000000in}}%
\pgfpathcurveto{\pgfqpoint{-0.018373in}{-0.004873in}}{\pgfqpoint{-0.016437in}{-0.009546in}}{\pgfqpoint{-0.012992in}{-0.012992in}}%
\pgfpathcurveto{\pgfqpoint{-0.009546in}{-0.016437in}}{\pgfqpoint{-0.004873in}{-0.018373in}}{\pgfqpoint{0.000000in}{-0.018373in}}%
\pgfpathclose%
\pgfusepath{stroke,fill}%
}%
\begin{pgfscope}%
\pgfsys@transformshift{3.021506in}{2.604746in}%
\pgfsys@useobject{currentmarker}{}%
\end{pgfscope}%
\begin{pgfscope}%
\pgfsys@transformshift{3.009619in}{2.610965in}%
\pgfsys@useobject{currentmarker}{}%
\end{pgfscope}%
\begin{pgfscope}%
\pgfsys@transformshift{3.018887in}{2.642049in}%
\pgfsys@useobject{currentmarker}{}%
\end{pgfscope}%
\begin{pgfscope}%
\pgfsys@transformshift{3.069656in}{2.587416in}%
\pgfsys@useobject{currentmarker}{}%
\end{pgfscope}%
\begin{pgfscope}%
\pgfsys@transformshift{3.050345in}{2.668062in}%
\pgfsys@useobject{currentmarker}{}%
\end{pgfscope}%
\begin{pgfscope}%
\pgfsys@transformshift{3.081278in}{2.634503in}%
\pgfsys@useobject{currentmarker}{}%
\end{pgfscope}%
\begin{pgfscope}%
\pgfsys@transformshift{2.983145in}{2.595159in}%
\pgfsys@useobject{currentmarker}{}%
\end{pgfscope}%
\begin{pgfscope}%
\pgfsys@transformshift{3.074851in}{2.663159in}%
\pgfsys@useobject{currentmarker}{}%
\end{pgfscope}%
\begin{pgfscope}%
\pgfsys@transformshift{3.098658in}{2.597833in}%
\pgfsys@useobject{currentmarker}{}%
\end{pgfscope}%
\begin{pgfscope}%
\pgfsys@transformshift{2.984381in}{2.675848in}%
\pgfsys@useobject{currentmarker}{}%
\end{pgfscope}%
\begin{pgfscope}%
\pgfsys@transformshift{3.045474in}{2.693191in}%
\pgfsys@useobject{currentmarker}{}%
\end{pgfscope}%
\begin{pgfscope}%
\pgfsys@transformshift{3.022118in}{2.699820in}%
\pgfsys@useobject{currentmarker}{}%
\end{pgfscope}%
\begin{pgfscope}%
\pgfsys@transformshift{3.015609in}{2.702288in}%
\pgfsys@useobject{currentmarker}{}%
\end{pgfscope}%
\begin{pgfscope}%
\pgfsys@transformshift{2.951934in}{2.600127in}%
\pgfsys@useobject{currentmarker}{}%
\end{pgfscope}%
\begin{pgfscope}%
\pgfsys@transformshift{2.990125in}{2.545969in}%
\pgfsys@useobject{currentmarker}{}%
\end{pgfscope}%
\begin{pgfscope}%
\pgfsys@transformshift{3.010441in}{2.710598in}%
\pgfsys@useobject{currentmarker}{}%
\end{pgfscope}%
\begin{pgfscope}%
\pgfsys@transformshift{2.939820in}{2.612396in}%
\pgfsys@useobject{currentmarker}{}%
\end{pgfscope}%
\begin{pgfscope}%
\pgfsys@transformshift{3.441894in}{2.170245in}%
\pgfsys@useobject{currentmarker}{}%
\end{pgfscope}%
\begin{pgfscope}%
\pgfsys@transformshift{2.942339in}{2.679866in}%
\pgfsys@useobject{currentmarker}{}%
\end{pgfscope}%
\begin{pgfscope}%
\pgfsys@transformshift{3.428978in}{2.136618in}%
\pgfsys@useobject{currentmarker}{}%
\end{pgfscope}%
\begin{pgfscope}%
\pgfsys@transformshift{3.038084in}{2.730931in}%
\pgfsys@useobject{currentmarker}{}%
\end{pgfscope}%
\begin{pgfscope}%
\pgfsys@transformshift{2.918428in}{2.631070in}%
\pgfsys@useobject{currentmarker}{}%
\end{pgfscope}%
\begin{pgfscope}%
\pgfsys@transformshift{2.998621in}{2.732027in}%
\pgfsys@useobject{currentmarker}{}%
\end{pgfscope}%
\begin{pgfscope}%
\pgfsys@transformshift{2.923882in}{2.582483in}%
\pgfsys@useobject{currentmarker}{}%
\end{pgfscope}%
\begin{pgfscope}%
\pgfsys@transformshift{3.455832in}{2.210247in}%
\pgfsys@useobject{currentmarker}{}%
\end{pgfscope}%
\begin{pgfscope}%
\pgfsys@transformshift{3.059775in}{2.741471in}%
\pgfsys@useobject{currentmarker}{}%
\end{pgfscope}%
\begin{pgfscope}%
\pgfsys@transformshift{3.421590in}{2.182776in}%
\pgfsys@useobject{currentmarker}{}%
\end{pgfscope}%
\begin{pgfscope}%
\pgfsys@transformshift{3.408426in}{2.143282in}%
\pgfsys@useobject{currentmarker}{}%
\end{pgfscope}%
\begin{pgfscope}%
\pgfsys@transformshift{3.409622in}{2.102582in}%
\pgfsys@useobject{currentmarker}{}%
\end{pgfscope}%
\begin{pgfscope}%
\pgfsys@transformshift{3.429322in}{2.203323in}%
\pgfsys@useobject{currentmarker}{}%
\end{pgfscope}%
\begin{pgfscope}%
\pgfsys@transformshift{3.416081in}{2.184684in}%
\pgfsys@useobject{currentmarker}{}%
\end{pgfscope}%
\begin{pgfscope}%
\pgfsys@transformshift{3.416886in}{2.192563in}%
\pgfsys@useobject{currentmarker}{}%
\end{pgfscope}%
\begin{pgfscope}%
\pgfsys@transformshift{3.499057in}{2.254026in}%
\pgfsys@useobject{currentmarker}{}%
\end{pgfscope}%
\begin{pgfscope}%
\pgfsys@transformshift{3.457462in}{2.236776in}%
\pgfsys@useobject{currentmarker}{}%
\end{pgfscope}%
\begin{pgfscope}%
\pgfsys@transformshift{2.972485in}{2.749780in}%
\pgfsys@useobject{currentmarker}{}%
\end{pgfscope}%
\begin{pgfscope}%
\pgfsys@transformshift{3.401035in}{2.174406in}%
\pgfsys@useobject{currentmarker}{}%
\end{pgfscope}%
\begin{pgfscope}%
\pgfsys@transformshift{3.393417in}{2.164338in}%
\pgfsys@useobject{currentmarker}{}%
\end{pgfscope}%
\begin{pgfscope}%
\pgfsys@transformshift{3.387745in}{2.132571in}%
\pgfsys@useobject{currentmarker}{}%
\end{pgfscope}%
\begin{pgfscope}%
\pgfsys@transformshift{3.402907in}{2.193613in}%
\pgfsys@useobject{currentmarker}{}%
\end{pgfscope}%
\begin{pgfscope}%
\pgfsys@transformshift{3.395123in}{2.179427in}%
\pgfsys@useobject{currentmarker}{}%
\end{pgfscope}%
\begin{pgfscope}%
\pgfsys@transformshift{3.422308in}{2.225725in}%
\pgfsys@useobject{currentmarker}{}%
\end{pgfscope}%
\begin{pgfscope}%
\pgfsys@transformshift{3.446402in}{2.250169in}%
\pgfsys@useobject{currentmarker}{}%
\end{pgfscope}%
\begin{pgfscope}%
\pgfsys@transformshift{2.872313in}{2.599845in}%
\pgfsys@useobject{currentmarker}{}%
\end{pgfscope}%
\begin{pgfscope}%
\pgfsys@transformshift{3.650728in}{2.769778in}%
\pgfsys@useobject{currentmarker}{}%
\end{pgfscope}%
\begin{pgfscope}%
\pgfsys@transformshift{3.385703in}{2.182308in}%
\pgfsys@useobject{currentmarker}{}%
\end{pgfscope}%
\begin{pgfscope}%
\pgfsys@transformshift{3.552798in}{2.280984in}%
\pgfsys@useobject{currentmarker}{}%
\end{pgfscope}%
\begin{pgfscope}%
\pgfsys@transformshift{3.150187in}{2.724315in}%
\pgfsys@useobject{currentmarker}{}%
\end{pgfscope}%
\begin{pgfscope}%
\pgfsys@transformshift{3.140142in}{2.739202in}%
\pgfsys@useobject{currentmarker}{}%
\end{pgfscope}%
\begin{pgfscope}%
\pgfsys@transformshift{3.372249in}{2.121401in}%
\pgfsys@useobject{currentmarker}{}%
\end{pgfscope}%
\begin{pgfscope}%
\pgfsys@transformshift{2.869300in}{2.703402in}%
\pgfsys@useobject{currentmarker}{}%
\end{pgfscope}%
\end{pgfscope}%
\begin{pgfscope}%
\pgfpathrectangle{\pgfqpoint{1.432000in}{0.528000in}}{\pgfqpoint{3.696000in}{3.696000in}}%
\pgfusepath{clip}%
\pgfsetbuttcap%
\pgfsetroundjoin%
\definecolor{currentfill}{rgb}{1.000000,0.000000,0.000000}%
\pgfsetfillcolor{currentfill}%
\pgfsetlinewidth{0.501875pt}%
\definecolor{currentstroke}{rgb}{1.000000,1.000000,1.000000}%
\pgfsetstrokecolor{currentstroke}%
\pgfsetdash{}{0pt}%
\pgfsys@defobject{currentmarker}{\pgfqpoint{-0.018373in}{-0.018373in}}{\pgfqpoint{0.018373in}{0.018373in}}{%
\pgfpathmoveto{\pgfqpoint{0.000000in}{-0.018373in}}%
\pgfpathcurveto{\pgfqpoint{0.004873in}{-0.018373in}}{\pgfqpoint{0.009546in}{-0.016437in}}{\pgfqpoint{0.012992in}{-0.012992in}}%
\pgfpathcurveto{\pgfqpoint{0.016437in}{-0.009546in}}{\pgfqpoint{0.018373in}{-0.004873in}}{\pgfqpoint{0.018373in}{0.000000in}}%
\pgfpathcurveto{\pgfqpoint{0.018373in}{0.004873in}}{\pgfqpoint{0.016437in}{0.009546in}}{\pgfqpoint{0.012992in}{0.012992in}}%
\pgfpathcurveto{\pgfqpoint{0.009546in}{0.016437in}}{\pgfqpoint{0.004873in}{0.018373in}}{\pgfqpoint{0.000000in}{0.018373in}}%
\pgfpathcurveto{\pgfqpoint{-0.004873in}{0.018373in}}{\pgfqpoint{-0.009546in}{0.016437in}}{\pgfqpoint{-0.012992in}{0.012992in}}%
\pgfpathcurveto{\pgfqpoint{-0.016437in}{0.009546in}}{\pgfqpoint{-0.018373in}{0.004873in}}{\pgfqpoint{-0.018373in}{0.000000in}}%
\pgfpathcurveto{\pgfqpoint{-0.018373in}{-0.004873in}}{\pgfqpoint{-0.016437in}{-0.009546in}}{\pgfqpoint{-0.012992in}{-0.012992in}}%
\pgfpathcurveto{\pgfqpoint{-0.009546in}{-0.016437in}}{\pgfqpoint{-0.004873in}{-0.018373in}}{\pgfqpoint{0.000000in}{-0.018373in}}%
\pgfpathclose%
\pgfusepath{stroke,fill}%
}%
\begin{pgfscope}%
\pgfsys@transformshift{3.050309in}{2.661601in}%
\pgfsys@useobject{currentmarker}{}%
\end{pgfscope}%
\begin{pgfscope}%
\pgfsys@transformshift{3.098016in}{2.689295in}%
\pgfsys@useobject{currentmarker}{}%
\end{pgfscope}%
\begin{pgfscope}%
\pgfsys@transformshift{3.413110in}{2.166836in}%
\pgfsys@useobject{currentmarker}{}%
\end{pgfscope}%
\begin{pgfscope}%
\pgfsys@transformshift{3.128731in}{2.707201in}%
\pgfsys@useobject{currentmarker}{}%
\end{pgfscope}%
\begin{pgfscope}%
\pgfsys@transformshift{3.573224in}{2.250984in}%
\pgfsys@useobject{currentmarker}{}%
\end{pgfscope}%
\end{pgfscope}%
\begin{pgfscope}%
\pgfsetbuttcap%
\pgfsetroundjoin%
\definecolor{currentfill}{rgb}{0.000000,0.000000,0.000000}%
\pgfsetfillcolor{currentfill}%
\pgfsetlinewidth{0.803000pt}%
\definecolor{currentstroke}{rgb}{0.000000,0.000000,0.000000}%
\pgfsetstrokecolor{currentstroke}%
\pgfsetdash{}{0pt}%
\pgfsys@defobject{currentmarker}{\pgfqpoint{0.000000in}{-0.048611in}}{\pgfqpoint{0.000000in}{0.000000in}}{%
\pgfpathmoveto{\pgfqpoint{0.000000in}{0.000000in}}%
\pgfpathlineto{\pgfqpoint{0.000000in}{-0.048611in}}%
\pgfusepath{stroke,fill}%
}%
\begin{pgfscope}%
\pgfsys@transformshift{1.432000in}{0.528000in}%
\pgfsys@useobject{currentmarker}{}%
\end{pgfscope}%
\end{pgfscope}%
\begin{pgfscope}%
\definecolor{textcolor}{rgb}{0.000000,0.000000,0.000000}%
\pgfsetstrokecolor{textcolor}%
\pgfsetfillcolor{textcolor}%
\pgftext[x=1.432000in,y=0.430778in,,top]{\color{textcolor}\rmfamily\fontsize{10.000000}{12.000000}\selectfont \(\displaystyle -15\)}%
\end{pgfscope}%
\begin{pgfscope}%
\pgfsetbuttcap%
\pgfsetroundjoin%
\definecolor{currentfill}{rgb}{0.000000,0.000000,0.000000}%
\pgfsetfillcolor{currentfill}%
\pgfsetlinewidth{0.803000pt}%
\definecolor{currentstroke}{rgb}{0.000000,0.000000,0.000000}%
\pgfsetstrokecolor{currentstroke}%
\pgfsetdash{}{0pt}%
\pgfsys@defobject{currentmarker}{\pgfqpoint{0.000000in}{-0.048611in}}{\pgfqpoint{0.000000in}{0.000000in}}{%
\pgfpathmoveto{\pgfqpoint{0.000000in}{0.000000in}}%
\pgfpathlineto{\pgfqpoint{0.000000in}{-0.048611in}}%
\pgfusepath{stroke,fill}%
}%
\begin{pgfscope}%
\pgfsys@transformshift{2.048000in}{0.528000in}%
\pgfsys@useobject{currentmarker}{}%
\end{pgfscope}%
\end{pgfscope}%
\begin{pgfscope}%
\definecolor{textcolor}{rgb}{0.000000,0.000000,0.000000}%
\pgfsetstrokecolor{textcolor}%
\pgfsetfillcolor{textcolor}%
\pgftext[x=2.048000in,y=0.430778in,,top]{\color{textcolor}\rmfamily\fontsize{10.000000}{12.000000}\selectfont \(\displaystyle -10\)}%
\end{pgfscope}%
\begin{pgfscope}%
\pgfsetbuttcap%
\pgfsetroundjoin%
\definecolor{currentfill}{rgb}{0.000000,0.000000,0.000000}%
\pgfsetfillcolor{currentfill}%
\pgfsetlinewidth{0.803000pt}%
\definecolor{currentstroke}{rgb}{0.000000,0.000000,0.000000}%
\pgfsetstrokecolor{currentstroke}%
\pgfsetdash{}{0pt}%
\pgfsys@defobject{currentmarker}{\pgfqpoint{0.000000in}{-0.048611in}}{\pgfqpoint{0.000000in}{0.000000in}}{%
\pgfpathmoveto{\pgfqpoint{0.000000in}{0.000000in}}%
\pgfpathlineto{\pgfqpoint{0.000000in}{-0.048611in}}%
\pgfusepath{stroke,fill}%
}%
\begin{pgfscope}%
\pgfsys@transformshift{2.664000in}{0.528000in}%
\pgfsys@useobject{currentmarker}{}%
\end{pgfscope}%
\end{pgfscope}%
\begin{pgfscope}%
\definecolor{textcolor}{rgb}{0.000000,0.000000,0.000000}%
\pgfsetstrokecolor{textcolor}%
\pgfsetfillcolor{textcolor}%
\pgftext[x=2.664000in,y=0.430778in,,top]{\color{textcolor}\rmfamily\fontsize{10.000000}{12.000000}\selectfont \(\displaystyle -5\)}%
\end{pgfscope}%
\begin{pgfscope}%
\pgfsetbuttcap%
\pgfsetroundjoin%
\definecolor{currentfill}{rgb}{0.000000,0.000000,0.000000}%
\pgfsetfillcolor{currentfill}%
\pgfsetlinewidth{0.803000pt}%
\definecolor{currentstroke}{rgb}{0.000000,0.000000,0.000000}%
\pgfsetstrokecolor{currentstroke}%
\pgfsetdash{}{0pt}%
\pgfsys@defobject{currentmarker}{\pgfqpoint{0.000000in}{-0.048611in}}{\pgfqpoint{0.000000in}{0.000000in}}{%
\pgfpathmoveto{\pgfqpoint{0.000000in}{0.000000in}}%
\pgfpathlineto{\pgfqpoint{0.000000in}{-0.048611in}}%
\pgfusepath{stroke,fill}%
}%
\begin{pgfscope}%
\pgfsys@transformshift{3.280000in}{0.528000in}%
\pgfsys@useobject{currentmarker}{}%
\end{pgfscope}%
\end{pgfscope}%
\begin{pgfscope}%
\definecolor{textcolor}{rgb}{0.000000,0.000000,0.000000}%
\pgfsetstrokecolor{textcolor}%
\pgfsetfillcolor{textcolor}%
\pgftext[x=3.280000in,y=0.430778in,,top]{\color{textcolor}\rmfamily\fontsize{10.000000}{12.000000}\selectfont \(\displaystyle 0\)}%
\end{pgfscope}%
\begin{pgfscope}%
\pgfsetbuttcap%
\pgfsetroundjoin%
\definecolor{currentfill}{rgb}{0.000000,0.000000,0.000000}%
\pgfsetfillcolor{currentfill}%
\pgfsetlinewidth{0.803000pt}%
\definecolor{currentstroke}{rgb}{0.000000,0.000000,0.000000}%
\pgfsetstrokecolor{currentstroke}%
\pgfsetdash{}{0pt}%
\pgfsys@defobject{currentmarker}{\pgfqpoint{0.000000in}{-0.048611in}}{\pgfqpoint{0.000000in}{0.000000in}}{%
\pgfpathmoveto{\pgfqpoint{0.000000in}{0.000000in}}%
\pgfpathlineto{\pgfqpoint{0.000000in}{-0.048611in}}%
\pgfusepath{stroke,fill}%
}%
\begin{pgfscope}%
\pgfsys@transformshift{3.896000in}{0.528000in}%
\pgfsys@useobject{currentmarker}{}%
\end{pgfscope}%
\end{pgfscope}%
\begin{pgfscope}%
\definecolor{textcolor}{rgb}{0.000000,0.000000,0.000000}%
\pgfsetstrokecolor{textcolor}%
\pgfsetfillcolor{textcolor}%
\pgftext[x=3.896000in,y=0.430778in,,top]{\color{textcolor}\rmfamily\fontsize{10.000000}{12.000000}\selectfont \(\displaystyle 5\)}%
\end{pgfscope}%
\begin{pgfscope}%
\pgfsetbuttcap%
\pgfsetroundjoin%
\definecolor{currentfill}{rgb}{0.000000,0.000000,0.000000}%
\pgfsetfillcolor{currentfill}%
\pgfsetlinewidth{0.803000pt}%
\definecolor{currentstroke}{rgb}{0.000000,0.000000,0.000000}%
\pgfsetstrokecolor{currentstroke}%
\pgfsetdash{}{0pt}%
\pgfsys@defobject{currentmarker}{\pgfqpoint{0.000000in}{-0.048611in}}{\pgfqpoint{0.000000in}{0.000000in}}{%
\pgfpathmoveto{\pgfqpoint{0.000000in}{0.000000in}}%
\pgfpathlineto{\pgfqpoint{0.000000in}{-0.048611in}}%
\pgfusepath{stroke,fill}%
}%
\begin{pgfscope}%
\pgfsys@transformshift{4.512000in}{0.528000in}%
\pgfsys@useobject{currentmarker}{}%
\end{pgfscope}%
\end{pgfscope}%
\begin{pgfscope}%
\definecolor{textcolor}{rgb}{0.000000,0.000000,0.000000}%
\pgfsetstrokecolor{textcolor}%
\pgfsetfillcolor{textcolor}%
\pgftext[x=4.512000in,y=0.430778in,,top]{\color{textcolor}\rmfamily\fontsize{10.000000}{12.000000}\selectfont \(\displaystyle 10\)}%
\end{pgfscope}%
\begin{pgfscope}%
\pgfsetbuttcap%
\pgfsetroundjoin%
\definecolor{currentfill}{rgb}{0.000000,0.000000,0.000000}%
\pgfsetfillcolor{currentfill}%
\pgfsetlinewidth{0.803000pt}%
\definecolor{currentstroke}{rgb}{0.000000,0.000000,0.000000}%
\pgfsetstrokecolor{currentstroke}%
\pgfsetdash{}{0pt}%
\pgfsys@defobject{currentmarker}{\pgfqpoint{0.000000in}{-0.048611in}}{\pgfqpoint{0.000000in}{0.000000in}}{%
\pgfpathmoveto{\pgfqpoint{0.000000in}{0.000000in}}%
\pgfpathlineto{\pgfqpoint{0.000000in}{-0.048611in}}%
\pgfusepath{stroke,fill}%
}%
\begin{pgfscope}%
\pgfsys@transformshift{5.128000in}{0.528000in}%
\pgfsys@useobject{currentmarker}{}%
\end{pgfscope}%
\end{pgfscope}%
\begin{pgfscope}%
\definecolor{textcolor}{rgb}{0.000000,0.000000,0.000000}%
\pgfsetstrokecolor{textcolor}%
\pgfsetfillcolor{textcolor}%
\pgftext[x=5.128000in,y=0.430778in,,top]{\color{textcolor}\rmfamily\fontsize{10.000000}{12.000000}\selectfont \(\displaystyle 15\)}%
\end{pgfscope}%
\begin{pgfscope}%
\pgfsetbuttcap%
\pgfsetroundjoin%
\definecolor{currentfill}{rgb}{0.000000,0.000000,0.000000}%
\pgfsetfillcolor{currentfill}%
\pgfsetlinewidth{0.803000pt}%
\definecolor{currentstroke}{rgb}{0.000000,0.000000,0.000000}%
\pgfsetstrokecolor{currentstroke}%
\pgfsetdash{}{0pt}%
\pgfsys@defobject{currentmarker}{\pgfqpoint{-0.048611in}{0.000000in}}{\pgfqpoint{0.000000in}{0.000000in}}{%
\pgfpathmoveto{\pgfqpoint{0.000000in}{0.000000in}}%
\pgfpathlineto{\pgfqpoint{-0.048611in}{0.000000in}}%
\pgfusepath{stroke,fill}%
}%
\begin{pgfscope}%
\pgfsys@transformshift{1.432000in}{0.528000in}%
\pgfsys@useobject{currentmarker}{}%
\end{pgfscope}%
\end{pgfscope}%
\begin{pgfscope}%
\definecolor{textcolor}{rgb}{0.000000,0.000000,0.000000}%
\pgfsetstrokecolor{textcolor}%
\pgfsetfillcolor{textcolor}%
\pgftext[x=1.087863in,y=0.475238in,left,base]{\color{textcolor}\rmfamily\fontsize{10.000000}{12.000000}\selectfont \(\displaystyle -15\)}%
\end{pgfscope}%
\begin{pgfscope}%
\pgfsetbuttcap%
\pgfsetroundjoin%
\definecolor{currentfill}{rgb}{0.000000,0.000000,0.000000}%
\pgfsetfillcolor{currentfill}%
\pgfsetlinewidth{0.803000pt}%
\definecolor{currentstroke}{rgb}{0.000000,0.000000,0.000000}%
\pgfsetstrokecolor{currentstroke}%
\pgfsetdash{}{0pt}%
\pgfsys@defobject{currentmarker}{\pgfqpoint{-0.048611in}{0.000000in}}{\pgfqpoint{0.000000in}{0.000000in}}{%
\pgfpathmoveto{\pgfqpoint{0.000000in}{0.000000in}}%
\pgfpathlineto{\pgfqpoint{-0.048611in}{0.000000in}}%
\pgfusepath{stroke,fill}%
}%
\begin{pgfscope}%
\pgfsys@transformshift{1.432000in}{1.144000in}%
\pgfsys@useobject{currentmarker}{}%
\end{pgfscope}%
\end{pgfscope}%
\begin{pgfscope}%
\definecolor{textcolor}{rgb}{0.000000,0.000000,0.000000}%
\pgfsetstrokecolor{textcolor}%
\pgfsetfillcolor{textcolor}%
\pgftext[x=1.087863in,y=1.091238in,left,base]{\color{textcolor}\rmfamily\fontsize{10.000000}{12.000000}\selectfont \(\displaystyle -10\)}%
\end{pgfscope}%
\begin{pgfscope}%
\pgfsetbuttcap%
\pgfsetroundjoin%
\definecolor{currentfill}{rgb}{0.000000,0.000000,0.000000}%
\pgfsetfillcolor{currentfill}%
\pgfsetlinewidth{0.803000pt}%
\definecolor{currentstroke}{rgb}{0.000000,0.000000,0.000000}%
\pgfsetstrokecolor{currentstroke}%
\pgfsetdash{}{0pt}%
\pgfsys@defobject{currentmarker}{\pgfqpoint{-0.048611in}{0.000000in}}{\pgfqpoint{0.000000in}{0.000000in}}{%
\pgfpathmoveto{\pgfqpoint{0.000000in}{0.000000in}}%
\pgfpathlineto{\pgfqpoint{-0.048611in}{0.000000in}}%
\pgfusepath{stroke,fill}%
}%
\begin{pgfscope}%
\pgfsys@transformshift{1.432000in}{1.760000in}%
\pgfsys@useobject{currentmarker}{}%
\end{pgfscope}%
\end{pgfscope}%
\begin{pgfscope}%
\definecolor{textcolor}{rgb}{0.000000,0.000000,0.000000}%
\pgfsetstrokecolor{textcolor}%
\pgfsetfillcolor{textcolor}%
\pgftext[x=1.157308in,y=1.707238in,left,base]{\color{textcolor}\rmfamily\fontsize{10.000000}{12.000000}\selectfont \(\displaystyle -5\)}%
\end{pgfscope}%
\begin{pgfscope}%
\pgfsetbuttcap%
\pgfsetroundjoin%
\definecolor{currentfill}{rgb}{0.000000,0.000000,0.000000}%
\pgfsetfillcolor{currentfill}%
\pgfsetlinewidth{0.803000pt}%
\definecolor{currentstroke}{rgb}{0.000000,0.000000,0.000000}%
\pgfsetstrokecolor{currentstroke}%
\pgfsetdash{}{0pt}%
\pgfsys@defobject{currentmarker}{\pgfqpoint{-0.048611in}{0.000000in}}{\pgfqpoint{0.000000in}{0.000000in}}{%
\pgfpathmoveto{\pgfqpoint{0.000000in}{0.000000in}}%
\pgfpathlineto{\pgfqpoint{-0.048611in}{0.000000in}}%
\pgfusepath{stroke,fill}%
}%
\begin{pgfscope}%
\pgfsys@transformshift{1.432000in}{2.376000in}%
\pgfsys@useobject{currentmarker}{}%
\end{pgfscope}%
\end{pgfscope}%
\begin{pgfscope}%
\definecolor{textcolor}{rgb}{0.000000,0.000000,0.000000}%
\pgfsetstrokecolor{textcolor}%
\pgfsetfillcolor{textcolor}%
\pgftext[x=1.265333in,y=2.323238in,left,base]{\color{textcolor}\rmfamily\fontsize{10.000000}{12.000000}\selectfont \(\displaystyle 0\)}%
\end{pgfscope}%
\begin{pgfscope}%
\pgfsetbuttcap%
\pgfsetroundjoin%
\definecolor{currentfill}{rgb}{0.000000,0.000000,0.000000}%
\pgfsetfillcolor{currentfill}%
\pgfsetlinewidth{0.803000pt}%
\definecolor{currentstroke}{rgb}{0.000000,0.000000,0.000000}%
\pgfsetstrokecolor{currentstroke}%
\pgfsetdash{}{0pt}%
\pgfsys@defobject{currentmarker}{\pgfqpoint{-0.048611in}{0.000000in}}{\pgfqpoint{0.000000in}{0.000000in}}{%
\pgfpathmoveto{\pgfqpoint{0.000000in}{0.000000in}}%
\pgfpathlineto{\pgfqpoint{-0.048611in}{0.000000in}}%
\pgfusepath{stroke,fill}%
}%
\begin{pgfscope}%
\pgfsys@transformshift{1.432000in}{2.992000in}%
\pgfsys@useobject{currentmarker}{}%
\end{pgfscope}%
\end{pgfscope}%
\begin{pgfscope}%
\definecolor{textcolor}{rgb}{0.000000,0.000000,0.000000}%
\pgfsetstrokecolor{textcolor}%
\pgfsetfillcolor{textcolor}%
\pgftext[x=1.265333in,y=2.939238in,left,base]{\color{textcolor}\rmfamily\fontsize{10.000000}{12.000000}\selectfont \(\displaystyle 5\)}%
\end{pgfscope}%
\begin{pgfscope}%
\pgfsetbuttcap%
\pgfsetroundjoin%
\definecolor{currentfill}{rgb}{0.000000,0.000000,0.000000}%
\pgfsetfillcolor{currentfill}%
\pgfsetlinewidth{0.803000pt}%
\definecolor{currentstroke}{rgb}{0.000000,0.000000,0.000000}%
\pgfsetstrokecolor{currentstroke}%
\pgfsetdash{}{0pt}%
\pgfsys@defobject{currentmarker}{\pgfqpoint{-0.048611in}{0.000000in}}{\pgfqpoint{0.000000in}{0.000000in}}{%
\pgfpathmoveto{\pgfqpoint{0.000000in}{0.000000in}}%
\pgfpathlineto{\pgfqpoint{-0.048611in}{0.000000in}}%
\pgfusepath{stroke,fill}%
}%
\begin{pgfscope}%
\pgfsys@transformshift{1.432000in}{3.608000in}%
\pgfsys@useobject{currentmarker}{}%
\end{pgfscope}%
\end{pgfscope}%
\begin{pgfscope}%
\definecolor{textcolor}{rgb}{0.000000,0.000000,0.000000}%
\pgfsetstrokecolor{textcolor}%
\pgfsetfillcolor{textcolor}%
\pgftext[x=1.195888in,y=3.555238in,left,base]{\color{textcolor}\rmfamily\fontsize{10.000000}{12.000000}\selectfont \(\displaystyle 10\)}%
\end{pgfscope}%
\begin{pgfscope}%
\pgfsetbuttcap%
\pgfsetroundjoin%
\definecolor{currentfill}{rgb}{0.000000,0.000000,0.000000}%
\pgfsetfillcolor{currentfill}%
\pgfsetlinewidth{0.803000pt}%
\definecolor{currentstroke}{rgb}{0.000000,0.000000,0.000000}%
\pgfsetstrokecolor{currentstroke}%
\pgfsetdash{}{0pt}%
\pgfsys@defobject{currentmarker}{\pgfqpoint{-0.048611in}{0.000000in}}{\pgfqpoint{0.000000in}{0.000000in}}{%
\pgfpathmoveto{\pgfqpoint{0.000000in}{0.000000in}}%
\pgfpathlineto{\pgfqpoint{-0.048611in}{0.000000in}}%
\pgfusepath{stroke,fill}%
}%
\begin{pgfscope}%
\pgfsys@transformshift{1.432000in}{4.224000in}%
\pgfsys@useobject{currentmarker}{}%
\end{pgfscope}%
\end{pgfscope}%
\begin{pgfscope}%
\definecolor{textcolor}{rgb}{0.000000,0.000000,0.000000}%
\pgfsetstrokecolor{textcolor}%
\pgfsetfillcolor{textcolor}%
\pgftext[x=1.195888in,y=4.171238in,left,base]{\color{textcolor}\rmfamily\fontsize{10.000000}{12.000000}\selectfont \(\displaystyle 15\)}%
\end{pgfscope}%
\begin{pgfscope}%
\pgfpathrectangle{\pgfqpoint{1.432000in}{0.528000in}}{\pgfqpoint{3.696000in}{3.696000in}}%
\pgfusepath{clip}%
\pgfsetbuttcap%
\pgfsetroundjoin%
\pgfsetlinewidth{1.505625pt}%
\definecolor{currentstroke}{rgb}{0.371035,0.000000,0.000000}%
\pgfsetstrokecolor{currentstroke}%
\pgfsetstrokeopacity{0.300000}%
\pgfsetdash{}{0pt}%
\pgfpathmoveto{\pgfqpoint{3.522667in}{2.082218in}}%
\pgfpathlineto{\pgfqpoint{3.560000in}{2.078844in}}%
\pgfpathlineto{\pgfqpoint{3.597333in}{2.085200in}}%
\pgfpathlineto{\pgfqpoint{3.624049in}{2.096000in}}%
\pgfpathlineto{\pgfqpoint{3.634667in}{2.100433in}}%
\pgfpathlineto{\pgfqpoint{3.671883in}{2.133333in}}%
\pgfpathlineto{\pgfqpoint{3.672000in}{2.133567in}}%
\pgfpathlineto{\pgfqpoint{3.686018in}{2.170667in}}%
\pgfpathlineto{\pgfqpoint{3.689560in}{2.208000in}}%
\pgfpathlineto{\pgfqpoint{3.690348in}{2.245333in}}%
\pgfpathlineto{\pgfqpoint{3.692129in}{2.282667in}}%
\pgfpathlineto{\pgfqpoint{3.696198in}{2.320000in}}%
\pgfpathlineto{\pgfqpoint{3.701769in}{2.357333in}}%
\pgfpathlineto{\pgfqpoint{3.707524in}{2.394667in}}%
\pgfpathlineto{\pgfqpoint{3.709333in}{2.406641in}}%
\pgfpathlineto{\pgfqpoint{3.713921in}{2.432000in}}%
\pgfpathlineto{\pgfqpoint{3.719999in}{2.469333in}}%
\pgfpathlineto{\pgfqpoint{3.725490in}{2.506667in}}%
\pgfpathlineto{\pgfqpoint{3.730690in}{2.544000in}}%
\pgfpathlineto{\pgfqpoint{3.735773in}{2.581333in}}%
\pgfpathlineto{\pgfqpoint{3.740814in}{2.618667in}}%
\pgfpathlineto{\pgfqpoint{3.745840in}{2.656000in}}%
\pgfpathlineto{\pgfqpoint{3.746667in}{2.662002in}}%
\pgfpathlineto{\pgfqpoint{3.752037in}{2.693333in}}%
\pgfpathlineto{\pgfqpoint{3.757852in}{2.730667in}}%
\pgfpathlineto{\pgfqpoint{3.763076in}{2.768000in}}%
\pgfpathlineto{\pgfqpoint{3.767788in}{2.805333in}}%
\pgfpathlineto{\pgfqpoint{3.772042in}{2.842667in}}%
\pgfpathlineto{\pgfqpoint{3.775851in}{2.880000in}}%
\pgfpathlineto{\pgfqpoint{3.779198in}{2.917333in}}%
\pgfpathlineto{\pgfqpoint{3.782037in}{2.954667in}}%
\pgfpathlineto{\pgfqpoint{3.784000in}{2.987639in}}%
\pgfpathlineto{\pgfqpoint{3.784312in}{2.992000in}}%
\pgfpathlineto{\pgfqpoint{3.785834in}{3.029333in}}%
\pgfpathlineto{\pgfqpoint{3.786119in}{3.066667in}}%
\pgfpathlineto{\pgfqpoint{3.784983in}{3.104000in}}%
\pgfpathlineto{\pgfqpoint{3.784000in}{3.117526in}}%
\pgfpathlineto{\pgfqpoint{3.781644in}{3.141333in}}%
\pgfpathlineto{\pgfqpoint{3.774243in}{3.178667in}}%
\pgfpathlineto{\pgfqpoint{3.759969in}{3.216000in}}%
\pgfpathlineto{\pgfqpoint{3.746667in}{3.237188in}}%
\pgfpathlineto{\pgfqpoint{3.709333in}{3.222958in}}%
\pgfpathlineto{\pgfqpoint{3.705102in}{3.216000in}}%
\pgfpathlineto{\pgfqpoint{3.681860in}{3.178667in}}%
\pgfpathlineto{\pgfqpoint{3.672000in}{3.163700in}}%
\pgfpathlineto{\pgfqpoint{3.662430in}{3.141333in}}%
\pgfpathlineto{\pgfqpoint{3.647558in}{3.104000in}}%
\pgfpathlineto{\pgfqpoint{3.634667in}{3.069706in}}%
\pgfpathlineto{\pgfqpoint{3.633597in}{3.066667in}}%
\pgfpathlineto{\pgfqpoint{3.621083in}{3.029333in}}%
\pgfpathlineto{\pgfqpoint{3.610052in}{2.992000in}}%
\pgfpathlineto{\pgfqpoint{3.599852in}{2.954667in}}%
\pgfpathlineto{\pgfqpoint{3.597333in}{2.945480in}}%
\pgfpathlineto{\pgfqpoint{3.588551in}{2.917333in}}%
\pgfpathlineto{\pgfqpoint{3.577553in}{2.880000in}}%
\pgfpathlineto{\pgfqpoint{3.566544in}{2.842667in}}%
\pgfpathlineto{\pgfqpoint{3.560000in}{2.822500in}}%
\pgfpathlineto{\pgfqpoint{3.551395in}{2.805333in}}%
\pgfpathlineto{\pgfqpoint{3.530114in}{2.768000in}}%
\pgfpathlineto{\pgfqpoint{3.522667in}{2.756459in}}%
\pgfpathlineto{\pgfqpoint{3.485333in}{2.740950in}}%
\pgfpathlineto{\pgfqpoint{3.448000in}{2.752433in}}%
\pgfpathlineto{\pgfqpoint{3.414912in}{2.768000in}}%
\pgfpathlineto{\pgfqpoint{3.410667in}{2.769981in}}%
\pgfpathlineto{\pgfqpoint{3.373333in}{2.787014in}}%
\pgfpathlineto{\pgfqpoint{3.336000in}{2.799110in}}%
\pgfpathlineto{\pgfqpoint{3.307860in}{2.805333in}}%
\pgfpathlineto{\pgfqpoint{3.298667in}{2.807968in}}%
\pgfpathlineto{\pgfqpoint{3.261333in}{2.815331in}}%
\pgfpathlineto{\pgfqpoint{3.224000in}{2.819273in}}%
\pgfpathlineto{\pgfqpoint{3.186667in}{2.820303in}}%
\pgfpathlineto{\pgfqpoint{3.149333in}{2.818924in}}%
\pgfpathlineto{\pgfqpoint{3.112000in}{2.816091in}}%
\pgfpathlineto{\pgfqpoint{3.074667in}{2.813665in}}%
\pgfpathlineto{\pgfqpoint{3.037333in}{2.813853in}}%
\pgfpathlineto{\pgfqpoint{3.000000in}{2.817447in}}%
\pgfpathlineto{\pgfqpoint{2.962667in}{2.823323in}}%
\pgfpathlineto{\pgfqpoint{2.925333in}{2.829920in}}%
\pgfpathlineto{\pgfqpoint{2.888000in}{2.836143in}}%
\pgfpathlineto{\pgfqpoint{2.850667in}{2.840639in}}%
\pgfpathlineto{\pgfqpoint{2.813333in}{2.837837in}}%
\pgfpathlineto{\pgfqpoint{2.784450in}{2.805333in}}%
\pgfpathlineto{\pgfqpoint{2.785181in}{2.768000in}}%
\pgfpathlineto{\pgfqpoint{2.788863in}{2.730667in}}%
\pgfpathlineto{\pgfqpoint{2.787520in}{2.693333in}}%
\pgfpathlineto{\pgfqpoint{2.778200in}{2.656000in}}%
\pgfpathlineto{\pgfqpoint{2.776000in}{2.649629in}}%
\pgfpathlineto{\pgfqpoint{2.761249in}{2.618667in}}%
\pgfpathlineto{\pgfqpoint{2.751291in}{2.581333in}}%
\pgfpathlineto{\pgfqpoint{2.756054in}{2.544000in}}%
\pgfpathlineto{\pgfqpoint{2.776000in}{2.514814in}}%
\pgfpathlineto{\pgfqpoint{2.782112in}{2.506667in}}%
\pgfpathlineto{\pgfqpoint{2.813333in}{2.482734in}}%
\pgfpathlineto{\pgfqpoint{2.836233in}{2.469333in}}%
\pgfpathlineto{\pgfqpoint{2.850667in}{2.461597in}}%
\pgfpathlineto{\pgfqpoint{2.888000in}{2.444615in}}%
\pgfpathlineto{\pgfqpoint{2.922845in}{2.432000in}}%
\pgfpathlineto{\pgfqpoint{2.925333in}{2.430819in}}%
\pgfpathlineto{\pgfqpoint{2.962667in}{2.413137in}}%
\pgfpathlineto{\pgfqpoint{3.000000in}{2.396528in}}%
\pgfpathlineto{\pgfqpoint{3.003916in}{2.394667in}}%
\pgfpathlineto{\pgfqpoint{3.037333in}{2.370265in}}%
\pgfpathlineto{\pgfqpoint{3.055076in}{2.357333in}}%
\pgfpathlineto{\pgfqpoint{3.074667in}{2.337788in}}%
\pgfpathlineto{\pgfqpoint{3.093120in}{2.320000in}}%
\pgfpathlineto{\pgfqpoint{3.112000in}{2.299777in}}%
\pgfpathlineto{\pgfqpoint{3.129436in}{2.282667in}}%
\pgfpathlineto{\pgfqpoint{3.149333in}{2.263334in}}%
\pgfpathlineto{\pgfqpoint{3.170642in}{2.245333in}}%
\pgfpathlineto{\pgfqpoint{3.186667in}{2.232117in}}%
\pgfpathlineto{\pgfqpoint{3.222090in}{2.208000in}}%
\pgfpathlineto{\pgfqpoint{3.224000in}{2.206697in}}%
\pgfpathlineto{\pgfqpoint{3.261333in}{2.184794in}}%
\pgfpathlineto{\pgfqpoint{3.293906in}{2.170667in}}%
\pgfpathlineto{\pgfqpoint{3.298667in}{2.168448in}}%
\pgfpathlineto{\pgfqpoint{3.336000in}{2.153733in}}%
\pgfpathlineto{\pgfqpoint{3.373333in}{2.141648in}}%
\pgfpathlineto{\pgfqpoint{3.399446in}{2.133333in}}%
\pgfpathlineto{\pgfqpoint{3.410667in}{2.129086in}}%
\pgfpathlineto{\pgfqpoint{3.448000in}{2.112015in}}%
\pgfpathlineto{\pgfqpoint{3.485333in}{2.096256in}}%
\pgfpathlineto{\pgfqpoint{3.486135in}{2.096000in}}%
\pgfpathlineto{\pgfqpoint{3.522667in}{2.082218in}}%
\pgfusepath{stroke}%
\end{pgfscope}%
\begin{pgfscope}%
\pgfpathrectangle{\pgfqpoint{1.432000in}{0.528000in}}{\pgfqpoint{3.696000in}{3.696000in}}%
\pgfusepath{clip}%
\pgfsetbuttcap%
\pgfsetroundjoin%
\pgfsetlinewidth{1.505625pt}%
\definecolor{currentstroke}{rgb}{0.700470,0.000000,0.000000}%
\pgfsetstrokecolor{currentstroke}%
\pgfsetstrokeopacity{0.300000}%
\pgfsetdash{}{0pt}%
\pgfpathmoveto{\pgfqpoint{3.448000in}{2.204569in}}%
\pgfpathlineto{\pgfqpoint{3.485333in}{2.187802in}}%
\pgfpathlineto{\pgfqpoint{3.522667in}{2.174936in}}%
\pgfpathlineto{\pgfqpoint{3.560000in}{2.171375in}}%
\pgfpathlineto{\pgfqpoint{3.597333in}{2.191301in}}%
\pgfpathlineto{\pgfqpoint{3.608127in}{2.208000in}}%
\pgfpathlineto{\pgfqpoint{3.617330in}{2.245333in}}%
\pgfpathlineto{\pgfqpoint{3.619745in}{2.282667in}}%
\pgfpathlineto{\pgfqpoint{3.621486in}{2.320000in}}%
\pgfpathlineto{\pgfqpoint{3.626210in}{2.357333in}}%
\pgfpathlineto{\pgfqpoint{3.634667in}{2.390353in}}%
\pgfpathlineto{\pgfqpoint{3.635572in}{2.394667in}}%
\pgfpathlineto{\pgfqpoint{3.644117in}{2.432000in}}%
\pgfpathlineto{\pgfqpoint{3.652528in}{2.469333in}}%
\pgfpathlineto{\pgfqpoint{3.660904in}{2.506667in}}%
\pgfpathlineto{\pgfqpoint{3.669882in}{2.544000in}}%
\pgfpathlineto{\pgfqpoint{3.672000in}{2.552260in}}%
\pgfpathlineto{\pgfqpoint{3.677183in}{2.581333in}}%
\pgfpathlineto{\pgfqpoint{3.683474in}{2.618667in}}%
\pgfpathlineto{\pgfqpoint{3.689432in}{2.656000in}}%
\pgfpathlineto{\pgfqpoint{3.695093in}{2.693333in}}%
\pgfpathlineto{\pgfqpoint{3.700562in}{2.730667in}}%
\pgfpathlineto{\pgfqpoint{3.706028in}{2.768000in}}%
\pgfpathlineto{\pgfqpoint{3.709333in}{2.791654in}}%
\pgfpathlineto{\pgfqpoint{3.710586in}{2.805333in}}%
\pgfpathlineto{\pgfqpoint{3.712853in}{2.842667in}}%
\pgfpathlineto{\pgfqpoint{3.713746in}{2.880000in}}%
\pgfpathlineto{\pgfqpoint{3.712664in}{2.917333in}}%
\pgfpathlineto{\pgfqpoint{3.709333in}{2.948208in}}%
\pgfpathlineto{\pgfqpoint{3.672000in}{2.949043in}}%
\pgfpathlineto{\pgfqpoint{3.660865in}{2.917333in}}%
\pgfpathlineto{\pgfqpoint{3.646271in}{2.880000in}}%
\pgfpathlineto{\pgfqpoint{3.634667in}{2.855985in}}%
\pgfpathlineto{\pgfqpoint{3.631002in}{2.842667in}}%
\pgfpathlineto{\pgfqpoint{3.620575in}{2.805333in}}%
\pgfpathlineto{\pgfqpoint{3.608539in}{2.768000in}}%
\pgfpathlineto{\pgfqpoint{3.597333in}{2.739588in}}%
\pgfpathlineto{\pgfqpoint{3.594138in}{2.730667in}}%
\pgfpathlineto{\pgfqpoint{3.580646in}{2.693333in}}%
\pgfpathlineto{\pgfqpoint{3.566411in}{2.656000in}}%
\pgfpathlineto{\pgfqpoint{3.560000in}{2.638497in}}%
\pgfpathlineto{\pgfqpoint{3.549910in}{2.618667in}}%
\pgfpathlineto{\pgfqpoint{3.534346in}{2.581333in}}%
\pgfpathlineto{\pgfqpoint{3.522667in}{2.550044in}}%
\pgfpathlineto{\pgfqpoint{3.515849in}{2.544000in}}%
\pgfpathlineto{\pgfqpoint{3.485333in}{2.517076in}}%
\pgfpathlineto{\pgfqpoint{3.463778in}{2.544000in}}%
\pgfpathlineto{\pgfqpoint{3.448000in}{2.577920in}}%
\pgfpathlineto{\pgfqpoint{3.447360in}{2.581333in}}%
\pgfpathlineto{\pgfqpoint{3.438368in}{2.618667in}}%
\pgfpathlineto{\pgfqpoint{3.419853in}{2.656000in}}%
\pgfpathlineto{\pgfqpoint{3.410667in}{2.666083in}}%
\pgfpathlineto{\pgfqpoint{3.386363in}{2.693333in}}%
\pgfpathlineto{\pgfqpoint{3.373333in}{2.702792in}}%
\pgfpathlineto{\pgfqpoint{3.336000in}{2.724557in}}%
\pgfpathlineto{\pgfqpoint{3.321437in}{2.730667in}}%
\pgfpathlineto{\pgfqpoint{3.298667in}{2.739225in}}%
\pgfpathlineto{\pgfqpoint{3.261333in}{2.748804in}}%
\pgfpathlineto{\pgfqpoint{3.224000in}{2.754485in}}%
\pgfpathlineto{\pgfqpoint{3.186667in}{2.757453in}}%
\pgfpathlineto{\pgfqpoint{3.149333in}{2.758896in}}%
\pgfpathlineto{\pgfqpoint{3.112000in}{2.760006in}}%
\pgfpathlineto{\pgfqpoint{3.074667in}{2.761673in}}%
\pgfpathlineto{\pgfqpoint{3.037333in}{2.764218in}}%
\pgfpathlineto{\pgfqpoint{3.000000in}{2.767441in}}%
\pgfpathlineto{\pgfqpoint{2.992727in}{2.768000in}}%
\pgfpathlineto{\pgfqpoint{2.962667in}{2.771351in}}%
\pgfpathlineto{\pgfqpoint{2.925333in}{2.772166in}}%
\pgfpathlineto{\pgfqpoint{2.902793in}{2.768000in}}%
\pgfpathlineto{\pgfqpoint{2.888000in}{2.763669in}}%
\pgfpathlineto{\pgfqpoint{2.860703in}{2.730667in}}%
\pgfpathlineto{\pgfqpoint{2.852282in}{2.693333in}}%
\pgfpathlineto{\pgfqpoint{2.850667in}{2.682814in}}%
\pgfpathlineto{\pgfqpoint{2.845846in}{2.656000in}}%
\pgfpathlineto{\pgfqpoint{2.839248in}{2.618667in}}%
\pgfpathlineto{\pgfqpoint{2.839011in}{2.581333in}}%
\pgfpathlineto{\pgfqpoint{2.850667in}{2.554383in}}%
\pgfpathlineto{\pgfqpoint{2.855492in}{2.544000in}}%
\pgfpathlineto{\pgfqpoint{2.888000in}{2.515155in}}%
\pgfpathlineto{\pgfqpoint{2.900269in}{2.506667in}}%
\pgfpathlineto{\pgfqpoint{2.925333in}{2.493172in}}%
\pgfpathlineto{\pgfqpoint{2.962667in}{2.476794in}}%
\pgfpathlineto{\pgfqpoint{2.982059in}{2.469333in}}%
\pgfpathlineto{\pgfqpoint{3.000000in}{2.461282in}}%
\pgfpathlineto{\pgfqpoint{3.037333in}{2.444507in}}%
\pgfpathlineto{\pgfqpoint{3.066021in}{2.432000in}}%
\pgfpathlineto{\pgfqpoint{3.074667in}{2.426751in}}%
\pgfpathlineto{\pgfqpoint{3.112000in}{2.403234in}}%
\pgfpathlineto{\pgfqpoint{3.125936in}{2.394667in}}%
\pgfpathlineto{\pgfqpoint{3.149333in}{2.375881in}}%
\pgfpathlineto{\pgfqpoint{3.175254in}{2.357333in}}%
\pgfpathlineto{\pgfqpoint{3.186667in}{2.347570in}}%
\pgfpathlineto{\pgfqpoint{3.224000in}{2.320612in}}%
\pgfpathlineto{\pgfqpoint{3.224947in}{2.320000in}}%
\pgfpathlineto{\pgfqpoint{3.261333in}{2.293625in}}%
\pgfpathlineto{\pgfqpoint{3.280930in}{2.282667in}}%
\pgfpathlineto{\pgfqpoint{3.298667in}{2.271474in}}%
\pgfpathlineto{\pgfqpoint{3.336000in}{2.252629in}}%
\pgfpathlineto{\pgfqpoint{3.353447in}{2.245333in}}%
\pgfpathlineto{\pgfqpoint{3.373333in}{2.235631in}}%
\pgfpathlineto{\pgfqpoint{3.410667in}{2.220004in}}%
\pgfpathlineto{\pgfqpoint{3.440748in}{2.208000in}}%
\pgfpathlineto{\pgfqpoint{3.448000in}{2.204569in}}%
\pgfusepath{stroke}%
\end{pgfscope}%
\begin{pgfscope}%
\pgfpathrectangle{\pgfqpoint{1.432000in}{0.528000in}}{\pgfqpoint{3.696000in}{3.696000in}}%
\pgfusepath{clip}%
\pgfsetbuttcap%
\pgfsetroundjoin%
\pgfsetlinewidth{1.505625pt}%
\definecolor{currentstroke}{rgb}{1.000000,0.029903,0.000000}%
\pgfsetstrokecolor{currentstroke}%
\pgfsetstrokeopacity{0.300000}%
\pgfsetdash{}{0pt}%
\pgfpathmoveto{\pgfqpoint{3.410667in}{2.276929in}}%
\pgfpathlineto{\pgfqpoint{3.448000in}{2.267834in}}%
\pgfpathlineto{\pgfqpoint{3.485333in}{2.262309in}}%
\pgfpathlineto{\pgfqpoint{3.522667in}{2.271774in}}%
\pgfpathlineto{\pgfqpoint{3.530118in}{2.282667in}}%
\pgfpathlineto{\pgfqpoint{3.523985in}{2.320000in}}%
\pgfpathlineto{\pgfqpoint{3.522667in}{2.321954in}}%
\pgfpathlineto{\pgfqpoint{3.497857in}{2.357333in}}%
\pgfpathlineto{\pgfqpoint{3.485333in}{2.368194in}}%
\pgfpathlineto{\pgfqpoint{3.464109in}{2.394667in}}%
\pgfpathlineto{\pgfqpoint{3.448000in}{2.410423in}}%
\pgfpathlineto{\pgfqpoint{3.431538in}{2.432000in}}%
\pgfpathlineto{\pgfqpoint{3.410667in}{2.457118in}}%
\pgfpathlineto{\pgfqpoint{3.401936in}{2.469333in}}%
\pgfpathlineto{\pgfqpoint{3.378296in}{2.506667in}}%
\pgfpathlineto{\pgfqpoint{3.373333in}{2.518630in}}%
\pgfpathlineto{\pgfqpoint{3.364075in}{2.544000in}}%
\pgfpathlineto{\pgfqpoint{3.357593in}{2.581333in}}%
\pgfpathlineto{\pgfqpoint{3.350329in}{2.618667in}}%
\pgfpathlineto{\pgfqpoint{3.336000in}{2.645619in}}%
\pgfpathlineto{\pgfqpoint{3.329945in}{2.656000in}}%
\pgfpathlineto{\pgfqpoint{3.298667in}{2.681142in}}%
\pgfpathlineto{\pgfqpoint{3.277035in}{2.693333in}}%
\pgfpathlineto{\pgfqpoint{3.261333in}{2.699689in}}%
\pgfpathlineto{\pgfqpoint{3.224000in}{2.710611in}}%
\pgfpathlineto{\pgfqpoint{3.186667in}{2.717684in}}%
\pgfpathlineto{\pgfqpoint{3.149333in}{2.722512in}}%
\pgfpathlineto{\pgfqpoint{3.112000in}{2.726313in}}%
\pgfpathlineto{\pgfqpoint{3.074667in}{2.729782in}}%
\pgfpathlineto{\pgfqpoint{3.064425in}{2.730667in}}%
\pgfpathlineto{\pgfqpoint{3.037333in}{2.733626in}}%
\pgfpathlineto{\pgfqpoint{3.000000in}{2.736037in}}%
\pgfpathlineto{\pgfqpoint{2.962667in}{2.734480in}}%
\pgfpathlineto{\pgfqpoint{2.947529in}{2.730667in}}%
\pgfpathlineto{\pgfqpoint{2.925333in}{2.721604in}}%
\pgfpathlineto{\pgfqpoint{2.901702in}{2.693333in}}%
\pgfpathlineto{\pgfqpoint{2.891714in}{2.656000in}}%
\pgfpathlineto{\pgfqpoint{2.890006in}{2.618667in}}%
\pgfpathlineto{\pgfqpoint{2.897690in}{2.581333in}}%
\pgfpathlineto{\pgfqpoint{2.925333in}{2.545059in}}%
\pgfpathlineto{\pgfqpoint{2.926273in}{2.544000in}}%
\pgfpathlineto{\pgfqpoint{2.962667in}{2.518857in}}%
\pgfpathlineto{\pgfqpoint{2.985195in}{2.506667in}}%
\pgfpathlineto{\pgfqpoint{3.000000in}{2.499308in}}%
\pgfpathlineto{\pgfqpoint{3.037333in}{2.481776in}}%
\pgfpathlineto{\pgfqpoint{3.065889in}{2.469333in}}%
\pgfpathlineto{\pgfqpoint{3.074667in}{2.464712in}}%
\pgfpathlineto{\pgfqpoint{3.112000in}{2.444511in}}%
\pgfpathlineto{\pgfqpoint{3.135422in}{2.432000in}}%
\pgfpathlineto{\pgfqpoint{3.149333in}{2.422529in}}%
\pgfpathlineto{\pgfqpoint{3.186667in}{2.398441in}}%
\pgfpathlineto{\pgfqpoint{3.192675in}{2.394667in}}%
\pgfpathlineto{\pgfqpoint{3.224000in}{2.371829in}}%
\pgfpathlineto{\pgfqpoint{3.246767in}{2.357333in}}%
\pgfpathlineto{\pgfqpoint{3.261333in}{2.347006in}}%
\pgfpathlineto{\pgfqpoint{3.298667in}{2.325050in}}%
\pgfpathlineto{\pgfqpoint{3.308800in}{2.320000in}}%
\pgfpathlineto{\pgfqpoint{3.336000in}{2.305356in}}%
\pgfpathlineto{\pgfqpoint{3.373333in}{2.289935in}}%
\pgfpathlineto{\pgfqpoint{3.394901in}{2.282667in}}%
\pgfpathlineto{\pgfqpoint{3.410667in}{2.276929in}}%
\pgfusepath{stroke}%
\end{pgfscope}%
\begin{pgfscope}%
\pgfpathrectangle{\pgfqpoint{1.432000in}{0.528000in}}{\pgfqpoint{3.696000in}{3.696000in}}%
\pgfusepath{clip}%
\pgfsetbuttcap%
\pgfsetroundjoin%
\pgfsetlinewidth{1.505625pt}%
\definecolor{currentstroke}{rgb}{1.000000,0.359314,0.000000}%
\pgfsetstrokecolor{currentstroke}%
\pgfsetstrokeopacity{0.300000}%
\pgfsetdash{}{0pt}%
\pgfpathmoveto{\pgfqpoint{3.336000in}{2.353462in}}%
\pgfpathlineto{\pgfqpoint{3.373333in}{2.348016in}}%
\pgfpathlineto{\pgfqpoint{3.394861in}{2.357333in}}%
\pgfpathlineto{\pgfqpoint{3.391838in}{2.394667in}}%
\pgfpathlineto{\pgfqpoint{3.373333in}{2.428674in}}%
\pgfpathlineto{\pgfqpoint{3.371760in}{2.432000in}}%
\pgfpathlineto{\pgfqpoint{3.347539in}{2.469333in}}%
\pgfpathlineto{\pgfqpoint{3.336000in}{2.488292in}}%
\pgfpathlineto{\pgfqpoint{3.325784in}{2.506667in}}%
\pgfpathlineto{\pgfqpoint{3.310175in}{2.544000in}}%
\pgfpathlineto{\pgfqpoint{3.298667in}{2.579673in}}%
\pgfpathlineto{\pgfqpoint{3.298148in}{2.581333in}}%
\pgfpathlineto{\pgfqpoint{3.284360in}{2.618667in}}%
\pgfpathlineto{\pgfqpoint{3.261333in}{2.645782in}}%
\pgfpathlineto{\pgfqpoint{3.250047in}{2.656000in}}%
\pgfpathlineto{\pgfqpoint{3.224000in}{2.669993in}}%
\pgfpathlineto{\pgfqpoint{3.186667in}{2.683862in}}%
\pgfpathlineto{\pgfqpoint{3.149333in}{2.693031in}}%
\pgfpathlineto{\pgfqpoint{3.147839in}{2.693333in}}%
\pgfpathlineto{\pgfqpoint{3.112000in}{2.700324in}}%
\pgfpathlineto{\pgfqpoint{3.074667in}{2.705591in}}%
\pgfpathlineto{\pgfqpoint{3.037333in}{2.708481in}}%
\pgfpathlineto{\pgfqpoint{3.000000in}{2.707429in}}%
\pgfpathlineto{\pgfqpoint{2.962667in}{2.697536in}}%
\pgfpathlineto{\pgfqpoint{2.955957in}{2.693333in}}%
\pgfpathlineto{\pgfqpoint{2.931235in}{2.656000in}}%
\pgfpathlineto{\pgfqpoint{2.929510in}{2.618667in}}%
\pgfpathlineto{\pgfqpoint{2.942864in}{2.581333in}}%
\pgfpathlineto{\pgfqpoint{2.962667in}{2.559756in}}%
\pgfpathlineto{\pgfqpoint{2.979630in}{2.544000in}}%
\pgfpathlineto{\pgfqpoint{3.000000in}{2.531032in}}%
\pgfpathlineto{\pgfqpoint{3.037333in}{2.510900in}}%
\pgfpathlineto{\pgfqpoint{3.045683in}{2.506667in}}%
\pgfpathlineto{\pgfqpoint{3.074667in}{2.492041in}}%
\pgfpathlineto{\pgfqpoint{3.112000in}{2.474473in}}%
\pgfpathlineto{\pgfqpoint{3.122819in}{2.469333in}}%
\pgfpathlineto{\pgfqpoint{3.149333in}{2.454513in}}%
\pgfpathlineto{\pgfqpoint{3.186667in}{2.434052in}}%
\pgfpathlineto{\pgfqpoint{3.190525in}{2.432000in}}%
\pgfpathlineto{\pgfqpoint{3.224000in}{2.411428in}}%
\pgfpathlineto{\pgfqpoint{3.252517in}{2.394667in}}%
\pgfpathlineto{\pgfqpoint{3.261333in}{2.388987in}}%
\pgfpathlineto{\pgfqpoint{3.298667in}{2.369807in}}%
\pgfpathlineto{\pgfqpoint{3.327172in}{2.357333in}}%
\pgfpathlineto{\pgfqpoint{3.336000in}{2.353462in}}%
\pgfusepath{stroke}%
\end{pgfscope}%
\begin{pgfscope}%
\pgfpathrectangle{\pgfqpoint{1.432000in}{0.528000in}}{\pgfqpoint{3.696000in}{3.696000in}}%
\pgfusepath{clip}%
\pgfsetbuttcap%
\pgfsetroundjoin%
\pgfsetlinewidth{1.505625pt}%
\definecolor{currentstroke}{rgb}{1.000000,0.688725,0.000000}%
\pgfsetstrokecolor{currentstroke}%
\pgfsetstrokeopacity{0.300000}%
\pgfsetdash{}{0pt}%
\pgfpathmoveto{\pgfqpoint{3.186667in}{2.466218in}}%
\pgfpathlineto{\pgfqpoint{3.224000in}{2.451429in}}%
\pgfpathlineto{\pgfqpoint{3.261333in}{2.438872in}}%
\pgfpathlineto{\pgfqpoint{3.290292in}{2.469333in}}%
\pgfpathlineto{\pgfqpoint{3.278373in}{2.506667in}}%
\pgfpathlineto{\pgfqpoint{3.266008in}{2.544000in}}%
\pgfpathlineto{\pgfqpoint{3.261333in}{2.556586in}}%
\pgfpathlineto{\pgfqpoint{3.252185in}{2.581333in}}%
\pgfpathlineto{\pgfqpoint{3.229727in}{2.618667in}}%
\pgfpathlineto{\pgfqpoint{3.224000in}{2.624010in}}%
\pgfpathlineto{\pgfqpoint{3.186667in}{2.650752in}}%
\pgfpathlineto{\pgfqpoint{3.176501in}{2.656000in}}%
\pgfpathlineto{\pgfqpoint{3.149333in}{2.666124in}}%
\pgfpathlineto{\pgfqpoint{3.112000in}{2.675725in}}%
\pgfpathlineto{\pgfqpoint{3.074667in}{2.681524in}}%
\pgfpathlineto{\pgfqpoint{3.037333in}{2.682962in}}%
\pgfpathlineto{\pgfqpoint{3.000000in}{2.675673in}}%
\pgfpathlineto{\pgfqpoint{2.972900in}{2.656000in}}%
\pgfpathlineto{\pgfqpoint{2.965671in}{2.618667in}}%
\pgfpathlineto{\pgfqpoint{2.982408in}{2.581333in}}%
\pgfpathlineto{\pgfqpoint{3.000000in}{2.564744in}}%
\pgfpathlineto{\pgfqpoint{3.025636in}{2.544000in}}%
\pgfpathlineto{\pgfqpoint{3.037333in}{2.537095in}}%
\pgfpathlineto{\pgfqpoint{3.074667in}{2.517611in}}%
\pgfpathlineto{\pgfqpoint{3.097401in}{2.506667in}}%
\pgfpathlineto{\pgfqpoint{3.112000in}{2.499588in}}%
\pgfpathlineto{\pgfqpoint{3.149333in}{2.483218in}}%
\pgfpathlineto{\pgfqpoint{3.180593in}{2.469333in}}%
\pgfpathlineto{\pgfqpoint{3.186667in}{2.466218in}}%
\pgfusepath{stroke}%
\end{pgfscope}%
\begin{pgfscope}%
\pgfpathrectangle{\pgfqpoint{1.432000in}{0.528000in}}{\pgfqpoint{3.696000in}{3.696000in}}%
\pgfusepath{clip}%
\pgfsetbuttcap%
\pgfsetroundjoin%
\pgfsetlinewidth{1.505625pt}%
\definecolor{currentstroke}{rgb}{1.000000,1.000000,0.027205}%
\pgfsetstrokecolor{currentstroke}%
\pgfsetstrokeopacity{0.300000}%
\pgfsetdash{}{0pt}%
\pgfpathmoveto{\pgfqpoint{3.186667in}{2.503780in}}%
\pgfpathlineto{\pgfqpoint{3.202492in}{2.506667in}}%
\pgfpathlineto{\pgfqpoint{3.221207in}{2.544000in}}%
\pgfpathlineto{\pgfqpoint{3.208784in}{2.581333in}}%
\pgfpathlineto{\pgfqpoint{3.186667in}{2.610436in}}%
\pgfpathlineto{\pgfqpoint{3.179018in}{2.618667in}}%
\pgfpathlineto{\pgfqpoint{3.149333in}{2.636003in}}%
\pgfpathlineto{\pgfqpoint{3.112000in}{2.650236in}}%
\pgfpathlineto{\pgfqpoint{3.084663in}{2.656000in}}%
\pgfpathlineto{\pgfqpoint{3.074667in}{2.657585in}}%
\pgfpathlineto{\pgfqpoint{3.039251in}{2.656000in}}%
\pgfpathlineto{\pgfqpoint{3.037333in}{2.655805in}}%
\pgfpathlineto{\pgfqpoint{3.004106in}{2.618667in}}%
\pgfpathlineto{\pgfqpoint{3.021879in}{2.581333in}}%
\pgfpathlineto{\pgfqpoint{3.037333in}{2.568626in}}%
\pgfpathlineto{\pgfqpoint{3.071927in}{2.544000in}}%
\pgfpathlineto{\pgfqpoint{3.074667in}{2.542546in}}%
\pgfpathlineto{\pgfqpoint{3.112000in}{2.526475in}}%
\pgfpathlineto{\pgfqpoint{3.149333in}{2.512625in}}%
\pgfpathlineto{\pgfqpoint{3.174585in}{2.506667in}}%
\pgfpathlineto{\pgfqpoint{3.186667in}{2.503780in}}%
\pgfusepath{stroke}%
\end{pgfscope}%
\begin{pgfscope}%
\pgfpathrectangle{\pgfqpoint{1.432000in}{0.528000in}}{\pgfqpoint{3.696000in}{3.696000in}}%
\pgfusepath{clip}%
\pgfsetbuttcap%
\pgfsetroundjoin%
\pgfsetlinewidth{1.505625pt}%
\definecolor{currentstroke}{rgb}{1.000000,1.000000,0.521323}%
\pgfsetstrokecolor{currentstroke}%
\pgfsetstrokeopacity{0.300000}%
\pgfsetdash{}{0pt}%
\pgfpathmoveto{\pgfqpoint{3.074667in}{2.579433in}}%
\pgfpathlineto{\pgfqpoint{3.112000in}{2.564169in}}%
\pgfpathlineto{\pgfqpoint{3.149333in}{2.567411in}}%
\pgfpathlineto{\pgfqpoint{3.155441in}{2.581333in}}%
\pgfpathlineto{\pgfqpoint{3.149333in}{2.587967in}}%
\pgfpathlineto{\pgfqpoint{3.112000in}{2.617706in}}%
\pgfpathlineto{\pgfqpoint{3.107637in}{2.618667in}}%
\pgfpathlineto{\pgfqpoint{3.074667in}{2.620988in}}%
\pgfpathlineto{\pgfqpoint{3.068561in}{2.618667in}}%
\pgfpathlineto{\pgfqpoint{3.071855in}{2.581333in}}%
\pgfpathlineto{\pgfqpoint{3.074667in}{2.579433in}}%
\pgfusepath{stroke}%
\end{pgfscope}%
\begin{pgfscope}%
\pgfsetrectcap%
\pgfsetmiterjoin%
\pgfsetlinewidth{0.803000pt}%
\definecolor{currentstroke}{rgb}{0.000000,0.000000,0.000000}%
\pgfsetstrokecolor{currentstroke}%
\pgfsetdash{}{0pt}%
\pgfpathmoveto{\pgfqpoint{1.432000in}{0.528000in}}%
\pgfpathlineto{\pgfqpoint{1.432000in}{4.224000in}}%
\pgfusepath{stroke}%
\end{pgfscope}%
\begin{pgfscope}%
\pgfsetrectcap%
\pgfsetmiterjoin%
\pgfsetlinewidth{0.803000pt}%
\definecolor{currentstroke}{rgb}{0.000000,0.000000,0.000000}%
\pgfsetstrokecolor{currentstroke}%
\pgfsetdash{}{0pt}%
\pgfpathmoveto{\pgfqpoint{5.128000in}{0.528000in}}%
\pgfpathlineto{\pgfqpoint{5.128000in}{4.224000in}}%
\pgfusepath{stroke}%
\end{pgfscope}%
\begin{pgfscope}%
\pgfsetrectcap%
\pgfsetmiterjoin%
\pgfsetlinewidth{0.803000pt}%
\definecolor{currentstroke}{rgb}{0.000000,0.000000,0.000000}%
\pgfsetstrokecolor{currentstroke}%
\pgfsetdash{}{0pt}%
\pgfpathmoveto{\pgfqpoint{1.432000in}{0.528000in}}%
\pgfpathlineto{\pgfqpoint{5.128000in}{0.528000in}}%
\pgfusepath{stroke}%
\end{pgfscope}%
\begin{pgfscope}%
\pgfsetrectcap%
\pgfsetmiterjoin%
\pgfsetlinewidth{0.803000pt}%
\definecolor{currentstroke}{rgb}{0.000000,0.000000,0.000000}%
\pgfsetstrokecolor{currentstroke}%
\pgfsetdash{}{0pt}%
\pgfpathmoveto{\pgfqpoint{1.432000in}{4.224000in}}%
\pgfpathlineto{\pgfqpoint{5.128000in}{4.224000in}}%
\pgfusepath{stroke}%
\end{pgfscope}%
\begin{pgfscope}%
\definecolor{textcolor}{rgb}{0.000000,0.000000,0.000000}%
\pgfsetstrokecolor{textcolor}%
\pgfsetfillcolor{textcolor}%
\pgftext[x=3.280000in,y=4.307333in,,base]{\color{textcolor}\rmfamily\fontsize{12.000000}{14.400000}\selectfont Experiment 1A: CE-mixture (\(\displaystyle k=5\))}%
\end{pgfscope}%
\end{pgfpicture}%
\makeatother%
\endgroup%
}
    }
  \caption{
    \label{fig:k5} Iteration $k=5$ illustrated for each algorithm. The covariance is shown by the contours.
  } 
\end{figure*}

Because the algorithms are stochastic, we run each experiment with 50 different random number generator seed values.
To evaluate the performance of the algorithms in their respective experiments, we define three metrics.
First, we define the average ``optimal'' value $\bar{b}_v$ to be the average of the best so-far objective function value (termed ``optimal'' in the context of each algorithm). Again, we emphasize that we average over the 50 seed values to gather meaningful statistics.
Another metric we monitor is the average distance to the true global optimal $\bar{b}_d = \norm{\vec{b}_{\vec{x}} - \vec{x}^*}$, where $\vec{b}_{\vec{x}}$ denotes the $\vec{x}$-value associated with the ``optimal''.
We make the distinction between these metrics to show both ``closeness'' in \textit{value} to the global minimum and ``closeness'' in the \textit{design space} to the global minimum.
Our final metric looks at the average runtime of each algorithm, noting that our goal is to off-load computationally expensive objective function calls to the surrogate model.

For all of the experiments, we use a common setting of the following parameters for the sierra test function (shown in the top-right plot in \cref{fig:sierra}):
\begin{equation*}
    (\mathbf{\tilde{\vec{\mu}}} =[0,0],\; \sigma=3,\; \delta=2,\; \eta=6,\; \text{decay} = 1)
\end{equation*}


\subsubsection{Algorithmic Experiments} \label{sec:alg_experiments}
We run three separate algorithmic experiments, each to test a specific feature.
For our first algorithmic experiment (1A), we want to test each algorithm when the user-defined mean is centered at the global minimum and the covariance is arbitrarily wide enough to cover the design space.
Let $\M$ be a distribution parameterized by $\vec{\theta} = (\vec{\mu}, \mat{\Sigma})$, and for experiment (1A) we set the following:
% CE-mixture mean and covariance (1A)
\begin{equation*}
    \vec\mu^{(\text{1A})} = [0, 0] \qquad
    \mat\Sigma^{(\text{1A})} = \begin{bmatrix}
        200 & 0\\
        0 & 200
    \end{bmatrix}
\end{equation*}

For our second algorithmic experiment (1B), we test a mean that is far off-centered with a wider covariance:
% CE-mixture mean and covariance (1B)
\begin{equation*}
    \vec\mu^{(\text{1B})} = [-50, -50] \qquad
    \mat\Sigma^{(\text{1B})} = \begin{bmatrix}
        2000 & 0\\
        0 & 2000
    \end{bmatrix}
\end{equation*}
This experiment is used to test the ``exploration'' of the CE-method variants introduced in this work.
In experiments (1A) and (1B), we set the following common parameters across each CE-method variant:
%% CE-mixture hyperparameter settings
\begin{equation*}
    (k_\text{max} = 10,\; m=10,\; m_\text{elite}=5)^{(\text{1A,1B})}
\end{equation*}
This results in $m\cdot k_\text{max} = 100$ objective function evaluations, which we define to be \textit{relatively} low.

For our third algorithmic experiment (1C), we want to test how each variant responds to an extremely low number of function evaluations.
This sparse experiment sets the common CE-method parameters to:
% CE-method params (1C)
\begin{equation*}
    (k_\text{max} = 10,\; m=5,\; m_\text{elite}=3)^{(\text{1C})}
\end{equation*}
This results in $m\cdot k_\text{max} = 50$ objective function evaluations, which we defined to be \textit{extremely} low.
We use the same mean and covariance defined for experiment (1A):
\begin{equation*}
    \vec\mu^{(\text{1C})} = [0, 0] \qquad
    \mat\Sigma^{(\text{1C})} = \begin{bmatrix}
        200 & 0\\
        0 & 200
    \end{bmatrix}
\end{equation*}


\subsubsection{Scheduling Experiments} \label{sec:schedule_experiments}
In our final experiment (2), we test the evaluation scheduling heuristics which are based on the Geometric distribution.
We sweep over the parameter $p$ that determines the Geometric distribution which controls the redistribution of objective function evaluations.
In this experiment, we compare the CE-surrogate methods using the same setup as experiment (1B), namely the far off-centered mean.
We chose this setup to analyze exploration schemes when given very little information about the true objective function.



\subsection{Results and Analysis} \label{sec:results}

\begin{figure}[!hb]
  % \centering
  \resizebox{0.9\columnwidth}{!}{\begin{tikzpicture}[]
\begin{axis}[height = {6cm}, legend style = {}, ylabel = {$\bar{b}_v$}, title = {Experiment 1A}, xmin = {1}, xmax = {10}, xlabel = {Iteration}, width = {10cm}]\addplot+ [mark = {none}, blue]coordinates {
(1.0, -0.0031201098066532405)
(2.0, -0.005870857044435953)
(3.0, -0.008029959839914747)
(4.0, -0.009931463697053772)
(5.0, -0.011237543181515833)
(6.0, -0.012262911549891264)
(7.0, -0.012592946518249947)
(8.0, -0.013108380477455581)
(9.0, -0.013229171614954224)
(10.0, -0.013384552992011585)
};
\addlegendentry{CE-method}
\addplot+ [mark = {none}, red]coordinates {
(1.0, -0.004867620923558716)
(2.0, -0.008359447446686907)
(3.0, -0.010922475433594345)
(4.0, -0.012512144731967456)
(5.0, -0.013955455243209767)
(6.0, -0.014683807740700328)
(7.0, -0.015815892823378606)
(8.0, -0.016625543566398643)
(9.0, -0.017259993731152726)
(10.0, -0.017879391835753086)
};
\addlegendentry{CE-surrogate}
\addplot+ [mark = {none}, green!50!black]coordinates {
(1.0, -0.004867620923558716)
(2.0, -0.007838400186402038)
(3.0, -0.010702093990244235)
(4.0, -0.012711538858683289)
(5.0, -0.013873504611916936)
(6.0, -0.014905868943295396)
(7.0, -0.015707648145471702)
(8.0, -0.016105669536789855)
(9.0, -0.016316854535311606)
(10.0, -0.016924208538882466)
};
\addlegendentry{CE-mixture}
\addplot+ [mark = {none}, dashed, blue, name path=Aplus, opacity=0.2]coordinates {
(1.0, -0.0006047180244641553)
(2.0, -0.0015997728585300944)
(3.0, -0.003961121055630016)
(4.0, -0.005585741144704704)
(5.0, -0.0065318460375573)
(6.0, -0.007015918213275647)
(7.0, -0.007277530944143609)
(8.0, -0.007347028008143184)
(9.0, -0.007393699423113323)
(10.0, -0.007469733398611503)
};
\addplot+ [mark = {none}, dashed, blue, name path=Aminus, opacity=0.2]coordinates {
(1.0, -0.005635501588842326)
(2.0, -0.010141941230341813)
(3.0, -0.012098798624199478)
(4.0, -0.01427718624940284)
(5.0, -0.015943240325474367)
(6.0, -0.01750990488650688)
(7.0, -0.017908362092356286)
(8.0, -0.018869732946767977)
(9.0, -0.019064643806795126)
(10.0, -0.019299372585411666)
};
\addplot+ [mark = {none}, dashed, red, name path=Bplus, opacity=0.2]coordinates {
(1.0, -0.0014620695871026)
(2.0, -0.004236985725206302)
(3.0, -0.008120707593571749)
(4.0, -0.009862374242251448)
(5.0, -0.01110182392044447)
(6.0, -0.011659906910015172)
(7.0, -0.012664071193490563)
(8.0, -0.013375680889737156)
(9.0, -0.014139327969562021)
(10.0, -0.014872302762209612)
};
\addplot+ [mark = {none}, dashed, red, name path=Bminus, opacity=0.2]coordinates {
(1.0, -0.008273172260014831)
(2.0, -0.012481909168167512)
(3.0, -0.013724243273616942)
(4.0, -0.015161915221683465)
(5.0, -0.016809086565975063)
(6.0, -0.017707708571385483)
(7.0, -0.01896771445326665)
(8.0, -0.01987540624306013)
(9.0, -0.020380659492743432)
(10.0, -0.02088648090929656)
};
\addplot+ [mark = {none}, dashed, green!50!black, name path=Cplus, opacity=0.2]coordinates {
(1.0, -0.0014620695871026)
(2.0, -0.004070861982928433)
(3.0, -0.006957718362033021)
(4.0, -0.009460796991808296)
(5.0, -0.011276196842049358)
(6.0, -0.012279100123204593)
(7.0, -0.013125278973715203)
(8.0, -0.01366560841237955)
(9.0, -0.013839465448164053)
(10.0, -0.014000450783072643)
};
\addplot+ [mark = {none}, dashed, green!50!black, name path=Cminus, opacity=0.2]coordinates {
(1.0, -0.008273172260014831)
(2.0, -0.011605938389875644)
(3.0, -0.01444646961845545)
(4.0, -0.015962280725558282)
(5.0, -0.016470812381784515)
(6.0, -0.017532637763386198)
(7.0, -0.0182900173172282)
(8.0, -0.01854573066120016)
(9.0, -0.01879424362245916)
(10.0, -0.01984796629469229)
};
\addplot[blue!80, fill opacity=0.1] fill between[of=Aplus and Aminus];
                    \addplot[red!80, fill opacity=0.1] fill between[of=Bplus and Bminus];
                    \addplot[green!80, fill opacity=0.1] fill between[of=Cplus and Cminus];
\end{axis}

\end{tikzpicture}
}
  \caption{
    \label{fig:experiment_1a}
    Average optimal value for experiment (1A) when the initial mean is centered at the global minimum and the covariance sufficiently covers the design space.
  }
\end{figure}

\Cref{fig:experiment_1a} shows the average value of the current optimal $\bar{b}_v$ for the three algorithms for experiment (1A). 
One standard deviation is plotted in the shaded region.
Notice that the standard CE-method converges to a local minima before $k_\text{max}$ is reached.
Both CE-surrogate method and CE-mixture stay below the standard CE-method curve, highlighting the mitigation of convergence to local minima.
Minor differences can be seen between CE-surrogate and CE-mixture, differing slightly towards the tail in favor of CE-surrogate.
The average runtime of the algorithms along with the performance metrics are shown together for each experiment in \cref{tab:results}.


\begin{table}[!ht]
    \centering
    \caption{\label{tab:results} Experimental results.}
    \begin{tabular}{cllll} % p{3cm}
    \toprule
    \textbf{Exper.} & \textbf{Algorithm} & \textbf{Runtime} & $\bar{b}_v$ & $\bar{b}_d$\\
    \midrule
    \multirow{3}{*}{1A} & CE-method & \textbf{0.029 $\operatorname{s}$} & $-$0.0134 & 23.48\\
    &CE-surrogate & 1.47 $\operatorname{s}$ & \textbf{\boldmath$-$0.0179} & \textbf{12.23}\\
    &CE-mixture & 9.17 $\operatorname{s}$ & $-$0.0169 & 16.87\\
    \midrule
    \multirow{3}{*}{1B} & CE-method & \textbf{0.046 $\operatorname{s}$} & $-$0.0032 & 138.87\\
    &CE-surrogate & 11.82 $\operatorname{s}$ & \textbf{\boldmath$-$0.0156} & \textbf{18.24}\\
    &CE-mixture & 28.10 $\operatorname{s}$ & $-$0.0146 & 33.30\\
    \midrule
    \multirow{3}{*}{1C} & CE-method & \textbf{0.052 $\operatorname{s}$} & $-$0.0065 & 43.14\\
    &CE-surrogate & 0.474 $\operatorname{s}$ & \textbf{\boldmath$-$0.0156} & \textbf{17.23}\\
    &CE-mixture & 2.57 $\operatorname{s}$ & $-$0.0146 & 22.17\\
    \midrule
    \multirow{3}{*}{2} & CE-surrogate, $\operatorname{Uniform}$ & --- & \textbf{\boldmath$-$0.0193} & \textbf{8.53}\\
    &CE-surrogate, $\Geo(0.1)$ & {\color{gray}---} & $-$0.0115 & 25.35\\
    &CE-surrogate, $\Geo(0.2)$ & {\color{gray}---} & $-$0.0099 & 27.59\\
    &CE-surrogate, $\Geo(0.3)$ & {\color{gray}---} & $-$0.0089 & 30.88\\
    \bottomrule
    & & & \multicolumn{2}{l}{$-\text{0.0220} \approx\vec{x}^*$}\\
    \end{tabular}
\end{table}

An apparent benefit of the standard CE-method is in its simplicity and speed.
As shown in \cref{tab:results}, the CE-method is the fastest approach by about 2-3 orders of magnitude compared to CE-surrogate and CE-mixture.
The CE-mixture method is notably the slowest approach.
Although the runtime is also based on the objective function being tested, recall that we are using the same number of true objective function calls in each algorithm, and the metrics we are concerned with in optimization are to minimize $\bar{b}_v$ and $\bar{b}_d$.
We can see that the CE-surrogate method consistently out performs the other methods.
Surprisingly, a uniform evaluation schedule performs the best even in the sparse scenario where the initial mean is far away from the global optimal.

\begin{figure}[!ht]
  % \centering
  \resizebox{0.9\columnwidth}{!}{\begin{tikzpicture}[]
\begin{axis}[height = {6cm}, legend style = {{at={(0.01,0.01)},anchor=south west}}, ylabel = {$\bar{b}_v$}, title = {Experiment 1B}, xmin = {1}, xmax = {10}, xlabel = {Iteration}, width = {10cm}]\addplot+ [mark = {none}, blue]coordinates {
(1.0, -0.00042122778028887684)
(2.0, -0.0008654798596108782)
(3.0, -0.0016348513086736408)
(4.0, -0.002010511770819231)
(5.0, -0.0023846629522817613)
(6.0, -0.002809885878600046)
(7.0, -0.002905905701518952)
(8.0, -0.0029916991837182685)
(9.0, -0.0029962938587671283)
(10.0, -0.003153108797753543)
(11.0, -0.003200678101250048)
(12.0, -0.0032007316164522825)
(13.0, -0.0032177866798048533)
(14.0, -0.0032191390738716737)
(15.0, -0.0032191390738716737)
};
\addlegendentry{CE-method}
\addplot+ [mark = {none}, red]coordinates {
(1.0, -0.0009160031451805364)
(2.0, -0.0018071173179996145)
(3.0, -0.0028698649829150672)
(4.0, -0.0036008686370057694)
(5.0, -0.004320368487047906)
(6.0, -0.0059393295586773285)
(7.0, -0.00678273558549222)
(8.0, -0.008091271531196513)
(9.0, -0.009476259763406172)
(10.0, -0.010690011244082556)
(11.0, -0.01147307819315019)
(12.0, -0.012321804398558118)
(13.0, -0.013183297254110658)
(14.0, -0.01367136706559999)
(15.0, -0.015608547105351183)
};
\addlegendentry{CE-surrogate}
\addplot+ [mark = {none}, green!50!black]coordinates {
(1.0, -0.000412120500565601)
(2.0, -0.001738946222088931)
(3.0, -0.003059216164423346)
(4.0, -0.004236829328663139)
(5.0, -0.00549335742167539)
(6.0, -0.006732282457237139)
(7.0, -0.007568310870316352)
(8.0, -0.008240658426026835)
(9.0, -0.008888142456563063)
(10.0, -0.009401616408731146)
(11.0, -0.009887198873674374)
(12.0, -0.010263372352459069)
(13.0, -0.010744544183763844)
(14.0, -0.01126495335687492)
};
\addlegendentry{CE-mixture}
\addplot+ [mark = {none}, dashed, blue, name path=Aplus, opacity=0.2]coordinates {
(1.0, 0.0005655849280861258)
(2.0, 0.00032175256846091627)
(3.0, 0.0008497895809695827)
(4.0, 0.0005963365316760945)
(5.0, 0.0007097290570611372)
(6.0, 0.0009103115533052007)
(7.0, 0.0009992775069641918)
(8.0, 0.0010856946682476802)
(9.0, 0.0010854983468129714)
(10.0, 0.0013713579241315613)
(11.0, 0.001440724016786289)
(12.0, 0.001440743614448835)
(13.0, 0.0014905872639687503)
(14.0, 0.0014947257311029132)
(15.0, 0.0014947257311029132)
};
\addplot+ [mark = {none}, dashed, blue, name path=Aminus, opacity=0.2]coordinates {
(1.0, -0.0014080404886638795)
(2.0, -0.0020527122876826728)
(3.0, -0.0041194921983168644)
(4.0, -0.004617360073314557)
(5.0, -0.00547905496162466)
(6.0, -0.006530083310505292)
(7.0, -0.006811088910002096)
(8.0, -0.007069093035684217)
(9.0, -0.007078086064347228)
(10.0, -0.007677575519638647)
(11.0, -0.007842080219286385)
(12.0, -0.0078422068473534)
(13.0, -0.007926160623578458)
(14.0, -0.00793300387884626)
(15.0, -0.00793300387884626)
};
\addplot+ [mark = {none}, dashed, red, name path=Bplus, opacity=0.2]coordinates {
(1.0, 0.0003587132390718365)
(2.0, 4.3909388608471536e-5)
(3.0, -0.0009563349640724199)
(4.0, -0.0011219410909393585)
(5.0, -0.0013518441148593534)
(6.0, -0.001982073189738983)
(7.0, -0.0026645191389055127)
(8.0, -0.003734694891290487)
(9.0, -0.004642661716129139)
(10.0, -0.006134757213424801)
(11.0, -0.00703998642994086)
(12.0, -0.007720644278804806)
(13.0, -0.008604485772167086)
(14.0, -0.009176232630158578)
(15.0, -0.011167141396342426)
};
\addplot+ [mark = {none}, dashed, red, name path=Bminus, opacity=0.2]coordinates {
(1.0, -0.0021907195294329092)
(2.0, -0.0036581440246077008)
(3.0, -0.004783395001757715)
(4.0, -0.006079796183072181)
(5.0, -0.007288892859236459)
(6.0, -0.009896585927615675)
(7.0, -0.010900952032078927)
(8.0, -0.01244784817110254)
(9.0, -0.014309857810683207)
(10.0, -0.015245265274740311)
(11.0, -0.01590616995635952)
(12.0, -0.016922964518311427)
(13.0, -0.01776210873605423)
(14.0, -0.0181665015010414)
(15.0, -0.02004995281435994)
};
\addplot+ [mark = {none}, dashed, green!50!black, name path=Cplus, opacity=0.2]coordinates {
(1.0, 0.000514959877286993)
(2.0, 0.0007179203270678671)
(3.0, 0.0005549019657232362)
(4.0, -0.0005752709235030138)
(5.0, -0.0011618180963984875)
(6.0, -0.001668216676329902)
(7.0, -0.0022504153689073573)
(8.0, -0.002992159232448518)
(9.0, -0.0034027054262190494)
(10.0, -0.003824399895506824)
(11.0, -0.004146653236500213)
(12.0, -0.00467764930747694)
(13.0, -0.0051020269786564935)
(14.0, -0.005600974474109703)
};
\addplot+ [mark = {none}, dashed, green!50!black, name path=Cminus, opacity=0.2]coordinates {
(1.0, -0.001339200878418195)
(2.0, -0.004195812771245729)
(3.0, -0.006673334294569929)
(4.0, -0.007898387733823266)
(5.0, -0.009824896746952291)
(6.0, -0.011796348238144376)
(7.0, -0.012886206371725346)
(8.0, -0.013489157619605152)
(9.0, -0.014373579486907078)
(10.0, -0.014978832921955468)
(11.0, -0.015627744510848536)
(12.0, -0.0158490953974412)
(13.0, -0.016387061388871194)
(14.0, -0.016928932239640135)
};
\addplot[blue!80, fill opacity=0.1] fill between[of=Aplus and Aminus];
                    \addplot[red!80, fill opacity=0.1] fill between[of=Bplus and Bminus];
                    \addplot[green!80, fill opacity=0.1] fill between[of=Cplus and Cminus];
\end{axis}

\end{tikzpicture}
}
  \caption{
    \label{fig:experiment_1b}
    Average optimal value for experiment (1B) when the initial mean is far from the global minimum with a wide covariance.
  }
\end{figure}

When the initial mean of the input distribution is placed far away from the global optimal, the CE-method tends to converge prematurely as shown in \cref{fig:experiment_1b}.
This scenario is illustrated in \cref{fig:example_1b}.
We can see that both CE-surrogate and CE-mixture perform well in this case.

\begin{figure}[!h]
  \centering
  \resizebox{0.7\columnwidth}{!}{%% Creator: Matplotlib, PGF backend
%%
%% To include the figure in your LaTeX document, write
%%   \input{<filename>.pgf}
%%
%% Make sure the required packages are loaded in your preamble
%%   \usepackage{pgf}
%%
%% Figures using additional raster images can only be included by \input if
%% they are in the same directory as the main LaTeX file. For loading figures
%% from other directories you can use the `import` package
%%   \usepackage{import}
%% and then include the figures with
%%   \import{<path to file>}{<filename>.pgf}
%%
%% Matplotlib used the following preamble
%%   \usepackage{fontspec}
%%   \setmainfont{DejaVuSans.ttf}[Path=C:/Users/mossr/.julia/conda/3/lib/site-packages/matplotlib/mpl-data/fonts/ttf/]
%%   \setsansfont{DejaVuSans.ttf}[Path=C:/Users/mossr/.julia/conda/3/lib/site-packages/matplotlib/mpl-data/fonts/ttf/]
%%   \setmonofont{DejaVuSansMono.ttf}[Path=C:/Users/mossr/.julia/conda/3/lib/site-packages/matplotlib/mpl-data/fonts/ttf/]
%%
\begingroup%
\makeatletter%
\begin{pgfpicture}%
% \pgfpathrectangle{\pgfpointorigin}{\pgfqpoint{6.400000in}{4.800000in}}%
\pgfpathrectangle{\pgfqpoint{1.0in}{0.0in}}{\pgfqpoint{4.0in}{4.8in}}%
\pgfusepath{use as bounding box, clip}%
\begin{pgfscope}%
\pgfsetbuttcap%
\pgfsetmiterjoin%
\definecolor{currentfill}{rgb}{1.000000,1.000000,1.000000}%
\pgfsetfillcolor{currentfill}%
\pgfsetlinewidth{0.000000pt}%
\definecolor{currentstroke}{rgb}{1.000000,1.000000,1.000000}%
\pgfsetstrokecolor{currentstroke}%
\pgfsetdash{}{0pt}%
\pgfpathmoveto{\pgfqpoint{0.000000in}{0.000000in}}%
\pgfpathlineto{\pgfqpoint{6.400000in}{0.000000in}}%
\pgfpathlineto{\pgfqpoint{6.400000in}{4.800000in}}%
\pgfpathlineto{\pgfqpoint{0.000000in}{4.800000in}}%
\pgfpathclose%
\pgfusepath{fill}%
\end{pgfscope}%
\begin{pgfscope}%
\pgfsetbuttcap%
\pgfsetmiterjoin%
\definecolor{currentfill}{rgb}{1.000000,1.000000,1.000000}%
\pgfsetfillcolor{currentfill}%
\pgfsetlinewidth{0.000000pt}%
\definecolor{currentstroke}{rgb}{0.000000,0.000000,0.000000}%
\pgfsetstrokecolor{currentstroke}%
\pgfsetstrokeopacity{0.000000}%
\pgfsetdash{}{0pt}%
\pgfpathmoveto{\pgfqpoint{1.432000in}{0.528000in}}%
\pgfpathlineto{\pgfqpoint{5.128000in}{0.528000in}}%
\pgfpathlineto{\pgfqpoint{5.128000in}{4.224000in}}%
\pgfpathlineto{\pgfqpoint{1.432000in}{4.224000in}}%
\pgfpathclose%
\pgfusepath{fill}%
\end{pgfscope}%
\begin{pgfscope}%
\pgfpathrectangle{\pgfqpoint{1.432000in}{0.528000in}}{\pgfqpoint{3.696000in}{3.696000in}}%
\pgfusepath{clip}%
\pgfsys@transformshift{1.432000in}{0.528000in}%
\pgftext[left,bottom]{\pgfimage[interpolate=true,width=3.700000in,height=3.700000in]{figures/cem_variants/example1b-img0.png}}%
\end{pgfscope}%
\begin{pgfscope}%
\pgfpathrectangle{\pgfqpoint{1.432000in}{0.528000in}}{\pgfqpoint{3.696000in}{3.696000in}}%
\pgfusepath{clip}%
\pgfsetbuttcap%
\pgfsetroundjoin%
\definecolor{currentfill}{rgb}{0.000000,0.000000,0.000000}%
\pgfsetfillcolor{currentfill}%
\pgfsetlinewidth{0.501875pt}%
\definecolor{currentstroke}{rgb}{1.000000,1.000000,1.000000}%
\pgfsetstrokecolor{currentstroke}%
\pgfsetdash{}{0pt}%
\pgfsys@defobject{currentmarker}{\pgfqpoint{-0.018373in}{-0.018373in}}{\pgfqpoint{0.018373in}{0.018373in}}{%
\pgfpathmoveto{\pgfqpoint{0.000000in}{-0.018373in}}%
\pgfpathcurveto{\pgfqpoint{0.004873in}{-0.018373in}}{\pgfqpoint{0.009546in}{-0.016437in}}{\pgfqpoint{0.012992in}{-0.012992in}}%
\pgfpathcurveto{\pgfqpoint{0.016437in}{-0.009546in}}{\pgfqpoint{0.018373in}{-0.004873in}}{\pgfqpoint{0.018373in}{0.000000in}}%
\pgfpathcurveto{\pgfqpoint{0.018373in}{0.004873in}}{\pgfqpoint{0.016437in}{0.009546in}}{\pgfqpoint{0.012992in}{0.012992in}}%
\pgfpathcurveto{\pgfqpoint{0.009546in}{0.016437in}}{\pgfqpoint{0.004873in}{0.018373in}}{\pgfqpoint{0.000000in}{0.018373in}}%
\pgfpathcurveto{\pgfqpoint{-0.004873in}{0.018373in}}{\pgfqpoint{-0.009546in}{0.016437in}}{\pgfqpoint{-0.012992in}{0.012992in}}%
\pgfpathcurveto{\pgfqpoint{-0.016437in}{0.009546in}}{\pgfqpoint{-0.018373in}{0.004873in}}{\pgfqpoint{-0.018373in}{0.000000in}}%
\pgfpathcurveto{\pgfqpoint{-0.018373in}{-0.004873in}}{\pgfqpoint{-0.016437in}{-0.009546in}}{\pgfqpoint{-0.012992in}{-0.012992in}}%
\pgfpathcurveto{\pgfqpoint{-0.009546in}{-0.016437in}}{\pgfqpoint{-0.004873in}{-0.018373in}}{\pgfqpoint{0.000000in}{-0.018373in}}%
\pgfpathclose%
\pgfusepath{stroke,fill}%
}%
\begin{pgfscope}%
\pgfsys@transformshift{3.466250in}{2.684576in}%
\pgfsys@useobject{currentmarker}{}%
\end{pgfscope}%
\end{pgfscope}%
\begin{pgfscope}%
\pgfpathrectangle{\pgfqpoint{1.432000in}{0.528000in}}{\pgfqpoint{3.696000in}{3.696000in}}%
\pgfusepath{clip}%
\pgfsetbuttcap%
\pgfsetroundjoin%
\definecolor{currentfill}{rgb}{0.000000,0.000000,0.000000}%
\pgfsetfillcolor{currentfill}%
\pgfsetlinewidth{0.501875pt}%
\definecolor{currentstroke}{rgb}{1.000000,1.000000,1.000000}%
\pgfsetstrokecolor{currentstroke}%
\pgfsetdash{}{0pt}%
\pgfsys@defobject{currentmarker}{\pgfqpoint{-0.018373in}{-0.018373in}}{\pgfqpoint{0.018373in}{0.018373in}}{%
\pgfpathmoveto{\pgfqpoint{0.000000in}{-0.018373in}}%
\pgfpathcurveto{\pgfqpoint{0.004873in}{-0.018373in}}{\pgfqpoint{0.009546in}{-0.016437in}}{\pgfqpoint{0.012992in}{-0.012992in}}%
\pgfpathcurveto{\pgfqpoint{0.016437in}{-0.009546in}}{\pgfqpoint{0.018373in}{-0.004873in}}{\pgfqpoint{0.018373in}{0.000000in}}%
\pgfpathcurveto{\pgfqpoint{0.018373in}{0.004873in}}{\pgfqpoint{0.016437in}{0.009546in}}{\pgfqpoint{0.012992in}{0.012992in}}%
\pgfpathcurveto{\pgfqpoint{0.009546in}{0.016437in}}{\pgfqpoint{0.004873in}{0.018373in}}{\pgfqpoint{0.000000in}{0.018373in}}%
\pgfpathcurveto{\pgfqpoint{-0.004873in}{0.018373in}}{\pgfqpoint{-0.009546in}{0.016437in}}{\pgfqpoint{-0.012992in}{0.012992in}}%
\pgfpathcurveto{\pgfqpoint{-0.016437in}{0.009546in}}{\pgfqpoint{-0.018373in}{0.004873in}}{\pgfqpoint{-0.018373in}{0.000000in}}%
\pgfpathcurveto{\pgfqpoint{-0.018373in}{-0.004873in}}{\pgfqpoint{-0.016437in}{-0.009546in}}{\pgfqpoint{-0.012992in}{-0.012992in}}%
\pgfpathcurveto{\pgfqpoint{-0.009546in}{-0.016437in}}{\pgfqpoint{-0.004873in}{-0.018373in}}{\pgfqpoint{0.000000in}{-0.018373in}}%
\pgfpathclose%
\pgfusepath{stroke,fill}%
}%
\begin{pgfscope}%
\pgfsys@transformshift{2.179973in}{2.119944in}%
\pgfsys@useobject{currentmarker}{}%
\end{pgfscope}%
\end{pgfscope}%
\begin{pgfscope}%
\pgfpathrectangle{\pgfqpoint{1.432000in}{0.528000in}}{\pgfqpoint{3.696000in}{3.696000in}}%
\pgfusepath{clip}%
\pgfsetbuttcap%
\pgfsetroundjoin%
\definecolor{currentfill}{rgb}{0.000000,0.000000,0.000000}%
\pgfsetfillcolor{currentfill}%
\pgfsetlinewidth{0.501875pt}%
\definecolor{currentstroke}{rgb}{1.000000,1.000000,1.000000}%
\pgfsetstrokecolor{currentstroke}%
\pgfsetdash{}{0pt}%
\pgfsys@defobject{currentmarker}{\pgfqpoint{-0.018373in}{-0.018373in}}{\pgfqpoint{0.018373in}{0.018373in}}{%
\pgfpathmoveto{\pgfqpoint{0.000000in}{-0.018373in}}%
\pgfpathcurveto{\pgfqpoint{0.004873in}{-0.018373in}}{\pgfqpoint{0.009546in}{-0.016437in}}{\pgfqpoint{0.012992in}{-0.012992in}}%
\pgfpathcurveto{\pgfqpoint{0.016437in}{-0.009546in}}{\pgfqpoint{0.018373in}{-0.004873in}}{\pgfqpoint{0.018373in}{0.000000in}}%
\pgfpathcurveto{\pgfqpoint{0.018373in}{0.004873in}}{\pgfqpoint{0.016437in}{0.009546in}}{\pgfqpoint{0.012992in}{0.012992in}}%
\pgfpathcurveto{\pgfqpoint{0.009546in}{0.016437in}}{\pgfqpoint{0.004873in}{0.018373in}}{\pgfqpoint{0.000000in}{0.018373in}}%
\pgfpathcurveto{\pgfqpoint{-0.004873in}{0.018373in}}{\pgfqpoint{-0.009546in}{0.016437in}}{\pgfqpoint{-0.012992in}{0.012992in}}%
\pgfpathcurveto{\pgfqpoint{-0.016437in}{0.009546in}}{\pgfqpoint{-0.018373in}{0.004873in}}{\pgfqpoint{-0.018373in}{0.000000in}}%
\pgfpathcurveto{\pgfqpoint{-0.018373in}{-0.004873in}}{\pgfqpoint{-0.016437in}{-0.009546in}}{\pgfqpoint{-0.012992in}{-0.012992in}}%
\pgfpathcurveto{\pgfqpoint{-0.009546in}{-0.016437in}}{\pgfqpoint{-0.004873in}{-0.018373in}}{\pgfqpoint{0.000000in}{-0.018373in}}%
\pgfpathclose%
\pgfusepath{stroke,fill}%
}%
\begin{pgfscope}%
\pgfsys@transformshift{1.833019in}{2.582119in}%
\pgfsys@useobject{currentmarker}{}%
\end{pgfscope}%
\end{pgfscope}%
\begin{pgfscope}%
\pgfpathrectangle{\pgfqpoint{1.432000in}{0.528000in}}{\pgfqpoint{3.696000in}{3.696000in}}%
\pgfusepath{clip}%
\pgfsetbuttcap%
\pgfsetroundjoin%
\definecolor{currentfill}{rgb}{0.000000,0.000000,0.000000}%
\pgfsetfillcolor{currentfill}%
\pgfsetlinewidth{0.501875pt}%
\definecolor{currentstroke}{rgb}{1.000000,1.000000,1.000000}%
\pgfsetstrokecolor{currentstroke}%
\pgfsetdash{}{0pt}%
\pgfsys@defobject{currentmarker}{\pgfqpoint{-0.018373in}{-0.018373in}}{\pgfqpoint{0.018373in}{0.018373in}}{%
\pgfpathmoveto{\pgfqpoint{0.000000in}{-0.018373in}}%
\pgfpathcurveto{\pgfqpoint{0.004873in}{-0.018373in}}{\pgfqpoint{0.009546in}{-0.016437in}}{\pgfqpoint{0.012992in}{-0.012992in}}%
\pgfpathcurveto{\pgfqpoint{0.016437in}{-0.009546in}}{\pgfqpoint{0.018373in}{-0.004873in}}{\pgfqpoint{0.018373in}{0.000000in}}%
\pgfpathcurveto{\pgfqpoint{0.018373in}{0.004873in}}{\pgfqpoint{0.016437in}{0.009546in}}{\pgfqpoint{0.012992in}{0.012992in}}%
\pgfpathcurveto{\pgfqpoint{0.009546in}{0.016437in}}{\pgfqpoint{0.004873in}{0.018373in}}{\pgfqpoint{0.000000in}{0.018373in}}%
\pgfpathcurveto{\pgfqpoint{-0.004873in}{0.018373in}}{\pgfqpoint{-0.009546in}{0.016437in}}{\pgfqpoint{-0.012992in}{0.012992in}}%
\pgfpathcurveto{\pgfqpoint{-0.016437in}{0.009546in}}{\pgfqpoint{-0.018373in}{0.004873in}}{\pgfqpoint{-0.018373in}{0.000000in}}%
\pgfpathcurveto{\pgfqpoint{-0.018373in}{-0.004873in}}{\pgfqpoint{-0.016437in}{-0.009546in}}{\pgfqpoint{-0.012992in}{-0.012992in}}%
\pgfpathcurveto{\pgfqpoint{-0.009546in}{-0.016437in}}{\pgfqpoint{-0.004873in}{-0.018373in}}{\pgfqpoint{0.000000in}{-0.018373in}}%
\pgfpathclose%
\pgfusepath{stroke,fill}%
}%
\begin{pgfscope}%
\pgfsys@transformshift{6.337699in}{-1.123539in}%
\pgfsys@useobject{currentmarker}{}%
\end{pgfscope}%
\end{pgfscope}%
\begin{pgfscope}%
\pgfpathrectangle{\pgfqpoint{1.432000in}{0.528000in}}{\pgfqpoint{3.696000in}{3.696000in}}%
\pgfusepath{clip}%
\pgfsetbuttcap%
\pgfsetroundjoin%
\definecolor{currentfill}{rgb}{0.000000,0.000000,0.000000}%
\pgfsetfillcolor{currentfill}%
\pgfsetlinewidth{0.501875pt}%
\definecolor{currentstroke}{rgb}{1.000000,1.000000,1.000000}%
\pgfsetstrokecolor{currentstroke}%
\pgfsetdash{}{0pt}%
\pgfsys@defobject{currentmarker}{\pgfqpoint{-0.018373in}{-0.018373in}}{\pgfqpoint{0.018373in}{0.018373in}}{%
\pgfpathmoveto{\pgfqpoint{0.000000in}{-0.018373in}}%
\pgfpathcurveto{\pgfqpoint{0.004873in}{-0.018373in}}{\pgfqpoint{0.009546in}{-0.016437in}}{\pgfqpoint{0.012992in}{-0.012992in}}%
\pgfpathcurveto{\pgfqpoint{0.016437in}{-0.009546in}}{\pgfqpoint{0.018373in}{-0.004873in}}{\pgfqpoint{0.018373in}{0.000000in}}%
\pgfpathcurveto{\pgfqpoint{0.018373in}{0.004873in}}{\pgfqpoint{0.016437in}{0.009546in}}{\pgfqpoint{0.012992in}{0.012992in}}%
\pgfpathcurveto{\pgfqpoint{0.009546in}{0.016437in}}{\pgfqpoint{0.004873in}{0.018373in}}{\pgfqpoint{0.000000in}{0.018373in}}%
\pgfpathcurveto{\pgfqpoint{-0.004873in}{0.018373in}}{\pgfqpoint{-0.009546in}{0.016437in}}{\pgfqpoint{-0.012992in}{0.012992in}}%
\pgfpathcurveto{\pgfqpoint{-0.016437in}{0.009546in}}{\pgfqpoint{-0.018373in}{0.004873in}}{\pgfqpoint{-0.018373in}{0.000000in}}%
\pgfpathcurveto{\pgfqpoint{-0.018373in}{-0.004873in}}{\pgfqpoint{-0.016437in}{-0.009546in}}{\pgfqpoint{-0.012992in}{-0.012992in}}%
\pgfpathcurveto{\pgfqpoint{-0.009546in}{-0.016437in}}{\pgfqpoint{-0.004873in}{-0.018373in}}{\pgfqpoint{0.000000in}{-0.018373in}}%
\pgfpathclose%
\pgfusepath{stroke,fill}%
}%
\begin{pgfscope}%
\pgfsys@transformshift{3.800679in}{2.755041in}%
\pgfsys@useobject{currentmarker}{}%
\end{pgfscope}%
\end{pgfscope}%
\begin{pgfscope}%
\pgfpathrectangle{\pgfqpoint{1.432000in}{0.528000in}}{\pgfqpoint{3.696000in}{3.696000in}}%
\pgfusepath{clip}%
\pgfsetbuttcap%
\pgfsetroundjoin%
\definecolor{currentfill}{rgb}{0.000000,0.000000,0.000000}%
\pgfsetfillcolor{currentfill}%
\pgfsetlinewidth{0.501875pt}%
\definecolor{currentstroke}{rgb}{1.000000,1.000000,1.000000}%
\pgfsetstrokecolor{currentstroke}%
\pgfsetdash{}{0pt}%
\pgfsys@defobject{currentmarker}{\pgfqpoint{-0.018373in}{-0.018373in}}{\pgfqpoint{0.018373in}{0.018373in}}{%
\pgfpathmoveto{\pgfqpoint{0.000000in}{-0.018373in}}%
\pgfpathcurveto{\pgfqpoint{0.004873in}{-0.018373in}}{\pgfqpoint{0.009546in}{-0.016437in}}{\pgfqpoint{0.012992in}{-0.012992in}}%
\pgfpathcurveto{\pgfqpoint{0.016437in}{-0.009546in}}{\pgfqpoint{0.018373in}{-0.004873in}}{\pgfqpoint{0.018373in}{0.000000in}}%
\pgfpathcurveto{\pgfqpoint{0.018373in}{0.004873in}}{\pgfqpoint{0.016437in}{0.009546in}}{\pgfqpoint{0.012992in}{0.012992in}}%
\pgfpathcurveto{\pgfqpoint{0.009546in}{0.016437in}}{\pgfqpoint{0.004873in}{0.018373in}}{\pgfqpoint{0.000000in}{0.018373in}}%
\pgfpathcurveto{\pgfqpoint{-0.004873in}{0.018373in}}{\pgfqpoint{-0.009546in}{0.016437in}}{\pgfqpoint{-0.012992in}{0.012992in}}%
\pgfpathcurveto{\pgfqpoint{-0.016437in}{0.009546in}}{\pgfqpoint{-0.018373in}{0.004873in}}{\pgfqpoint{-0.018373in}{0.000000in}}%
\pgfpathcurveto{\pgfqpoint{-0.018373in}{-0.004873in}}{\pgfqpoint{-0.016437in}{-0.009546in}}{\pgfqpoint{-0.012992in}{-0.012992in}}%
\pgfpathcurveto{\pgfqpoint{-0.009546in}{-0.016437in}}{\pgfqpoint{-0.004873in}{-0.018373in}}{\pgfqpoint{0.000000in}{-0.018373in}}%
\pgfpathclose%
\pgfusepath{stroke,fill}%
}%
\begin{pgfscope}%
\pgfsys@transformshift{3.877924in}{3.519472in}%
\pgfsys@useobject{currentmarker}{}%
\end{pgfscope}%
\end{pgfscope}%
\begin{pgfscope}%
\pgfpathrectangle{\pgfqpoint{1.432000in}{0.528000in}}{\pgfqpoint{3.696000in}{3.696000in}}%
\pgfusepath{clip}%
\pgfsetbuttcap%
\pgfsetroundjoin%
\definecolor{currentfill}{rgb}{0.000000,0.000000,0.000000}%
\pgfsetfillcolor{currentfill}%
\pgfsetlinewidth{0.501875pt}%
\definecolor{currentstroke}{rgb}{1.000000,1.000000,1.000000}%
\pgfsetstrokecolor{currentstroke}%
\pgfsetdash{}{0pt}%
\pgfsys@defobject{currentmarker}{\pgfqpoint{-0.018373in}{-0.018373in}}{\pgfqpoint{0.018373in}{0.018373in}}{%
\pgfpathmoveto{\pgfqpoint{0.000000in}{-0.018373in}}%
\pgfpathcurveto{\pgfqpoint{0.004873in}{-0.018373in}}{\pgfqpoint{0.009546in}{-0.016437in}}{\pgfqpoint{0.012992in}{-0.012992in}}%
\pgfpathcurveto{\pgfqpoint{0.016437in}{-0.009546in}}{\pgfqpoint{0.018373in}{-0.004873in}}{\pgfqpoint{0.018373in}{0.000000in}}%
\pgfpathcurveto{\pgfqpoint{0.018373in}{0.004873in}}{\pgfqpoint{0.016437in}{0.009546in}}{\pgfqpoint{0.012992in}{0.012992in}}%
\pgfpathcurveto{\pgfqpoint{0.009546in}{0.016437in}}{\pgfqpoint{0.004873in}{0.018373in}}{\pgfqpoint{0.000000in}{0.018373in}}%
\pgfpathcurveto{\pgfqpoint{-0.004873in}{0.018373in}}{\pgfqpoint{-0.009546in}{0.016437in}}{\pgfqpoint{-0.012992in}{0.012992in}}%
\pgfpathcurveto{\pgfqpoint{-0.016437in}{0.009546in}}{\pgfqpoint{-0.018373in}{0.004873in}}{\pgfqpoint{-0.018373in}{0.000000in}}%
\pgfpathcurveto{\pgfqpoint{-0.018373in}{-0.004873in}}{\pgfqpoint{-0.016437in}{-0.009546in}}{\pgfqpoint{-0.012992in}{-0.012992in}}%
\pgfpathcurveto{\pgfqpoint{-0.009546in}{-0.016437in}}{\pgfqpoint{-0.004873in}{-0.018373in}}{\pgfqpoint{0.000000in}{-0.018373in}}%
\pgfpathclose%
\pgfusepath{stroke,fill}%
}%
\begin{pgfscope}%
\pgfsys@transformshift{3.698379in}{1.384199in}%
\pgfsys@useobject{currentmarker}{}%
\end{pgfscope}%
\end{pgfscope}%
\begin{pgfscope}%
\pgfpathrectangle{\pgfqpoint{1.432000in}{0.528000in}}{\pgfqpoint{3.696000in}{3.696000in}}%
\pgfusepath{clip}%
\pgfsetbuttcap%
\pgfsetroundjoin%
\definecolor{currentfill}{rgb}{0.000000,0.000000,0.000000}%
\pgfsetfillcolor{currentfill}%
\pgfsetlinewidth{0.501875pt}%
\definecolor{currentstroke}{rgb}{1.000000,1.000000,1.000000}%
\pgfsetstrokecolor{currentstroke}%
\pgfsetdash{}{0pt}%
\pgfsys@defobject{currentmarker}{\pgfqpoint{-0.018373in}{-0.018373in}}{\pgfqpoint{0.018373in}{0.018373in}}{%
\pgfpathmoveto{\pgfqpoint{0.000000in}{-0.018373in}}%
\pgfpathcurveto{\pgfqpoint{0.004873in}{-0.018373in}}{\pgfqpoint{0.009546in}{-0.016437in}}{\pgfqpoint{0.012992in}{-0.012992in}}%
\pgfpathcurveto{\pgfqpoint{0.016437in}{-0.009546in}}{\pgfqpoint{0.018373in}{-0.004873in}}{\pgfqpoint{0.018373in}{0.000000in}}%
\pgfpathcurveto{\pgfqpoint{0.018373in}{0.004873in}}{\pgfqpoint{0.016437in}{0.009546in}}{\pgfqpoint{0.012992in}{0.012992in}}%
\pgfpathcurveto{\pgfqpoint{0.009546in}{0.016437in}}{\pgfqpoint{0.004873in}{0.018373in}}{\pgfqpoint{0.000000in}{0.018373in}}%
\pgfpathcurveto{\pgfqpoint{-0.004873in}{0.018373in}}{\pgfqpoint{-0.009546in}{0.016437in}}{\pgfqpoint{-0.012992in}{0.012992in}}%
\pgfpathcurveto{\pgfqpoint{-0.016437in}{0.009546in}}{\pgfqpoint{-0.018373in}{0.004873in}}{\pgfqpoint{-0.018373in}{0.000000in}}%
\pgfpathcurveto{\pgfqpoint{-0.018373in}{-0.004873in}}{\pgfqpoint{-0.016437in}{-0.009546in}}{\pgfqpoint{-0.012992in}{-0.012992in}}%
\pgfpathcurveto{\pgfqpoint{-0.009546in}{-0.016437in}}{\pgfqpoint{-0.004873in}{-0.018373in}}{\pgfqpoint{0.000000in}{-0.018373in}}%
\pgfpathclose%
\pgfusepath{stroke,fill}%
}%
\begin{pgfscope}%
\pgfsys@transformshift{3.625946in}{2.062443in}%
\pgfsys@useobject{currentmarker}{}%
\end{pgfscope}%
\end{pgfscope}%
\begin{pgfscope}%
\pgfpathrectangle{\pgfqpoint{1.432000in}{0.528000in}}{\pgfqpoint{3.696000in}{3.696000in}}%
\pgfusepath{clip}%
\pgfsetbuttcap%
\pgfsetroundjoin%
\definecolor{currentfill}{rgb}{0.000000,0.000000,0.000000}%
\pgfsetfillcolor{currentfill}%
\pgfsetlinewidth{0.501875pt}%
\definecolor{currentstroke}{rgb}{1.000000,1.000000,1.000000}%
\pgfsetstrokecolor{currentstroke}%
\pgfsetdash{}{0pt}%
\pgfsys@defobject{currentmarker}{\pgfqpoint{-0.018373in}{-0.018373in}}{\pgfqpoint{0.018373in}{0.018373in}}{%
\pgfpathmoveto{\pgfqpoint{0.000000in}{-0.018373in}}%
\pgfpathcurveto{\pgfqpoint{0.004873in}{-0.018373in}}{\pgfqpoint{0.009546in}{-0.016437in}}{\pgfqpoint{0.012992in}{-0.012992in}}%
\pgfpathcurveto{\pgfqpoint{0.016437in}{-0.009546in}}{\pgfqpoint{0.018373in}{-0.004873in}}{\pgfqpoint{0.018373in}{0.000000in}}%
\pgfpathcurveto{\pgfqpoint{0.018373in}{0.004873in}}{\pgfqpoint{0.016437in}{0.009546in}}{\pgfqpoint{0.012992in}{0.012992in}}%
\pgfpathcurveto{\pgfqpoint{0.009546in}{0.016437in}}{\pgfqpoint{0.004873in}{0.018373in}}{\pgfqpoint{0.000000in}{0.018373in}}%
\pgfpathcurveto{\pgfqpoint{-0.004873in}{0.018373in}}{\pgfqpoint{-0.009546in}{0.016437in}}{\pgfqpoint{-0.012992in}{0.012992in}}%
\pgfpathcurveto{\pgfqpoint{-0.016437in}{0.009546in}}{\pgfqpoint{-0.018373in}{0.004873in}}{\pgfqpoint{-0.018373in}{0.000000in}}%
\pgfpathcurveto{\pgfqpoint{-0.018373in}{-0.004873in}}{\pgfqpoint{-0.016437in}{-0.009546in}}{\pgfqpoint{-0.012992in}{-0.012992in}}%
\pgfpathcurveto{\pgfqpoint{-0.009546in}{-0.016437in}}{\pgfqpoint{-0.004873in}{-0.018373in}}{\pgfqpoint{0.000000in}{-0.018373in}}%
\pgfpathclose%
\pgfusepath{stroke,fill}%
}%
\begin{pgfscope}%
\pgfsys@transformshift{2.041964in}{-0.414584in}%
\pgfsys@useobject{currentmarker}{}%
\end{pgfscope}%
\end{pgfscope}%
\begin{pgfscope}%
\pgfpathrectangle{\pgfqpoint{1.432000in}{0.528000in}}{\pgfqpoint{3.696000in}{3.696000in}}%
\pgfusepath{clip}%
\pgfsetbuttcap%
\pgfsetroundjoin%
\definecolor{currentfill}{rgb}{0.000000,0.000000,0.000000}%
\pgfsetfillcolor{currentfill}%
\pgfsetlinewidth{0.501875pt}%
\definecolor{currentstroke}{rgb}{1.000000,1.000000,1.000000}%
\pgfsetstrokecolor{currentstroke}%
\pgfsetdash{}{0pt}%
\pgfsys@defobject{currentmarker}{\pgfqpoint{-0.018373in}{-0.018373in}}{\pgfqpoint{0.018373in}{0.018373in}}{%
\pgfpathmoveto{\pgfqpoint{0.000000in}{-0.018373in}}%
\pgfpathcurveto{\pgfqpoint{0.004873in}{-0.018373in}}{\pgfqpoint{0.009546in}{-0.016437in}}{\pgfqpoint{0.012992in}{-0.012992in}}%
\pgfpathcurveto{\pgfqpoint{0.016437in}{-0.009546in}}{\pgfqpoint{0.018373in}{-0.004873in}}{\pgfqpoint{0.018373in}{0.000000in}}%
\pgfpathcurveto{\pgfqpoint{0.018373in}{0.004873in}}{\pgfqpoint{0.016437in}{0.009546in}}{\pgfqpoint{0.012992in}{0.012992in}}%
\pgfpathcurveto{\pgfqpoint{0.009546in}{0.016437in}}{\pgfqpoint{0.004873in}{0.018373in}}{\pgfqpoint{0.000000in}{0.018373in}}%
\pgfpathcurveto{\pgfqpoint{-0.004873in}{0.018373in}}{\pgfqpoint{-0.009546in}{0.016437in}}{\pgfqpoint{-0.012992in}{0.012992in}}%
\pgfpathcurveto{\pgfqpoint{-0.016437in}{0.009546in}}{\pgfqpoint{-0.018373in}{0.004873in}}{\pgfqpoint{-0.018373in}{0.000000in}}%
\pgfpathcurveto{\pgfqpoint{-0.018373in}{-0.004873in}}{\pgfqpoint{-0.016437in}{-0.009546in}}{\pgfqpoint{-0.012992in}{-0.012992in}}%
\pgfpathcurveto{\pgfqpoint{-0.009546in}{-0.016437in}}{\pgfqpoint{-0.004873in}{-0.018373in}}{\pgfqpoint{0.000000in}{-0.018373in}}%
\pgfpathclose%
\pgfusepath{stroke,fill}%
}%
\begin{pgfscope}%
\pgfsys@transformshift{3.212534in}{1.046006in}%
\pgfsys@useobject{currentmarker}{}%
\end{pgfscope}%
\end{pgfscope}%
\begin{pgfscope}%
\pgfpathrectangle{\pgfqpoint{1.432000in}{0.528000in}}{\pgfqpoint{3.696000in}{3.696000in}}%
\pgfusepath{clip}%
\pgfsetbuttcap%
\pgfsetroundjoin%
\definecolor{currentfill}{rgb}{1.000000,1.000000,1.000000}%
\pgfsetfillcolor{currentfill}%
\pgfsetlinewidth{0.501875pt}%
\definecolor{currentstroke}{rgb}{0.000000,0.000000,0.000000}%
\pgfsetstrokecolor{currentstroke}%
\pgfsetdash{}{0pt}%
\pgfsys@defobject{currentmarker}{\pgfqpoint{-0.018373in}{-0.018373in}}{\pgfqpoint{0.018373in}{0.018373in}}{%
\pgfpathmoveto{\pgfqpoint{0.000000in}{-0.018373in}}%
\pgfpathcurveto{\pgfqpoint{0.004873in}{-0.018373in}}{\pgfqpoint{0.009546in}{-0.016437in}}{\pgfqpoint{0.012992in}{-0.012992in}}%
\pgfpathcurveto{\pgfqpoint{0.016437in}{-0.009546in}}{\pgfqpoint{0.018373in}{-0.004873in}}{\pgfqpoint{0.018373in}{0.000000in}}%
\pgfpathcurveto{\pgfqpoint{0.018373in}{0.004873in}}{\pgfqpoint{0.016437in}{0.009546in}}{\pgfqpoint{0.012992in}{0.012992in}}%
\pgfpathcurveto{\pgfqpoint{0.009546in}{0.016437in}}{\pgfqpoint{0.004873in}{0.018373in}}{\pgfqpoint{0.000000in}{0.018373in}}%
\pgfpathcurveto{\pgfqpoint{-0.004873in}{0.018373in}}{\pgfqpoint{-0.009546in}{0.016437in}}{\pgfqpoint{-0.012992in}{0.012992in}}%
\pgfpathcurveto{\pgfqpoint{-0.016437in}{0.009546in}}{\pgfqpoint{-0.018373in}{0.004873in}}{\pgfqpoint{-0.018373in}{0.000000in}}%
\pgfpathcurveto{\pgfqpoint{-0.018373in}{-0.004873in}}{\pgfqpoint{-0.016437in}{-0.009546in}}{\pgfqpoint{-0.012992in}{-0.012992in}}%
\pgfpathcurveto{\pgfqpoint{-0.009546in}{-0.016437in}}{\pgfqpoint{-0.004873in}{-0.018373in}}{\pgfqpoint{0.000000in}{-0.018373in}}%
\pgfpathclose%
\pgfusepath{stroke,fill}%
}%
\begin{pgfscope}%
\pgfsys@transformshift{3.767108in}{3.388022in}%
\pgfsys@useobject{currentmarker}{}%
\end{pgfscope}%
\begin{pgfscope}%
\pgfsys@transformshift{3.854908in}{3.232338in}%
\pgfsys@useobject{currentmarker}{}%
\end{pgfscope}%
\begin{pgfscope}%
\pgfsys@transformshift{3.489383in}{3.552502in}%
\pgfsys@useobject{currentmarker}{}%
\end{pgfscope}%
\begin{pgfscope}%
\pgfsys@transformshift{4.153498in}{3.221645in}%
\pgfsys@useobject{currentmarker}{}%
\end{pgfscope}%
\begin{pgfscope}%
\pgfsys@transformshift{3.721939in}{3.052866in}%
\pgfsys@useobject{currentmarker}{}%
\end{pgfscope}%
\begin{pgfscope}%
\pgfsys@transformshift{3.561904in}{3.105633in}%
\pgfsys@useobject{currentmarker}{}%
\end{pgfscope}%
\begin{pgfscope}%
\pgfsys@transformshift{3.727049in}{3.004720in}%
\pgfsys@useobject{currentmarker}{}%
\end{pgfscope}%
\begin{pgfscope}%
\pgfsys@transformshift{3.459884in}{3.094646in}%
\pgfsys@useobject{currentmarker}{}%
\end{pgfscope}%
\begin{pgfscope}%
\pgfsys@transformshift{4.002040in}{2.926700in}%
\pgfsys@useobject{currentmarker}{}%
\end{pgfscope}%
\begin{pgfscope}%
\pgfsys@transformshift{3.228311in}{3.985264in}%
\pgfsys@useobject{currentmarker}{}%
\end{pgfscope}%
\begin{pgfscope}%
\pgfsys@transformshift{3.198559in}{3.058850in}%
\pgfsys@useobject{currentmarker}{}%
\end{pgfscope}%
\begin{pgfscope}%
\pgfsys@transformshift{3.207265in}{3.035694in}%
\pgfsys@useobject{currentmarker}{}%
\end{pgfscope}%
\begin{pgfscope}%
\pgfsys@transformshift{3.411530in}{2.795680in}%
\pgfsys@useobject{currentmarker}{}%
\end{pgfscope}%
\begin{pgfscope}%
\pgfsys@transformshift{2.958975in}{3.226432in}%
\pgfsys@useobject{currentmarker}{}%
\end{pgfscope}%
\begin{pgfscope}%
\pgfsys@transformshift{3.340771in}{4.324225in}%
\pgfsys@useobject{currentmarker}{}%
\end{pgfscope}%
\begin{pgfscope}%
\pgfsys@transformshift{3.555160in}{2.594930in}%
\pgfsys@useobject{currentmarker}{}%
\end{pgfscope}%
\begin{pgfscope}%
\pgfsys@transformshift{2.942991in}{3.892008in}%
\pgfsys@useobject{currentmarker}{}%
\end{pgfscope}%
\begin{pgfscope}%
\pgfsys@transformshift{2.937226in}{3.931172in}%
\pgfsys@useobject{currentmarker}{}%
\end{pgfscope}%
\begin{pgfscope}%
\pgfsys@transformshift{3.342402in}{2.602734in}%
\pgfsys@useobject{currentmarker}{}%
\end{pgfscope}%
\begin{pgfscope}%
\pgfsys@transformshift{3.256282in}{2.645427in}%
\pgfsys@useobject{currentmarker}{}%
\end{pgfscope}%
\begin{pgfscope}%
\pgfsys@transformshift{2.947868in}{4.056552in}%
\pgfsys@useobject{currentmarker}{}%
\end{pgfscope}%
\begin{pgfscope}%
\pgfsys@transformshift{3.140240in}{2.626219in}%
\pgfsys@useobject{currentmarker}{}%
\end{pgfscope}%
\begin{pgfscope}%
\pgfsys@transformshift{4.994570in}{3.185911in}%
\pgfsys@useobject{currentmarker}{}%
\end{pgfscope}%
\begin{pgfscope}%
\pgfsys@transformshift{4.800802in}{4.275234in}%
\pgfsys@useobject{currentmarker}{}%
\end{pgfscope}%
\begin{pgfscope}%
\pgfsys@transformshift{3.966170in}{4.874295in}%
\pgfsys@useobject{currentmarker}{}%
\end{pgfscope}%
\begin{pgfscope}%
\pgfsys@transformshift{2.445224in}{3.575034in}%
\pgfsys@useobject{currentmarker}{}%
\end{pgfscope}%
\begin{pgfscope}%
\pgfsys@transformshift{2.655100in}{4.289596in}%
\pgfsys@useobject{currentmarker}{}%
\end{pgfscope}%
\begin{pgfscope}%
\pgfsys@transformshift{4.044445in}{5.023445in}%
\pgfsys@useobject{currentmarker}{}%
\end{pgfscope}%
\begin{pgfscope}%
\pgfsys@transformshift{3.474291in}{1.988683in}%
\pgfsys@useobject{currentmarker}{}%
\end{pgfscope}%
\begin{pgfscope}%
\pgfsys@transformshift{4.184926in}{1.888257in}%
\pgfsys@useobject{currentmarker}{}%
\end{pgfscope}%
\begin{pgfscope}%
\pgfsys@transformshift{4.742332in}{4.968778in}%
\pgfsys@useobject{currentmarker}{}%
\end{pgfscope}%
\begin{pgfscope}%
\pgfsys@transformshift{5.532920in}{4.003333in}%
\pgfsys@useobject{currentmarker}{}%
\end{pgfscope}%
\begin{pgfscope}%
\pgfsys@transformshift{2.131186in}{3.031635in}%
\pgfsys@useobject{currentmarker}{}%
\end{pgfscope}%
\begin{pgfscope}%
\pgfsys@transformshift{2.364746in}{2.447749in}%
\pgfsys@useobject{currentmarker}{}%
\end{pgfscope}%
\begin{pgfscope}%
\pgfsys@transformshift{4.849666in}{1.897394in}%
\pgfsys@useobject{currentmarker}{}%
\end{pgfscope}%
\begin{pgfscope}%
\pgfsys@transformshift{3.613073in}{1.636745in}%
\pgfsys@useobject{currentmarker}{}%
\end{pgfscope}%
\begin{pgfscope}%
\pgfsys@transformshift{2.932568in}{1.844445in}%
\pgfsys@useobject{currentmarker}{}%
\end{pgfscope}%
\begin{pgfscope}%
\pgfsys@transformshift{3.126227in}{1.745650in}%
\pgfsys@useobject{currentmarker}{}%
\end{pgfscope}%
\begin{pgfscope}%
\pgfsys@transformshift{3.985167in}{1.489676in}%
\pgfsys@useobject{currentmarker}{}%
\end{pgfscope}%
\begin{pgfscope}%
\pgfsys@transformshift{2.465834in}{2.019474in}%
\pgfsys@useobject{currentmarker}{}%
\end{pgfscope}%
\begin{pgfscope}%
\pgfsys@transformshift{1.817167in}{3.758263in}%
\pgfsys@useobject{currentmarker}{}%
\end{pgfscope}%
\begin{pgfscope}%
\pgfsys@transformshift{2.995454in}{1.615970in}%
\pgfsys@useobject{currentmarker}{}%
\end{pgfscope}%
\begin{pgfscope}%
\pgfsys@transformshift{4.474361in}{1.422929in}%
\pgfsys@useobject{currentmarker}{}%
\end{pgfscope}%
\begin{pgfscope}%
\pgfsys@transformshift{4.697693in}{1.437104in}%
\pgfsys@useobject{currentmarker}{}%
\end{pgfscope}%
\begin{pgfscope}%
\pgfsys@transformshift{4.688725in}{1.409013in}%
\pgfsys@useobject{currentmarker}{}%
\end{pgfscope}%
\begin{pgfscope}%
\pgfsys@transformshift{2.906429in}{1.467928in}%
\pgfsys@useobject{currentmarker}{}%
\end{pgfscope}%
\begin{pgfscope}%
\pgfsys@transformshift{2.853665in}{1.487905in}%
\pgfsys@useobject{currentmarker}{}%
\end{pgfscope}%
\begin{pgfscope}%
\pgfsys@transformshift{5.087238in}{1.562018in}%
\pgfsys@useobject{currentmarker}{}%
\end{pgfscope}%
\begin{pgfscope}%
\pgfsys@transformshift{1.804908in}{2.506164in}%
\pgfsys@useobject{currentmarker}{}%
\end{pgfscope}%
\begin{pgfscope}%
\pgfsys@transformshift{3.583119in}{1.207036in}%
\pgfsys@useobject{currentmarker}{}%
\end{pgfscope}%
\end{pgfscope}%
\begin{pgfscope}%
\pgfpathrectangle{\pgfqpoint{1.432000in}{0.528000in}}{\pgfqpoint{3.696000in}{3.696000in}}%
\pgfusepath{clip}%
\pgfsetbuttcap%
\pgfsetroundjoin%
\definecolor{currentfill}{rgb}{1.000000,0.000000,0.000000}%
\pgfsetfillcolor{currentfill}%
\pgfsetlinewidth{0.501875pt}%
\definecolor{currentstroke}{rgb}{1.000000,1.000000,1.000000}%
\pgfsetstrokecolor{currentstroke}%
\pgfsetdash{}{0pt}%
\pgfsys@defobject{currentmarker}{\pgfqpoint{-0.018373in}{-0.018373in}}{\pgfqpoint{0.018373in}{0.018373in}}{%
\pgfpathmoveto{\pgfqpoint{0.000000in}{-0.018373in}}%
\pgfpathcurveto{\pgfqpoint{0.004873in}{-0.018373in}}{\pgfqpoint{0.009546in}{-0.016437in}}{\pgfqpoint{0.012992in}{-0.012992in}}%
\pgfpathcurveto{\pgfqpoint{0.016437in}{-0.009546in}}{\pgfqpoint{0.018373in}{-0.004873in}}{\pgfqpoint{0.018373in}{0.000000in}}%
\pgfpathcurveto{\pgfqpoint{0.018373in}{0.004873in}}{\pgfqpoint{0.016437in}{0.009546in}}{\pgfqpoint{0.012992in}{0.012992in}}%
\pgfpathcurveto{\pgfqpoint{0.009546in}{0.016437in}}{\pgfqpoint{0.004873in}{0.018373in}}{\pgfqpoint{0.000000in}{0.018373in}}%
\pgfpathcurveto{\pgfqpoint{-0.004873in}{0.018373in}}{\pgfqpoint{-0.009546in}{0.016437in}}{\pgfqpoint{-0.012992in}{0.012992in}}%
\pgfpathcurveto{\pgfqpoint{-0.016437in}{0.009546in}}{\pgfqpoint{-0.018373in}{0.004873in}}{\pgfqpoint{-0.018373in}{0.000000in}}%
\pgfpathcurveto{\pgfqpoint{-0.018373in}{-0.004873in}}{\pgfqpoint{-0.016437in}{-0.009546in}}{\pgfqpoint{-0.012992in}{-0.012992in}}%
\pgfpathcurveto{\pgfqpoint{-0.009546in}{-0.016437in}}{\pgfqpoint{-0.004873in}{-0.018373in}}{\pgfqpoint{0.000000in}{-0.018373in}}%
\pgfpathclose%
\pgfusepath{stroke,fill}%
}%
\begin{pgfscope}%
\pgfsys@transformshift{3.877924in}{3.519472in}%
\pgfsys@useobject{currentmarker}{}%
\end{pgfscope}%
\begin{pgfscope}%
\pgfsys@transformshift{3.800679in}{2.755041in}%
\pgfsys@useobject{currentmarker}{}%
\end{pgfscope}%
\begin{pgfscope}%
\pgfsys@transformshift{3.466250in}{2.684576in}%
\pgfsys@useobject{currentmarker}{}%
\end{pgfscope}%
\begin{pgfscope}%
\pgfsys@transformshift{3.625946in}{2.062443in}%
\pgfsys@useobject{currentmarker}{}%
\end{pgfscope}%
\begin{pgfscope}%
\pgfsys@transformshift{2.179973in}{2.119944in}%
\pgfsys@useobject{currentmarker}{}%
\end{pgfscope}%
\end{pgfscope}%
\begin{pgfscope}%
\pgfsetbuttcap%
\pgfsetroundjoin%
\definecolor{currentfill}{rgb}{0.000000,0.000000,0.000000}%
\pgfsetfillcolor{currentfill}%
\pgfsetlinewidth{0.803000pt}%
\definecolor{currentstroke}{rgb}{0.000000,0.000000,0.000000}%
\pgfsetstrokecolor{currentstroke}%
\pgfsetdash{}{0pt}%
\pgfsys@defobject{currentmarker}{\pgfqpoint{0.000000in}{-0.048611in}}{\pgfqpoint{0.000000in}{0.000000in}}{%
\pgfpathmoveto{\pgfqpoint{0.000000in}{0.000000in}}%
\pgfpathlineto{\pgfqpoint{0.000000in}{-0.048611in}}%
\pgfusepath{stroke,fill}%
}%
\begin{pgfscope}%
\pgfsys@transformshift{1.432000in}{0.528000in}%
\pgfsys@useobject{currentmarker}{}%
\end{pgfscope}%
\end{pgfscope}%
\begin{pgfscope}%
\definecolor{textcolor}{rgb}{0.000000,0.000000,0.000000}%
\pgfsetstrokecolor{textcolor}%
\pgfsetfillcolor{textcolor}%
\pgftext[x=1.432000in,y=0.430778in,,top]{\color{textcolor}\rmfamily\fontsize{10.000000}{12.000000}\selectfont \(\displaystyle -100\)}%
\end{pgfscope}%
\begin{pgfscope}%
\pgfsetbuttcap%
\pgfsetroundjoin%
\definecolor{currentfill}{rgb}{0.000000,0.000000,0.000000}%
\pgfsetfillcolor{currentfill}%
\pgfsetlinewidth{0.803000pt}%
\definecolor{currentstroke}{rgb}{0.000000,0.000000,0.000000}%
\pgfsetstrokecolor{currentstroke}%
\pgfsetdash{}{0pt}%
\pgfsys@defobject{currentmarker}{\pgfqpoint{0.000000in}{-0.048611in}}{\pgfqpoint{0.000000in}{0.000000in}}{%
\pgfpathmoveto{\pgfqpoint{0.000000in}{0.000000in}}%
\pgfpathlineto{\pgfqpoint{0.000000in}{-0.048611in}}%
\pgfusepath{stroke,fill}%
}%
\begin{pgfscope}%
\pgfsys@transformshift{2.074783in}{0.528000in}%
\pgfsys@useobject{currentmarker}{}%
\end{pgfscope}%
\end{pgfscope}%
\begin{pgfscope}%
\definecolor{textcolor}{rgb}{0.000000,0.000000,0.000000}%
\pgfsetstrokecolor{textcolor}%
\pgfsetfillcolor{textcolor}%
\pgftext[x=2.074783in,y=0.430778in,,top]{\color{textcolor}\rmfamily\fontsize{10.000000}{12.000000}\selectfont \(\displaystyle -80\)}%
\end{pgfscope}%
\begin{pgfscope}%
\pgfsetbuttcap%
\pgfsetroundjoin%
\definecolor{currentfill}{rgb}{0.000000,0.000000,0.000000}%
\pgfsetfillcolor{currentfill}%
\pgfsetlinewidth{0.803000pt}%
\definecolor{currentstroke}{rgb}{0.000000,0.000000,0.000000}%
\pgfsetstrokecolor{currentstroke}%
\pgfsetdash{}{0pt}%
\pgfsys@defobject{currentmarker}{\pgfqpoint{0.000000in}{-0.048611in}}{\pgfqpoint{0.000000in}{0.000000in}}{%
\pgfpathmoveto{\pgfqpoint{0.000000in}{0.000000in}}%
\pgfpathlineto{\pgfqpoint{0.000000in}{-0.048611in}}%
\pgfusepath{stroke,fill}%
}%
\begin{pgfscope}%
\pgfsys@transformshift{2.717565in}{0.528000in}%
\pgfsys@useobject{currentmarker}{}%
\end{pgfscope}%
\end{pgfscope}%
\begin{pgfscope}%
\definecolor{textcolor}{rgb}{0.000000,0.000000,0.000000}%
\pgfsetstrokecolor{textcolor}%
\pgfsetfillcolor{textcolor}%
\pgftext[x=2.717565in,y=0.430778in,,top]{\color{textcolor}\rmfamily\fontsize{10.000000}{12.000000}\selectfont \(\displaystyle -60\)}%
\end{pgfscope}%
\begin{pgfscope}%
\pgfsetbuttcap%
\pgfsetroundjoin%
\definecolor{currentfill}{rgb}{0.000000,0.000000,0.000000}%
\pgfsetfillcolor{currentfill}%
\pgfsetlinewidth{0.803000pt}%
\definecolor{currentstroke}{rgb}{0.000000,0.000000,0.000000}%
\pgfsetstrokecolor{currentstroke}%
\pgfsetdash{}{0pt}%
\pgfsys@defobject{currentmarker}{\pgfqpoint{0.000000in}{-0.048611in}}{\pgfqpoint{0.000000in}{0.000000in}}{%
\pgfpathmoveto{\pgfqpoint{0.000000in}{0.000000in}}%
\pgfpathlineto{\pgfqpoint{0.000000in}{-0.048611in}}%
\pgfusepath{stroke,fill}%
}%
\begin{pgfscope}%
\pgfsys@transformshift{3.360348in}{0.528000in}%
\pgfsys@useobject{currentmarker}{}%
\end{pgfscope}%
\end{pgfscope}%
\begin{pgfscope}%
\definecolor{textcolor}{rgb}{0.000000,0.000000,0.000000}%
\pgfsetstrokecolor{textcolor}%
\pgfsetfillcolor{textcolor}%
\pgftext[x=3.360348in,y=0.430778in,,top]{\color{textcolor}\rmfamily\fontsize{10.000000}{12.000000}\selectfont \(\displaystyle -40\)}%
\end{pgfscope}%
\begin{pgfscope}%
\pgfsetbuttcap%
\pgfsetroundjoin%
\definecolor{currentfill}{rgb}{0.000000,0.000000,0.000000}%
\pgfsetfillcolor{currentfill}%
\pgfsetlinewidth{0.803000pt}%
\definecolor{currentstroke}{rgb}{0.000000,0.000000,0.000000}%
\pgfsetstrokecolor{currentstroke}%
\pgfsetdash{}{0pt}%
\pgfsys@defobject{currentmarker}{\pgfqpoint{0.000000in}{-0.048611in}}{\pgfqpoint{0.000000in}{0.000000in}}{%
\pgfpathmoveto{\pgfqpoint{0.000000in}{0.000000in}}%
\pgfpathlineto{\pgfqpoint{0.000000in}{-0.048611in}}%
\pgfusepath{stroke,fill}%
}%
\begin{pgfscope}%
\pgfsys@transformshift{4.003130in}{0.528000in}%
\pgfsys@useobject{currentmarker}{}%
\end{pgfscope}%
\end{pgfscope}%
\begin{pgfscope}%
\definecolor{textcolor}{rgb}{0.000000,0.000000,0.000000}%
\pgfsetstrokecolor{textcolor}%
\pgfsetfillcolor{textcolor}%
\pgftext[x=4.003130in,y=0.430778in,,top]{\color{textcolor}\rmfamily\fontsize{10.000000}{12.000000}\selectfont \(\displaystyle -20\)}%
\end{pgfscope}%
\begin{pgfscope}%
\pgfsetbuttcap%
\pgfsetroundjoin%
\definecolor{currentfill}{rgb}{0.000000,0.000000,0.000000}%
\pgfsetfillcolor{currentfill}%
\pgfsetlinewidth{0.803000pt}%
\definecolor{currentstroke}{rgb}{0.000000,0.000000,0.000000}%
\pgfsetstrokecolor{currentstroke}%
\pgfsetdash{}{0pt}%
\pgfsys@defobject{currentmarker}{\pgfqpoint{0.000000in}{-0.048611in}}{\pgfqpoint{0.000000in}{0.000000in}}{%
\pgfpathmoveto{\pgfqpoint{0.000000in}{0.000000in}}%
\pgfpathlineto{\pgfqpoint{0.000000in}{-0.048611in}}%
\pgfusepath{stroke,fill}%
}%
\begin{pgfscope}%
\pgfsys@transformshift{4.645913in}{0.528000in}%
\pgfsys@useobject{currentmarker}{}%
\end{pgfscope}%
\end{pgfscope}%
\begin{pgfscope}%
\definecolor{textcolor}{rgb}{0.000000,0.000000,0.000000}%
\pgfsetstrokecolor{textcolor}%
\pgfsetfillcolor{textcolor}%
\pgftext[x=4.645913in,y=0.430778in,,top]{\color{textcolor}\rmfamily\fontsize{10.000000}{12.000000}\selectfont \(\displaystyle 0\)}%
\end{pgfscope}%
\begin{pgfscope}%
\pgfsetbuttcap%
\pgfsetroundjoin%
\definecolor{currentfill}{rgb}{0.000000,0.000000,0.000000}%
\pgfsetfillcolor{currentfill}%
\pgfsetlinewidth{0.803000pt}%
\definecolor{currentstroke}{rgb}{0.000000,0.000000,0.000000}%
\pgfsetstrokecolor{currentstroke}%
\pgfsetdash{}{0pt}%
\pgfsys@defobject{currentmarker}{\pgfqpoint{-0.048611in}{0.000000in}}{\pgfqpoint{0.000000in}{0.000000in}}{%
\pgfpathmoveto{\pgfqpoint{0.000000in}{0.000000in}}%
\pgfpathlineto{\pgfqpoint{-0.048611in}{0.000000in}}%
\pgfusepath{stroke,fill}%
}%
\begin{pgfscope}%
\pgfsys@transformshift{1.432000in}{0.528000in}%
\pgfsys@useobject{currentmarker}{}%
\end{pgfscope}%
\end{pgfscope}%
\begin{pgfscope}%
\definecolor{textcolor}{rgb}{0.000000,0.000000,0.000000}%
\pgfsetstrokecolor{textcolor}%
\pgfsetfillcolor{textcolor}%
\pgftext[x=1.018419in,y=0.475238in,left,base]{\color{textcolor}\rmfamily\fontsize{10.000000}{12.000000}\selectfont \(\displaystyle -100\)}%
\end{pgfscope}%
\begin{pgfscope}%
\pgfsetbuttcap%
\pgfsetroundjoin%
\definecolor{currentfill}{rgb}{0.000000,0.000000,0.000000}%
\pgfsetfillcolor{currentfill}%
\pgfsetlinewidth{0.803000pt}%
\definecolor{currentstroke}{rgb}{0.000000,0.000000,0.000000}%
\pgfsetstrokecolor{currentstroke}%
\pgfsetdash{}{0pt}%
\pgfsys@defobject{currentmarker}{\pgfqpoint{-0.048611in}{0.000000in}}{\pgfqpoint{0.000000in}{0.000000in}}{%
\pgfpathmoveto{\pgfqpoint{0.000000in}{0.000000in}}%
\pgfpathlineto{\pgfqpoint{-0.048611in}{0.000000in}}%
\pgfusepath{stroke,fill}%
}%
\begin{pgfscope}%
\pgfsys@transformshift{1.432000in}{1.170783in}%
\pgfsys@useobject{currentmarker}{}%
\end{pgfscope}%
\end{pgfscope}%
\begin{pgfscope}%
\definecolor{textcolor}{rgb}{0.000000,0.000000,0.000000}%
\pgfsetstrokecolor{textcolor}%
\pgfsetfillcolor{textcolor}%
\pgftext[x=1.087863in,y=1.118021in,left,base]{\color{textcolor}\rmfamily\fontsize{10.000000}{12.000000}\selectfont \(\displaystyle -80\)}%
\end{pgfscope}%
\begin{pgfscope}%
\pgfsetbuttcap%
\pgfsetroundjoin%
\definecolor{currentfill}{rgb}{0.000000,0.000000,0.000000}%
\pgfsetfillcolor{currentfill}%
\pgfsetlinewidth{0.803000pt}%
\definecolor{currentstroke}{rgb}{0.000000,0.000000,0.000000}%
\pgfsetstrokecolor{currentstroke}%
\pgfsetdash{}{0pt}%
\pgfsys@defobject{currentmarker}{\pgfqpoint{-0.048611in}{0.000000in}}{\pgfqpoint{0.000000in}{0.000000in}}{%
\pgfpathmoveto{\pgfqpoint{0.000000in}{0.000000in}}%
\pgfpathlineto{\pgfqpoint{-0.048611in}{0.000000in}}%
\pgfusepath{stroke,fill}%
}%
\begin{pgfscope}%
\pgfsys@transformshift{1.432000in}{1.813565in}%
\pgfsys@useobject{currentmarker}{}%
\end{pgfscope}%
\end{pgfscope}%
\begin{pgfscope}%
\definecolor{textcolor}{rgb}{0.000000,0.000000,0.000000}%
\pgfsetstrokecolor{textcolor}%
\pgfsetfillcolor{textcolor}%
\pgftext[x=1.087863in,y=1.760804in,left,base]{\color{textcolor}\rmfamily\fontsize{10.000000}{12.000000}\selectfont \(\displaystyle -60\)}%
\end{pgfscope}%
\begin{pgfscope}%
\pgfsetbuttcap%
\pgfsetroundjoin%
\definecolor{currentfill}{rgb}{0.000000,0.000000,0.000000}%
\pgfsetfillcolor{currentfill}%
\pgfsetlinewidth{0.803000pt}%
\definecolor{currentstroke}{rgb}{0.000000,0.000000,0.000000}%
\pgfsetstrokecolor{currentstroke}%
\pgfsetdash{}{0pt}%
\pgfsys@defobject{currentmarker}{\pgfqpoint{-0.048611in}{0.000000in}}{\pgfqpoint{0.000000in}{0.000000in}}{%
\pgfpathmoveto{\pgfqpoint{0.000000in}{0.000000in}}%
\pgfpathlineto{\pgfqpoint{-0.048611in}{0.000000in}}%
\pgfusepath{stroke,fill}%
}%
\begin{pgfscope}%
\pgfsys@transformshift{1.432000in}{2.456348in}%
\pgfsys@useobject{currentmarker}{}%
\end{pgfscope}%
\end{pgfscope}%
\begin{pgfscope}%
\definecolor{textcolor}{rgb}{0.000000,0.000000,0.000000}%
\pgfsetstrokecolor{textcolor}%
\pgfsetfillcolor{textcolor}%
\pgftext[x=1.087863in,y=2.403586in,left,base]{\color{textcolor}\rmfamily\fontsize{10.000000}{12.000000}\selectfont \(\displaystyle -40\)}%
\end{pgfscope}%
\begin{pgfscope}%
\pgfsetbuttcap%
\pgfsetroundjoin%
\definecolor{currentfill}{rgb}{0.000000,0.000000,0.000000}%
\pgfsetfillcolor{currentfill}%
\pgfsetlinewidth{0.803000pt}%
\definecolor{currentstroke}{rgb}{0.000000,0.000000,0.000000}%
\pgfsetstrokecolor{currentstroke}%
\pgfsetdash{}{0pt}%
\pgfsys@defobject{currentmarker}{\pgfqpoint{-0.048611in}{0.000000in}}{\pgfqpoint{0.000000in}{0.000000in}}{%
\pgfpathmoveto{\pgfqpoint{0.000000in}{0.000000in}}%
\pgfpathlineto{\pgfqpoint{-0.048611in}{0.000000in}}%
\pgfusepath{stroke,fill}%
}%
\begin{pgfscope}%
\pgfsys@transformshift{1.432000in}{3.099130in}%
\pgfsys@useobject{currentmarker}{}%
\end{pgfscope}%
\end{pgfscope}%
\begin{pgfscope}%
\definecolor{textcolor}{rgb}{0.000000,0.000000,0.000000}%
\pgfsetstrokecolor{textcolor}%
\pgfsetfillcolor{textcolor}%
\pgftext[x=1.087863in,y=3.046369in,left,base]{\color{textcolor}\rmfamily\fontsize{10.000000}{12.000000}\selectfont \(\displaystyle -20\)}%
\end{pgfscope}%
\begin{pgfscope}%
\pgfsetbuttcap%
\pgfsetroundjoin%
\definecolor{currentfill}{rgb}{0.000000,0.000000,0.000000}%
\pgfsetfillcolor{currentfill}%
\pgfsetlinewidth{0.803000pt}%
\definecolor{currentstroke}{rgb}{0.000000,0.000000,0.000000}%
\pgfsetstrokecolor{currentstroke}%
\pgfsetdash{}{0pt}%
\pgfsys@defobject{currentmarker}{\pgfqpoint{-0.048611in}{0.000000in}}{\pgfqpoint{0.000000in}{0.000000in}}{%
\pgfpathmoveto{\pgfqpoint{0.000000in}{0.000000in}}%
\pgfpathlineto{\pgfqpoint{-0.048611in}{0.000000in}}%
\pgfusepath{stroke,fill}%
}%
\begin{pgfscope}%
\pgfsys@transformshift{1.432000in}{3.741913in}%
\pgfsys@useobject{currentmarker}{}%
\end{pgfscope}%
\end{pgfscope}%
\begin{pgfscope}%
\definecolor{textcolor}{rgb}{0.000000,0.000000,0.000000}%
\pgfsetstrokecolor{textcolor}%
\pgfsetfillcolor{textcolor}%
\pgftext[x=1.265333in,y=3.689152in,left,base]{\color{textcolor}\rmfamily\fontsize{10.000000}{12.000000}\selectfont \(\displaystyle 0\)}%
\end{pgfscope}%
\begin{pgfscope}%
\pgfpathrectangle{\pgfqpoint{1.432000in}{0.528000in}}{\pgfqpoint{3.696000in}{3.696000in}}%
\pgfusepath{clip}%
\pgfsetbuttcap%
\pgfsetroundjoin%
\pgfsetlinewidth{1.505625pt}%
\definecolor{currentstroke}{rgb}{0.371035,0.000000,0.000000}%
\pgfsetstrokecolor{currentstroke}%
\pgfsetstrokeopacity{0.300000}%
\pgfsetdash{}{0pt}%
\pgfpathmoveto{\pgfqpoint{5.128000in}{4.185730in}}%
\pgfpathlineto{\pgfqpoint{5.127080in}{4.186667in}}%
\pgfpathlineto{\pgfqpoint{5.090667in}{4.223080in}}%
\pgfpathlineto{\pgfqpoint{5.089730in}{4.224000in}}%
\pgfusepath{stroke}%
\end{pgfscope}%
\begin{pgfscope}%
\pgfpathrectangle{\pgfqpoint{1.432000in}{0.528000in}}{\pgfqpoint{3.696000in}{3.696000in}}%
\pgfusepath{clip}%
\pgfsetbuttcap%
\pgfsetroundjoin%
\pgfsetlinewidth{1.505625pt}%
\definecolor{currentstroke}{rgb}{0.700470,0.000000,0.000000}%
\pgfsetstrokecolor{currentstroke}%
\pgfsetstrokeopacity{0.300000}%
\pgfsetdash{}{0pt}%
\pgfpathmoveto{\pgfqpoint{4.806363in}{0.528000in}}%
\pgfpathlineto{\pgfqpoint{4.829333in}{0.553666in}}%
\pgfpathlineto{\pgfqpoint{4.839598in}{0.565333in}}%
\pgfpathlineto{\pgfqpoint{4.866667in}{0.596981in}}%
\pgfpathlineto{\pgfqpoint{4.871455in}{0.602667in}}%
\pgfpathlineto{\pgfqpoint{4.902023in}{0.640000in}}%
\pgfpathlineto{\pgfqpoint{4.904000in}{0.642481in}}%
\pgfpathlineto{\pgfqpoint{4.931415in}{0.677333in}}%
\pgfpathlineto{\pgfqpoint{4.941333in}{0.690370in}}%
\pgfpathlineto{\pgfqpoint{4.959612in}{0.714667in}}%
\pgfpathlineto{\pgfqpoint{4.978667in}{0.740907in}}%
\pgfpathlineto{\pgfqpoint{4.986646in}{0.752000in}}%
\pgfpathlineto{\pgfqpoint{5.012599in}{0.789333in}}%
\pgfpathlineto{\pgfqpoint{5.016000in}{0.794382in}}%
\pgfpathlineto{\pgfqpoint{5.037596in}{0.826667in}}%
\pgfpathlineto{\pgfqpoint{5.053333in}{0.851177in}}%
\pgfpathlineto{\pgfqpoint{5.061522in}{0.864000in}}%
\pgfpathlineto{\pgfqpoint{5.084497in}{0.901333in}}%
\pgfpathlineto{\pgfqpoint{5.090667in}{0.911735in}}%
\pgfpathlineto{\pgfqpoint{5.106593in}{0.938667in}}%
\pgfpathlineto{\pgfqpoint{5.127664in}{0.976000in}}%
\pgfpathlineto{\pgfqpoint{5.128000in}{0.976615in}}%
\pgfusepath{stroke}%
\end{pgfscope}%
\begin{pgfscope}%
\pgfpathrectangle{\pgfqpoint{1.432000in}{0.528000in}}{\pgfqpoint{3.696000in}{3.696000in}}%
\pgfusepath{clip}%
\pgfsetbuttcap%
\pgfsetroundjoin%
\pgfsetlinewidth{1.505625pt}%
\definecolor{currentstroke}{rgb}{0.700470,0.000000,0.000000}%
\pgfsetstrokecolor{currentstroke}%
\pgfsetstrokeopacity{0.300000}%
\pgfsetdash{}{0pt}%
\pgfpathmoveto{\pgfqpoint{5.128000in}{3.293288in}}%
\pgfpathlineto{\pgfqpoint{5.108470in}{3.328000in}}%
\pgfpathlineto{\pgfqpoint{5.090667in}{3.358186in}}%
\pgfpathlineto{\pgfqpoint{5.086438in}{3.365333in}}%
\pgfpathlineto{\pgfqpoint{5.063568in}{3.402667in}}%
\pgfpathlineto{\pgfqpoint{5.053333in}{3.418733in}}%
\pgfpathlineto{\pgfqpoint{5.039714in}{3.440000in}}%
\pgfpathlineto{\pgfqpoint{5.016000in}{3.475537in}}%
\pgfpathlineto{\pgfqpoint{5.014793in}{3.477333in}}%
\pgfpathlineto{\pgfqpoint{4.988957in}{3.514667in}}%
\pgfpathlineto{\pgfqpoint{4.978667in}{3.529005in}}%
\pgfpathlineto{\pgfqpoint{4.962007in}{3.552000in}}%
\pgfpathlineto{\pgfqpoint{4.941333in}{3.579542in}}%
\pgfpathlineto{\pgfqpoint{4.933901in}{3.589333in}}%
\pgfpathlineto{\pgfqpoint{4.904615in}{3.626667in}}%
\pgfpathlineto{\pgfqpoint{4.904000in}{3.627431in}}%
\pgfpathlineto{\pgfqpoint{4.874177in}{3.664000in}}%
\pgfpathlineto{\pgfqpoint{4.866667in}{3.672936in}}%
\pgfpathlineto{\pgfqpoint{4.842429in}{3.701333in}}%
\pgfpathlineto{\pgfqpoint{4.829333in}{3.716250in}}%
\pgfpathlineto{\pgfqpoint{4.809312in}{3.738667in}}%
\pgfpathlineto{\pgfqpoint{4.792000in}{3.757547in}}%
\pgfpathlineto{\pgfqpoint{4.774756in}{3.776000in}}%
\pgfpathlineto{\pgfqpoint{4.754667in}{3.796979in}}%
\pgfpathlineto{\pgfqpoint{4.738678in}{3.813333in}}%
\pgfpathlineto{\pgfqpoint{4.717333in}{3.834678in}}%
\pgfpathlineto{\pgfqpoint{4.700979in}{3.850667in}}%
\pgfpathlineto{\pgfqpoint{4.680000in}{3.870756in}}%
\pgfpathlineto{\pgfqpoint{4.661547in}{3.888000in}}%
\pgfpathlineto{\pgfqpoint{4.642667in}{3.905312in}}%
\pgfpathlineto{\pgfqpoint{4.620250in}{3.925333in}}%
\pgfpathlineto{\pgfqpoint{4.605333in}{3.938429in}}%
\pgfpathlineto{\pgfqpoint{4.576936in}{3.962667in}}%
\pgfpathlineto{\pgfqpoint{4.568000in}{3.970177in}}%
\pgfpathlineto{\pgfqpoint{4.531431in}{4.000000in}}%
\pgfpathlineto{\pgfqpoint{4.530667in}{4.000615in}}%
\pgfpathlineto{\pgfqpoint{4.493333in}{4.029901in}}%
\pgfpathlineto{\pgfqpoint{4.483542in}{4.037333in}}%
\pgfpathlineto{\pgfqpoint{4.456000in}{4.058007in}}%
\pgfpathlineto{\pgfqpoint{4.433005in}{4.074667in}}%
\pgfpathlineto{\pgfqpoint{4.418667in}{4.084957in}}%
\pgfpathlineto{\pgfqpoint{4.381333in}{4.110793in}}%
\pgfpathlineto{\pgfqpoint{4.379537in}{4.112000in}}%
\pgfpathlineto{\pgfqpoint{4.344000in}{4.135714in}}%
\pgfpathlineto{\pgfqpoint{4.322733in}{4.149333in}}%
\pgfpathlineto{\pgfqpoint{4.306667in}{4.159568in}}%
\pgfpathlineto{\pgfqpoint{4.269333in}{4.182438in}}%
\pgfpathlineto{\pgfqpoint{4.262186in}{4.186667in}}%
\pgfpathlineto{\pgfqpoint{4.232000in}{4.204470in}}%
\pgfpathlineto{\pgfqpoint{4.197288in}{4.224000in}}%
\pgfusepath{stroke}%
\end{pgfscope}%
\begin{pgfscope}%
\pgfpathrectangle{\pgfqpoint{1.432000in}{0.528000in}}{\pgfqpoint{3.696000in}{3.696000in}}%
\pgfusepath{clip}%
\pgfsetbuttcap%
\pgfsetroundjoin%
\pgfsetlinewidth{1.505625pt}%
\definecolor{currentstroke}{rgb}{0.700470,0.000000,0.000000}%
\pgfsetstrokecolor{currentstroke}%
\pgfsetstrokeopacity{0.300000}%
\pgfsetdash{}{0pt}%
\pgfpathmoveto{\pgfqpoint{1.880615in}{4.224000in}}%
\pgfpathlineto{\pgfqpoint{1.880000in}{4.223664in}}%
\pgfpathlineto{\pgfqpoint{1.842667in}{4.202593in}}%
\pgfpathlineto{\pgfqpoint{1.815735in}{4.186667in}}%
\pgfpathlineto{\pgfqpoint{1.805333in}{4.180497in}}%
\pgfpathlineto{\pgfqpoint{1.768000in}{4.157522in}}%
\pgfpathlineto{\pgfqpoint{1.755177in}{4.149333in}}%
\pgfpathlineto{\pgfqpoint{1.730667in}{4.133596in}}%
\pgfpathlineto{\pgfqpoint{1.698382in}{4.112000in}}%
\pgfpathlineto{\pgfqpoint{1.693333in}{4.108599in}}%
\pgfpathlineto{\pgfqpoint{1.656000in}{4.082646in}}%
\pgfpathlineto{\pgfqpoint{1.644907in}{4.074667in}}%
\pgfpathlineto{\pgfqpoint{1.618667in}{4.055612in}}%
\pgfpathlineto{\pgfqpoint{1.594370in}{4.037333in}}%
\pgfpathlineto{\pgfqpoint{1.581333in}{4.027415in}}%
\pgfpathlineto{\pgfqpoint{1.546481in}{4.000000in}}%
\pgfpathlineto{\pgfqpoint{1.544000in}{3.998023in}}%
\pgfpathlineto{\pgfqpoint{1.506667in}{3.967455in}}%
\pgfpathlineto{\pgfqpoint{1.500981in}{3.962667in}}%
\pgfpathlineto{\pgfqpoint{1.469333in}{3.935598in}}%
\pgfpathlineto{\pgfqpoint{1.457666in}{3.925333in}}%
\pgfpathlineto{\pgfqpoint{1.432000in}{3.902363in}}%
\pgfusepath{stroke}%
\end{pgfscope}%
\begin{pgfscope}%
\pgfpathrectangle{\pgfqpoint{1.432000in}{0.528000in}}{\pgfqpoint{3.696000in}{3.696000in}}%
\pgfusepath{clip}%
\pgfsetbuttcap%
\pgfsetroundjoin%
\pgfsetlinewidth{1.505625pt}%
\definecolor{currentstroke}{rgb}{1.000000,0.029903,0.000000}%
\pgfsetstrokecolor{currentstroke}%
\pgfsetstrokeopacity{0.300000}%
\pgfsetdash{}{0pt}%
\pgfpathmoveto{\pgfqpoint{4.242375in}{0.528000in}}%
\pgfpathlineto{\pgfqpoint{4.306667in}{0.578174in}}%
\pgfpathlineto{\pgfqpoint{4.344000in}{0.609344in}}%
\pgfpathlineto{\pgfqpoint{4.381333in}{0.642095in}}%
\pgfpathlineto{\pgfqpoint{4.419514in}{0.677333in}}%
\pgfpathlineto{\pgfqpoint{4.457894in}{0.714667in}}%
\pgfpathlineto{\pgfqpoint{4.494305in}{0.752000in}}%
\pgfpathlineto{\pgfqpoint{4.530667in}{0.791305in}}%
\pgfpathlineto{\pgfqpoint{4.568000in}{0.833960in}}%
\pgfpathlineto{\pgfqpoint{4.622893in}{0.901333in}}%
\pgfpathlineto{\pgfqpoint{4.651273in}{0.938667in}}%
\pgfpathlineto{\pgfqpoint{4.704092in}{1.013333in}}%
\pgfpathlineto{\pgfqpoint{4.728642in}{1.050667in}}%
\pgfpathlineto{\pgfqpoint{4.774303in}{1.125333in}}%
\pgfpathlineto{\pgfqpoint{4.815648in}{1.200000in}}%
\pgfpathlineto{\pgfqpoint{4.852985in}{1.274667in}}%
\pgfpathlineto{\pgfqpoint{4.886558in}{1.349333in}}%
\pgfpathlineto{\pgfqpoint{4.916553in}{1.424000in}}%
\pgfpathlineto{\pgfqpoint{4.943108in}{1.498667in}}%
\pgfpathlineto{\pgfqpoint{4.966502in}{1.573333in}}%
\pgfpathlineto{\pgfqpoint{4.986693in}{1.648000in}}%
\pgfpathlineto{\pgfqpoint{5.003871in}{1.722667in}}%
\pgfpathlineto{\pgfqpoint{5.018018in}{1.797333in}}%
\pgfpathlineto{\pgfqpoint{5.029373in}{1.872000in}}%
\pgfpathlineto{\pgfqpoint{5.037824in}{1.946667in}}%
\pgfpathlineto{\pgfqpoint{5.043439in}{2.021333in}}%
\pgfpathlineto{\pgfqpoint{5.046266in}{2.096000in}}%
\pgfpathlineto{\pgfqpoint{5.046326in}{2.170667in}}%
\pgfpathlineto{\pgfqpoint{5.043620in}{2.245333in}}%
\pgfpathlineto{\pgfqpoint{5.038126in}{2.320000in}}%
\pgfpathlineto{\pgfqpoint{5.029800in}{2.394667in}}%
\pgfpathlineto{\pgfqpoint{5.018573in}{2.469333in}}%
\pgfpathlineto{\pgfqpoint{5.004547in}{2.544000in}}%
\pgfpathlineto{\pgfqpoint{4.987506in}{2.618667in}}%
\pgfpathlineto{\pgfqpoint{4.967442in}{2.693333in}}%
\pgfpathlineto{\pgfqpoint{4.944196in}{2.768000in}}%
\pgfpathlineto{\pgfqpoint{4.917782in}{2.842667in}}%
\pgfpathlineto{\pgfqpoint{4.887934in}{2.917333in}}%
\pgfpathlineto{\pgfqpoint{4.854518in}{2.992000in}}%
\pgfpathlineto{\pgfqpoint{4.817350in}{3.066667in}}%
\pgfpathlineto{\pgfqpoint{4.776187in}{3.141333in}}%
\pgfpathlineto{\pgfqpoint{4.730724in}{3.216000in}}%
\pgfpathlineto{\pgfqpoint{4.706271in}{3.253333in}}%
\pgfpathlineto{\pgfqpoint{4.653680in}{3.328000in}}%
\pgfpathlineto{\pgfqpoint{4.605333in}{3.390742in}}%
\pgfpathlineto{\pgfqpoint{4.564565in}{3.440000in}}%
\pgfpathlineto{\pgfqpoint{4.530667in}{3.478609in}}%
\pgfpathlineto{\pgfqpoint{4.493333in}{3.518932in}}%
\pgfpathlineto{\pgfqpoint{4.456000in}{3.557136in}}%
\pgfpathlineto{\pgfqpoint{4.418667in}{3.593385in}}%
\pgfpathlineto{\pgfqpoint{4.381333in}{3.627818in}}%
\pgfpathlineto{\pgfqpoint{4.339959in}{3.664000in}}%
\pgfpathlineto{\pgfqpoint{4.269333in}{3.721417in}}%
\pgfpathlineto{\pgfqpoint{4.232000in}{3.749680in}}%
\pgfpathlineto{\pgfqpoint{4.157333in}{3.802271in}}%
\pgfpathlineto{\pgfqpoint{4.120000in}{3.826724in}}%
\pgfpathlineto{\pgfqpoint{4.045333in}{3.872187in}}%
\pgfpathlineto{\pgfqpoint{3.970667in}{3.913350in}}%
\pgfpathlineto{\pgfqpoint{3.896000in}{3.950518in}}%
\pgfpathlineto{\pgfqpoint{3.821333in}{3.983934in}}%
\pgfpathlineto{\pgfqpoint{3.746667in}{4.013782in}}%
\pgfpathlineto{\pgfqpoint{3.672000in}{4.040196in}}%
\pgfpathlineto{\pgfqpoint{3.597333in}{4.063442in}}%
\pgfpathlineto{\pgfqpoint{3.522667in}{4.083506in}}%
\pgfpathlineto{\pgfqpoint{3.448000in}{4.100547in}}%
\pgfpathlineto{\pgfqpoint{3.373333in}{4.114573in}}%
\pgfpathlineto{\pgfqpoint{3.298667in}{4.125800in}}%
\pgfpathlineto{\pgfqpoint{3.224000in}{4.134126in}}%
\pgfpathlineto{\pgfqpoint{3.149333in}{4.139620in}}%
\pgfpathlineto{\pgfqpoint{3.074667in}{4.142326in}}%
\pgfpathlineto{\pgfqpoint{3.000000in}{4.142266in}}%
\pgfpathlineto{\pgfqpoint{2.925333in}{4.139439in}}%
\pgfpathlineto{\pgfqpoint{2.850667in}{4.133824in}}%
\pgfpathlineto{\pgfqpoint{2.776000in}{4.125373in}}%
\pgfpathlineto{\pgfqpoint{2.701333in}{4.114018in}}%
\pgfpathlineto{\pgfqpoint{2.626667in}{4.099871in}}%
\pgfpathlineto{\pgfqpoint{2.552000in}{4.082693in}}%
\pgfpathlineto{\pgfqpoint{2.477333in}{4.062502in}}%
\pgfpathlineto{\pgfqpoint{2.397522in}{4.037333in}}%
\pgfpathlineto{\pgfqpoint{2.328000in}{4.012553in}}%
\pgfpathlineto{\pgfqpoint{2.253333in}{3.982558in}}%
\pgfpathlineto{\pgfqpoint{2.178667in}{3.948985in}}%
\pgfpathlineto{\pgfqpoint{2.104000in}{3.911648in}}%
\pgfpathlineto{\pgfqpoint{2.029333in}{3.870303in}}%
\pgfpathlineto{\pgfqpoint{1.954667in}{3.824642in}}%
\pgfpathlineto{\pgfqpoint{1.917333in}{3.800092in}}%
\pgfpathlineto{\pgfqpoint{1.842667in}{3.747273in}}%
\pgfpathlineto{\pgfqpoint{1.783174in}{3.701333in}}%
\pgfpathlineto{\pgfqpoint{1.730667in}{3.657785in}}%
\pgfpathlineto{\pgfqpoint{1.693333in}{3.624893in}}%
\pgfpathlineto{\pgfqpoint{1.654979in}{3.589333in}}%
\pgfpathlineto{\pgfqpoint{1.616777in}{3.552000in}}%
\pgfpathlineto{\pgfqpoint{1.580530in}{3.514667in}}%
\pgfpathlineto{\pgfqpoint{1.544000in}{3.474994in}}%
\pgfpathlineto{\pgfqpoint{1.506667in}{3.432126in}}%
\pgfpathlineto{\pgfqpoint{1.452496in}{3.365333in}}%
\pgfpathlineto{\pgfqpoint{1.432000in}{3.338375in}}%
\pgfpathlineto{\pgfqpoint{1.432000in}{3.338375in}}%
\pgfusepath{stroke}%
\end{pgfscope}%
\begin{pgfscope}%
\pgfpathrectangle{\pgfqpoint{1.432000in}{0.528000in}}{\pgfqpoint{3.696000in}{3.696000in}}%
\pgfusepath{clip}%
\pgfsetbuttcap%
\pgfsetroundjoin%
\pgfsetlinewidth{1.505625pt}%
\definecolor{currentstroke}{rgb}{1.000000,0.029903,0.000000}%
\pgfsetstrokecolor{currentstroke}%
\pgfsetstrokeopacity{0.300000}%
\pgfsetdash{}{0pt}%
\pgfpathmoveto{\pgfqpoint{1.432000in}{0.931530in}}%
\pgfpathlineto{\pgfqpoint{1.455021in}{0.901333in}}%
\pgfpathlineto{\pgfqpoint{1.469333in}{0.883235in}}%
\pgfpathlineto{\pgfqpoint{1.484822in}{0.864000in}}%
\pgfpathlineto{\pgfqpoint{1.506667in}{0.837791in}}%
\pgfpathlineto{\pgfqpoint{1.516125in}{0.826667in}}%
\pgfpathlineto{\pgfqpoint{1.544000in}{0.794924in}}%
\pgfpathlineto{\pgfqpoint{1.549018in}{0.789333in}}%
\pgfpathlineto{\pgfqpoint{1.581333in}{0.754401in}}%
\pgfpathlineto{\pgfqpoint{1.583608in}{0.752000in}}%
\pgfpathlineto{\pgfqpoint{1.618667in}{0.716020in}}%
\pgfpathlineto{\pgfqpoint{1.620020in}{0.714667in}}%
\pgfpathlineto{\pgfqpoint{1.656000in}{0.679608in}}%
\pgfpathlineto{\pgfqpoint{1.658401in}{0.677333in}}%
\pgfpathlineto{\pgfqpoint{1.693333in}{0.645018in}}%
\pgfpathlineto{\pgfqpoint{1.698924in}{0.640000in}}%
\pgfpathlineto{\pgfqpoint{1.730667in}{0.612125in}}%
\pgfpathlineto{\pgfqpoint{1.741791in}{0.602667in}}%
\pgfpathlineto{\pgfqpoint{1.768000in}{0.580822in}}%
\pgfpathlineto{\pgfqpoint{1.787235in}{0.565333in}}%
\pgfpathlineto{\pgfqpoint{1.805333in}{0.551021in}}%
\pgfpathlineto{\pgfqpoint{1.835530in}{0.528000in}}%
\pgfusepath{stroke}%
\end{pgfscope}%
\begin{pgfscope}%
\pgfpathrectangle{\pgfqpoint{1.432000in}{0.528000in}}{\pgfqpoint{3.696000in}{3.696000in}}%
\pgfusepath{clip}%
\pgfsetbuttcap%
\pgfsetroundjoin%
\pgfsetlinewidth{1.505625pt}%
\definecolor{currentstroke}{rgb}{1.000000,0.359314,0.000000}%
\pgfsetstrokecolor{currentstroke}%
\pgfsetstrokeopacity{0.300000}%
\pgfsetdash{}{0pt}%
\pgfpathmoveto{\pgfqpoint{3.548290in}{0.528000in}}%
\pgfpathlineto{\pgfqpoint{3.597333in}{0.544270in}}%
\pgfpathlineto{\pgfqpoint{3.672000in}{0.572495in}}%
\pgfpathlineto{\pgfqpoint{3.746667in}{0.604875in}}%
\pgfpathlineto{\pgfqpoint{3.821333in}{0.641677in}}%
\pgfpathlineto{\pgfqpoint{3.896000in}{0.683243in}}%
\pgfpathlineto{\pgfqpoint{3.947018in}{0.714667in}}%
\pgfpathlineto{\pgfqpoint{4.008000in}{0.755487in}}%
\pgfpathlineto{\pgfqpoint{4.054419in}{0.789333in}}%
\pgfpathlineto{\pgfqpoint{4.102071in}{0.826667in}}%
\pgfpathlineto{\pgfqpoint{4.146451in}{0.864000in}}%
\pgfpathlineto{\pgfqpoint{4.194667in}{0.907654in}}%
\pgfpathlineto{\pgfqpoint{4.232000in}{0.943913in}}%
\pgfpathlineto{\pgfqpoint{4.269333in}{0.982528in}}%
\pgfpathlineto{\pgfqpoint{4.306667in}{1.023746in}}%
\pgfpathlineto{\pgfqpoint{4.344000in}{1.067858in}}%
\pgfpathlineto{\pgfqpoint{4.389004in}{1.125333in}}%
\pgfpathlineto{\pgfqpoint{4.418667in}{1.166285in}}%
\pgfpathlineto{\pgfqpoint{4.456000in}{1.221841in}}%
\pgfpathlineto{\pgfqpoint{4.488748in}{1.274667in}}%
\pgfpathlineto{\pgfqpoint{4.510266in}{1.312000in}}%
\pgfpathlineto{\pgfqpoint{4.549616in}{1.386667in}}%
\pgfpathlineto{\pgfqpoint{4.584362in}{1.461333in}}%
\pgfpathlineto{\pgfqpoint{4.614805in}{1.536000in}}%
\pgfpathlineto{\pgfqpoint{4.641182in}{1.610667in}}%
\pgfpathlineto{\pgfqpoint{4.663742in}{1.685333in}}%
\pgfpathlineto{\pgfqpoint{4.682559in}{1.760000in}}%
\pgfpathlineto{\pgfqpoint{4.697852in}{1.834667in}}%
\pgfpathlineto{\pgfqpoint{4.709634in}{1.909333in}}%
\pgfpathlineto{\pgfqpoint{4.718008in}{1.984000in}}%
\pgfpathlineto{\pgfqpoint{4.723071in}{2.058667in}}%
\pgfpathlineto{\pgfqpoint{4.724802in}{2.133333in}}%
\pgfpathlineto{\pgfqpoint{4.723216in}{2.208000in}}%
\pgfpathlineto{\pgfqpoint{4.718298in}{2.282667in}}%
\pgfpathlineto{\pgfqpoint{4.710069in}{2.357333in}}%
\pgfpathlineto{\pgfqpoint{4.698437in}{2.432000in}}%
\pgfpathlineto{\pgfqpoint{4.683299in}{2.506667in}}%
\pgfpathlineto{\pgfqpoint{4.664636in}{2.581333in}}%
\pgfpathlineto{\pgfqpoint{4.642238in}{2.656000in}}%
\pgfpathlineto{\pgfqpoint{4.616037in}{2.730667in}}%
\pgfpathlineto{\pgfqpoint{4.585774in}{2.805333in}}%
\pgfpathlineto{\pgfqpoint{4.551219in}{2.880000in}}%
\pgfpathlineto{\pgfqpoint{4.512078in}{2.954667in}}%
\pgfpathlineto{\pgfqpoint{4.490669in}{2.992000in}}%
\pgfpathlineto{\pgfqpoint{4.456000in}{3.048068in}}%
\pgfpathlineto{\pgfqpoint{4.418430in}{3.104000in}}%
\pgfpathlineto{\pgfqpoint{4.381333in}{3.154692in}}%
\pgfpathlineto{\pgfqpoint{4.344000in}{3.202059in}}%
\pgfpathlineto{\pgfqpoint{4.300376in}{3.253333in}}%
\pgfpathlineto{\pgfqpoint{4.266269in}{3.290667in}}%
\pgfpathlineto{\pgfqpoint{4.230014in}{3.328000in}}%
\pgfpathlineto{\pgfqpoint{4.191399in}{3.365333in}}%
\pgfpathlineto{\pgfqpoint{4.150179in}{3.402667in}}%
\pgfpathlineto{\pgfqpoint{4.106059in}{3.440000in}}%
\pgfpathlineto{\pgfqpoint{4.058692in}{3.477333in}}%
\pgfpathlineto{\pgfqpoint{4.007660in}{3.514667in}}%
\pgfpathlineto{\pgfqpoint{3.952068in}{3.552000in}}%
\pgfpathlineto{\pgfqpoint{3.896000in}{3.586669in}}%
\pgfpathlineto{\pgfqpoint{3.858667in}{3.608078in}}%
\pgfpathlineto{\pgfqpoint{3.784000in}{3.647219in}}%
\pgfpathlineto{\pgfqpoint{3.709333in}{3.681774in}}%
\pgfpathlineto{\pgfqpoint{3.634667in}{3.712037in}}%
\pgfpathlineto{\pgfqpoint{3.558633in}{3.738667in}}%
\pgfpathlineto{\pgfqpoint{3.485333in}{3.760636in}}%
\pgfpathlineto{\pgfqpoint{3.410667in}{3.779299in}}%
\pgfpathlineto{\pgfqpoint{3.336000in}{3.794437in}}%
\pgfpathlineto{\pgfqpoint{3.261333in}{3.806069in}}%
\pgfpathlineto{\pgfqpoint{3.186667in}{3.814298in}}%
\pgfpathlineto{\pgfqpoint{3.112000in}{3.819216in}}%
\pgfpathlineto{\pgfqpoint{3.037333in}{3.820802in}}%
\pgfpathlineto{\pgfqpoint{2.962667in}{3.819071in}}%
\pgfpathlineto{\pgfqpoint{2.888000in}{3.814008in}}%
\pgfpathlineto{\pgfqpoint{2.813333in}{3.805634in}}%
\pgfpathlineto{\pgfqpoint{2.738667in}{3.793852in}}%
\pgfpathlineto{\pgfqpoint{2.664000in}{3.778559in}}%
\pgfpathlineto{\pgfqpoint{2.589333in}{3.759742in}}%
\pgfpathlineto{\pgfqpoint{2.514667in}{3.737182in}}%
\pgfpathlineto{\pgfqpoint{2.440000in}{3.710805in}}%
\pgfpathlineto{\pgfqpoint{2.365333in}{3.680362in}}%
\pgfpathlineto{\pgfqpoint{2.290667in}{3.645616in}}%
\pgfpathlineto{\pgfqpoint{2.216000in}{3.606266in}}%
\pgfpathlineto{\pgfqpoint{2.178667in}{3.584748in}}%
\pgfpathlineto{\pgfqpoint{2.125841in}{3.552000in}}%
\pgfpathlineto{\pgfqpoint{2.066667in}{3.512141in}}%
\pgfpathlineto{\pgfqpoint{2.019210in}{3.477333in}}%
\pgfpathlineto{\pgfqpoint{1.971858in}{3.440000in}}%
\pgfpathlineto{\pgfqpoint{1.917333in}{3.393484in}}%
\pgfpathlineto{\pgfqpoint{1.880000in}{3.359194in}}%
\pgfpathlineto{\pgfqpoint{1.842667in}{3.322739in}}%
\pgfpathlineto{\pgfqpoint{1.805333in}{3.283908in}}%
\pgfpathlineto{\pgfqpoint{1.768000in}{3.242451in}}%
\pgfpathlineto{\pgfqpoint{1.730667in}{3.198071in}}%
\pgfpathlineto{\pgfqpoint{1.686486in}{3.141333in}}%
\pgfpathlineto{\pgfqpoint{1.656000in}{3.098976in}}%
\pgfpathlineto{\pgfqpoint{1.618667in}{3.043018in}}%
\pgfpathlineto{\pgfqpoint{1.587243in}{2.992000in}}%
\pgfpathlineto{\pgfqpoint{1.565835in}{2.954667in}}%
\pgfpathlineto{\pgfqpoint{1.526693in}{2.880000in}}%
\pgfpathlineto{\pgfqpoint{1.492140in}{2.805333in}}%
\pgfpathlineto{\pgfqpoint{1.461880in}{2.730667in}}%
\pgfpathlineto{\pgfqpoint{1.435668in}{2.656000in}}%
\pgfpathlineto{\pgfqpoint{1.432000in}{2.644290in}}%
\pgfpathlineto{\pgfqpoint{1.432000in}{2.644290in}}%
\pgfusepath{stroke}%
\end{pgfscope}%
\begin{pgfscope}%
\pgfpathrectangle{\pgfqpoint{1.432000in}{0.528000in}}{\pgfqpoint{3.696000in}{3.696000in}}%
\pgfusepath{clip}%
\pgfsetbuttcap%
\pgfsetroundjoin%
\pgfsetlinewidth{1.505625pt}%
\definecolor{currentstroke}{rgb}{1.000000,0.359314,0.000000}%
\pgfsetstrokecolor{currentstroke}%
\pgfsetstrokeopacity{0.300000}%
\pgfsetdash{}{0pt}%
\pgfpathmoveto{\pgfqpoint{1.432000in}{1.625653in}}%
\pgfpathlineto{\pgfqpoint{1.436724in}{1.610667in}}%
\pgfpathlineto{\pgfqpoint{1.449413in}{1.573333in}}%
\pgfpathlineto{\pgfqpoint{1.463112in}{1.536000in}}%
\pgfpathlineto{\pgfqpoint{1.469333in}{1.520141in}}%
\pgfpathlineto{\pgfqpoint{1.477813in}{1.498667in}}%
\pgfpathlineto{\pgfqpoint{1.493552in}{1.461333in}}%
\pgfpathlineto{\pgfqpoint{1.506667in}{1.432178in}}%
\pgfpathlineto{\pgfqpoint{1.510379in}{1.424000in}}%
\pgfpathlineto{\pgfqpoint{1.528297in}{1.386667in}}%
\pgfpathlineto{\pgfqpoint{1.544000in}{1.355884in}}%
\pgfpathlineto{\pgfqpoint{1.547381in}{1.349333in}}%
\pgfpathlineto{\pgfqpoint{1.567647in}{1.312000in}}%
\pgfpathlineto{\pgfqpoint{1.581333in}{1.288129in}}%
\pgfpathlineto{\pgfqpoint{1.589165in}{1.274667in}}%
\pgfpathlineto{\pgfqpoint{1.611979in}{1.237333in}}%
\pgfpathlineto{\pgfqpoint{1.618667in}{1.226880in}}%
\pgfpathlineto{\pgfqpoint{1.636162in}{1.200000in}}%
\pgfpathlineto{\pgfqpoint{1.656000in}{1.170960in}}%
\pgfpathlineto{\pgfqpoint{1.661775in}{1.162667in}}%
\pgfpathlineto{\pgfqpoint{1.688905in}{1.125333in}}%
\pgfpathlineto{\pgfqpoint{1.693333in}{1.119476in}}%
\pgfpathlineto{\pgfqpoint{1.717659in}{1.088000in}}%
\pgfpathlineto{\pgfqpoint{1.730667in}{1.071840in}}%
\pgfpathlineto{\pgfqpoint{1.748127in}{1.050667in}}%
\pgfpathlineto{\pgfqpoint{1.768000in}{1.027469in}}%
\pgfpathlineto{\pgfqpoint{1.780433in}{1.013333in}}%
\pgfpathlineto{\pgfqpoint{1.805333in}{0.986015in}}%
\pgfpathlineto{\pgfqpoint{1.814727in}{0.976000in}}%
\pgfpathlineto{\pgfqpoint{1.842667in}{0.947184in}}%
\pgfpathlineto{\pgfqpoint{1.851184in}{0.938667in}}%
\pgfpathlineto{\pgfqpoint{1.880000in}{0.910727in}}%
\pgfpathlineto{\pgfqpoint{1.890015in}{0.901333in}}%
\pgfpathlineto{\pgfqpoint{1.917333in}{0.876433in}}%
\pgfpathlineto{\pgfqpoint{1.931469in}{0.864000in}}%
\pgfpathlineto{\pgfqpoint{1.954667in}{0.844127in}}%
\pgfpathlineto{\pgfqpoint{1.975840in}{0.826667in}}%
\pgfpathlineto{\pgfqpoint{1.992000in}{0.813659in}}%
\pgfpathlineto{\pgfqpoint{2.023476in}{0.789333in}}%
\pgfpathlineto{\pgfqpoint{2.029333in}{0.784905in}}%
\pgfpathlineto{\pgfqpoint{2.066667in}{0.757775in}}%
\pgfpathlineto{\pgfqpoint{2.074960in}{0.752000in}}%
\pgfpathlineto{\pgfqpoint{2.104000in}{0.732162in}}%
\pgfpathlineto{\pgfqpoint{2.130880in}{0.714667in}}%
\pgfpathlineto{\pgfqpoint{2.141333in}{0.707979in}}%
\pgfpathlineto{\pgfqpoint{2.178667in}{0.685165in}}%
\pgfpathlineto{\pgfqpoint{2.192129in}{0.677333in}}%
\pgfpathlineto{\pgfqpoint{2.216000in}{0.663647in}}%
\pgfpathlineto{\pgfqpoint{2.253333in}{0.643381in}}%
\pgfpathlineto{\pgfqpoint{2.259884in}{0.640000in}}%
\pgfpathlineto{\pgfqpoint{2.290667in}{0.624297in}}%
\pgfpathlineto{\pgfqpoint{2.328000in}{0.606379in}}%
\pgfpathlineto{\pgfqpoint{2.336178in}{0.602667in}}%
\pgfpathlineto{\pgfqpoint{2.365333in}{0.589552in}}%
\pgfpathlineto{\pgfqpoint{2.402667in}{0.573813in}}%
\pgfpathlineto{\pgfqpoint{2.424141in}{0.565333in}}%
\pgfpathlineto{\pgfqpoint{2.440000in}{0.559112in}}%
\pgfpathlineto{\pgfqpoint{2.477333in}{0.545413in}}%
\pgfpathlineto{\pgfqpoint{2.514667in}{0.532724in}}%
\pgfpathlineto{\pgfqpoint{2.529653in}{0.528000in}}%
\pgfusepath{stroke}%
\end{pgfscope}%
\begin{pgfscope}%
\pgfpathrectangle{\pgfqpoint{1.432000in}{0.528000in}}{\pgfqpoint{3.696000in}{3.696000in}}%
\pgfusepath{clip}%
\pgfsetbuttcap%
\pgfsetroundjoin%
\pgfsetlinewidth{1.505625pt}%
\definecolor{currentstroke}{rgb}{1.000000,0.688725,0.000000}%
\pgfsetstrokecolor{currentstroke}%
\pgfsetstrokeopacity{0.300000}%
\pgfsetdash{}{0pt}%
\pgfpathmoveto{\pgfqpoint{2.962667in}{0.751407in}}%
\pgfpathlineto{\pgfqpoint{3.037333in}{0.749307in}}%
\pgfpathlineto{\pgfqpoint{3.112000in}{0.751232in}}%
\pgfpathlineto{\pgfqpoint{3.186667in}{0.757206in}}%
\pgfpathlineto{\pgfqpoint{3.261333in}{0.767277in}}%
\pgfpathlineto{\pgfqpoint{3.336000in}{0.781526in}}%
\pgfpathlineto{\pgfqpoint{3.410667in}{0.800108in}}%
\pgfpathlineto{\pgfqpoint{3.485333in}{0.823179in}}%
\pgfpathlineto{\pgfqpoint{3.522667in}{0.836494in}}%
\pgfpathlineto{\pgfqpoint{3.590797in}{0.864000in}}%
\pgfpathlineto{\pgfqpoint{3.634667in}{0.883926in}}%
\pgfpathlineto{\pgfqpoint{3.672000in}{0.902367in}}%
\pgfpathlineto{\pgfqpoint{3.738050in}{0.938667in}}%
\pgfpathlineto{\pgfqpoint{3.784000in}{0.966686in}}%
\pgfpathlineto{\pgfqpoint{3.821333in}{0.991369in}}%
\pgfpathlineto{\pgfqpoint{3.858667in}{1.017796in}}%
\pgfpathlineto{\pgfqpoint{3.901647in}{1.050667in}}%
\pgfpathlineto{\pgfqpoint{3.946549in}{1.088000in}}%
\pgfpathlineto{\pgfqpoint{3.987902in}{1.125333in}}%
\pgfpathlineto{\pgfqpoint{4.026124in}{1.162667in}}%
\pgfpathlineto{\pgfqpoint{4.061561in}{1.200000in}}%
\pgfpathlineto{\pgfqpoint{4.094498in}{1.237333in}}%
\pgfpathlineto{\pgfqpoint{4.125172in}{1.274667in}}%
\pgfpathlineto{\pgfqpoint{4.157333in}{1.316937in}}%
\pgfpathlineto{\pgfqpoint{4.194667in}{1.370678in}}%
\pgfpathlineto{\pgfqpoint{4.228297in}{1.424000in}}%
\pgfpathlineto{\pgfqpoint{4.249809in}{1.461333in}}%
\pgfpathlineto{\pgfqpoint{4.269876in}{1.498667in}}%
\pgfpathlineto{\pgfqpoint{4.288435in}{1.536000in}}%
\pgfpathlineto{\pgfqpoint{4.306667in}{1.575605in}}%
\pgfpathlineto{\pgfqpoint{4.321565in}{1.610667in}}%
\pgfpathlineto{\pgfqpoint{4.344000in}{1.669565in}}%
\pgfpathlineto{\pgfqpoint{4.361831in}{1.722667in}}%
\pgfpathlineto{\pgfqpoint{4.382845in}{1.797333in}}%
\pgfpathlineto{\pgfqpoint{4.399416in}{1.872000in}}%
\pgfpathlineto{\pgfqpoint{4.411749in}{1.946667in}}%
\pgfpathlineto{\pgfqpoint{4.419941in}{2.021333in}}%
\pgfpathlineto{\pgfqpoint{4.424057in}{2.096000in}}%
\pgfpathlineto{\pgfqpoint{4.424145in}{2.170667in}}%
\pgfpathlineto{\pgfqpoint{4.420204in}{2.245333in}}%
\pgfpathlineto{\pgfqpoint{4.412190in}{2.320000in}}%
\pgfpathlineto{\pgfqpoint{4.400039in}{2.394667in}}%
\pgfpathlineto{\pgfqpoint{4.383655in}{2.469333in}}%
\pgfpathlineto{\pgfqpoint{4.373809in}{2.506667in}}%
\pgfpathlineto{\pgfqpoint{4.350729in}{2.581333in}}%
\pgfpathlineto{\pgfqpoint{4.337424in}{2.618667in}}%
\pgfpathlineto{\pgfqpoint{4.322887in}{2.656000in}}%
\pgfpathlineto{\pgfqpoint{4.306667in}{2.694342in}}%
\pgfpathlineto{\pgfqpoint{4.289987in}{2.730667in}}%
\pgfpathlineto{\pgfqpoint{4.269333in}{2.772211in}}%
\pgfpathlineto{\pgfqpoint{4.232000in}{2.839655in}}%
\pgfpathlineto{\pgfqpoint{4.207219in}{2.880000in}}%
\pgfpathlineto{\pgfqpoint{4.182549in}{2.917333in}}%
\pgfpathlineto{\pgfqpoint{4.156136in}{2.954667in}}%
\pgfpathlineto{\pgfqpoint{4.120000in}{3.001679in}}%
\pgfpathlineto{\pgfqpoint{4.082667in}{3.046271in}}%
\pgfpathlineto{\pgfqpoint{4.045333in}{3.087345in}}%
\pgfpathlineto{\pgfqpoint{4.008000in}{3.125313in}}%
\pgfpathlineto{\pgfqpoint{3.970667in}{3.160518in}}%
\pgfpathlineto{\pgfqpoint{3.933333in}{3.193243in}}%
\pgfpathlineto{\pgfqpoint{3.896000in}{3.223723in}}%
\pgfpathlineto{\pgfqpoint{3.857018in}{3.253333in}}%
\pgfpathlineto{\pgfqpoint{3.803231in}{3.290667in}}%
\pgfpathlineto{\pgfqpoint{3.784000in}{3.303219in}}%
\pgfpathlineto{\pgfqpoint{3.743655in}{3.328000in}}%
\pgfpathlineto{\pgfqpoint{3.676211in}{3.365333in}}%
\pgfpathlineto{\pgfqpoint{3.634667in}{3.385987in}}%
\pgfpathlineto{\pgfqpoint{3.597333in}{3.403122in}}%
\pgfpathlineto{\pgfqpoint{3.560000in}{3.418887in}}%
\pgfpathlineto{\pgfqpoint{3.504340in}{3.440000in}}%
\pgfpathlineto{\pgfqpoint{3.448000in}{3.458838in}}%
\pgfpathlineto{\pgfqpoint{3.373333in}{3.479655in}}%
\pgfpathlineto{\pgfqpoint{3.298667in}{3.496039in}}%
\pgfpathlineto{\pgfqpoint{3.224000in}{3.508190in}}%
\pgfpathlineto{\pgfqpoint{3.149333in}{3.516204in}}%
\pgfpathlineto{\pgfqpoint{3.074667in}{3.520145in}}%
\pgfpathlineto{\pgfqpoint{3.000000in}{3.520057in}}%
\pgfpathlineto{\pgfqpoint{2.925333in}{3.515941in}}%
\pgfpathlineto{\pgfqpoint{2.850667in}{3.507749in}}%
\pgfpathlineto{\pgfqpoint{2.776000in}{3.495416in}}%
\pgfpathlineto{\pgfqpoint{2.701333in}{3.478845in}}%
\pgfpathlineto{\pgfqpoint{2.664000in}{3.468899in}}%
\pgfpathlineto{\pgfqpoint{2.589333in}{3.445622in}}%
\pgfpathlineto{\pgfqpoint{2.552000in}{3.432208in}}%
\pgfpathlineto{\pgfqpoint{2.479605in}{3.402667in}}%
\pgfpathlineto{\pgfqpoint{2.440000in}{3.384435in}}%
\pgfpathlineto{\pgfqpoint{2.401638in}{3.365333in}}%
\pgfpathlineto{\pgfqpoint{2.334309in}{3.328000in}}%
\pgfpathlineto{\pgfqpoint{2.290667in}{3.301147in}}%
\pgfpathlineto{\pgfqpoint{2.253333in}{3.276317in}}%
\pgfpathlineto{\pgfqpoint{2.216000in}{3.249732in}}%
\pgfpathlineto{\pgfqpoint{2.172209in}{3.216000in}}%
\pgfpathlineto{\pgfqpoint{2.127634in}{3.178667in}}%
\pgfpathlineto{\pgfqpoint{2.086568in}{3.141333in}}%
\pgfpathlineto{\pgfqpoint{2.048601in}{3.104000in}}%
\pgfpathlineto{\pgfqpoint{2.013393in}{3.066667in}}%
\pgfpathlineto{\pgfqpoint{1.980661in}{3.029333in}}%
\pgfpathlineto{\pgfqpoint{1.950174in}{2.992000in}}%
\pgfpathlineto{\pgfqpoint{1.917333in}{2.948505in}}%
\pgfpathlineto{\pgfqpoint{1.880000in}{2.894308in}}%
\pgfpathlineto{\pgfqpoint{1.847686in}{2.842667in}}%
\pgfpathlineto{\pgfqpoint{1.826303in}{2.805333in}}%
\pgfpathlineto{\pgfqpoint{1.805333in}{2.765945in}}%
\pgfpathlineto{\pgfqpoint{1.787926in}{2.730667in}}%
\pgfpathlineto{\pgfqpoint{1.768000in}{2.686797in}}%
\pgfpathlineto{\pgfqpoint{1.740494in}{2.618667in}}%
\pgfpathlineto{\pgfqpoint{1.727179in}{2.581333in}}%
\pgfpathlineto{\pgfqpoint{1.704108in}{2.506667in}}%
\pgfpathlineto{\pgfqpoint{1.685526in}{2.432000in}}%
\pgfpathlineto{\pgfqpoint{1.671277in}{2.357333in}}%
\pgfpathlineto{\pgfqpoint{1.661206in}{2.282667in}}%
\pgfpathlineto{\pgfqpoint{1.655232in}{2.208000in}}%
\pgfpathlineto{\pgfqpoint{1.653307in}{2.133333in}}%
\pgfpathlineto{\pgfqpoint{1.655407in}{2.058667in}}%
\pgfpathlineto{\pgfqpoint{1.661558in}{1.984000in}}%
\pgfpathlineto{\pgfqpoint{1.671809in}{1.909333in}}%
\pgfpathlineto{\pgfqpoint{1.686242in}{1.834667in}}%
\pgfpathlineto{\pgfqpoint{1.695069in}{1.797333in}}%
\pgfpathlineto{\pgfqpoint{1.716081in}{1.722667in}}%
\pgfpathlineto{\pgfqpoint{1.730667in}{1.678639in}}%
\pgfpathlineto{\pgfqpoint{1.756345in}{1.610667in}}%
\pgfpathlineto{\pgfqpoint{1.772237in}{1.573333in}}%
\pgfpathlineto{\pgfqpoint{1.805333in}{1.504032in}}%
\pgfpathlineto{\pgfqpoint{1.828102in}{1.461333in}}%
\pgfpathlineto{\pgfqpoint{1.849627in}{1.424000in}}%
\pgfpathlineto{\pgfqpoint{1.880000in}{1.375586in}}%
\pgfpathlineto{\pgfqpoint{1.917333in}{1.321438in}}%
\pgfpathlineto{\pgfqpoint{1.924197in}{1.312000in}}%
\pgfpathlineto{\pgfqpoint{1.954667in}{1.272232in}}%
\pgfpathlineto{\pgfqpoint{1.992000in}{1.227349in}}%
\pgfpathlineto{\pgfqpoint{2.029333in}{1.186005in}}%
\pgfpathlineto{\pgfqpoint{2.066667in}{1.147784in}}%
\pgfpathlineto{\pgfqpoint{2.104000in}{1.112345in}}%
\pgfpathlineto{\pgfqpoint{2.141333in}{1.079403in}}%
\pgfpathlineto{\pgfqpoint{2.178667in}{1.048722in}}%
\pgfpathlineto{\pgfqpoint{2.225438in}{1.013333in}}%
\pgfpathlineto{\pgfqpoint{2.279586in}{0.976000in}}%
\pgfpathlineto{\pgfqpoint{2.328000in}{0.945627in}}%
\pgfpathlineto{\pgfqpoint{2.365333in}{0.924102in}}%
\pgfpathlineto{\pgfqpoint{2.408032in}{0.901333in}}%
\pgfpathlineto{\pgfqpoint{2.440000in}{0.885476in}}%
\pgfpathlineto{\pgfqpoint{2.487169in}{0.864000in}}%
\pgfpathlineto{\pgfqpoint{2.552000in}{0.837710in}}%
\pgfpathlineto{\pgfqpoint{2.589333in}{0.824286in}}%
\pgfpathlineto{\pgfqpoint{2.664000in}{0.801017in}}%
\pgfpathlineto{\pgfqpoint{2.738667in}{0.782242in}}%
\pgfpathlineto{\pgfqpoint{2.813333in}{0.767809in}}%
\pgfpathlineto{\pgfqpoint{2.888000in}{0.757558in}}%
\pgfpathlineto{\pgfqpoint{2.962667in}{0.751407in}}%
\pgfpathlineto{\pgfqpoint{2.962667in}{0.751407in}}%
\pgfusepath{stroke}%
\end{pgfscope}%
\begin{pgfscope}%
\pgfpathrectangle{\pgfqpoint{1.432000in}{0.528000in}}{\pgfqpoint{3.696000in}{3.696000in}}%
\pgfusepath{clip}%
\pgfsetbuttcap%
\pgfsetroundjoin%
\pgfsetlinewidth{1.505625pt}%
\definecolor{currentstroke}{rgb}{1.000000,1.000000,0.027205}%
\pgfsetstrokecolor{currentstroke}%
\pgfsetstrokeopacity{0.300000}%
\pgfsetdash{}{0pt}%
\pgfpathmoveto{\pgfqpoint{2.776000in}{1.087303in}}%
\pgfpathlineto{\pgfqpoint{2.813333in}{1.078684in}}%
\pgfpathlineto{\pgfqpoint{2.850667in}{1.071412in}}%
\pgfpathlineto{\pgfqpoint{2.888000in}{1.065472in}}%
\pgfpathlineto{\pgfqpoint{2.925333in}{1.060852in}}%
\pgfpathlineto{\pgfqpoint{2.962667in}{1.057543in}}%
\pgfpathlineto{\pgfqpoint{3.000000in}{1.055537in}}%
\pgfpathlineto{\pgfqpoint{3.037333in}{1.054831in}}%
\pgfpathlineto{\pgfqpoint{3.074667in}{1.055424in}}%
\pgfpathlineto{\pgfqpoint{3.112000in}{1.057317in}}%
\pgfpathlineto{\pgfqpoint{3.149333in}{1.060512in}}%
\pgfpathlineto{\pgfqpoint{3.186667in}{1.065018in}}%
\pgfpathlineto{\pgfqpoint{3.224000in}{1.070843in}}%
\pgfpathlineto{\pgfqpoint{3.261333in}{1.077999in}}%
\pgfpathlineto{\pgfqpoint{3.304368in}{1.088000in}}%
\pgfpathlineto{\pgfqpoint{3.336000in}{1.096505in}}%
\pgfpathlineto{\pgfqpoint{3.373333in}{1.107943in}}%
\pgfpathlineto{\pgfqpoint{3.422514in}{1.125333in}}%
\pgfpathlineto{\pgfqpoint{3.448000in}{1.135319in}}%
\pgfpathlineto{\pgfqpoint{3.509168in}{1.162667in}}%
\pgfpathlineto{\pgfqpoint{3.522667in}{1.169220in}}%
\pgfpathlineto{\pgfqpoint{3.579597in}{1.200000in}}%
\pgfpathlineto{\pgfqpoint{3.597333in}{1.210411in}}%
\pgfpathlineto{\pgfqpoint{3.639685in}{1.237333in}}%
\pgfpathlineto{\pgfqpoint{3.691919in}{1.274667in}}%
\pgfpathlineto{\pgfqpoint{3.738451in}{1.312000in}}%
\pgfpathlineto{\pgfqpoint{3.746667in}{1.319031in}}%
\pgfpathlineto{\pgfqpoint{3.784000in}{1.352935in}}%
\pgfpathlineto{\pgfqpoint{3.821333in}{1.390288in}}%
\pgfpathlineto{\pgfqpoint{3.858667in}{1.431653in}}%
\pgfpathlineto{\pgfqpoint{3.896000in}{1.477736in}}%
\pgfpathlineto{\pgfqpoint{3.911686in}{1.498667in}}%
\pgfpathlineto{\pgfqpoint{3.937788in}{1.536000in}}%
\pgfpathlineto{\pgfqpoint{3.970667in}{1.588732in}}%
\pgfpathlineto{\pgfqpoint{3.983244in}{1.610667in}}%
\pgfpathlineto{\pgfqpoint{4.008000in}{1.658045in}}%
\pgfpathlineto{\pgfqpoint{4.020999in}{1.685333in}}%
\pgfpathlineto{\pgfqpoint{4.045333in}{1.743012in}}%
\pgfpathlineto{\pgfqpoint{4.051894in}{1.760000in}}%
\pgfpathlineto{\pgfqpoint{4.064907in}{1.797333in}}%
\pgfpathlineto{\pgfqpoint{4.082667in}{1.857316in}}%
\pgfpathlineto{\pgfqpoint{4.086587in}{1.872000in}}%
\pgfpathlineto{\pgfqpoint{4.095218in}{1.909333in}}%
\pgfpathlineto{\pgfqpoint{4.102500in}{1.946667in}}%
\pgfpathlineto{\pgfqpoint{4.108448in}{1.984000in}}%
\pgfpathlineto{\pgfqpoint{4.113074in}{2.021333in}}%
\pgfpathlineto{\pgfqpoint{4.116388in}{2.058667in}}%
\pgfpathlineto{\pgfqpoint{4.118397in}{2.096000in}}%
\pgfpathlineto{\pgfqpoint{4.119103in}{2.133333in}}%
\pgfpathlineto{\pgfqpoint{4.118510in}{2.170667in}}%
\pgfpathlineto{\pgfqpoint{4.116615in}{2.208000in}}%
\pgfpathlineto{\pgfqpoint{4.113415in}{2.245333in}}%
\pgfpathlineto{\pgfqpoint{4.108903in}{2.282667in}}%
\pgfpathlineto{\pgfqpoint{4.103070in}{2.320000in}}%
\pgfpathlineto{\pgfqpoint{4.095904in}{2.357333in}}%
\pgfpathlineto{\pgfqpoint{4.082667in}{2.412570in}}%
\pgfpathlineto{\pgfqpoint{4.077425in}{2.432000in}}%
\pgfpathlineto{\pgfqpoint{4.065970in}{2.469333in}}%
\pgfpathlineto{\pgfqpoint{4.045333in}{2.526900in}}%
\pgfpathlineto{\pgfqpoint{4.038611in}{2.544000in}}%
\pgfpathlineto{\pgfqpoint{4.022475in}{2.581333in}}%
\pgfpathlineto{\pgfqpoint{4.004719in}{2.618667in}}%
\pgfpathlineto{\pgfqpoint{3.970667in}{2.681206in}}%
\pgfpathlineto{\pgfqpoint{3.963522in}{2.693333in}}%
\pgfpathlineto{\pgfqpoint{3.933333in}{2.740435in}}%
\pgfpathlineto{\pgfqpoint{3.896000in}{2.792192in}}%
\pgfpathlineto{\pgfqpoint{3.885802in}{2.805333in}}%
\pgfpathlineto{\pgfqpoint{3.854919in}{2.842667in}}%
\pgfpathlineto{\pgfqpoint{3.821029in}{2.880000in}}%
\pgfpathlineto{\pgfqpoint{3.783680in}{2.917333in}}%
\pgfpathlineto{\pgfqpoint{3.742305in}{2.954667in}}%
\pgfpathlineto{\pgfqpoint{3.696192in}{2.992000in}}%
\pgfpathlineto{\pgfqpoint{3.644435in}{3.029333in}}%
\pgfpathlineto{\pgfqpoint{3.634667in}{3.035924in}}%
\pgfpathlineto{\pgfqpoint{3.585206in}{3.066667in}}%
\pgfpathlineto{\pgfqpoint{3.560000in}{3.081034in}}%
\pgfpathlineto{\pgfqpoint{3.515931in}{3.104000in}}%
\pgfpathlineto{\pgfqpoint{3.485333in}{3.118475in}}%
\pgfpathlineto{\pgfqpoint{3.430900in}{3.141333in}}%
\pgfpathlineto{\pgfqpoint{3.410667in}{3.149083in}}%
\pgfpathlineto{\pgfqpoint{3.373333in}{3.161970in}}%
\pgfpathlineto{\pgfqpoint{3.316570in}{3.178667in}}%
\pgfpathlineto{\pgfqpoint{3.298667in}{3.183391in}}%
\pgfpathlineto{\pgfqpoint{3.261333in}{3.191904in}}%
\pgfpathlineto{\pgfqpoint{3.224000in}{3.199070in}}%
\pgfpathlineto{\pgfqpoint{3.186667in}{3.204903in}}%
\pgfpathlineto{\pgfqpoint{3.149333in}{3.209415in}}%
\pgfpathlineto{\pgfqpoint{3.112000in}{3.212615in}}%
\pgfpathlineto{\pgfqpoint{3.074667in}{3.214510in}}%
\pgfpathlineto{\pgfqpoint{3.037333in}{3.215103in}}%
\pgfpathlineto{\pgfqpoint{3.000000in}{3.214397in}}%
\pgfpathlineto{\pgfqpoint{2.962667in}{3.212388in}}%
\pgfpathlineto{\pgfqpoint{2.925333in}{3.209074in}}%
\pgfpathlineto{\pgfqpoint{2.888000in}{3.204448in}}%
\pgfpathlineto{\pgfqpoint{2.850667in}{3.198500in}}%
\pgfpathlineto{\pgfqpoint{2.813333in}{3.191218in}}%
\pgfpathlineto{\pgfqpoint{2.761316in}{3.178667in}}%
\pgfpathlineto{\pgfqpoint{2.738667in}{3.172486in}}%
\pgfpathlineto{\pgfqpoint{2.701333in}{3.160907in}}%
\pgfpathlineto{\pgfqpoint{2.647012in}{3.141333in}}%
\pgfpathlineto{\pgfqpoint{2.626667in}{3.133269in}}%
\pgfpathlineto{\pgfqpoint{2.562045in}{3.104000in}}%
\pgfpathlineto{\pgfqpoint{2.552000in}{3.099072in}}%
\pgfpathlineto{\pgfqpoint{2.492732in}{3.066667in}}%
\pgfpathlineto{\pgfqpoint{2.477333in}{3.057540in}}%
\pgfpathlineto{\pgfqpoint{2.433432in}{3.029333in}}%
\pgfpathlineto{\pgfqpoint{2.381736in}{2.992000in}}%
\pgfpathlineto{\pgfqpoint{2.335653in}{2.954667in}}%
\pgfpathlineto{\pgfqpoint{2.328000in}{2.948059in}}%
\pgfpathlineto{\pgfqpoint{2.290667in}{2.913874in}}%
\pgfpathlineto{\pgfqpoint{2.253333in}{2.876196in}}%
\pgfpathlineto{\pgfqpoint{2.216000in}{2.834451in}}%
\pgfpathlineto{\pgfqpoint{2.178667in}{2.787919in}}%
\pgfpathlineto{\pgfqpoint{2.163867in}{2.768000in}}%
\pgfpathlineto{\pgfqpoint{2.137959in}{2.730667in}}%
\pgfpathlineto{\pgfqpoint{2.104000in}{2.675597in}}%
\pgfpathlineto{\pgfqpoint{2.092867in}{2.656000in}}%
\pgfpathlineto{\pgfqpoint{2.066667in}{2.605168in}}%
\pgfpathlineto{\pgfqpoint{2.055428in}{2.581333in}}%
\pgfpathlineto{\pgfqpoint{2.029333in}{2.518514in}}%
\pgfpathlineto{\pgfqpoint{2.024810in}{2.506667in}}%
\pgfpathlineto{\pgfqpoint{2.011943in}{2.469333in}}%
\pgfpathlineto{\pgfqpoint{2.000505in}{2.432000in}}%
\pgfpathlineto{\pgfqpoint{1.990500in}{2.394667in}}%
\pgfpathlineto{\pgfqpoint{1.981999in}{2.357333in}}%
\pgfpathlineto{\pgfqpoint{1.974843in}{2.320000in}}%
\pgfpathlineto{\pgfqpoint{1.969018in}{2.282667in}}%
\pgfpathlineto{\pgfqpoint{1.964512in}{2.245333in}}%
\pgfpathlineto{\pgfqpoint{1.961317in}{2.208000in}}%
\pgfpathlineto{\pgfqpoint{1.959424in}{2.170667in}}%
\pgfpathlineto{\pgfqpoint{1.958831in}{2.133333in}}%
\pgfpathlineto{\pgfqpoint{1.959537in}{2.096000in}}%
\pgfpathlineto{\pgfqpoint{1.961543in}{2.058667in}}%
\pgfpathlineto{\pgfqpoint{1.964852in}{2.021333in}}%
\pgfpathlineto{\pgfqpoint{1.969472in}{1.984000in}}%
\pgfpathlineto{\pgfqpoint{1.975412in}{1.946667in}}%
\pgfpathlineto{\pgfqpoint{1.982684in}{1.909333in}}%
\pgfpathlineto{\pgfqpoint{1.992000in}{1.869383in}}%
\pgfpathlineto{\pgfqpoint{2.001444in}{1.834667in}}%
\pgfpathlineto{\pgfqpoint{2.013004in}{1.797333in}}%
\pgfpathlineto{\pgfqpoint{2.029333in}{1.751334in}}%
\pgfpathlineto{\pgfqpoint{2.040659in}{1.722667in}}%
\pgfpathlineto{\pgfqpoint{2.066667in}{1.664766in}}%
\pgfpathlineto{\pgfqpoint{2.074864in}{1.648000in}}%
\pgfpathlineto{\pgfqpoint{2.104000in}{1.594312in}}%
\pgfpathlineto{\pgfqpoint{2.116389in}{1.573333in}}%
\pgfpathlineto{\pgfqpoint{2.141333in}{1.534162in}}%
\pgfpathlineto{\pgfqpoint{2.178667in}{1.481992in}}%
\pgfpathlineto{\pgfqpoint{2.216000in}{1.435493in}}%
\pgfpathlineto{\pgfqpoint{2.253333in}{1.393760in}}%
\pgfpathlineto{\pgfqpoint{2.290667in}{1.356079in}}%
\pgfpathlineto{\pgfqpoint{2.328000in}{1.321882in}}%
\pgfpathlineto{\pgfqpoint{2.365333in}{1.290716in}}%
\pgfpathlineto{\pgfqpoint{2.402667in}{1.262216in}}%
\pgfpathlineto{\pgfqpoint{2.440000in}{1.236091in}}%
\pgfpathlineto{\pgfqpoint{2.498312in}{1.200000in}}%
\pgfpathlineto{\pgfqpoint{2.514667in}{1.190654in}}%
\pgfpathlineto{\pgfqpoint{2.568766in}{1.162667in}}%
\pgfpathlineto{\pgfqpoint{2.589333in}{1.152901in}}%
\pgfpathlineto{\pgfqpoint{2.655334in}{1.125333in}}%
\pgfpathlineto{\pgfqpoint{2.664000in}{1.121997in}}%
\pgfpathlineto{\pgfqpoint{2.701333in}{1.109004in}}%
\pgfpathlineto{\pgfqpoint{2.738667in}{1.097444in}}%
\pgfpathlineto{\pgfqpoint{2.776000in}{1.087303in}}%
\pgfpathlineto{\pgfqpoint{2.776000in}{1.087303in}}%
\pgfusepath{stroke}%
\end{pgfscope}%
\begin{pgfscope}%
\pgfpathrectangle{\pgfqpoint{1.432000in}{0.528000in}}{\pgfqpoint{3.696000in}{3.696000in}}%
\pgfusepath{clip}%
\pgfsetbuttcap%
\pgfsetroundjoin%
\pgfsetlinewidth{1.505625pt}%
\definecolor{currentstroke}{rgb}{1.000000,1.000000,0.521323}%
\pgfsetstrokecolor{currentstroke}%
\pgfsetstrokeopacity{0.300000}%
\pgfsetdash{}{0pt}%
\pgfpathmoveto{\pgfqpoint{2.888000in}{1.422907in}}%
\pgfpathlineto{\pgfqpoint{2.925333in}{1.416104in}}%
\pgfpathlineto{\pgfqpoint{2.962667in}{1.411231in}}%
\pgfpathlineto{\pgfqpoint{3.000000in}{1.408278in}}%
\pgfpathlineto{\pgfqpoint{3.037333in}{1.407239in}}%
\pgfpathlineto{\pgfqpoint{3.074667in}{1.408112in}}%
\pgfpathlineto{\pgfqpoint{3.112000in}{1.410898in}}%
\pgfpathlineto{\pgfqpoint{3.149333in}{1.415604in}}%
\pgfpathlineto{\pgfqpoint{3.194350in}{1.424000in}}%
\pgfpathlineto{\pgfqpoint{3.224000in}{1.431092in}}%
\pgfpathlineto{\pgfqpoint{3.261333in}{1.442055in}}%
\pgfpathlineto{\pgfqpoint{3.314161in}{1.461333in}}%
\pgfpathlineto{\pgfqpoint{3.336000in}{1.470584in}}%
\pgfpathlineto{\pgfqpoint{3.391999in}{1.498667in}}%
\pgfpathlineto{\pgfqpoint{3.410667in}{1.509312in}}%
\pgfpathlineto{\pgfqpoint{3.452417in}{1.536000in}}%
\pgfpathlineto{\pgfqpoint{3.501670in}{1.573333in}}%
\pgfpathlineto{\pgfqpoint{3.543585in}{1.610667in}}%
\pgfpathlineto{\pgfqpoint{3.579702in}{1.648000in}}%
\pgfpathlineto{\pgfqpoint{3.611143in}{1.685333in}}%
\pgfpathlineto{\pgfqpoint{3.638738in}{1.722667in}}%
\pgfpathlineto{\pgfqpoint{3.672000in}{1.776116in}}%
\pgfpathlineto{\pgfqpoint{3.683630in}{1.797333in}}%
\pgfpathlineto{\pgfqpoint{3.709333in}{1.851950in}}%
\pgfpathlineto{\pgfqpoint{3.717558in}{1.872000in}}%
\pgfpathlineto{\pgfqpoint{3.730811in}{1.909333in}}%
\pgfpathlineto{\pgfqpoint{3.746667in}{1.965740in}}%
\pgfpathlineto{\pgfqpoint{3.750952in}{1.984000in}}%
\pgfpathlineto{\pgfqpoint{3.757778in}{2.021333in}}%
\pgfpathlineto{\pgfqpoint{3.762667in}{2.058667in}}%
\pgfpathlineto{\pgfqpoint{3.765630in}{2.096000in}}%
\pgfpathlineto{\pgfqpoint{3.766672in}{2.133333in}}%
\pgfpathlineto{\pgfqpoint{3.765797in}{2.170667in}}%
\pgfpathlineto{\pgfqpoint{3.763001in}{2.208000in}}%
\pgfpathlineto{\pgfqpoint{3.758280in}{2.245333in}}%
\pgfpathlineto{\pgfqpoint{3.746667in}{2.304195in}}%
\pgfpathlineto{\pgfqpoint{3.742868in}{2.320000in}}%
\pgfpathlineto{\pgfqpoint{3.731866in}{2.357333in}}%
\pgfpathlineto{\pgfqpoint{3.718794in}{2.394667in}}%
\pgfpathlineto{\pgfqpoint{3.703372in}{2.432000in}}%
\pgfpathlineto{\pgfqpoint{3.685307in}{2.469333in}}%
\pgfpathlineto{\pgfqpoint{3.664642in}{2.506667in}}%
\pgfpathlineto{\pgfqpoint{3.634667in}{2.552962in}}%
\pgfpathlineto{\pgfqpoint{3.597333in}{2.601579in}}%
\pgfpathlineto{\pgfqpoint{3.582620in}{2.618667in}}%
\pgfpathlineto{\pgfqpoint{3.546959in}{2.656000in}}%
\pgfpathlineto{\pgfqpoint{3.505579in}{2.693333in}}%
\pgfpathlineto{\pgfqpoint{3.485333in}{2.709666in}}%
\pgfpathlineto{\pgfqpoint{3.448000in}{2.736916in}}%
\pgfpathlineto{\pgfqpoint{3.397832in}{2.768000in}}%
\pgfpathlineto{\pgfqpoint{3.373333in}{2.781307in}}%
\pgfpathlineto{\pgfqpoint{3.321997in}{2.805333in}}%
\pgfpathlineto{\pgfqpoint{3.298667in}{2.814794in}}%
\pgfpathlineto{\pgfqpoint{3.261333in}{2.827866in}}%
\pgfpathlineto{\pgfqpoint{3.208195in}{2.842667in}}%
\pgfpathlineto{\pgfqpoint{3.186667in}{2.847623in}}%
\pgfpathlineto{\pgfqpoint{3.149333in}{2.854280in}}%
\pgfpathlineto{\pgfqpoint{3.112000in}{2.859001in}}%
\pgfpathlineto{\pgfqpoint{3.074667in}{2.861797in}}%
\pgfpathlineto{\pgfqpoint{3.037333in}{2.862672in}}%
\pgfpathlineto{\pgfqpoint{3.000000in}{2.861630in}}%
\pgfpathlineto{\pgfqpoint{2.962667in}{2.858667in}}%
\pgfpathlineto{\pgfqpoint{2.925333in}{2.853778in}}%
\pgfpathlineto{\pgfqpoint{2.869740in}{2.842667in}}%
\pgfpathlineto{\pgfqpoint{2.850667in}{2.837993in}}%
\pgfpathlineto{\pgfqpoint{2.813333in}{2.826811in}}%
\pgfpathlineto{\pgfqpoint{2.755950in}{2.805333in}}%
\pgfpathlineto{\pgfqpoint{2.738667in}{2.797890in}}%
\pgfpathlineto{\pgfqpoint{2.701333in}{2.779630in}}%
\pgfpathlineto{\pgfqpoint{2.664000in}{2.758676in}}%
\pgfpathlineto{\pgfqpoint{2.620873in}{2.730667in}}%
\pgfpathlineto{\pgfqpoint{2.572333in}{2.693333in}}%
\pgfpathlineto{\pgfqpoint{2.552000in}{2.675702in}}%
\pgfpathlineto{\pgfqpoint{2.514667in}{2.639585in}}%
\pgfpathlineto{\pgfqpoint{2.477333in}{2.597670in}}%
\pgfpathlineto{\pgfqpoint{2.464228in}{2.581333in}}%
\pgfpathlineto{\pgfqpoint{2.436936in}{2.544000in}}%
\pgfpathlineto{\pgfqpoint{2.402667in}{2.487999in}}%
\pgfpathlineto{\pgfqpoint{2.392578in}{2.469333in}}%
\pgfpathlineto{\pgfqpoint{2.374584in}{2.432000in}}%
\pgfpathlineto{\pgfqpoint{2.359080in}{2.394667in}}%
\pgfpathlineto{\pgfqpoint{2.346055in}{2.357333in}}%
\pgfpathlineto{\pgfqpoint{2.335092in}{2.320000in}}%
\pgfpathlineto{\pgfqpoint{2.326239in}{2.282667in}}%
\pgfpathlineto{\pgfqpoint{2.319604in}{2.245333in}}%
\pgfpathlineto{\pgfqpoint{2.314898in}{2.208000in}}%
\pgfpathlineto{\pgfqpoint{2.312112in}{2.170667in}}%
\pgfpathlineto{\pgfqpoint{2.311239in}{2.133333in}}%
\pgfpathlineto{\pgfqpoint{2.312278in}{2.096000in}}%
\pgfpathlineto{\pgfqpoint{2.315231in}{2.058667in}}%
\pgfpathlineto{\pgfqpoint{2.320104in}{2.021333in}}%
\pgfpathlineto{\pgfqpoint{2.328000in}{1.979324in}}%
\pgfpathlineto{\pgfqpoint{2.335964in}{1.946667in}}%
\pgfpathlineto{\pgfqpoint{2.347105in}{1.909333in}}%
\pgfpathlineto{\pgfqpoint{2.365333in}{1.859699in}}%
\pgfpathlineto{\pgfqpoint{2.376061in}{1.834667in}}%
\pgfpathlineto{\pgfqpoint{2.402667in}{1.781900in}}%
\pgfpathlineto{\pgfqpoint{2.415270in}{1.760000in}}%
\pgfpathlineto{\pgfqpoint{2.440000in}{1.721386in}}%
\pgfpathlineto{\pgfqpoint{2.477333in}{1.672218in}}%
\pgfpathlineto{\pgfqpoint{2.514667in}{1.630331in}}%
\pgfpathlineto{\pgfqpoint{2.552000in}{1.594207in}}%
\pgfpathlineto{\pgfqpoint{2.589333in}{1.562739in}}%
\pgfpathlineto{\pgfqpoint{2.626667in}{1.535105in}}%
\pgfpathlineto{\pgfqpoint{2.664000in}{1.511270in}}%
\pgfpathlineto{\pgfqpoint{2.701333in}{1.490248in}}%
\pgfpathlineto{\pgfqpoint{2.738667in}{1.472061in}}%
\pgfpathlineto{\pgfqpoint{2.776000in}{1.456311in}}%
\pgfpathlineto{\pgfqpoint{2.813333in}{1.443105in}}%
\pgfpathlineto{\pgfqpoint{2.850667in}{1.431964in}}%
\pgfpathlineto{\pgfqpoint{2.888000in}{1.422907in}}%
\pgfpathlineto{\pgfqpoint{2.888000in}{1.422907in}}%
\pgfusepath{stroke}%
\end{pgfscope}%
\begin{pgfscope}%
\pgfsetrectcap%
\pgfsetmiterjoin%
\pgfsetlinewidth{0.803000pt}%
\definecolor{currentstroke}{rgb}{0.000000,0.000000,0.000000}%
\pgfsetstrokecolor{currentstroke}%
\pgfsetdash{}{0pt}%
\pgfpathmoveto{\pgfqpoint{1.432000in}{0.528000in}}%
\pgfpathlineto{\pgfqpoint{1.432000in}{4.224000in}}%
\pgfusepath{stroke}%
\end{pgfscope}%
\begin{pgfscope}%
\pgfsetrectcap%
\pgfsetmiterjoin%
\pgfsetlinewidth{0.803000pt}%
\definecolor{currentstroke}{rgb}{0.000000,0.000000,0.000000}%
\pgfsetstrokecolor{currentstroke}%
\pgfsetdash{}{0pt}%
\pgfpathmoveto{\pgfqpoint{5.128000in}{0.528000in}}%
\pgfpathlineto{\pgfqpoint{5.128000in}{4.224000in}}%
\pgfusepath{stroke}%
\end{pgfscope}%
\begin{pgfscope}%
\pgfsetrectcap%
\pgfsetmiterjoin%
\pgfsetlinewidth{0.803000pt}%
\definecolor{currentstroke}{rgb}{0.000000,0.000000,0.000000}%
\pgfsetstrokecolor{currentstroke}%
\pgfsetdash{}{0pt}%
\pgfpathmoveto{\pgfqpoint{1.432000in}{0.528000in}}%
\pgfpathlineto{\pgfqpoint{5.128000in}{0.528000in}}%
\pgfusepath{stroke}%
\end{pgfscope}%
\begin{pgfscope}%
\pgfsetrectcap%
\pgfsetmiterjoin%
\pgfsetlinewidth{0.803000pt}%
\definecolor{currentstroke}{rgb}{0.000000,0.000000,0.000000}%
\pgfsetstrokecolor{currentstroke}%
\pgfsetdash{}{0pt}%
\pgfpathmoveto{\pgfqpoint{1.432000in}{4.224000in}}%
\pgfpathlineto{\pgfqpoint{5.128000in}{4.224000in}}%
\pgfusepath{stroke}%
\end{pgfscope}%
\begin{pgfscope}%
\definecolor{textcolor}{rgb}{0.000000,0.000000,0.000000}%
\pgfsetstrokecolor{textcolor}%
\pgfsetfillcolor{textcolor}%
\pgftext[x=3.280000in,y=4.307333in,,base]{\color{textcolor}\rmfamily\fontsize{12.000000}{14.400000}\selectfont Experiment 1B: \(\displaystyle k=1\)}%
\end{pgfscope}%
\end{pgfpicture}%
\makeatother%
\endgroup%
}
  \caption{
    \label{fig:example_1b}
    First iteration of the scenario in experiment (1B) where the initial distribution is far away form the global optimal. The red dots indicate the true-elites, the black dots with white outlines indicate the ``non-elites'' evaluated from the true objective function, and the white dots with black outlines indicate the samples evaluated using the surrogate model.
  }
\end{figure}



\begin{figure}[!ht]
  % \centering
  \resizebox{0.9\columnwidth}{!}{\begin{tikzpicture}[]
\begin{axis}[height = {6cm}, legend style = {{at={(0.01,0.01)},anchor=south west}}, ylabel = {$\bar{b}_v$}, title = {Experiment 1C}, xmin = {1}, xmax = {10}, xlabel = {Iteration}, width = {10cm}]\addplot+ [mark = {none}, blue]coordinates {
(1.0, -0.002317341737835217)
(2.0, -0.004409663148327077)
(3.0, -0.005597712634610402)
(4.0, -0.005924407326871117)
(5.0, -0.006240681168005196)
(6.0, -0.006336869777484001)
(7.0, -0.00645335964644521)
(8.0, -0.006498832536568549)
(9.0, -0.006512999909560328)
(10.0, -0.006515684357317801)
};
\addlegendentry{CE-method}
\addplot+ [mark = {none}, red]coordinates {
(1.0, -0.0032523297583967304)
(2.0, -0.005055935736024911)
(3.0, -0.007254343579734129)
(4.0, -0.009425793030696876)
(5.0, -0.010381223749346389)
(6.0, -0.011697303259215135)
(7.0, -0.012359244989274664)
(8.0, -0.013041466197121826)
(9.0, -0.015631139919847138)
};
\addlegendentry{CE-surrogate}
\addplot+ [mark = {none}, green!50!black]coordinates {
(1.0, -0.003718899914679905)
(2.0, -0.005975318324663225)
(3.0, -0.008091989292448459)
(4.0, -0.009279493682692748)
(5.0, -0.010588537390066141)
(6.0, -0.011487546373826613)
(7.0, -0.012119978015104808)
(8.0, -0.012484799647570616)
(9.0, -0.013261146870849947)
(10.0, -0.014645709933698618)
};
\addlegendentry{CE-mixture}
\addplot+ [mark = {none}, dashed, blue, name path=Aplus, opacity=0.2]coordinates {
(1.0, -2.505055686322869e-5)
(2.0, -0.0006614193413984025)
(3.0, -0.0011266247205687892)
(4.0, -0.0013803909598864546)
(5.0, -0.001658432456737093)
(6.0, -0.0017166511411978603)
(7.0, -0.0017967264523970272)
(8.0, -0.001838149454357168)
(9.0, -0.0018469228764020481)
(10.0, -0.0018487907766789472)
};
\addplot+ [mark = {none}, dashed, blue, name path=Aminus, opacity=0.2]coordinates {
(1.0, -0.0046096329188072055)
(2.0, -0.00815790695525575)
(3.0, -0.010068800548652015)
(4.0, -0.01046842369385578)
(5.0, -0.010822929879273298)
(6.0, -0.010957088413770142)
(7.0, -0.011109992840493393)
(8.0, -0.011159515618779928)
(9.0, -0.011179076942718608)
(10.0, -0.011182577937956656)
};
\addplot+ [mark = {none}, dashed, red, name path=Bplus, opacity=0.2]coordinates {
(1.0, -0.0005911030782760943)
(2.0, -0.001976609661316075)
(3.0, -0.0031050508422256345)
(4.0, -0.0051624457056261884)
(5.0, -0.006323763373178248)
(6.0, -0.007750121884409469)
(7.0, -0.008445510156223423)
(8.0, -0.00921161194337762)
(9.0, -0.012753813024120047)
};
\addplot+ [mark = {none}, dashed, red, name path=Bminus, opacity=0.2]coordinates {
(1.0, -0.005913556438517367)
(2.0, -0.008135261810733747)
(3.0, -0.011403636317242623)
(4.0, -0.013689140355767564)
(5.0, -0.01443868412551453)
(6.0, -0.0156444846340208)
(7.0, -0.016272979822325905)
(8.0, -0.01687132045086603)
(9.0, -0.01850846681557423)
};
\addplot+ [mark = {none}, dashed, green!50!black, name path=Cplus, opacity=0.2]coordinates {
(1.0, -0.0006097905928366437)
(2.0, -0.0023709995253624262)
(3.0, -0.004127557677415865)
(4.0, -0.0053789760978201805)
(5.0, -0.006856131057669737)
(6.0, -0.007556847315585175)
(7.0, -0.008035559101084024)
(8.0, -0.008303268643850135)
(9.0, -0.00918373936480171)
(10.0, -0.009792935462045668)
};
\addplot+ [mark = {none}, dashed, green!50!black, name path=Cminus, opacity=0.2]coordinates {
(1.0, -0.006828009236523167)
(2.0, -0.009579637123964025)
(3.0, -0.012056420907481052)
(4.0, -0.013180011267565316)
(5.0, -0.014320943722462546)
(6.0, -0.015418245432068052)
(7.0, -0.016204396929125592)
(8.0, -0.016666330651291097)
(9.0, -0.017338554376898185)
(10.0, -0.019498484405351568)
};
\addplot[blue!80, fill opacity=0.1] fill between[of=Aplus and Aminus];
                    \addplot[red!80, fill opacity=0.1] fill between[of=Bplus and Bminus];
                    \addplot[green!80, fill opacity=0.1] fill between[of=Cplus and Cminus];
\end{axis}

\end{tikzpicture}
}
  \caption{
    \label{fig:experiment_1c}
    Average optimal value for experiment (1C) when we restrict the number of objective function calls.
  }
\end{figure}


Given the same centered mean as before, when we restrict the number of objective function calls even further to just 50 we see interesting behavior.
Notice that the results of experiment (1C) shown in \cref{fig:experiment_1c} follow a curve closer to the far away mean from experiment (1B) than from the same setup as experiment (1A). Also notice that the CE-surrogate results cap out at iteration 9 due to the evaluation schedule front-loading the objective function calls, thus leaving none for the final iteration (while still maintaining the same total number of evaluations of 50).




\section{Conclusion} \label{sec:conclusion}
We presented variants of the popular cross-entropy method for optimization of objective functions with multiple local minima.
Using a Gaussian processes-based surrogate model, we can use the same number of true objective function evaluations and achieve better performance than the standard CE-method on average.
We also explored the use of a Gaussian mixture model to help find global minimum in multimodal objective functions.
We introduce a parameterized test objective function with a controllable global minimum and spread of local minima.
Using this test function, we showed that the CE-surrogate algorithm achieves the best performance relative to the standard CE-method, each using the same number of true objective function evaluations.


% \printbibliography