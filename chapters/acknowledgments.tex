% \begin{itemize}
%     \item Thank Mykel.
%     \begin{itemize}
%         \item Saw inspiration in 2012 at MIT Lincoln Laboratory
%         \item Always available
%         \item High integrity
%         \item An model leader
%     \end{itemize}
%     \item Thank Dorsa for being a secondary adviser.
%     \item Thank Eva Moss.
%     \begin{itemize}
%         \item Supportive
%         \item Interested
%         \item Flexible
%     \end{itemize}
%     \begin{itemize}
%         \item SISL: Ritchie (for AST), Anthony, Mark, Shushman Choudhury, Jayesh Gupta, Bernard Lange, Alex Koufos, Sydney Katz, SISL as a whole. Zachary Sunberg for POMDPs.jl and MCTS.jl. Tomer Arnon (CEM).
%     \end{itemize}
%     \begin{itemize}
%         \item Sponsors: GE (Jerry, Joachim, Nick, etc.)
%         \item NASA Ames: Edward Balaban.
%         \item AI Center for Safety
%     \end{itemize}
% \end{itemize}

I would especially like to thank my advisor and friend, Mykel Kochenderfer, for his continued guidance over the past decade.
Our time together at MIT Lincoln Laboratory has certainly shaped me into the researcher am I today.
The positivity in Mykel's leadership is inspirational and his high level of integrity and honesty encouraged me to always do my best.
He is a model leader and I thank him for the incredible opportunity both at MIT Lincoln Laboratory at here at Stanford.
I'd also like to thank Professor Dorsa Sadigh for being my secondary research adviser and for her advice regarding this thesis.

As with many theses, I am standing on the shoulders of giants.
I would like to thank Dr. Ritchie Lee from NASA Ames for his original development of the adaptive stress testing idea and for his patience and guidance as he helped shape my ideas.
I want to thank Dr. Anthony Corso, Dr. Mark Koren, and Dr. Alex Koufos for always listening to my ideas, encouraging my excitement in the AI safety field and always providing constructive feedback. Without their advice, this work would not have been possible.
I'd also like to thank members of the Stanford Intelligent Systems Laboratory (SISL) for their encouragement and willingness to listen; particularly Bernard Lange, Dr. Shushman Choudhury, Dr. Jayesh Gupta, Sydney Katz, and Tomer Arnon.
Because this work is built off of other open source tools, I'm forever indebted to the SISL members that developed the POMDPs.jl ecosystem; this includes Dr. Zachary Sunberg, Maxim Egorov, and Dr. Tim Wheeler.
I want to extend a thank you to Dr. Edward Balaban at NASA Ames for the opportunity to work on a decision making under uncertainty system in a high-profile NASA mission.

% Episodic AST
Part of this work had the support from GE's Global Research Center and GE Aviation through the Stanford Center for AI Safety.
I want to thank each of these organizations for their fascinating problems and allowing me to explore research ideas that fit not only my interests but had large industrial impacts.
I also want to thank the NASA AOSP System-Wide Safety Project for partially supporting this work and Dr. Jerry Lopez, Nicholas Visser, and Joachim Hochwarth for their engineering guidance.

My family and friends have always been there for me, even as we are physically distant.
My Mom, Dad, brothers Travis and Jake, and sister Emily are a big reason I have core values that have helped me succeed.
Their love and support is infinite and I could not thank you enough for the life you've provided for me.
Everyone back in Rockport, MA and beyond have seen me grown through every phase in my life, and that bond is irreplaceable; so thank you.

Lastly---but most importantly---I want to thank my wife, Eva Moss, for always being supportive and growing with me during my graduate studies.
Eva, you always make me laugh and smile and have shaped me into a better person because of it.
Your logical thinking always helps me to check my opinions at the door.
Your flexibility in leaving our home back in Massachusetts and moving out to California tremendously helped in reducing the stress of graduate school---I love you and I am forever grateful.